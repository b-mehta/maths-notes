\documentclass{article}

\usepackage{inputenc}
\usepackage{amsmath}
\usepackage{amsthm}
\usepackage{amssymb}
\usepackage{enumerate}
\usepackage{mathtools}
% \usepackage{graphicsx}
\usepackage{xcolor}
\usepackage{wrapfig}
\usepackage{float}
\usepackage{hyperref}
\usepackage{hypcap}
\usepackage{geometry}

\title{Graph Theory}
\author{Russell, notes taken by Bhavik Mehta}

\begin{document}
\maketitle

\section{Introduction}

\subsection{Preliminary}
This course has no prerequistes, being the first course in the tripos in this field.  However, introductory facts about, say, eigenvalues and limits will be used.
As usual, books are not required, but the most relevant is Modern Graph Theory, B. Bollobas


This course consists of six chapters: corresponding to paragraphs 2 to 7 in the schedules but in a different order
Paragraph 1: Introductory - split between chapters

\subsection{Informal definitions}
A \emph{graph} consists of `vertices' with some pairs of vertices joined by `edges'. 
(drawing goes here)
% do the tikz bs later

In this course, we make the following assumptions:
the number of vertices is finite
no `multiple edges' - every pair of vertices is connected by at most one edge
no `loops' - no vertex can be joined to itself - edges have to go between different vertices

\subsection{Where do such structures arise?}
konigsberg (insert thing here)

Problem: can we walk across each bridge precisely once, returning to our starting point?

(Multigraph) The vertices of this graph are the four bits of the city, while the edges are the seven bridges
(insert other thing here)
In terms of this graph, the problem becomes to walk around the edges of this graph precisely once, and return to the starting vertex

example two
Map colouring problem (1850s)
How many colours are required to colour a map such that neighbouring countries get different colours?
(picture)
Here, the vertices are countries, and the edges are neighbouring pairs.
If we can draw a graph with no crossing edges, how many colours are needed to colour vertices with all edges having two different colours?

example three
Cosets of finite subgroups of finite groups

Let $G$ be a finite group, $H \leq G$, $|G : H| = n$.  By Lagrange, $\exists a_1, \dots, a_n \in G$ such that the left cosets of $H$ are $a_1 H, \dots a_n H$.  Similarly, $\exists b_1, \dots, b_n \in G$ such that the right cosets of $H$ are $H a_1, \dots, H a_n$.  
Can we do this simultaneously? In particular, are there $a_1, \dots, a_n \in G$ such that $a_1 H, \dots, a_n H \in G$ are the left cosets, and $a_1, \dots, a_n \in G$ are the right cosets? 

In the case where $H \tri G$, this is easy. We have $a H = H a \forall a \in G$.  However it is less obvious if $H \notri G$.  Let $L$ be the set of left cosets, and let $R$ be the set of right cosets. We can create a graph where the vertices are $L \cup R$.  Join $X \in L$ to $Y \in R$ by an edge if we can find some $a$ such that $X = aH$ and $Y = Ha$, that is, if $X$ and $Y$ have a commmon element.  Ideally, we would like a graph like this
(insert bipartite graph here)

Formally, we ask: In this graph, can we find a set of edges meeting each vertex precisely once?

example four
Fermat equation mod p
Consider the equation $x^n + y^n = z^n$. Does this have solutions mod $p$ for $p$ a prime?  Rule out trivial solutions such as $x = 0, y = z$, in particular we such for solutions in $\mathbb{Z}_p$ where $xyz \ne 0$.
Let $G = \mathbb{Z} \\ {0}$ under multiplication, and $H = \{g^n : g \in G\}$.  We can check that $H \leq G$, and that $|G : H| \leq n$, by considering the number of $n$th roots an element can have.  So, $G$ can be partitioned into $g_1 H, g_2 H, \dots, g_mH$ for some $g_1, g_2, \dots, g_m$ and $m \leq n$.  Suppose in some $g_i H$, we have $a, b, c$ with $a + b = c$.  Then $a = g_i x^n, b = g_i y^n, c = g_i z^n$ for some $x, y, z \in G$.  Then
\begin{align}
  g_i x^n + g_i y^n = g_i z^n 
  x^n + y^n = z^n
\end{align}

It finally remains to show Schur's Theorem:
(thm) Let $n$ be a positive integer. Then if $p$ is a sufficiently large positive integer, whenever $\{1, 2, \dots, p\}$ is parititioned into $n$ parts, we can solve $a + b = c$ with $a, b, c$ all in some part.

\end{document}