\documentclass{article}

\def\npart {II}
\def\nyear {2017}
\def\nterm {Michaelmas}
\def\nlecturer{Dr H.\ Wilton}
\def\ncourse{Algebraic Topology}
\ifx \nauthor\undefined
  \def\nauthor{Bhavik Mehta}
\else
\fi

\author{Based on lectures by \nlecturer \\\small Notes taken by \nauthor}
\date{\nterm\ \nyear}
\title{Part \npart\ -- \ncourse}

\usepackage[utf8]{inputenc}
\usepackage{amsmath}
\usepackage{amsthm}
\usepackage{amssymb}
\usepackage{enumerate}
\usepackage{mathtools}
\usepackage{graphicx}
\usepackage[dvipsnames]{xcolor}
\usepackage{tikz}
\usepackage{wrapfig}
\usepackage{centernot}
\usepackage{float}
\usepackage{braket}
\usepackage[hypcap=true]{caption}
\usepackage{enumitem}
\usepackage[colorlinks=true, linkcolor=mblue]{hyperref}
\usepackage[nameinlink,noabbrev]{cleveref}
\usepackage{nameref}
\usepackage[margin=1.5in]{geometry}

% Theorems
\theoremstyle{definition}
\newtheorem*{aim}{Aim}
\newtheorem*{axiom}{Axiom}
\newtheorem*{claim}{Claim}
\newtheorem*{cor}{Corollary}
\newtheorem*{conjecture}{Conjecture}
\newtheorem*{defi}{Definition}
\newtheorem*{eg}{Example}
\newtheorem*{ex}{Exercise}
\newtheorem*{fact}{Fact}
\newtheorem*{law}{Law}
\newtheorem*{lemma}{Lemma}
\newtheorem*{notation}{Notation}
\newtheorem*{prop}{Proposition}
\newtheorem*{question}{Question}
\newtheorem*{rrule}{Rule}
\newtheorem*{thm}{Theorem}
\newtheorem*{assumption}{Assumption}

\newtheorem*{remark}{Remark}
\newtheorem*{warning}{Warning}
\newtheorem*{exercise}{Exercise}

% \newcommand{\nthmautorefname}{Theorem}

\newtheorem{nthm}{Theorem}[section]
\newtheorem{nlemma}[nthm]{Lemma}
\newtheorem{nprop}[nthm]{Proposition}
\newtheorem{ncor}[nthm]{Corollary}
\newtheorem{ndef}[nthm]{Definition}

% Special sets
\newcommand{\C}{\mathbb{C}}
\newcommand{\N}{\mathbb{N}}
\newcommand{\Q}{\mathbb{Q}}
\newcommand{\R}{\mathbb{R}}
\newcommand{\Z}{\mathbb{Z}}

\newcommand{\abs}[1]{\left\lvert #1\right\rvert}
\newcommand{\norm}[1]{\left\lVert #1\right\rVert}
\renewcommand{\vec}[1]{\boldsymbol{\mathbf{#1}}}

\let\Im\relax
\let\Re\relax

\DeclareMathOperator{\Im}{Im}
\DeclareMathOperator{\Re}{Re}
\DeclareMathOperator{\id}{id}

\definecolor{mblue}{rgb}{0., 0.05, 0.6}


\usetikzlibrary{knots}

% preamble
\newcommand{\id}{\mathrm{id}}
\numberwithin{nthm}{subsection}
% and here we go!

% hjrw2
% orw notes
% hatchers book
\begin{document}
\maketitle
\section{Introduction}
% trefoil picture goes here (trefoil ~> unknot)

% \begin{figure}[!h]
%     \centering
%     \begin{tikzpicture}
%         \begin{knot}[
%           clip width=6,
%           consider self intersections=true,
%           flip crossing=2,
%           ]
%           \strand[ultra thick, red] (0,2)   .. controls +(2.5,0)   and +(120:-2.5) ..
%                                     (210:2) .. controls +(120:2.5) and +(60:2.5)  ..
%                                     (-30:2) .. controls +(60:-2.5) and +(-2.5,0)  ..
%                                     (0,2);
%         \end{knot}
%     \end{tikzpicture}
%     \caption{Trefoil diagram}
% \end{figure}
\paragraph{Question 1} Is the trefoil really a knot?

How do we make this precise?
We can think of the trefoil and the unknot as continuous embeddings
\begin{equation*}
    f_i: S^1 \hookrightarrow \R^3
\end{equation*}
where $f_0$ corresponds to the trefoil, and $f_1$ corresponds to the unknot.

\paragraph{Question 1 (precise)} Is there a continuous map
\begin{equation*}
    F:S^1 \times [0, 1] \to \R^3
\end{equation*}
such that
\begin{enumerate}[(i)]
    \item $F(\theta, 0) = f_0(\theta)$, $F(\theta, 1) = f_1(\theta)$ $\forall \theta \in S^1$
    \item $F(\cdot, t_0): S^1 \to \R^3$ is injective, $\forall t_0 \in [0, 1]$
\end{enumerate}

We phrased unknotting as an \emph{extension problem}.  We can think about other, easier to state extension problems.
\paragraph{Question 2}
Consider $S^2 = \set{\vec{x} \in \R^3 | \norm{x} = 1}$, and $D^3 = \Set{\vec{x} \in \R^3 | \norm{x} \leq 1}$ along with the natural inclusion $i: S^2 \hookrightarrow D^3$.
Does there exist a continuous $f: D^3 \to S^2$ such that $f \circ i = \id_{S^2}$?
% picture if i can be bothered

This seems like a hard question!  So, let's instead consider this analogous question in algebra.
We take $S^2$ analogous to $\Z$, and $D^3$ analogous to $\{0\}$, the trivial group. From here, we get another question:
\paragraph{Question 3} Does there exist a group homomorphism $f: \{0\} \to \Z$ such that $\id_\Z = f \circ 0$?

This question seems much easier to solve! This is the essence of algebraic topology - we turn difficult questions in topology into easy questions about algebra.

\subsection{Examples and conventions}
Zoo of examples

\begin{itemize}
    \item Point $*$
    \item Circle $S^1$
    \item The circle generalises to the $n$-sphere, $S^{n-1} = \Set{x \in \R^n | \norm{x} = 1}$
    \item But it also can be used to produce the torus $T^2 = S^1 \times S^1$.
        % more damn pictures
    \item Which itself generalises into the genus-$g$ surface, $\Sigma_g$
    \item The torus can alternatively be found by identifying edges of a square, but by identifying them differently we can make a Klein bottle $K$
    \item Or the real projective plane $\mathbb{RP}^2$
\end{itemize}

In this course, the term map always refers to a continuous map.
To check that a map is continuous, we'll almost always use
\begin{lemma}[The Gluing Lemma]
    If $X = C_1 \cup C_2$, with $C_i$ closed, and $f: X \to Y$ is a function such that $f|_{C_1}$ and $f|_{C_2}$ are both continuous, then $f$ is continuous.
\end{lemma}

\begin{proof}
    Let $D \subseteq Y$ be closed.  Then
    \begin{equation*}
        f^{-1}(D) = (f|_{C_1})^{-1} (D) \cup (f|_{C_2})^{-1} (D)
    \end{equation*}
    is a finite union of closed sets, hence is closed.  Therefore, $f$ is continuous.
\end{proof}

\subsection{Cell complexes}
There are the spaces we can build by gluing.  The \emph{basic operation} we use is to attach an $n$-dimensional disc, as shown in the non-existent diagram.
Formally, if $f: S^{n-1} \to X$ is a continuous map, then we produce
\begin{equation}
    (X \sqcup D^n) / \sim
\end{equation}
where $\sim$ is an equivalence relation: $Y \sim f(y)$, for $y \in S^{n-1}$ and $f(y) \in X$.
\begin{defi}[Cell complexes]
    We define cell complexes by induction:
    \begin{enumerate}[(i)]
        \item A zero-dimensional cell complex is a finite discrete topological space
        \item An $n$-dimensional cell complex is constructed from an $(n-1)$ dimensional cell complex $X$ by attaching finitely many $n$-dimensional discs to $X$
    \end{enumerate}
\end{defi}

\section{Homotopy and the Fundamental Group}
\paragraph{Basic question} How can we prove that $X \ncong Y$?
The basic idea is to associate algebraic objects (for instance groups) to $X, Y$.
\clearpage
\subsection{Homotopy}
Throughout, we will denote $I = [0, 1] \subseteq \R$.

% def 2.1.1
\begin{ndef}\hypertarget{def:homotopy}
    Let $f, g: X \to Y$ be maps. A \textbf{homotopy} from $f$ to $g$ is a continuous map $H: X \times I \to Y$ such that $H(x, 0) = f(x)$ and $H(x, 1) = g(x) \; \forall x \in X$.
    We say that $f$ and $g$ are \textbf{homotopic}, and write $f \simeq g$ (or $f \simeq_H g$).
\end{ndef}

\begin{ndef}\hypertarget{def:homotopyRel}
    If $A \subseteq X$ and $\forall a \in A, \forall t \in I$, $H(a, t) = f(a) = g(a)$, then we say \emph{$f$ is homotopic to $g$ relative to $A$} and we write $f \simeq g \, \mathrm{rel} A$.
\end{ndef}

% prop 2.1.2
\begin{nprop}
    The relation $\simeq$  (rel $A$) is an equivalence relation.
\end{nprop}

\begin{proof}
    Reflexivity and symmetry are easy exercises, so they are omitted.  Transitivity: we have $f, g, h: X \to Y$, where $f \simeq_H g$ and $g \simeq_H' h$, and our goal is to build a homotopy $f \simeq h$.

    colourful pictures go here

    From the diagram, it's clear that the required homotopy is given by
    \begin{align*}
        H'' : X \times I &\to Y \\
        (x, t) &\mapsto
        \begin{cases}
            H(x, 2t) & 0 \leq t \leq \frac12 \\
            H'(x, 2t-1) & \frac12 \leq t \leq 1
        \end{cases}
    \end{align*}
    This is continuous from the gluing lemma.
\end{proof}

\begin{ndef}\hypertarget{def:homotopyEq}
    If we have $f: X \to Y$ and $g: Y \to X$ (diagram) with $g \circ f \simeq \id_X$ and $f \circ g \simeq \id_Y$, then we say that $X$ and $Y$ are \textbf{homotopy equivalent}, and $f, g$ are \textbf{homotopy equivalences}, and we write $X \simeq Y$.
\end{ndef}

\begin{eg}
    Take $X = \R^n$ and $Y = *$. Consider the trivial map $r: \R^n \to *$ and $i: * \to \R^n$, $i(*) = 0$. Then $r \circ i = \id_*$ and $i \circ r = 0: \R^n \to \R^n$
    Let $H: R^n \times I \to \R^n$ $(x, t) \mapsto tx$, so for $t=0$, H(., 0) = 0 and for $t=1$, $H(., 1) = \id_{\R^n}$.
\end{eg}

\begin{eg}
    This time, consider the spaces $\R^2 \setminus {0}$ and $S^1$, and the maps $r(x) = \frac{x}{\norm{x}}$  and inclusion $i$.  Then $r \circ i = \id_{S^1}$ and we can use the homotopy $H(x, t) = tx + (1-t) \frac{x}{\norm{x}}$ to show $i \circ r \simeq \id_{\R^2 \setminus {0}}$.  So, $\R^2 \setminus \{0\}$ and $S^1$ are homotopy equivalent.
\end{eg}

\begin{ndef}
    If $X$ is homotopy equivalent to a point, we say that $X$ is \textbf{contractible}.
\end{ndef}

% def 2.1.7
\begin{nlemma}
    If
    [diagram]
    are maps with $f_0 \simeq_H f_1$ and $g_0 \simeq_H'$ then $g_0 \circ f_0 \simeq g_1 \circ f_1$.
\end{nlemma}

\begin{proof}
    We show that both are homotopic to $g_0 \circ f_1$.  $g_0 \circ f_0 \simeq g_0 \circ f_1$ via $g_0 \circ H$ and $g_0 \circ f_1 \simeq g_1 \circ f_1$ via $H'(f_1(\cdot), \cdot)$.
\end{proof}

% prop 2.1.8
\begin{prop}
    Homotopy equivalence is an equivalence relation of spaces.
\end{prop}

\begin{proof}
    Symmetry and reflexivity are trivial. Transitivity [diagram]
    Suppose these are homotopy equivalences. We need to show $g \circ g' \circ f' \circ f \simeq \id_X$ and $f' \circ f \circ g \circ g' \simeq \id_Z$.
    Now,
    \begin{equation*}
        g \circ (g' \circ f') \circ f \simeq g \circ \id_Y \circ f \simeq g \circ f \simeq \id_X \\
    \end{equation*}
    and
    \begin{equation*}
        f' \circ (f \circ g) \circ g' \simeq f' \circ \id_Y \circ g' \simeq f' \circ g' \simeq \id_Z \\
    \end{equation*}
\end{proof}
\end{document}

