\documentclass{article}

\def\npart {II}
\def\nyear {2017}
\def\nterm {Michaelmas}
\def\nlecturer{Dr H.\ Wilton}
\def\ncourse{Galois Theory}
\ifx \nauthor\undefined
  \def\nauthor{Bhavik Mehta}
\else
\fi

\author{Based on lectures by \nlecturer \\\small Notes taken by \nauthor}
\date{\nterm\ \nyear}
\title{Part \npart\ -- \ncourse}

\usepackage[utf8]{inputenc}
\usepackage{amsmath}
\usepackage{amsthm}
\usepackage{amssymb}
\usepackage{enumerate}
\usepackage{mathtools}
\usepackage{graphicx}
\usepackage[dvipsnames]{xcolor}
\usepackage{tikz}
\usepackage{wrapfig}
\usepackage{centernot}
\usepackage{float}
\usepackage{braket}
\usepackage[hypcap=true]{caption}
\usepackage{enumitem}
\usepackage[colorlinks=true, linkcolor=mblue]{hyperref}
\usepackage[nameinlink,noabbrev]{cleveref}
\usepackage{nameref}
\usepackage[margin=1.5in]{geometry}

% Theorems
\theoremstyle{definition}
\newtheorem*{aim}{Aim}
\newtheorem*{axiom}{Axiom}
\newtheorem*{claim}{Claim}
\newtheorem*{cor}{Corollary}
\newtheorem*{conjecture}{Conjecture}
\newtheorem*{defi}{Definition}
\newtheorem*{eg}{Example}
\newtheorem*{ex}{Exercise}
\newtheorem*{fact}{Fact}
\newtheorem*{law}{Law}
\newtheorem*{lemma}{Lemma}
\newtheorem*{notation}{Notation}
\newtheorem*{prop}{Proposition}
\newtheorem*{question}{Question}
\newtheorem*{rrule}{Rule}
\newtheorem*{thm}{Theorem}
\newtheorem*{assumption}{Assumption}

\newtheorem*{remark}{Remark}
\newtheorem*{warning}{Warning}
\newtheorem*{exercise}{Exercise}

% \newcommand{\nthmautorefname}{Theorem}

\newtheorem{nthm}{Theorem}[section]
\newtheorem{nlemma}[nthm]{Lemma}
\newtheorem{nprop}[nthm]{Proposition}
\newtheorem{ncor}[nthm]{Corollary}
\newtheorem{ndef}[nthm]{Definition}

% Special sets
\newcommand{\C}{\mathbb{C}}
\newcommand{\N}{\mathbb{N}}
\newcommand{\Q}{\mathbb{Q}}
\newcommand{\R}{\mathbb{R}}
\newcommand{\Z}{\mathbb{Z}}

\newcommand{\abs}[1]{\left\lvert #1\right\rvert}
\newcommand{\norm}[1]{\left\lVert #1\right\rVert}
\renewcommand{\vec}[1]{\boldsymbol{\mathbf{#1}}}

\let\Im\relax
\let\Re\relax

\DeclareMathOperator{\Im}{Im}
\DeclareMathOperator{\Re}{Re}
\DeclareMathOperator{\id}{id}

\definecolor{mblue}{rgb}{0., 0.05, 0.6}


% preamble
\setcounter{section}{-1}
% and here we go!

\begin{document}
\maketitle

\section{Introduction}
The primary motivation of this course is to study the solutions of polynomial equations in one variable to wander whether there is a formula involving roots, a solution by radicals.  Quadratics were typically studied in school, while the solution in radicals for cubics and quartics has been known for a long time and studied in particular in 1770 by Lagrange.

In 1799, Ruffini claimed that there were some quintics that can't be solved by radicals, that is, there is no general formula, but it took until 1824 before Abel used existing ideas about permutations to produce the first accepted proof of insolubility, before dying in 1829.  Galois' main contribution was in 1831, when he gave the first explanation as to why some polynomials are soluble by radicals and others are not. He made use of the group of permutations of the roots of a polynomial, and realised in particular the importance of \emph{normal} subgroups.

Galois' work was not known generally in his lifetime - it was only published by Liouville in 1846, who realised that it tied in well with the work of Cauchy on permutations. Galois had submitted his work for various competitions and for entry into the Ecole Polytechnique in Paris.  Unfortunately Galois died in a duel in 1832, leaving a six and a half page letter indicating his thoughts about the future development of his theory.

\subsection{Course overview}
Most of this course is Galois Theory, but presented in a more modern fashion, namely in terms of field extensions.  Recall from GRM that if $f(t)$ is an irreducible polynomial in $k[t]$ where $k$ is a field, then $k[t]/(f(t))$ is a field, where $f(t)$ denotes the ideal of $k[t]$ generated by $f(t)$, and this new field contains $k$.  In this way, we can see the field $k[t]/(f(t))$ as a field extension of $k$.

Galois' papers have been studied by Peter Neumann:
The math writings of Evariste Galois, European Math Soc
% TODO: sort this out with sensible formatting
Different books: I. Steward Galois Theory, (something) and Hall
contains a historcal introduction and covers almost all the syllabus.
Artin Galois Theory
Van der Waerden Modern Algebra (covers a lot more than Galois theory)
Lang Algebra (late editions are preferred, covers a lot of algebra)
Kaplansky Fields and Rings

\paragraph{Prerequisites} Quite a lot of GRM, but no modules except in one place where it's useful to know the structure of finite abelian groups. The dpmms website has a Galois Theory page - long history of example sheets and notes, in particular see Tony Scholl's 2013-4 course page - summary of the course.

\section{Field Extensions}
\begin{nthm}
  A field extension $K \leq L$ is the inclusion of a field $K$ into another field $L$ (with the same $0$, $1$, and restriction of $+$ and $.$ (in L) to $K$ gives the $+$ and $.$ of $K$).
\end{nthm}

(examples): $\Q \leq \R$
$\R \leq \C$
$\Q \leq \Q(\sqrt{2}) = \{\lambda + \mu \sqrt{2} \vert \lambda, \mu \in \Q\}$

$\{\lambda + \mu i \vert \lambda, \mu \in \Q\} = \Q(i) \leq \C$


Suppose $K \leq L$ is a field extension. Then $L$ is a $K$-vector space using the addition form the field structure and the scalar multiplcation given by the multiplication in the field $L$.

(definition): The \emph{Degree of L over K} is $dim_k L$, the $K$-vector space dimension of $L$. This may not be finite. Notation $|L:K|$. If $|L:K| \leq \infty$, then the extension is finite. Otherwise the extension is infinite.
(examples): $|\C:R| = 2$. R-bases 1, i
|Q(i):Q| = 2. Q-basis 1, i
$Q \leq R$ is an infinite extension.
(theorem 1.3) (tower law)
Suppose $K \leq L \leq M$ are field extensions. Then $|M:K|$ = $|M:L||L:K|$.
Proof: Assume that $|M:L| \le \infty$, and $|L:K| \le \infty$
Take an $L$-basis of $M$, given by $\{f_1, ..., f_b\}$, and a $K$-basis of $L$ given by $\{e_1, ..., e_a\}$.
Take $m \in M$. Then $m = \sum_{i=1}^b \mu_i f_i$ for some $\mu_i \in L$.  Each $\mu_i = \sum_{j=1}^a \lambda_{ij} e_j$ for some $\lambda_{ij} \in K$.
\begin{equation}
    m = \sum_{i=1}^b \sum{j=1}^a \lambda_{ij} e_j f_i
\end{equation}
Thus $\{e_j f_i \vert 1 \leq j \leq a, 1 \leq i \leq a\}$ span $M$.
Linear indepenence:
    It's enough to show that if $0 = m = \sum \sum \lambda_{ij} e_j f_i$ then $\lambda_{ij}$ are all zero. However if $m = 0$ the linear independence of $f_i$ forces each $\mu_i = 0$. Then the linear indepedence of $e_j$ forces $\lambda_{ij}$ all to be zero, as required.
The tower law will not be proved for infinite extensions, but observe that if $m$ is an infinite extension of $L$, then it is an infinite extensions of $K$, and if $l$ is an finintei extension sof $K$ then the larger field $M$ must also be an ifninte extensions of $K$.
\subsection{Motivatory Example}
\begin{equation}
    \Q \leq \Q(\sqrt{2}) \leq \Q(\sqrt{2}, i)
\end{equation}
$\Q(\sqrt{2})$ has basis $1, \sqrt{2}$ over $Q$.
$\Q(\sqrt{2}, i)$ has basis $1, i$ as a $Q(\sqrt{2})$ - vector space.
$\Q(\sqrt{2}, i)$ has basis $1, \sqrt{2}, i, i\sqrt{2}$ over $Q$.
Any in
\end{document}
