\documentclass{article}

\def\npart {II}
\def\nyear {2017}
\def\nterm {Michaelmas}
\def\nlecturer{E.\ Brieuillard}
\def\ncourse{Probability and Measure}
\ifx \nauthor\undefined
  \def\nauthor{Bhavik Mehta}
\else
\fi

\author{Based on lectures by \nlecturer \\\small Notes taken by \nauthor}
\date{\nterm\ \nyear}
\title{Part \npart\ -- \ncourse}

\usepackage[utf8]{inputenc}
\usepackage{amsmath}
\usepackage{amsthm}
\usepackage{amssymb}
\usepackage{enumerate}
\usepackage{mathtools}
\usepackage{graphicx}
\usepackage[dvipsnames]{xcolor}
\usepackage{tikz}
\usepackage{wrapfig}
\usepackage{centernot}
\usepackage{float}
\usepackage{braket}
\usepackage[hypcap=true]{caption}
\usepackage{enumitem}
\usepackage[colorlinks=true, linkcolor=mblue]{hyperref}
\usepackage[nameinlink,noabbrev]{cleveref}
\usepackage{nameref}
\usepackage[margin=1.5in]{geometry}

% Theorems
\theoremstyle{definition}
\newtheorem*{aim}{Aim}
\newtheorem*{axiom}{Axiom}
\newtheorem*{claim}{Claim}
\newtheorem*{cor}{Corollary}
\newtheorem*{conjecture}{Conjecture}
\newtheorem*{defi}{Definition}
\newtheorem*{eg}{Example}
\newtheorem*{ex}{Exercise}
\newtheorem*{fact}{Fact}
\newtheorem*{law}{Law}
\newtheorem*{lemma}{Lemma}
\newtheorem*{notation}{Notation}
\newtheorem*{prop}{Proposition}
\newtheorem*{question}{Question}
\newtheorem*{rrule}{Rule}
\newtheorem*{thm}{Theorem}
\newtheorem*{assumption}{Assumption}

\newtheorem*{remark}{Remark}
\newtheorem*{warning}{Warning}
\newtheorem*{exercise}{Exercise}

% \newcommand{\nthmautorefname}{Theorem}

\newtheorem{nthm}{Theorem}[section]
\newtheorem{nlemma}[nthm]{Lemma}
\newtheorem{nprop}[nthm]{Proposition}
\newtheorem{ncor}[nthm]{Corollary}
\newtheorem{ndef}[nthm]{Definition}

% Special sets
\newcommand{\C}{\mathbb{C}}
\newcommand{\N}{\mathbb{N}}
\newcommand{\Q}{\mathbb{Q}}
\newcommand{\R}{\mathbb{R}}
\newcommand{\Z}{\mathbb{Z}}

\newcommand{\abs}[1]{\left\lvert #1\right\rvert}
\newcommand{\norm}[1]{\left\lVert #1\right\rVert}
\renewcommand{\vec}[1]{\boldsymbol{\mathbf{#1}}}

\let\Im\relax
\let\Re\relax

\DeclareMathOperator{\Im}{Im}
\DeclareMathOperator{\Re}{Re}
\DeclareMathOperator{\id}{id}

\definecolor{mblue}{rgb}{0., 0.05, 0.6}

% efjb2@cam.ac.uk

% preamble
\usepackage{bbm}
\setcounter{section}{-1}
\newcommand{\1}[1]{\mathbbm{1}_{#1}}

\renewcommand{\thenthm}{\arabic{nthm}}
% and here we go!

\begin{document}
\maketitle
\section{Introduction}
\subsection{Course structure}
\begin{itemize}
    \item Week 1: Lebesgue measure
    \item Week 2: Abstract measure theory
    \item Week 3: Integration
    \item Week 4: Foundations of probability theory
    \item Week 5: $L^p$ spaces
    \item Week 6: Modes of convergence
    \item Week 7: Fourier transform and gaussians
    \item Week 8: Ergodic Theory
\end{itemize}

\subsection{Historical motivation}
Suppose we have a subset $E \subset \R^d$.
\begin{enumerate}
    \item What does it mean to measure this subset?

        In one dimension, we have some intuition of length, and in two and three dimensions we are familiar with the notions of surface area and volume.
    \item Does it make sense to measure every subset?

        This seems reasonable, but it turns out that assigning a measure to every subset can lead to logical contradictions.
    \item What is a measure? It should be a function defined on subsets, in particular some assignment $E \to m(E) \in \R$.
        It should satisfy some properties:
        \begin{itemize}
            \item Non-negativity: $m(E) \geq 0$ for all $E$
            \item Empty set: $m(\emptyset) = 0$
            \item Additivity: $m(E \sqcup F) = m(E) + m(F)$ for any two disjoint sets $E$ and $F$
            \item Normalisation: $m([0, 1]^d) = 1$
            \item Translation invariant: $m(E + x) = m(E)$ for all $E$ and all $x \in \R^d$
        \end{itemize}
        It's possible to construct pathological `measures' satisfying all these axioms and defined \emph{on all} subsets of $\R^d$, but they won't be `nice'.
        When mathematicians construct such measures, they usually do so on a restricted class of sets, otherwise this leads to contradictions.

        If $d \geq 3$, it can be shown that there is no $m: P(\R^d) \to [0, \infty)$ that is also rotation invariant.
        This is referred to as the Hausdorff--Banach--Tarski paradox.
        Namely, if we take $B = B(0, 1) = \Set{\vec{x} \in \R^d: x_1^2 + \dots x_d^2 \leq 1}$, then there is a partition
        \begin{equation*}
            B = X_1 \sqcup \dots \sqcup X_k \sqcup Y_1 \sqcup \dots \sqcup Y_k
        \end{equation*}
        and isometries $g_1, \dotsc, g_k$, $h_1, \dotsc, h_k$ such that
        \begin{equation*}
            \bigcup g_i X_i = B = \bigcup h_i Y_i
        \end{equation*}
\end{enumerate}

\clearpage
\section{Lebesgue measure}

\begin{defi}[\hypertarget{def:boxJMeasure}{Jordan measure of a box}]
    We define the \textbf{Jordan measure of the box} by $m(B) \coloneqq \prod_{i=1}^d |b_i - a_i|$.
\end{defi}
Consider a box $B \subset \R^d$, given by $B = \prod_{i=1}^d I_i$, where $I_i = [a_i, b_i]$ are intervals in $\R$
From here, we can define the Jordan measure of the box by $m(B) \coloneqq \prod_{i=1}^d |b_i - a_i|$.

\begin{defi}\hypertarget{def:elemSubs}
    An \textbf{elementary subset} of $\R^d$ is a finite union of boxes.
\end{defi}

\begin{remark}
    Every elementary set can be written as a finite union of disjoint boxes.
    The family of elementary sets is stable under finite unions, finite intersections and set difference.
    The concern may arise that if the disjoint union can be taken in two different ways, then perhaps we could get different measures, but
    \begin{align*}
        E = \bigsqcup_{i=1}^N B_i &= \bigsqcup_{j=1}^M B'_j \\
        \implies \sum_{i=1}^N m(B_i) &= \sum_{j=1}^M m(B'_j)
    \end{align*}

    This means that \emph{makes sense} to define $m(E)$ as $\sum_1^N m(B_i)$, for elementary subsets.
\end{remark}

\begin{defi}\hypertarget{def:jMeasurable}
    A subset $E \subset \R^d$ is \textbf{Jordan measurable} if $\forall \epsilon > 0 \; \exists$ \hyperlink{def:elemSubs}{elementary sets} $A, B$ such that $A \subset E \subset B$ and $m(B\setminus A) < \epsilon$.
\end{defi}

\begin{remark}
    If $E$ is Jordan measurable, then
    \begin{equation*}
        \inf\Set{m(B)| E \subset B, B \,\; \text{elementary}} = \sup\Set{m(A)| A \subset E, A \;\, \text{elementary}}
    \end{equation*}
    Proof is left as an exercise for the reader.
\end{remark}

\begin{defi}\hypertarget{def:jMeasure}
    We define the \textbf{Jordan measure} of $E$ as this supremum or infimum, and denote it by $m(E)$.
\end{defi}

\begin{exercise}
    $m$ so defined satisfies all the axioms defined earlier.
\end{exercise}

\begin{defi}[Riemann integrable function\hypertarget{def:riemannIntegrable}]
    A function $f:[a, b] \to \R$ is \textbf{Riemann integrable} if all its Riemann sums converge.
    Formally, the integral $I(f) \in \R$ exists if $\forall \epsilon > 0$, we can find $\delta > 0$ such that for every partition $P$ of $[a, b]$ of width $\tau(P) < \delta$, we have $\abs{S(f, P) - I(f)} < \epsilon$, where we recall the following

    A partition $P$ given by $a = t_0 < t_1 < \dots < t_n = b$ has width
    \begin{equation*}
        \tau(P) = \max_{0 \leq i < N} \abs{t_{i+1} - t_i}
    \end{equation*}
    and
    \begin{equation*}
        S(f, P) = \sum_{i=1}^{N-1} f(x_i) (t_{i+1} - t_i)
    \end{equation*}
    where $x_i \in [t_i, t_{i+1}]$
\end{defi}

\begin{prop}
    $f$ is Riemann integrable if and only if
    \begin{align*}
        E^+ &= \set{(x, t) \in \R^2 | 0 \leq t \leq f(x)} \\
        E^- &= \set{(x, t) \in \R^2 | f(x) \leq t \leq 0}
    \end{align*}
    are both Jordan measurable.
\end{prop}

However the \hyperlink{def:jMeasure}{Jordan measure} is not perfect. For instance, the complement of a \hyperlink{def:jMeasurable}{Jordan measurable} set is not Jordan measurable.
Also, we can take an (infinite) union of Jordan measurable sets and produce a set which is not Jordan measurable.
In addition, there are simple sets that are not Jordan measurable, and simple functions that are not \hyperlink{def:riemannIntegrable}{Riemann integrable}.

\begin{eg}
    \leavevmode
    \begin{enumerate}[label=(\emph{\roman*})]
        \item
            \begin{equation*}
                \1{\Q}(x) = \begin{cases}
                    1 & x \in \Q \\
                    0 & \text{otherwise}
                \end{cases}
             \end{equation*}
            This is not Riemann integrable, as seen in earlier Analysis courses.
        \item $\Q$ or even $\Q \cap [0, 1]$ is not a Jordan measurable subset of $\R$.
            In fact no dense countable subset of an interval in $\R$ can be Jordan measurable.
    \end{enumerate}
\end{eg}

Another problem is with limits of functions: a pointwise limit of \hyperlink{def:riemannIntegrable}{Riemann integrable} functions is not always Riemann integrable.
\begin{eg}
    $f_n(x) = \1{\frac1{n!}\Z \cap [0, 1]}$ is Riemann integrable, but $f(x) = \lim_{n \to 0} f_n(x) = \1{\Q \cap [0, 1]}$ is not.
\end{eg}

In comparison to the \hyperlink{def:jMeasure}{Jordan measure}, the ideas of Lebesgue were to remove the containment $A \subset E$, and to allow a countable union of boxes instead of just a finite union.

We begin by defining $m^*$, the \emph{Lebesgue outer measure}:
\begin{defi}[\hypertarget{def:lebOutMeas}{Lebesgue outer measure}]
    For a subset $F \subseteq \R^d$, the \textbf{Lebesgue outer measure} $m^*$ is given by
    \begin{equation*}
        m^*(F) = \inf\Set{\sum_{n \geq 1} m(B_n) | F \subset \bigcup_{n \geq 1} B_n, \; B_n \; \text{are boxes}}
    \end{equation*}
\end{defi}

This is defined for all subsets of $\R^d$, but it is not additive on all subsets.
\begin{remark}
    \leavevmode
    \begin{itemize}
        \item If you let $m^{*, J}(F)$ (the Jordan outer measure) be defined similarly, but for finitely many boxes $B_n$ instead of countably many, then $m^*(F) \leq m^{*, J}(F)$.
        \item But the inequality can be strict, for instance for $F = \Q \cap [0, 1]$ we have $m^*(F) = 0$ but $m^{*, J}(F) = 1$.
    \end{itemize}
\end{remark}

The \hyperlink{def:lebOutMeas}{Lebesgue outer measure} satisfies
\begin{enumerate}
    \item $m^*(\emptyset) = 0$
    \item $m^*(E) \leq m^*(F)$ if $E \subseteq F$ (monotone)
    \item $m^*\left(\bigcup_{i \geq 1} E_i\right) \leq \sum_{i \geq 1} m^* (E_i)$ (countable subadditivity)
\end{enumerate}

\begin{eg}[Vitali's example]
    Let $E$ be a set of representations of the cosets of the subgroup $(\Q, +) \subset (\R, +)$. We can choose $E \subset [0, 1]$, such that
    \begin{equation*}
        \forall x \in \R \; \exists! e \in E \; \text{such that} \; x - e \in \Q
    \end{equation*}
    So the family $\{E + r\}_{r \in \Q}$ is a disjoint family of subsets of $\R$, a partition.
    By translational invariance, \begin{equation*}m^*(E + r) = m^*(E) \quad \forall r \in \Q\end{equation*}

    If $m^*$ were additive, we can consider distinct $r_1, \dotsc, r_N \in \Q \cap [0, 1]$ so
    \begin{equation*}
        m^*\left(\bigcup_{i=1}^N E + r_i\right) = N m^*(E)
    \end{equation*}
    but
    \begin{align*}
        \bigcup_{i=1}^N E + r_i &\subseteq [0, 2] \\
        \implies m^*\left(\bigcup E + r_i\right) &\leq m^*([0, 2]) \leq 2
    \end{align*}
    So for any $N \in \N$, we have $N m^*(E) \leq 2$, hence $m^*(E) = 0$.
    On the other hand, $[0, 1] \subseteq \bigcup_{r \in \Q} E + r$, so by countable subadditivity $1 = m^*([0, 1]) \leq \sum_{r \in \Q} m^*(E + r) = 0$, so we have a contradiction.
    This shows that $m^*$ is not additive on all subsets.
\end{eg}

\begin{remark} \leavevmode
    \begin{itemize}
        \item This contruction uses the axiom of choice to define $E$.
        \item We will soon define the Lebesgue measure of a Lebesgue measurable set as the \hyperlink{def:lebOutMeas}{outer measure} of that set, and this will be additive on Lebesgue measurable sets.  This means Vitali's set $E$ was \emph{not} Lebesgue measurable.
    \end{itemize}
\end{remark}

\begin{defi}[\hypertarget{def:lebMAble}{Lebesgue measurable subset}]
    $E \subseteq \R^d$ is \textbf{Lebesgue measurable} if $\forall \epsilon > 0$, $\exists C \coloneqq \bigcup_{n} B_n$, a countable union of boxes such that $m^*(C \setminus E) < \epsilon$ and $E \subseteq C$.
\end{defi}

\begin{eg} [Middle-thirds Cantor set]
    This is a compact subset $C$ of $[0, 1]$, defined as follows.  Start with the interval $I_0 = [0, 1]$ and remove the middle third $\left(\frac13, \frac23\right)$, leaving $I_1 = \left[0, \frac13\right] \cup \left[\frac23, 1\right]$.
    Then remove the middle third of each interval here, giving $I_2 = \left[0, \frac19\right] \cup \left[\frac29, \frac13\right] \cup \left[\frac23, \frac79\right] \cup \left[\frac89, 1\right]$.
    Repeat, then define $C = \bigcap_{n \geq 0} I_n$.
    Equivalently, take a ternary expansion of $x \in [0, 1]$, and define $C$ as the set of $x$ which has \emph{none} of its digits equal to $2$.
\end{eg}

\begin{remark}
    $I_n$ is a finite union of intervals, so it is an \hyperlink{def:elemSubs}{elementary set} (link).
    In particular, $m(I_n) = 2^n / 3^n \to 0$ as $n \to \infty$, so $C$ is \hyperlink{def:jMeasurable}{Jordan measurable} with measure $0$.
    Every Jordan measurable set is \hyperlink{def:lebMAble}{Lebesgue measurable} (clear from the definition).
\end{remark}

We can define a `fat' Cantor set which is \hyperlink{def:jMeasurable}{Jordan measurable} but not \hyperlink{def:lebMAble}{Lebesgue measurable}.

\begin{remark}
    What is the \hyperlink{def:lebOutMeas}{outer measure} of a Vitali set?
    $m^*(E) > 0$ by the argument given above, but we can create $E \subseteq [0, \frac1n]$, or inside any interval. So, $m^*(E)$ depends on the choice of $E$ and could be arbitrarily small, but must be positive.
\end{remark}

\begin{remark}
    In the definition of the \hyperlink{def:lebOutMeas}{Lebesgue outer measure}, we used closed boxes, but open or half-open boxes would not change the definition (same for Jordan measure).
\end{remark}

\begin{remark}
    Every null set is \hyperlink{def:lebMAble}{Lebesgue measurable}.
\end{remark}

\begin{defi}[Null set\hypertarget{def:null}]
    A \textbf{null set} $E \subseteq \R^d$ is a subset $E$ such that $m^*(E) = 0$.
\end{defi}

We state two propositions which will be proved together.

\begin{nprop}\label{prop:bigProp}
    The family $\mathcal{L}$ of \hyperlink{def:lebMAble}{Lebesgue measurable} subsets of $\R^d$ is stable under
    \begin{enumerate}[label=\alph*)]
        \item countable unions: if $E_n \in \mathcal{L}$ for all $n \in \N$, then $\bigcup_{n \geq 1} E_n \in \mathcal{L}$.
        \item complementation: if $E \in \mathcal{L}$, then $E^c \in \mathcal{L}$.
        \item countable intersections: if $E_n \in \mathcal{L}$ for all $n \in \N$, then $\bigcap_{n \geq 1} E_n \in \mathcal{L}$.
    \end{enumerate}
\end{nprop}

\begin{nprop}\label{prop:otherBigProp}
    Every closed (open) subset of $\R^d$ is \hyperlink{def:lebMAble}{Lebesgue measurable}.
\end{nprop}

First note that part c) follows from parts a) and b), since $(\bigcap_n E_n)^c = \bigcup_n E_n^c$. We'll first show a).
\begin{proof}[Proof of \cref{prop:bigProp}a)]
    Let $(E_n)_{n \geq 1}$ be a countable family in $\mathcal{L}$, and pick $\epsilon > 0$.  By definition, $\exists C_n \coloneqq \bigcup_{i \geq 1} B_i^{(n)}$ such that $m^*(C_n \setminus E_n) < \frac{\epsilon}{2^n}$.
    Then $\bigcup E_n \subseteq \bigcup C_n = \bigcup_{n, i} B_i^{(n)}$ is still a countable union of boxes.

    Finally,
    \begin{align*}
        m^*\left(\bigcup B_i^{(n)} \setminus \bigcup E_n\right) \leq \sum_n m^*(C_n \setminus E_n) \leq \sum_{n \geq 1} \frac{\epsilon}{2^n} \leq \epsilon
    \end{align*}
    by countable subadditivity of $m^*$, proving a).
\end{proof}

Next, prove a lemma which will help to prove \cref{prop:otherBigProp}.

\begin{lemma}
    Every open subset of $\R^d$ is a countable union of open boxes.
\end{lemma}

\begin{proof}
    For $r \in \Q_{\geq 0}$ and $s \in \Q^d$, let $B_{s, r} = \Set{x \in \R^d | \abs{x_i - s_i} < \frac1r \ \text{for} \ i=1,\dotsc,d}$
    $(B_{s,r})_{s,r}$ form a countable family and any open set $U \subseteq \R^d$ is the union of those $B_{s,r}$'s it contains.
\end{proof}

Now we prove open and closed sets are Lebesgue measurable.
\begin{proof}[Proof of \cref{prop:otherBigProp}]
    By part a) of \cref{prop:bigProp} and the lemma, we see that every open set in $\R^d$ is Lebesgue measurable. Now, let's show that closed sets are Lebesgue measurable.

    It's enough to show that \emph{compact} subsets of $\R^d$ are in $\mathcal{L}$ because every closed subset $F \subseteq \R^d$ is a countable union of compact sets:

    $\R^d = \bigcup_{n \geq 1} A_n$, where $A_n$ is an annulus $\set{x \in \R^d \mid n-1 \leq \norm{x} \leq n}$, so we can write $F = \bigcup_{n \geq 1} (A_n \cap F)$, where $A_n \cap F$ is compact.

    So, let $F \subseteq \R^d$ be a compact subset. By definition of $m^*(F)$, $\forall k \geq 1$, $\exists $ a countable union of open boxes $\bigcup_n B_n^{(k)}$ such that $F \subseteq \bigcup_n B_n^{(k)}$ and $m^*(F) + \frac{1}{2^k} \geq \sum_n m(B_n^{(k)})$.

    Note:
    \begin{itemize}
        \item Up to subdividing each $B_n^{(k)}$ into a finite number of smaller boxes, we can assume that each $B_n^{(k)}$ has diameter less than $\frac{1}{2^k}$.
        \item Without loss of generality we can assume that each box meets $F$.
        \item Finally, since $F$ is compact there is a finite subcover, so we can assume that there are only finite many boxes at each step, that is, $F \subseteq \bigcup_{n=1}^{N_k} B_n^{(k)}$ for $N_k \in \N$.
    \end{itemize}

    Let $U_k = \bigcup_{n=1}^{N_k} B_n^{(k)}$, so $F \subseteq U_k$ for all $k$ and $F$ meets each box $B_n^{(k)}$.
    Then we must have $F = \bigcap_{k \geq 1} U_k$, because if $x \in \bigcap U_k$, for any $k$ we can find some $x_k \in F$ such that $x$ and $x_k$ lie in the same box $B_n^{(k)}$, and so $\norm{x - x_k}_\infty \leq \frac1{2^k}$.
    $F$ is compact, and $x_k \in F$ has a limit point, so $x \in F$.

    We need to show that $m^*(U_k \setminus F)$ tends to $0$, because this implies that $F$ is Lebesgue measurable.

    Claim that if $A, B$ are two disjoint compact subsets of $\R^d$ then $m^*(A \cup B) = m^*(A) + m^*(B)$. This is clear from definition as we can choose disjoint covers by open boxes.

    Apply this to $A = F$ and $B = \overline{U_k} \setminus U_{k+1}$ to get
    \begin{align*}
        m^*(\overline{U_k} \setminus U_{k+1}) + m^*(F) &\leq m^*(\overline{U_k} \setminus U_{k+1} \cup F) \\
                                                       &\leq m^*(\overline{U_k}) \\
                                                       &= m^*(U_k) \\
                                                       &\leq m^*(F) + \frac1{2^k}
    \end{align*}
    so $m^*(U_k \setminus U_{k+1}) \leq \frac1{2^k}$
    Since $F = \bigcap_k U_k$, by countable subadditivity of $m^*$, we get
    \begin{align*}
        m^*(U_k \setminus F) &\leq \sum_{i \geq k} m^* (U_i \setminus U_{i+1}) \\
                             &\leq \sum_{i \geq k} \frac{1}{2^i} \\
                             &\leq \frac{1}{2^{k-1}} \to 0
    \end{align*}
\end{proof}

Finally, we show that the complement of a set in $\mathcal{L}$ is in $\mathcal{L}$.

\begin{proof}[Proof of \cref{prop:bigProp}b)]
    We start with $E \in \mathcal{L}$.  By definition, $\forall n$, there is a countable family of open boxes $C_n$ with $E \subseteq C_n$ and $m^* (C_n \setminus E) < \frac1n$.

    Taking complements, we see $C_n^c \subseteq E^c$ and $C_n \setminus E = E^c \setminus C_n^c$, and note $C_n$ is open and $C_n^c$ is closed.

    By \cref{prop:otherBigProp}, $C_n^c$ is Lebesgue measurable, and by part a) of \cref{prop:bigProp}, so is $\bigcup_n C_n^c$. But $\bigcup_n C_n^c \subseteq E^c$ and \begin{equation*}m^*\left(E^c \setminus \bigcup_n C_n^c\right) \leq m^*(E^c \setminus C_n^c) = m^* (C_n \setminus E) < \frac1n\end{equation*}.

    Hence $m^* (E^c \setminus \bigcup_n C_n^c) = 0$, so
    \begin{equation*}
        E^c = \underbrace{\bigcup_n C_n^c}_{\in \mathcal{L}} \cup \underbrace{E^c \setminus \bigcup_n C_n^c}_{\text{is null hence is in} \; \mathcal{L}}
    \end{equation*}
    so by \cref{prop:bigProp} part a), $E \in \mathcal{L}$.
\end{proof}

\clearpage

\section{Abstract Measure Theory}

\begin{defi}[\hypertarget{def:sigAlg}{$\sigma$-algebra}]
    Let $X$ be a set. A family $\mathcal{A}$ of subsets of $X$ which contains the empty set $\emptyset$ and is stable under countable unions and complementation
    is called a \textbf{$\sigma$-algebra}.
\end{defi}

\begin{remark}
    $\sigma$ stands for `countable'
\end{remark}

\begin{defi}[Measurable space\hypertarget{def:measurableSpace}]
    A \textbf{measurable space} is a couple $(X, \mathcal{A})$ where $X$ is a set and $\mathcal{A}$ is a \hyperlink{def:sigAlg}{$\sigma$-algebra} of subsets of $X$.
\end{defi}

\begin{defi}[Measure\hypertarget{def:measure}]
    A \textbf{measure} on $(X, \mathcal{A})$ is a function $\mu: \mathcal{A} \to [0, \infty]$ such that $\mu(\emptyset) = 0$ and $\mu$ is countably additive.
    That is, if $\{A_n\}$ is a pairwise disjoint countable family of subsets from $\mathcal{A}$, then \begin{equation*}\mu\left(\bigsqcup_{n \geq 1} A_n\right) = \sum_{n \geq 1} \mu(A_n)\end{equation*}
\end{defi}

\begin{remark}
    Sometimes people call a \textbf{set function} a function $\mu$ from a family of subsets of $X$ to $[0, \infty]$ such that $\mu(\emptyset) = 0$.
\end{remark}

\begin{defi}[Measure space\hypertarget{def:measureSpace}]
    If $X$ is a set, $\mathcal{A}$ a \hyperlink{def:sigAlg}{$\sigma$-algebra} on $X$ and $\mu$ a \hyperlink{def:measure}{measure} on $\mathcal{A}$ then $(X, \mathcal{A}, \mu)$ is called a \textbf{measure space}.
\end{defi}

\begin{defi}[Lebesgue measure\hypertarget{def:lebMeas}]
    If $E \in \mathcal{L}$ ($E$ is \hyperlink{def:lebMAble}{Lebesgue measurable}), we define the \textbf{Lebesgue measure} of $E$ as $m(E) = m^*(E)$, the \hyperlink{def:lebOutMeas}{outer measure} of $E$.
\end{defi}

We've already shown $(\R^d, \mathcal{L})$ is a \hyperlink{def:measurableSpace}{measurable space}, but we would like to show that $(\R^d, \mathcal{L}, m)$ is a \hyperlink{def:measureSpace}{measure space}.
To do this, it only remains to show countable additivity of the \hyperlink{def:lebMeas}{Lebesgue measure} on \hyperlink{def:lebMAble}{Lebesgue measurable sets}, so it would be useful to know a little more about them.  So, let's prove a small lemma which helps with this.

\begin{lemma}\hypertarget{lem:lebChar}
    If $E$ is a \hyperlink{def:lebMAble}{Lebesgue measurable} subset of $\R^d$ then
    \begin{enumerate}[label=(\arabic*)]
        \item $\forall \epsilon > 0$, $\exists U \subseteq \R^d$ an open set with $E \subseteq U$ and $m^*(U \setminus E) < \epsilon$
        \item $\forall \epsilon > 0$, $\exists F \subseteq E$ a closed set with $m^*(E \setminus F) < \epsilon$.
    \end{enumerate}
\end{lemma}

\begin{proof}
    Note (2) follows from (1) applied to $E^c$: we get $E^c \subseteq U$ and $m^*(U \setminus E^c) \leq \epsilon$.
    So, set $F = U^c$ and direct calculation gives $U \setminus E^c = E \setminus F$.

    So let's prove (1). Since $E$ is \hyperlink{def:lebMAble}{Lebesgue measurable}, $\forall \epsilon > 0$, $\exists\bigcup B_n$ with $E \subseteq \bigcup B_n$ such that $m^*(\bigcup B_n \setminus E) < \epsilon$.
    Recall we can take the $B_n$ to be open boxes, then just set $U  = \bigcup_n B_n$.
\end{proof}

With that done, let's state and prove our proposition.

\begin{prop}
    $m$ is countably additive on $\mathcal{L}$, hence $(\R^d, \mathcal{L}, m)$ is a \hyperlink{def:measureSpace}{measure space}.
\end{prop}

\begin{proof}
    Recall we need to show additivity for a countable family of pairwise disjoint \hyperlink{def:lebMAble}{Lebesgue measurable} subsets.
    We do this by proving a series of special cases, getting more general each time.

    In turn, we prove additivity for
    \begin{enumerate}[label=(\roman*)]
        \item Two compact subsets
        \item Countably many compact subsets
        \item Countably many bounded subsets
        \item Countably many measurable subsets
    \end{enumerate}
    In each case we can take the sets pairwise disjoint.

    \begin{enumerate}[label=(\roman*)]
        \item For two disjoint compact subsets $A, B \subseteq \R^d$, we've seen $m^*(A) + m^*(B) = m^*(A \cup B)$, as required.

        \item Suppose now $\{E_n\}_n$ is a family of pairwise disjoint compact subsets of $\R^d$. By iterating (i),
            \begin{equation*}
                m^*\left(\bigcup_{i=1}^N E_i\right) = \sum_{i=1}^N m^* (E_i)
            \end{equation*}
            so,
            \begin{equation*}
                \sum_{i=1}^N m^*(E_i) \leq m^*\left(\bigcup_{i=1}^\infty E_i\right) \leq \sum_{i=1}^\infty m^*(E_i)
            \end{equation*}
            Let $N \to \infty$, and therefore
            \begin{equation*}
                \sum_{i=1}^\infty m^*(E_i) = m^*\left(\bigcup_{i=1}^\infty E_i\right)
            \end{equation*}

        \item By (2) of the lemma, $\exists F_n \subseteq E_n$ with $m^*(E_n \setminus F_n) < \epsilon/2^n$ and $F_n$ closed.
            But $E_n$ is bounded by assumption, so $F_n$ is closed and bounded, hence compact.  $E_n$'s are pairwise disjoint, so the $F_n$'s are also.
            Then $m^*(E_n) \leq m^*(F_n) + m^*(E_n \setminus F_n)$ and
            \begin{align*}
                \sum_{n = 1}^\infty m^*(E_n) &\leq \sum_{n=1}^\infty m^*(F_n) + \epsilon \sum_{n = 1}^\infty \frac{1}{2^n} \\
                                             &= m^*\left(\bigcup_{n = 1}^\infty F_n\right) + \epsilon \\
                                             &\leq m^*\left(\bigcup_{n = 1}^\infty E_n\right) + \epsilon
            \end{align*}
            This holds $\forall \epsilon > 0$, so $\sum_{n \geq 1} m^* (E_n) \leq m^*(\bigcup_1^\infty E_n)$
            and equality follows because the other inequality holds by countable subadditivity.

        \item In the general case, we reduce to the bounded case:

            Let $A_m = \set{x \in \R^d | m - 1 \leq \norm{x} < m}$, so we can write
            $\R^d = \bigcup_{m \geq 1} A_m$, with $A_m$ compact. Apply the previous case to the countable family $(A_m \cup E_n)_{n, m}$.
    \end{enumerate}
\end{proof}

\begin{remark}
    Of course there are many other measures on $(\R^d, \mathcal{L})$, for example $f \in C_c(\R^d)$, $f \geq 0$, $\mu_f(E) \coloneqq \int_{\R^d} f(x) \1{E}(x) dx$.
\end{remark}

\begin{prop}
    On $(\R^d, \mathcal{L})$, there is a unique measure $\mu$ which is
    \begin{itemize}
        \item Translation invariant: $\forall E \in \mathcal{L}, \ \forall x \in \R^d, \; \mu(E + x) = \mu(E)$
        \item $\mu([0, 1]^d) = 1$.
    \end{itemize}
\end{prop}

\begin{proof}
    We'll skip uniqueness for now, and see another proof later.
\end{proof}

\begin{remark}
    The countable additivity here is crucial. There are lots of finitely additive measures of $(\R^d, \mathcal{L})$ which are translation invariant and unrelated to $m$.  There are even such $\mu$ with $\mu(\R^d) < \infty$.
\end{remark}

In an \hyperlink{lem:lebChar}{earlier lemma} we started to characterise \hyperlink{def:lebMAble}{Lebesgue measurable sets}, and we can easily extend this further.

\begin{lemma}
    If $E \in \mathcal{L}$,
    \begin{itemize}
        \item $\exists$ a sequence $U_n$ of open sets with $U_{n+1} \subseteq U_n$ and $E \subseteq U_n$ such that $m(\bigcap U_n \setminus E) = 0$.
        \item $\exists$ a sequence $F_n$ of closed sets with $F_n \subseteq F_{n+1}$ and $F_n \subseteq E$ such that $m(E \setminus \bigcup_n F_n) = 0$.
    \end{itemize}
    So, we can write
    \begin{align*}
        E = \bigcap_n U_n \setminus N_1 \\
        E = \bigcup_n F_n \cup N_2
    \end{align*}
    where $N_1$ and $N_2$ are \hyperlink{def:null}{null sets}.
\end{lemma}

\begin{remark}
    \leavevmode
    \begin{itemize}
        \item A countable intersection of open sets is called a $G_{\delta}$-set.
        \item A countable union of closed sets is called a $F_{\sigma}$-set.
        \item So every Lebesgue measurable set is the difference of a $G_{\delta}$-set and a null set, alternatively it is the union of a $F_{\sigma}$-set and a null set.
    \end{itemize}
\end{remark}

\begin{defi}[Boolean algebra\hypertarget{def:boolAlg}]
    If $X$ is a set, a \textbf{Boolean algebra} of subsets of $X$ is a family $\mathcal{B}$ of subsets of $X$ which contains $\emptyset$ and $X$ and is stable under finite union and complementation.
\end{defi}

\begin{remark}
    \leavevmode
    \begin{itemize}
        \item A \hyperlink{def:sigAlg}{$\sigma$-algebra} is a \hyperlink{def:boolAlg}{Boolean algebra} that is also stable under countable unions.
        \item Every finite Boolean algebra is a $\sigma$-algebra.
    \end{itemize}
\end{remark}

\begin{itemize}
    \item Trivial Boolean algebra, $\mathcal{B} = \{\emptyset, X\}$.
    \item Discrete Boolean algebra, $\mathcal{B} = \mathcal{P}(X)$.
    \item The elementary Boolean algebra, with $\mathcal{B}=$\{finite unions of boxes in $\R^d$, or complements of such\}.
    \item Take a partition $X$ into finitely many pieces $X = \bigsqcup_{i=1}^N P_i$.
        \begin{equation*}
            \mathcal{B} \coloneqq \Set{A \subseteq X \mid \exists I \subseteq \set{1, \dotsc, N}, \; A = \bigcup_{i\in I} P_i}
        \end{equation*}
    \item $X$ is a topological space, and $\mathcal{B}=${ finite unions of sets of the form $U \cap F$, with $U$ open and $F$ closed}.
        This is called the Boolean algebra of constructible subsets of $X$.

    \item Null algebra,
        \begin{equation*}
            \mathcal{B} = \set{E \subseteq \R^d | m^*(E) = 0 \text{ or } m^*(E^c) = 0}
        \end{equation*}
\end{itemize}

\begin{remark}
    The intersection of any number of Boolean algebras on the same set $X$ is again a Boolean algebra. The same holds for $\sigma$-algebras.
\end{remark}

\begin{defi}
    \leavevmode
    \begin{itemize}
        \item If $\mathcal{F}$ is a family of subsets of $X$, the Boolean algebra \textbf{generated} by $\mathcal{F}$ is the inserction of all Boolean algebras containing $\mathcal{F}$.
        \item Same for $\sigma$-algebras: we denote this by $\sigma(\mathcal{F})$.
    \end{itemize}
\end{defi}

\begin{remark}
    The Boolean algebra generated by a family $\mathcal{F}$ of subsets of $X$ is the family of sets of the form
    \begin{equation*}
        \bigcup_{\text{finite}} \bigcap_{\text{finite}} F \text{ or } F^c
    \end{equation*}
\end{remark}

Note a finite intersection of a finite union of sets from a family $\mathcal{F}$ is always a finite union of finite intersections of sets from $\mathcal{F}$. But, this is no longer true if you change finite to countable:
\begin{equation*}
    \bigcap_{i \in \N} \bigcup_{j \in \N} E_j^{(i)} = \bigcup_{\substack{i \mapsto j_i \\ \N \to \N}} \bigcap_{i \in \N} E_{ji}^{(i)}
\end{equation*}
and there are more than countably many functions $\N \to \N$.

\begin{defi}
    The $\sigma$-algebra generated by the family of boxes in $\R^d$ is called the \textbf{Borel $\sigma$-algebra}.
    Elements of the Borel $\sigma$-algebra are called Borel sets.
\end{defi}

\begin{remark}
    It is also the $\sigma$-algebra generated by all open subsets of $\R^d$ (or by all closed subsets), because every open set is a countable union of boxes.

    More generally, if $X$ is an arbitrary topological space, the Borel $\sigma$-algebra of $X$ is the $\sigma$-algebra generated by open subsets of $X$. We denote it by $\mathcal{B}(X)$.
\end{remark}

Note, we've shown that every open subset $E \subseteq \R^d$ is \hyperlink{def:lebMAble}{Lebesgue measurable}, hence $\mathcal{B}(\R^d) \subseteq \mathcal{L}$. We might ask if we have equality.
No. One can show that $\abs{\mathcal{B}(\R^d)} = 2^{\abs{\N}}$. But, remember that every null set is in $\mathcal{L}$, and the middle-thirds Cantor set is null.
TODO: add end of the argument

% back to measures

\begin{defi}
    \leavevmode
    \begin{itemize}
        \item \textbf{Null sets} of a measure space $(X, \mathcal{A}, \mu)$ are subsets in $\mathcal{A}$ with $\mu$-measure $0$.
        \item The \textbf{sub-null} sets are the subsets of $X$ contained in a null set.
    \end{itemize}
\end{defi}

\begin{defi}
    The family $\mathcal{A}^*$ of subsets of $X$ of the form $A \cup N$ or $A \setminus N$ where $A \mathcal{A}$ and $N$ is a \hyperlink{def:null}{sub-null} set (with respect to $\mu$) is called the \textbf{completion}. If all sub-null sets are in $\mathcal{A}$, then $\mathcal{A}$ is called \textbf{complete}.
\end{defi}

\begin{prop}
    The completion forms a $\sigma$-algebra.
\end{prop}

\begin{eg}
    The completion of the \emph{Borel $\sigma$-algebra} $\mathcal{B}(\R^d)$ is $\mathcal{L}$.
\end{eg}

We can prove basic properties of measures on an arbitrary measure spacec.


\begin{prop}
    If $(X, \mathcal{A}, \mu)$ is a measure space,
    \begin{enumerate}[label=(\alph*)]
        \item Upwards monotone convergence for sets: If we have $E_1 \subseteq E_2 \subseteq \dotsb \subseteq E_n \subseteq \dotsb$, with $E_i \in \mathcal{A}$, then
            \begin{equation*}
                \lim_n \mu(E_n) = \sup_n \mu(E_n) = \mu\left(\bigcup_n E_n\right)
            \end{equation*}
        \item Downwards monotone convergence: If we have $E_1 \supseteq E_2 \supseteq \dotsb \supseteq E_n \supseteq \dotsb$ with $E_i \in \mathcal{A}$ and $\mu(E_1) < \infty$, then
            \begin{equation*}
                \lim_n \mu(E_n) = \inf_n \mu(E_n) = \mu\left(\bigcap_n E_n\right)
            \end{equation*}
    \end{enumerate}
\end{prop}

\begin{proof}
    This follows from countable additivity of $\mu$, say for b) consider
    \begin{equation*}
        E_1 = \bigsqcup_{i \geq 1} E_i \setminus E_{i+1} \sqcup \bigcap_n E_n
    \end{equation*}
\end{proof}
\end{document}
