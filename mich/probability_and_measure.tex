\documentclass{article}

\def\npart {II}
\def\nyear {2017}
\def\nterm {Michaelmas}
\def\nlecturer{E.\ Brieuillard}
\def\ncourse{Probability and Measure}
\ifx \nauthor\undefined
  \def\nauthor{Bhavik Mehta}
\else
\fi

\author{Based on lectures by \nlecturer \\\small Notes taken by \nauthor}
\date{\nterm\ \nyear}
\title{Part \npart\ -- \ncourse}

\usepackage[utf8]{inputenc}
\usepackage{amsmath}
\usepackage{amsthm}
\usepackage{amssymb}
\usepackage{enumerate}
\usepackage{mathtools}
\usepackage{graphicx}
\usepackage[dvipsnames]{xcolor}
\usepackage{tikz}
\usepackage{wrapfig}
\usepackage{centernot}
\usepackage{float}
\usepackage{braket}
\usepackage[hypcap=true]{caption}
\usepackage{enumitem}
\usepackage[colorlinks=true, linkcolor=mblue]{hyperref}
\usepackage[nameinlink,noabbrev]{cleveref}
\usepackage{nameref}
\usepackage[margin=1.5in]{geometry}

% Theorems
\theoremstyle{definition}
\newtheorem*{aim}{Aim}
\newtheorem*{axiom}{Axiom}
\newtheorem*{claim}{Claim}
\newtheorem*{cor}{Corollary}
\newtheorem*{conjecture}{Conjecture}
\newtheorem*{defi}{Definition}
\newtheorem*{eg}{Example}
\newtheorem*{ex}{Exercise}
\newtheorem*{fact}{Fact}
\newtheorem*{law}{Law}
\newtheorem*{lemma}{Lemma}
\newtheorem*{notation}{Notation}
\newtheorem*{prop}{Proposition}
\newtheorem*{question}{Question}
\newtheorem*{rrule}{Rule}
\newtheorem*{thm}{Theorem}
\newtheorem*{assumption}{Assumption}

\newtheorem*{remark}{Remark}
\newtheorem*{warning}{Warning}
\newtheorem*{exercise}{Exercise}

% \newcommand{\nthmautorefname}{Theorem}

\newtheorem{nthm}{Theorem}[section]
\newtheorem{nlemma}[nthm]{Lemma}
\newtheorem{nprop}[nthm]{Proposition}
\newtheorem{ncor}[nthm]{Corollary}
\newtheorem{ndef}[nthm]{Definition}

% Special sets
\newcommand{\C}{\mathbb{C}}
\newcommand{\N}{\mathbb{N}}
\newcommand{\Q}{\mathbb{Q}}
\newcommand{\R}{\mathbb{R}}
\newcommand{\Z}{\mathbb{Z}}

\newcommand{\abs}[1]{\left\lvert #1\right\rvert}
\newcommand{\norm}[1]{\left\lVert #1\right\rVert}
\renewcommand{\vec}[1]{\boldsymbol{\mathbf{#1}}}

\let\Im\relax
\let\Re\relax

\DeclareMathOperator{\Im}{Im}
\DeclareMathOperator{\Re}{Re}
\DeclareMathOperator{\id}{id}

\definecolor{mblue}{rgb}{0., 0.05, 0.6}

% efjb2@cam.ac.uk
% preamble
\setcounter{section}{-1}
% and here we go!

\begin{document}
\maketitle
\section{Introduction}
\subsection{Course structure}
\begin{itemize}
    \item Week 1: Lebesgue measure
    \item Week 2: Abstract measure theory
    \item Week 3: Integration
    \item Week 4: Foundations of probability theory
    \item Week 5: $L^p$ spaces
    \item Week 6: Modes of convergence
    \item Week 7: Fourier transform and gaussians
   \item Week 8: Ergodic Theory
\end{itemize}
\subsection{Historical motivation}
Suppose we have a subset $E \subset \R^d$.
\begin{enumerate}
    \item What does it mean to measure this subset?

        In one dimension, we have some intuition of length, and in two and three dimensions we are familiar with the notions of surface area and volumne

    \item Does it make sense to measure every subset?

        This seems reasonable, but it turns out that assigning a measure to every subset can lead to logical contradictions.
        What is a measure? It should be a function defined on subsets, in particular some assignment $E \to m(E) \in \R$.  It should satisfy some properties:
        \begin{itemize}
            \item Non-negativity: $m(E) \geq 0$ for all $E$
            \item Empty set: $m(\varnothing) = 0$
            \item Additivity: $m(E \sqcup F) = m(E) + m(F)$ for any two disjoint sets $E$ and $F$
            \item Normalisation: $m([0, 1]^d) = 1$
            \item Translation invariant: $m(E + x) = m(E)$ for all $E$ and all $x \in \R^d$
        \end{itemize}
        It's possible to construct pathological `measures' satisfying all these axioms and defined \emph{on all} subsets of $\R^d$, but they won't be `nice'
        When mathematicians construct such measures, they usually do so on a restricted class of sets. Otherwise this leads to contradictions.
        If $d \geq 3$, it can be shown that there is no $m: P(\R^d) \to [0, \infty)$ that are also rotation invariant.  This is referred to as the Hausdorff-Banach-Tarski paradox.
        Namely, if we take $B = B(0, 1) = \{\vec{x} \in \R^d: x_1^2 + \dots x_d^2 \leq 1\}$, then there is a partition (fill in later)

    \item Jordan Measure

        Consider a box $B \subset \R^d$, given by $B = \prod_{i=1}^d I_i$, where $I_i = [a_i, b_i]$ are intervals in $\R$
        From here, we can define the Jordan Measure $m(B) := \prod_{i=1}^d |b_i - a_i|$.

        \begin{defi}
            An \textbf{elementary subset} of $\R^d$ is a finite union of boxes.
        \end{defi}
        \begin{remark}
            Every elementary set can be written as a finite union of disjoint boxes
            The family of elementary sets is stable under finite unions, finite intersections and difference.
            If E = $\sqcup_{i=1}^N B_i = \sqcup_{j=1}^M B'_j$, that is $E$ is a disjoint union of boxes in two different ways, then
            $\sum_{i=1}^N m(B_i) = \sum_{j=1}^M m(B'_j)$
            This says that \emph{makes sense} to define $m(E) as \sum_1^N m(B_i)$, for elementary subsets.
        \end{remark}
\end{enumerate}

\begin{defi}
    A subset $E \in \R^d$ is \textbf{Jordan measurable} if $\forall \epsilon > 0 \exists$ elementary sets $A, B$ such that $A \subset E \subset B$ and $m(B\setminus A) < \epsilon$.
\end{defi}
\begin{remark}
    If $E$ is Jordan measurable, then
    \begin{equation}
        \inf\{m(B): E \subset B, B \mathrm{elementary}\} = \sup\{m(A): A \subset E, A \mathrm{elementary}\}
    \end{equation}
    (exercise: check)
\end{remark}
\begin{defi}
    We define the \textbf{Jordan measure} of $E$ as this supremum or infimum.
\end{defi}
\begin{exercise}
    $m$ so defined satisfies all the axioms defined earlier.
\end{exercise}
However the Jordan measure is not perfect. For instance, the complement of a Jordan measurable set is not Jordan measurable.  Similarly, we can take an (infinite) union of Jordan measurable sets and produce a set which is not Jordan measurable.

\begin{defi}[Riemann integrable function]
    A function $f:[a, b] \to \R$ is Riemann integrable if all its Riemann sums converge.
    Formally, the integral $I(f) \in \R$, exists, if $\forall \epsilon > 0$, $\exists \delta > 0$ for every partition $P$ of $[a, b]$ of width $\tau(P) < \delta$,
    $|S(f, P) - I(f)| < \epsilon$.
$P$ a partition $a = t_0 < t_1 < \dots < t_n = b$ width $\tau(P) = \max_{0 \leq 1 \leq N_1} |t_{i+1} - t_i|$
$S(f, P) = \sum_{i=1}^{N-1} P$
% sort this out you slow idiot
\end{defi}
\end{document}
