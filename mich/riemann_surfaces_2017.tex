\documentclass{article}

\ifx \nauthor\undefined
  \def\nauthor{Bhavik Mehta}
\else
\fi

\author{Based on lectures by \nlecturer \\\small Notes taken by \nauthor}
\date{\nterm\ \nyear}
\title{Part \npart\ -- \ncourse}

\usepackage[utf8]{inputenc}
\usepackage{amsmath}
\usepackage{amsthm}
\usepackage{amssymb}
\usepackage{enumerate}
\usepackage{mathtools}
\usepackage{graphicx}
\usepackage[dvipsnames]{xcolor}
\usepackage{tikz}
\usepackage{wrapfig}
\usepackage{centernot}
\usepackage{float}
\usepackage{braket}
\usepackage[hypcap=true]{caption}
\usepackage{enumitem}
\usepackage[colorlinks=true, linkcolor=mblue]{hyperref}
\usepackage[nameinlink,noabbrev]{cleveref}
\usepackage{nameref}
\usepackage[margin=1.5in]{geometry}

% Theorems
\theoremstyle{definition}
\newtheorem*{aim}{Aim}
\newtheorem*{axiom}{Axiom}
\newtheorem*{claim}{Claim}
\newtheorem*{cor}{Corollary}
\newtheorem*{conjecture}{Conjecture}
\newtheorem*{defi}{Definition}
\newtheorem*{eg}{Example}
\newtheorem*{ex}{Exercise}
\newtheorem*{fact}{Fact}
\newtheorem*{law}{Law}
\newtheorem*{lemma}{Lemma}
\newtheorem*{notation}{Notation}
\newtheorem*{prop}{Proposition}
\newtheorem*{question}{Question}
\newtheorem*{rrule}{Rule}
\newtheorem*{thm}{Theorem}
\newtheorem*{assumption}{Assumption}

\newtheorem*{remark}{Remark}
\newtheorem*{warning}{Warning}
\newtheorem*{exercise}{Exercise}

% \newcommand{\nthmautorefname}{Theorem}

\newtheorem{nthm}{Theorem}[section]
\newtheorem{nlemma}[nthm]{Lemma}
\newtheorem{nprop}[nthm]{Proposition}
\newtheorem{ncor}[nthm]{Corollary}
\newtheorem{ndef}[nthm]{Definition}

% Special sets
\newcommand{\C}{\mathbb{C}}
\newcommand{\N}{\mathbb{N}}
\newcommand{\Q}{\mathbb{Q}}
\newcommand{\R}{\mathbb{R}}
\newcommand{\Z}{\mathbb{Z}}

\newcommand{\abs}[1]{\left\lvert #1\right\rvert}
\newcommand{\norm}[1]{\left\lVert #1\right\rVert}
\renewcommand{\vec}[1]{\boldsymbol{\mathbf{#1}}}

\let\Im\relax
\let\Re\relax

\DeclareMathOperator{\Im}{Im}
\DeclareMathOperator{\Re}{Re}
\DeclareMathOperator{\id}{id}

\definecolor{mblue}{rgb}{0., 0.05, 0.6}

\usepackage{inputenc}
\usepackage{amsmath}
\usepackage{amsthm}
\usepackage{amssymb}
\usepackage{enumerate}
\usepackage{mathtools}
\usepackage{graphicx}
\usepackage{xcolor}
\usepackage{wrapfig}
\usepackage{float}
\usepackage{hyperref}
\usepackage{hypcap}
\usepackage{geometry}

\title{Riemann Surfaces}

\begin{document}
\maketitle

\section{Complex analysis and complex log}

Sketch of equivalence:
1 => 2 Use CIF to construct $a_n$, and convergence.

2 => 1 Show directly term-by-term derivative exists, and agrees with limit definition.

Note that a power series tells you about local behaviour. If $f(z)$ is not identically zero near $a \in U$, there exists some minimal $n_0$ such that $a_{n_0} \neq 0$.  We can write $f$ locally as
\begin{equation}
    f(z) = a_{n_0} (z-a)^(n_0) + \sum_{n \ge n_0} a_n (z-a)^n = a_{n_0} (z-a)^n_0 \left(1 + \sum_{n>n_0} \frac{a_n}{a_{n_0}} (z-a)^{n-n_0}\right)
\end{equation}
As $z \to a$, the sum $\to 0$ so the quantity in the brackets tends to 1.  So, $f(z) = a_{n_0} (z-a)^{n_0} g(z)$, where $g$ is analytic and nonzero on a neighbourhood of $a$.
Consequently, we have the \emph{principle of isolated zeros}: for $f$ analytic on domain $U$, then for all $a \in U$ such that $f(a) = 0$, either $f$ is identically $0$ on a neighbourhood of $a$, or $f$ is never $0$ on a punctured disk centred at $a$.  But recall a domain refers to an open \emph{and connected} set.  So, if $f$ is identically $0$ (improve this writing) on a neighbourhood of $a$, call it $D_a$.  If $f \ne 0$ on a punctured disk about $a$, call it $P_a^*$. Construct
\begin{align}
V = \cup_{a \in U such that f equiv 0 on a neighbourhood of a} D_a
W = \cup_{a \in U such that f \ne 0 on a punctured neighbourhood of a} P_a^*
\end{align}

$V$ and $W$ are open, disjoint and $V \cup W = U$. Since $U$ is connected, $U=V$ or $U=W$. So either $f \equiv 0$ on $U$, or $f$ has only isolated zeros.
This will both be referred to as the principle of isolated zeros.

(corollary) (identity principle)
If $f$ and $g$ are analytic on $U$, then either $f \equiv g$ on $U$, or ${z \in U : f(z) = g(z)}$ consists of isolated points.
Proof (clear)

(definition)
If f is analytic on a punctured disk $\mathbb{D}^*(a,r)$, then we say $a$ is an isolated singularity of $f$.
If so, then there exists a Laurent expansion $\sum_{n=-\infty}^{\infty} a_n (z-a)^n$ near $a$, which come in three types.

I. Removable singularity. $f$ extends to an analytic function on $\mathbb{D}^*{a, r}$. Phrased in terms of Laurent expansions, $a_n = 0 \forall n < 0$.
(thm) (Removable singularities theorem) $f$ has a removable singularity at $a$ if and only if $f$ is bounded on a punctured neighbourhood about $a$.
(proof sketch) (=>) follows from continuity of analytic functions
(<=) Cauchy's theorem and integral formula still hold for punctured neighbourhoods so long as $f(z) (z-a) \to 0$ as $z \to a$.  With a small circle about $a$, we can show directly that $a_n = 0$ for $n < 0$.

II. Poles: $f$ has a pole at $a$ if $a_n = 0$ for all $n < n_0$ for some $n_0$. Locally, this occurs if and only if $|f(z)| \to \infty$ as $z \to a$ (using the Laurent series).

III. Essential singularity: $a_n \neq 0$ for finitely many $n < 0$, f has an essential singularity at $a$.
thm [Casorati-Weierstrass] If $f$ has an essential singularity at $a$, then the image of $f$ on any punctured neighbourhood of $a$ is dense in $\mathbb{C}$.
(proof sketch) Examine $\frac1{f(z) - \gamma}$ if the image of $f$ misses a neighbourhood of $\gamma$.

\paragraph{Examples}
$f(z) = \frac1{e^z - 1}$ has poles wherever $e^z = 1$.
At $\infty$, we also have an isolated singularity, recalling that punctured neighbourhoods of $\infty$ are $\mathbb{C}\\ \mathbb{D}(0, R)$. Since $e^z$ takes all nonzero values or strips of $\mathbb{C}$, we cannot have $e^z \to 1$ as $z \to \infty$, hence $f$ cannot have a pole.  On the other hand, there exists arbitrarily large solutions (in modulus) to $e^z = 1$, and $f$ cannot have a removable singularity at $\infty$.  Hence, this singularity is essential.

\subsection{Complex logarithm}
The complex logarithm is an example of a multivalued function, which arises as the inverse of an analytic function. Given nonzero $z$, if $e^w = z$, $z = r e^{i \theta}$, then $w = \log r + (2 \pi n + \theta i)$ for some $n \in \mathbb{Z}$.  We cannot make a continuous choice of $n$ on all of $\mathbb{C}$, so we define the complex log on domains like $\mathbb{C} \\ \mathbb{R}_{\leq 0} =: U$. We have for each $n$, a choice of logarithm which can be analytically defined on $U$.  Recall a \emph{continuous} inverse of an analytic function is analytic.


\end{document}
