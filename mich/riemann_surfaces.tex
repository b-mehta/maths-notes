\documentclass{article}

\def\npart {II}
\def\nyear {2017}
\def\nterm {Michaelmas}
\def\nlecturer{Dr H.\ Krieger}
\def\ncourse{Riemann Surfaces}
\ifx \nauthor\undefined
  \def\nauthor{Bhavik Mehta}
\else
\fi

\author{Based on lectures by \nlecturer \\\small Notes taken by \nauthor}
\date{\nterm\ \nyear}
\title{Part \npart\ -- \ncourse}

\usepackage[utf8]{inputenc}
\usepackage{amsmath}
\usepackage{amsthm}
\usepackage{amssymb}
\usepackage{enumerate}
\usepackage{mathtools}
\usepackage{graphicx}
\usepackage[dvipsnames]{xcolor}
\usepackage{tikz}
\usepackage{wrapfig}
\usepackage{centernot}
\usepackage{float}
\usepackage{braket}
\usepackage[hypcap=true]{caption}
\usepackage{enumitem}
\usepackage[colorlinks=true, linkcolor=mblue]{hyperref}
\usepackage[nameinlink,noabbrev]{cleveref}
\usepackage{nameref}
\usepackage[margin=1.5in]{geometry}

% Theorems
\theoremstyle{definition}
\newtheorem*{aim}{Aim}
\newtheorem*{axiom}{Axiom}
\newtheorem*{claim}{Claim}
\newtheorem*{cor}{Corollary}
\newtheorem*{conjecture}{Conjecture}
\newtheorem*{defi}{Definition}
\newtheorem*{eg}{Example}
\newtheorem*{ex}{Exercise}
\newtheorem*{fact}{Fact}
\newtheorem*{law}{Law}
\newtheorem*{lemma}{Lemma}
\newtheorem*{notation}{Notation}
\newtheorem*{prop}{Proposition}
\newtheorem*{question}{Question}
\newtheorem*{rrule}{Rule}
\newtheorem*{thm}{Theorem}
\newtheorem*{assumption}{Assumption}

\newtheorem*{remark}{Remark}
\newtheorem*{warning}{Warning}
\newtheorem*{exercise}{Exercise}

% \newcommand{\nthmautorefname}{Theorem}

\newtheorem{nthm}{Theorem}[section]
\newtheorem{nlemma}[nthm]{Lemma}
\newtheorem{nprop}[nthm]{Proposition}
\newtheorem{ncor}[nthm]{Corollary}
\newtheorem{ndef}[nthm]{Definition}

% Special sets
\newcommand{\C}{\mathbb{C}}
\newcommand{\N}{\mathbb{N}}
\newcommand{\Q}{\mathbb{Q}}
\newcommand{\R}{\mathbb{R}}
\newcommand{\Z}{\mathbb{Z}}

\newcommand{\abs}[1]{\left\lvert #1\right\rvert}
\newcommand{\norm}[1]{\left\lVert #1\right\rVert}
\renewcommand{\vec}[1]{\boldsymbol{\mathbf{#1}}}

\let\Im\relax
\let\Re\relax

\DeclareMathOperator{\Im}{Im}
\DeclareMathOperator{\Re}{Re}
\DeclareMathOperator{\id}{id}

\definecolor{mblue}{rgb}{0., 0.05, 0.6}


% preamble
\setcounter{section}{-1}
% and here we go!

\begin{document}
\maketitle

\section{Complex analysis and complex log}

\begin{defi}
    A smooth function $f: U \to \C$ (where $U$ is a domain in $\C$) is \textbf{holomorphic} or \textbf{analytic} if either of the following equivalent statements hold:
    \begin{enumerate}[(1)]
        \item $f$ is differentiable at all points of $U$, where differentiability is defined by limits, and checked by the Cauchy-Riemann equations
        \item $\forall a \in U$, $f$ has a power series expansion on a neighbourhood of $a$:
            \begin{equation*}
                f(z) = \sum_{n \geq 0} a_n (z - a)^n
            \end{equation*}
            and the series converges on some disk about $a$ with positive radius.
    \end{enumerate}
\end{defi}
\begin{proof}[Sketch of proof of equivalence]

    \leavevmode

    $(\Rightarrow)$ Use the Cauchy Integral Formula to construct $a_n$, and convergence.

    $(\Leftarrow)$ Show directly that the term-by-term derivative exists, and that it agrees with the limit definition of the derivative.
\end{proof}

Note that a power series tells you about local behaviour.
If $f(z)$ is not identically zero near $a \in U$, there exists some minimal $n_0$ such that $a_{n_0} \neq 0$.
We can write $f$ locally as
\begin{align*}
    f(z) &= a_{n_0} (z-a)^{n_0} + \sum_{n \ge n_0} a_n (z-a)^n \\
         &= a_{n_0} (z-a)^{n_0} \left(1 + \sum_{n>n_0} \frac{a_n}{a_{n_0}} (z-a)^{n-n_0}\right)
\end{align*}
As $z \to a$, the sum $\to 0$ so the quantity in the brackets tends to 1.  So, $f(z) = a_{n_0} (z-a)^{n_0} g(z)$, where $g$ is analytic and nonzero on a neighbourhood of $a$.
Consequently, we have the \emph{principle of isolated zeros}: for $f$ analytic on domain $U$, then for all $a \in U$ such that $f(a) = 0$, either $f$ is identically $0$ on a neighbourhood of $a$, or $f$ is never $0$ on a punctured disk centred at $a$.  But recall a domain refers to an open \emph{and connected} set.  So, if $f$ is identically $0$ (improve this writing) on a neighbourhood of $a$, call it $D_a$.  If $f \ne 0$ on a punctured disk about $a$, call it $P_a^*$. Construct
\begin{align}
    V = \bigcup_{\substack{a \in U \text{such that} \\ f \equiv 0 \text{on a neighbourhood of} a}} D_a
    % W = \cup_{a \in U such that f \ne 0 on a punctured neighbourhood of a} P_a^*
\end{align}

$V$ and $W$ are open, disjoint and $V \cup W = U$. Since $U$ is connected, $U=V$ or $U=W$. So either $f \equiv 0$ on $U$, or $f$ has only isolated zeros.
This will both be referred to as the principle of isolated zeros.

(corollary) (identity principle)
If $f$ and $g$ are analytic on $U$, then either $f \equiv g$ on $U$, or ${z \in U : f(z) = g(z)}$ consists of isolated points.
Proof (clear)

(definition)
If f is analytic on a punctured disk $\mathbb{D}^*(a,r)$, then we say $a$ is an isolated singularity of $f$.
If so, then there exists a Laurent expansion $\sum_{n=-\infty}^{\infty} a_n (z-a)^n$ near $a$, which come in three types.

I. Removable singularity. $f$ extends to an analytic function on $\mathbb{D}^*{a, r}$. Phrased in terms of Laurent expansions, $a_n = 0 \forall n < 0$.
(thm) (Removable singularities theorem) $f$ has a removable singularity at $a$ if and only if $f$ is bounded on a punctured neighbourhood about $a$.
(proof sketch) (=>) follows from continuity of analytic functions
(<=) Cauchy's theorem and integral formula still hold for punctured neighbourhoods so long as $f(z) (z-a) \to 0$ as $z \to a$.  With a small circle about $a$, we can show directly that $a_n = 0$ for $n < 0$.

II. Poles: $f$ has a pole at $a$ if $a_n = 0$ for all $n < n_0$ for some $n_0$. Locally, this occurs if and only if $|f(z)| \to \infty$ as $z \to a$ (using the Laurent series).

III. Essential singularity: $a_n \neq 0$ for finitely many $n < 0$, f has an essential singularity at $a$.
thm [Casorati-Weierstrass] If $f$ has an essential singularity at $a$, then the image of $f$ on any punctured neighbourhood of $a$ is dense in $\mathbb{C}$.
(proof sketch) Examine $\frac1{f(z) - \gamma}$ if the image of $f$ misses a neighbourhood of $\gamma$.

\paragraph{Examples}
$f(z) = \frac1{e^z - 1}$ has poles wherever $e^z = 1$.
At $\infty$, we also have an isolated singularity, recalling that punctured neighbourhoods of $\infty$ are $\mathbb{C}\\ \mathbb{D}(0, R)$. Since $e^z$ takes all nonzero values or strips of $\mathbb{C}$, we cannot have $e^z \to 1$ as $z \to \infty$, hence $f$ cannot have a pole.  On the other hand, there exists arbitrarily large solutions (in modulus) to $e^z = 1$, and $f$ cannot have a removable singularity at $\infty$.  Hence, this singularity is essential.

\subsection{Complex logarithm}
The complex logarithm is an example of a multivalued function, which arises as the inverse of an analytic function. Given nonzero $z$, if $e^w = z$, $z = r e^{i \theta}$, then $w = \log r + (2 \pi n + \theta i)$ for some $n \in \mathbb{Z}$.  We cannot make a continuous choice of $n$ on all of $\mathbb{C}$, so we define the complex log on domains like $\mathbb{C} \\ \mathbb{R}_{\leq 0} =: U$. We have for each $n$, a choice of logarithm which can be analytically defined on $U$.  Recall a \emph{continuous} inverse of an analytic function is analytic.

% Lecture 2

Let $U_1 = \C \setminus \R_{\geq 0}$.

\begin{prop}
    For $n \in \Z$, define $h(z)$ on $U$, by
    \begin{equation*}
        h(z) = \int_{-1}^z \frac{dw}{w} + (2n+1) \pi i
    \end{equation*}
    with integral along straight line joining $-1$ and $z$.

    Then $h$ is analytic on $U_1$ and is the inverse to the exponential function on $U_1$.
\end{prop}

\begin{proof}
    Let $z \in U_1$. $\tau \in \C$ with $\abs{\tau}$ sufficiently small, such that the triangle is entirely in the domain.

    Then claim $\frac{h(z + \tau) - h(z)}{\tau} = \frac1{\tau} \int_z^{z+\tau} \frac{dw}{w} \to \frac1z$.
    The first equality follows from Cauchy's theorem, since $h$ is continuous in the triangle.

    \begin{align*}
        \abs{\frac{1}{\tau} \int_z^{z + \tau} \frac{dw}{w} - \frac12} &= \abs{\frac1{\tau} \int_z^{z+\tau} \frac{z-w}{zw} dw} \\
        &\leq C \tau \to 0
    \end{align*}


    Thus $h$ is analytic on $U_1$, with $h'(z) = \frac{1}{z}$.

    Look at $g(z) = \frac{e^{h(z)}}{z}$, so $g'(z) = \frac{z h'(z) e^{h(z)} - e^{h(z)}}{z^2} = 0$, thus $g$ is constant. But, we still need to find out what it's value us, so consider $g(-1)$. $g(-1) = \frac{e^{h(-1)}}{-1} = -e^{(2n+1) \pi i} = 1$.
    Thus, $e^{h(z)} \equiv z$ on $U_1$, so $h$ is the inverse to the exponential.
\end{proof}

\begin{remark}
    We can't extend the function $H$ continuously across the positive real axis.
\end{remark}

\subsection{Analytic continuation}
Fix a domain $U \in \C$ which is path connected.

\begin{defi}[\hypertarget{def:directAnalCont}{Direct Analytic Continuation}]
    A \textbf{function element} (or \textbf{function germ}) on $U$ is a pair, $(f, D)$ where $f$ is analytic on the domain $D \subseteq U$.

    Two function elements $(f, D)$ and $g, E)$ are \textbf{equivalent} if $D \cap E \neq \emptyset$ and $f = g$ on $D \cap E$. In this case, we say $(g, E)$ is a \textbf{direct analytic continuation} of $(f, D)$.
\end{defi}

\begin{remark}
    This is not an equivalence relation.  In the diagram, $(f_1, D_1)$ and $(f_3, D_3)$ are not equivalent since $D_1 \cap D_3 = \emptyset$.
\end{remark}
% why do they do this to me :(

\begin{defi}[\hypertarget{def:pathAnalCont}{Analytic continuation along a path}]
    We say $(g, E)$ is an analytic continuation of $(f, D)$ along a path $\gamma: [0, 1] \to U$, written as $(f, D) \sim_\gamma (g, E)$.  If there exists $(f_1, D_1), \dotsc, (f_n, D_n)$ and $0 = t_0 < t_1 < \dots < t_n = 1$ with $\gamma([t_{i-1}, t_i]) \subseteq D_i$ for $1 \leq i \leq n$ and $(f_1, D_1) = (f, D)$, $(f_n, D_n) = (g, E)$ and $(F_{i-1}, D_{i-1}) \sim (F_i, D_i)$, that is $(f_i, D_i)$ is a \hyperlink{def:directAnalCont}{direct analytic continuation} of $f_{i-1}, D_{i-1}$.
\end{defi}


\begin{defi}[\hypertarget{def:analCont}{Analytic continuation}]
    We say $(g, E)$ is an analytic continuation of $(f, D)$ if there exists a path $\gamma$ with $(f, D) \sim_\gamma (g, E)$.  We write $(f, D) \approx (g, E)$.
\end{defi}

\begin{remark}
    $\approx$ is an equivalence relation.  Reflexivity and symmetry are easy, and transitivity can be seen from the diagram.
\end{remark}


\begin{defi}
    An equivalence class $\mathcal{F}$ under $\approx$ is a \textbf{complete analytic function}.
\end{defi}

\begin{eg}
    Set $U = \C^* = \C \setminus {0}$.  Fix $(\alpha, \beta) \subseteq \R$, with $\abs{\beta-\alpha} \leq 2 \pi$, and define
    \begin{equation*}E_{\alpha, \beta} = \set{z = r e^{i \theta} | \alpha < \theta < \beta \, , \; r > 0}\end{equation*}
    So we can see $U_1 = E_(0, 2 \pi)$.
    Define $f_{(\alpha, \beta)} : E_{(\alpha, \beta)} \to \C$ by $f_{(\alpha, \beta)} (r e^{i \theta}) = \log r + i \theta$, $\theta \in (\alpha, \beta)$.

    Write $L_{(\alpha, \beta)}$ for the function element $(f_{(\alpha, \beta)}, E_{(\alpha, \beta)})$.

    Consider the three function elements $L_{-\frac\pi2, \frac\pi2}$, $L_{\frac\pi6, \frac{7\pi}6}$, $L_{\frac{5\pi}6, \frac{11\pi}{6}}$
    We can see the direct analytic continuations, but we do not have direct analytic continuation (), but ().
\end{eg}

\begin{goal}
    Construct a surface on which $\log$ is well-defined by gluing together a bunch of domains on which it is well-defined.
\end{goal}

FOr each $n \in \Z$, take a copy of $U_1 = E_{(0, 2\pi)}$.  Each has a well-defined choice of logarithm: $f_{(2\pi n, 2\pi (n+1))}$, $r e^{i \theta} \mapsto \log r + (\theta + 2 \pi n) i$, for $\theta \in (0, 2\pi)$.

We can glue together these copies of $U$, so that the functions $f_{(2\pi n, 2\pi (n+1))}$ glue to give a continuous function $L : \tilde{U} \to \C$.
\end{document}
