\documentclass{article}

\def\npart{II}
\def\nyear{2017}
\def\nterm{Michaelmas}
\def\nlecturer{Prof. P. Russell}
\def\ncourse{Graph Theory}
\ifx \nauthor\undefined
  \def\nauthor{Bhavik Mehta}
\else
\fi

\author{Based on lectures by \nlecturer \\\small Notes taken by \nauthor}
\date{\nterm\ \nyear}
\title{Part \npart\ -- \ncourse}

\usepackage[utf8]{inputenc}
\usepackage{amsmath}
\usepackage{amsthm}
\usepackage{amssymb}
\usepackage{enumerate}
\usepackage{mathtools}
\usepackage{graphicx}
\usepackage[dvipsnames]{xcolor}
\usepackage{tikz}
\usepackage{wrapfig}
\usepackage{centernot}
\usepackage{float}
\usepackage{braket}
\usepackage[hypcap=true]{caption}
\usepackage{enumitem}
\usepackage[colorlinks=true, linkcolor=mblue]{hyperref}
\usepackage[nameinlink,noabbrev]{cleveref}
\usepackage{nameref}
\usepackage[margin=1.5in]{geometry}

% Theorems
\theoremstyle{definition}
\newtheorem*{aim}{Aim}
\newtheorem*{axiom}{Axiom}
\newtheorem*{claim}{Claim}
\newtheorem*{cor}{Corollary}
\newtheorem*{conjecture}{Conjecture}
\newtheorem*{defi}{Definition}
\newtheorem*{eg}{Example}
\newtheorem*{ex}{Exercise}
\newtheorem*{fact}{Fact}
\newtheorem*{law}{Law}
\newtheorem*{lemma}{Lemma}
\newtheorem*{notation}{Notation}
\newtheorem*{prop}{Proposition}
\newtheorem*{question}{Question}
\newtheorem*{rrule}{Rule}
\newtheorem*{thm}{Theorem}
\newtheorem*{assumption}{Assumption}

\newtheorem*{remark}{Remark}
\newtheorem*{warning}{Warning}
\newtheorem*{exercise}{Exercise}

% \newcommand{\nthmautorefname}{Theorem}

\newtheorem{nthm}{Theorem}[section]
\newtheorem{nlemma}[nthm]{Lemma}
\newtheorem{nprop}[nthm]{Proposition}
\newtheorem{ncor}[nthm]{Corollary}
\newtheorem{ndef}[nthm]{Definition}

% Special sets
\newcommand{\C}{\mathbb{C}}
\newcommand{\N}{\mathbb{N}}
\newcommand{\Q}{\mathbb{Q}}
\newcommand{\R}{\mathbb{R}}
\newcommand{\Z}{\mathbb{Z}}

\newcommand{\abs}[1]{\left\lvert #1\right\rvert}
\newcommand{\norm}[1]{\left\lVert #1\right\rVert}
\renewcommand{\vec}[1]{\boldsymbol{\mathbf{#1}}}

\let\Im\relax
\let\Re\relax

\DeclareMathOperator{\Im}{Im}
\DeclareMathOperator{\Re}{Re}
\DeclareMathOperator{\id}{id}

\definecolor{mblue}{rgb}{0., 0.05, 0.6}

\hypersetup{unicode=true}

% preamble
\usepackage{chngcntr}
\usepackage{ifthen}
\usepackage{pifont}
\usepackage{bbm}

\newcommand{\xmark}{\ding{55}}

\setcounter{section}{-1}
\usetikzlibrary{positioning,decorations.pathmorphing, calc, backgrounds, fadings}
\tikzset{node/.style = {circle,draw,inner sep=0.8mm}}

\counterwithout{nthm}{section}

\DeclareMathOperator{\ext}{ex}
\DeclareMathOperator{\ud}{ud}
\DeclareMathOperator{\Var}{Var}
\DeclareMathOperator{\Tr}{Tr}
\DeclarePairedDelimiter\ceil{\lceil}{\rceil}
\DeclarePairedDelimiter\floor{\lfloor}{\rfloor}

\newtheorem{manualtheoreminner}{Theorem}
\newenvironment{manualtheorem}[1]{%
    \renewcommand\themanualtheoreminner{#1}%
    \manualtheoreminner
}{\endmanualtheoreminner}
% and here we go!


\begin{document}
\maketitle



\clearpage
\section{Introduction}
\subsection{Preliminary}

\subsection{Informal definitions}



\subsection{Where do such structures arise?}

\clearpage
\section{Ramsey Theory}





















































\begin{nprop}\label{prop:1}
    Let $k$ be a positive integer.
    Then there is a positive integer $n$ such that whenever the edges of $K_n$ are coloured with $k$ colours we can find a monochromatic triangle.
\end{nprop}









\begin{nthm}[Ramsey's Theorem]\label{thm:ramsey}
    $R(s, t)$ exists for all $s, t \geq 2$.
    Moreover, if $s, t > 2$ then $R(s, t) \leq R(s-1, t) + R(s, t-1)$.
\end{nthm}

% cor 3
\begin{ncor}\label{cor:3}
    For all $s, t \geq 2$, $R(s, t) \leq 2^{s+t}$, so $R(s) \leq 4^s$.
\end{ncor}


\begin{nthm}[Multicolour Ramsey Theorem]\label{thm:multiRamsey}
    Let $k \geq 1$ and $s \geq 2$.
    Then there exists some $n$ such that whenever the edges of $K_n$ are coloured with $k$ colours, we can find a monochromatic $K_s$.
\end{nthm}


















\begin{nthm}[Infinite Ramsey Theorem]\label{thm:infRamsey}
    Let $k \geq 1$. Whenever the edges of $K_\infty$ are $k$-coloured, we have a monochromatic $K_\infty$ subgraph.
\end{nthm}


\begin{ncor}
    Any bounded sequence has a convergent subsequence.
\end{ncor}

\subsection{Basic Terminology}



















% TODO thing about transitivity










\clearpage
\section{Extremal Graph Theory}



























\subsection{Forbidden Subgraph Problem}

























\begin{nthm}\label{thm:7}
    A graph is bipartite iff it contains no odd cycles.
\end{nthm}












\begin{nthm}[Mantel's Theorem]\label{thm:8}
    Let $n \geq 3$. Suppose $\abs{G} = n$, $e(g) \geq \floor{\frac{n^2}{4}}$ and $\triangle \not\subset G$.
    Then $G \cong K_{\ceil{\frac{n}{2}}, \floor{\frac{n}{2}}}$.
\end{nthm}











\begin{nthm}[Tur\'{a}n's Theorem]\label{thm:9}
    Let $r \geq 2$ and $\abs{G} = n \geq r+1$. If $e(G) \geq t_r(n)$ and $K_{r+1} \not\subset G$ then $G \cong T_r(n)$.
\end{nthm}








\begin{ncor}\label{cor:10}
    Let $r \geq 2$. As $n \to \infty$, $\ext(n; K_{r+1}) \sim (1 - \frac{1}{r}) \binom{r}{2}$.
\end{ncor}












\begin{nthm}\label{thm:11}
    Let $t \geq 2$. Then $\ext(n; K_{t,t}) = \mathcal{O}\left(n^{2 - \frac{1}{t}}\right)$.
\end{nthm}








\begin{nthm}\label{thm:12}
    Let $t \geq 2$. Then $z(n, t)  = \mathcal{O}(n^{2 - \frac{1}{t}})$.
\end{nthm}







\begin{nprop}\label{prop:13}
    Let $H$ be a graph with at least one edge, and for $n \geq \abs{H}$, let $x_n = \frac{\ext(n; H)}{\binom{n}{2}}$. Then $(x_n)$ converges.
\end{nprop}





\begin{nthm}[Erd\H{o}s-Stone Theorem]\label{thm:14}
    Let $r, t \geq 1$ be integers, and let $\epsilon > 0$ be real.
    Then $\exists n_0$ such that $\forall n \geq n_0$,
    \begin{equation*}
        \abs{G} = n,\ e(G) \geq \left(1 - \frac{1}{r} + \epsilon\right) \binom{n}{2} \implies K_{r+1}(t) \subset G.
    \end{equation*}
\end{nthm}





\begin{ncor}\label{cor:15}
    Let $H$ be a graph with at least one edge. Then
    \begin{equation*}
        \ext(H) = 1 - \frac{1}{\chi(H) - 1}.
    \end{equation*}
\end{ncor}






















\begin{ncor}
    For any infinite graph $G$,
    \begin{equation*}
        \ud(G) \in \left\{0, 1, \frac{1}{2}, \frac{2}{3}, \frac{3}{4}, \dotsc\right\}.
    \end{equation*}
\end{ncor}





















\subsection{Hamiltonian graphs}









\begin{nthm}[Dirac's Theorem]\label{thm:17}
    Let $\abs{G} = n \geq 3$ and $\delta(G) \geq \frac{n}{2}$. Then $G$ is Hamiltonian.
\end{nthm}









\begin{nprop}\label{thm:18}
    Let $G$ be a connected graph. Then
    \begin{equation*}
        G\text{ Eulerian if and only if } \forall v \in G,\ d(v)\text{ is even.}
    \end{equation*}
\end{nprop}

\clearpage
\section{Graph Colouring}







\subsection{Planar Graphs}










\begin{nthm}[Kuratowski's Theorem]\label{thm:19}
    Let $G$ be a graph. Then $G$ planar iff $G$ contains no subdivision of $K_5$ or $K_{3,3}$.
\end{nthm}





\begin{nprop}\label{prop:20}
    Every tree of order at least $2$ has a leaf.
\end{nprop}








\begin{nprop}\label{prop:21}
    Let $T$ be a tree, $\abs{T} = n \geq 1$. Then $e(T) = n-1$.
\end{nprop}

\begin{nprop}\label{prop:22}
    Every tree is planar.
\end{nprop}



\begin{nthm}[Euler's Formula]\label{thm:23}
    Take $G$ connected and planar.
    Take $\abs{G} = n \geq 1$, $e(G) = m$ with $l$ faces.
    Then $n - m + l = 2$.
\end{nthm}

\begin{ncor}\label{cor:24}
    Let $G$ be planar, $\abs{G} = n \geq 3$. Then $e(G) \leq 3n-6$.
\end{ncor}

\begin{nprop}[Six colour theorem]\label{prop:25}
    Any planar graph is 6-colourable.
\end{nprop}

\begin{nthm}[Five colour theorem]\label{thm:26}
    Any planar graph is 5-colourable.
\end{nthm}

\begin{nthm}[Four colour theorem]\label{thm:27}
    Any planar graph is 4-colourable.
\end{nthm}

% new lec


\subsection{General Graphs}









{
}


{
}















\begin{nthm}[Brooks' theorem]\label{thm:28}
    Let $G$ be a connected graph that is neither complete nor an odd cycle.
    Then $\chi(G) \leq \Delta(G)$.
\end{nthm}

\subsection{Graphs on surfaces}






















\begin{nthm}[Heawood's Theorem]\label{thm:29}
    Let $S$ be a closed boundaryless surface of Euler characteristic $E \leq 1$. Then
    \begin{equation*}
        \chi(S) \leq \floor*{\frac{7 + \sqrt{49-24E}}{2}}.
    \end{equation*}
\end{nthm}





























\subsection{Edge Colouring}




\begin{nthm}[Vizing's theorem]\label{thm:30}
    Let $G$ be a graph. Then
    \begin{equation*}
        \chi'(G) \leq \Delta(G) + 1
    \end{equation*}
\end{nthm}

\clearpage
\section{Connectivity}
\subsection{The Marriage Problem}















\begin{nthm}[Hall's Marriage Theorem]\label{thm:31}
    Let $G$ be a bipartite graph with bipartition $(X,Y)$.
    Then $G$ has a matching from $X$ to $Y$ iff $G$ satisfies \textbf{Hall's condition}:
    \begin{equation*}
        \hypertarget{def:hall}\forall A \subset X, \abs{\Gamma(A)} \geq |A|
    \end{equation*}
\end{nthm}






\begin{ncor}[Defect Hall]\label{cor:32}
    Let $G$ be a bipartite graph with bipartition $(X,Y)$ and let $d \geq 1$.
    Then $G$ contains $|X|-d$ independent edges if and only if $\forall A \subset X, \ |\Gamma(A)| \geq |A| - d$.
\end{ncor}

\begin{ncor}[Polyandrous Hall]\label{cor:33}
    Let $G$ be a bipartite graph, bipartition $(X,Y)$, $d \geq 2$.
    Then $G$ contains a set of $d |X|$ edges, each vertex in $X$ in precisely $d$ of them, each vertex in $Y$ in at most one $\iff \forall A \subset X, |\Gamma(A)| \geq d |A|$.
\end{ncor}



\subsection{Connectivity}
























\begin{nthm}\label{thm:34}
    Let $G$ be a graph and $A,B \subset V(G)$.
    Let
    \begin{equation*}
        k = \min \set{\abs{W} | W \text{ is an $AB$-separator}}.
    \end{equation*}
    Then $G$ contains $k$ vertex-disjoint $AB$-paths.
\end{nthm}





\begin{ncor}[Menger's Theorem]\label{cor:35}
    Let $G$ be an incomplete $k$-connected graph and let $a,b \in V(G)$, $a \neq b$.
    Then $G$ contains $k$ independent $ab$-paths.
\end{ncor}































\subsection{Edge connectivity}

\begin{ncor}[Edge Menger]\label{cor:36}
    Let $G$ be $l$-edge connected and $a,b \in V(G)$ be distinct.
    Then $G$ has $l$ edge-disjoint $ab$-paths.
\end{ncor}

\clearpage
\section{Probabilistic Techniques}
\subsection{The Probabilistic Method}







\begin{nthm}[Erd\H{o}s]\label{thm:37}
    \begin{equation*}
        R(s) = \Omega(\sqrt{2}^s)
    \end{equation*}
\end{nthm}

\subsection{Modifying a Random Graph}













\begin{nthm}\label{thm:38}
    If $t \geq 2$ then $z(n,t) = \Omega(n^{2 - \frac{2}{t+1}})$.
\end{nthm}




\begin{nthm}\label{thm:39}
    Let $g \geq 3$, $k \geq 2$. Then there is a graph $G$ with no cycles of length $\leq g$ and $\chi(G) \geq k$.
\end{nthm}


{
}

\subsection{The Structure of Random Graphs}






































































\begin{nprop}\label{prop:40}
    $p = \frac{1}{n}$ is a sharp threshold for $G \in \mathcal{G}(n,p)$ to contain a $\triangle$, in the sense that:
    \begin{itemize}
        \item if $p = o(\frac{1}{n})$ then almost every $G \in \mathcal{G}(n,p)$ has no $\triangle$, whereas
        \item if $p = \omega(\frac{1}{n})$ then almost every $G \in \mathcal{G}(n,p)$ has a $\triangle$.
    \end{itemize}
\end{nprop}





\begin{nthm}\label{thm:41}
    There exists a function $d: \mathbb{N} \to \mathbb{N}$ such that a.e.\ $G \in \mathcal{G}(n,p)$ has $\omega(G) \in \{d-1,d,d+1\}$ (where $d = d(n)$).
\end{nthm}


\begin{ncor}\label{cor:42}
    Almost every $G \in \mathcal{G}(n,p)$ has
    \begin{equation*}
        \chi(G) \geq (1 + o(1))\frac{n \log \frac{1}{q}}{2 \log n}
    \end{equation*}
    where $q = 1-p.$
\end{ncor}




\clearpage
\section{Algebraic Methods}



























\subsection{The Chromatic Polynomial}










\begin{nthm}[Cut-fuse relation]\label{thm:43}
    Let $G$ be a graph, $e \in E(G)$, $k \geq 1$. Then $f_G(k) = f_{G-e}(k) - f_{G/e} (k)$.
\end{nthm}

\begin{ncor}\label{cor:44}
    Let $G$ be a graph. Then $f_G$ is a polynomial.
\end{ncor}














\begin{ncor}\label{cor:45}
    If $|G| = n$, $e(G) = m$ then
    \begin{equation*}
        f_G(X) = X^n - m X^{n-1} + \dotsc
    \end{equation*}
\end{ncor}

\subsection{Eigenvalues}































\begin{nthm}\label{thm:46}
    Let $G$ be a graph, $\Delta(G) = \Delta$, $\lambda$ an eigenvalue of $G$.
    Then $|\lambda| \leq \Delta$.
    Moreover if $G$ is connected then $\Delta$ is an eigenvalue $\iff G$ is $\Delta$-regular; in this case $\Delta$ has multiplicity 1 and eigenvector $\left(\begin{smallmatrix}1\\1\\  \vphantom{\int\limits^x}\smash{\vdots}\\1\end{smallmatrix}\right)$.
\end{nthm}

\subsection{Strongly Regular Graphs}










































\begin{nthm}[Rationality condition]\label{thm:47}
    Let $G$ be $(k,a,b)$-strongly regular.
    Then
    \begin{equation*}
        \frac{1}{2} \left\{ (n-1) \pm \frac{(a-b)(n-1) + 2k}{\sqrt{(b-a)^2 - 4(b-k)}}\right\} \in \mathbb{Z}.
    \end{equation*}
\end{nthm}


\end{document}