\documentclass{article}

\def\npart {II}
\def\nyear {2017}
\def\nterm {Michaelmas}
\def\nlecturer{Dr C. Brookes}
\def\ncourse{Galois Theory}
\usepackage{imakeidx}
\ifx \nauthor\undefined
  \def\nauthor{Bhavik Mehta}
\else
\fi

\author{Based on lectures by \nlecturer \\\small Notes taken by \nauthor}
\date{\nterm\ \nyear}
\title{Part \npart\ -- \ncourse}

\usepackage[utf8]{inputenc}
\usepackage{amsmath}
\usepackage{amsthm}
\usepackage{amssymb}
\usepackage{enumerate}
\usepackage{mathtools}
\usepackage{graphicx}
\usepackage[dvipsnames]{xcolor}
\usepackage{tikz}
\usepackage{wrapfig}
\usepackage{centernot}
\usepackage{float}
\usepackage{braket}
\usepackage[hypcap=true]{caption}
\usepackage{enumitem}
\usepackage[colorlinks=true, linkcolor=mblue]{hyperref}
\usepackage[nameinlink,noabbrev]{cleveref}
\usepackage{nameref}
\usepackage[margin=1.5in]{geometry}

% Theorems
\theoremstyle{definition}
\newtheorem*{aim}{Aim}
\newtheorem*{axiom}{Axiom}
\newtheorem*{claim}{Claim}
\newtheorem*{cor}{Corollary}
\newtheorem*{conjecture}{Conjecture}
\newtheorem*{defi}{Definition}
\newtheorem*{eg}{Example}
\newtheorem*{ex}{Exercise}
\newtheorem*{fact}{Fact}
\newtheorem*{law}{Law}
\newtheorem*{lemma}{Lemma}
\newtheorem*{notation}{Notation}
\newtheorem*{prop}{Proposition}
\newtheorem*{question}{Question}
\newtheorem*{rrule}{Rule}
\newtheorem*{thm}{Theorem}
\newtheorem*{assumption}{Assumption}

\newtheorem*{remark}{Remark}
\newtheorem*{warning}{Warning}
\newtheorem*{exercise}{Exercise}

% \newcommand{\nthmautorefname}{Theorem}

\newtheorem{nthm}{Theorem}[section]
\newtheorem{nlemma}[nthm]{Lemma}
\newtheorem{nprop}[nthm]{Proposition}
\newtheorem{ncor}[nthm]{Corollary}
\newtheorem{ndef}[nthm]{Definition}

% Special sets
\newcommand{\C}{\mathbb{C}}
\newcommand{\N}{\mathbb{N}}
\newcommand{\Q}{\mathbb{Q}}
\newcommand{\R}{\mathbb{R}}
\newcommand{\Z}{\mathbb{Z}}

\newcommand{\abs}[1]{\left\lvert #1\right\rvert}
\newcommand{\norm}[1]{\left\lVert #1\right\rVert}
\renewcommand{\vec}[1]{\boldsymbol{\mathbf{#1}}}

\let\Im\relax
\let\Re\relax

\DeclareMathOperator{\Im}{Im}
\DeclareMathOperator{\Re}{Re}
\DeclareMathOperator{\id}{id}

\definecolor{mblue}{rgb}{0., 0.05, 0.6}

\makeindex[intoc]

% preamble
\setcounter{section}{-1}
\usepackage{tkz-euclide}
\usepackage{xfrac}
\usepackage{stmaryrd}
\SetSymbolFont{stmry}{bold}{U}{stmry}{m}{n}
\usetkzobj{all}
\usetikzlibrary{cd, backgrounds}

\DeclareMathOperator{\Aut}{Aut}
\DeclareMathOperator{\chara}{char}
\DeclareMathOperator{\Tr}{Tr}
\DeclareMathOperator{\Gal}{Gal}
\DeclareMathOperator{\Ker}{Ker}

\newtheorem{nexample}[nthm]{Example}
\newtheorem{nremark}[nthm]{Remark}
\newcommand{\F}{\mathbb{F}}

\newtheorem{manualinner}{}
\newenvironment{manual}[1]{%
    \renewcommand\themanualinner{#1}%
    \manualinner
}{\endmanualinner}
% and here we go!


\begin{document}
\maketitle

% lecture 1

\clearpage
\section{Introduction}

\subsection{Course overview}










% Galois' papers have been studied by Peter Neumann:
% The math writings of Evariste Galois, European Math Soc
% Different books: I. Steward Galois Theory, (something) and Hall
% contains a historical introduction and covers almost all the syllabus.
% Artin Galois Theory
% Van der Waerden Modern Algebra (covers a lot more than Galois theory)
% Lang Algebra (late editions are preferred, covers a lot of algebra)
% Kaplansky Fields and Rings


\clearpage
\section{Field Extensions}\label{sec:1}









\begin{nthm}[Tower law]\index{tower law}\label{thm:towerLaw}
    Suppose $K \leq L \leq M$ are field extensions. Then $\abs{M:K}$ = $\abs{M:L}\abs{L:K}$.
\end{nthm}










\subsection{Motivatory Example}\label{sec:motivEg}







% new lec (2)



























\subsection{Review of GRM}



\begin{nlemma}\label{lem:1.5}
    Let $K \leq L$ be a finite field extension. Then $L$ is algebraic over $K$.
\end{nlemma}




\begin{nlemma}\label{lem:1.7}
    Suppose $K \leq L$ is a field extension, $\alpha \in L$ and $\alpha$ is algebraic over $K$.
    Then the minimal polynomial $f_\alpha(t)$ of $\alpha$ over $K$ is irreducible in $K[t]$ and $I_\alpha$ is a prime ideal.
\end{nlemma}


% new lec (3)

\begin{nthm}\label{thm:1.9}
    Suppose $K \leq L$ is a field extension and $\alpha \in L$ is algebraic over $K$.  Then
    \begin{enumerate}[label=(\roman*)]
        \item  $K(\alpha) = K[\alpha]$
        \item $\abs{K(\alpha) : K} = \deg f_\alpha(t)$ where $f_\alpha(t)$ is the minimal polynomial of $\alpha$ over $K$.
    \end{enumerate}
\end{nthm}

\begin{ncor}\label{cor:1.10}
    If $K \leq L$ is a field extension and $\alpha \in L$, then $\alpha$ is algebraic over $K$ if and only if $K \leq K(\alpha)$ is finite.
\end{ncor}

\begin{ncor}\label{cor:1.11}
    Let $K \leq L$ be a field extension with $\abs{L:K} = n$. Let $\alpha \in L$, then $\deg f_\alpha(t) \mid n$.
\end{ncor}

\subsection{Digression on (Non-)Constructibility}















\begin{nlemma}\label{lem:1.13}
    $x_i, y_i$ are both roots in $K_i$ of quadratic polynomials in $K_{i-1}[t]$.
\end{nlemma}

% new lec (4)

\begin{nthm}\label{thm:1.14}
    If $\vec{r} = (x, y)$ is constructible from a set $P_0$ of points in $\R^2$ and if $K_0$ is the subfield of $\R$ generated by $\Q$ and the coordinates of the points in $P_0$, then the degrees $\abs{K_0(x) : K_0}$ and $\abs{K_0(y):K_0}$ are powers of two.
\end{nthm}



\begin{nthm}\label{thm:1.15}
    Let $f(t)$ be a primitive integral polynomial.  Then $f(t)$ is irreducible in $\Q[t]$ if and only if it is irreducible in $\Z[t]$.
\end{nthm}

\begin{nthm}[Eisenstein's criterion]\index{Eisenstein's criterion}\label{thm:1.16}
    Let $f(t) = a_n t^n + a_{n-1} t^{n-1} + \dots + a_0 \in \Z[t]$.
    Suppose there is a prime $p$ such that
    \begin{enumerate}[label=(\roman*)]
        \item $p \nmid a_n$
        \item $p \mid a_{n-1}, \, p \mid a_{n-2}, \dotsc, p \mid a_0$
        \item $p^2 \nmid a_0$
    \end{enumerate}
    Then $f(t)$ is irreducible in $\Z[t]$
\end{nthm}












% back to non-constructibility
\begin{nthm}\label{thm:1.17}
    The cube cannot be duplicated by ruler and compasses.
\end{nthm}

\begin{nthm}\label{thm:1.18}
    The circle cannot be squared using ruler and compasses.
\end{nthm}

\subsection{Return to theory development}
\begin{nlemma}\label{lem:1.19}
    Let $K \leq L$ be a field extension. Then
    \begin{enumerate}[label=(\roman*)]
        \item $\alpha_1, \dotsc, \alpha_n \in L$ are algebraic over $K$ if and only if $K \leq K(\alpha_1, \dotsc, \alpha_n)$ is a finite field extension.
        \item If $K \leq M \leq L$ such that $K \leq M$ is finite, then there exist $\alpha_1, \dotsc, \alpha_n \in L$ such that $K(\alpha_1, \dotsc, \alpha_n) = M$.
    \end{enumerate}
\end{nlemma}

% new lec (5)


\begin{nlemma}\label{lem:1.21}
    Suppose $K \leq L$, $K \leq L'$ are field extensions. Then
    \begin{enumerate}[label=(\roman*)]
        \item Any $K$-homomorphism $\phi:L \to L'$ is injective and $K \leq \phi(L)$ is a field extension.
        \item If $|L:K| = |L':K| < \infty$ then any $K$-homomorphism $\phi:L \to L'$ is a $K$-isomorphism.
    \end{enumerate}
\end{nlemma}

% \begin{notation}
%     If $K \leq L$ is a \hyperlink{def:fieldExt}{field extension} and $f(t) \in K[t]$, we denote the set of roots of $f$ in $L$ by $\Root_f(L)$.
% \end{notation}




\begin{nthm}[Existence of splitting fields]\index{splitting field!existence}\label{thm:1.23}
    Let $K$ be a field and $f(t) \in K[t]$. Then there exists a splitting field for $f$ over $K$.
\end{nthm}

\begin{nthm}[Uniqueness of splitting fields]\index{splitting field!uniqueness}\label{thm:1.24}
    If $K$ is a field and $f(t) \in K[t]$, then the splitting field for $f$ over $K$ is unique up to $K$-isomorphism, that is, if there are two such splitting fields $L$ and $L'$, there is a $K$-isomorphism $\phi: L \to L'$.
\end{nthm}
% new lec (6)

% need to make sure this makes sense


\begin{nthm}\label{thm:1.26}
    Let $K \leq L$ be a finite field extension. Then $K \leq L$ is normal $\iff$ $L$ is the splitting field for some $f(t) \in K[t]$.
\end{nthm}


\begin{nthm}\label{thm:1.28}
    Let $G$ be a finite subgroup of the multiplicative group of a field $K$. Then $G$ is cyclic. In particular, the multiplicative group of a finite field is cyclic.
\end{nthm}

\clearpage
\section{Separable, normal and Galois extensions}

% new lec (7)


\begin{nlemma}\label{lem:2.3}
    Let $K$ be a field and $f(t), g(t) \in K[t]$. Then:
    \begin{enumerate}[label=(\alph*)]
        \item $D(f(t) g(t)) = f'(t) g(t) + f(t) g'(t)$ (Leibniz' rule)
        \item Assume $f(t) \neq 0$. Then $f(t)$ has a repeated root in a splitting field $L$ if and only if $f(t)$ and $f'(t)$ have a common irreducible factor in $K[t]$.
    \end{enumerate}
\end{nlemma}

\begin{ncor}\label{cor:2.4}
    If $K$ is a field and $f(t) \in K[t]$ is irreducible:
    \begin{enumerate}[label=(\roman*)]
        \item If the characteristic of $K$ is 0, then $f(t)$ is separable over $K$.
        \item If the characteristic of $K$ is $p>0$, then $f(t)$ is not separable if and only if $f(t) \in K[t^p]$.
    \end{enumerate}
\end{ncor}




\begin{nlemma}\label{lem:2.6}
    Let $M = K(\alpha)$, where $\alpha$ is algebraic over $K$ and let $f_\alpha(t)$ be the minimal polynomial of $\alpha$ over $K$.

    Then, for any field extension $K \leq L$, the number of $K$-homomorphisms of $M$ to $L$ is equal to the number of distinct roots of $f_\alpha(t)$ in $L$.
    Thus this number is $\leq \deg f_\alpha(t) = \abs{K(\alpha):K} = \abs{M:K}$.

    \begin{center}
        \begin{tikzpicture}[scale=0.8]
            \node (K) at ( 0,-2)    {$K$};
            \node (L) at ( 2, 2)    {$L$};
            \node (M) at (-2, 0)    {$M = K(\alpha)$};

            \draw (M) -- (K) -- (L);
        \end{tikzpicture}
    \end{center}
\end{nlemma}

% new lec (8)


\begin{ncor}\label{cor:2.7}
    The number of $K$-homomorphisms $K(\alpha) \to L = \deg f_\alpha(t) \iff L$ is large enough, in particular $L$ contains a splitting field for $f_\alpha(t)$ and $\alpha$ is separable over $K$.
\end{ncor}

\begin{nlemma}\label{lem:2.8}
    Let $K \leq M$ be a field extension and $M_1 = M(\alpha_1)$ (where $\alpha_1$ is algebraic over $M$).
    Let $f(t)$ be the minimal polynomial of $\alpha_1$ over $M$ and let $K \leq L$.
    Let $\phi: M \to L$ be a $K$-homomorphism. Then there is a correspondence
    \begin{equation*}
        \{\text{Extensions } \phi_1:M_1 \to L \text{ of } \phi\} \longleftrightarrow \{\text{roots of} \ \phi(f(t)) \in L\}.
    \end{equation*}
    \begin{center}
        \begin{tikzpicture}
            \node (K)  at ( 0,-2)    {$K$};
            \node (M)  at (-1, -0.5)    {$M$};
            \node (M1) at (-2, 1)    {$M_1$};
            \node (L)  at ( 2, 2)    {$L$};

            \draw (M1) -- (M) -- (K) -- (L);
        \end{tikzpicture}
    \end{center}
\end{nlemma}


\begin{ncor}\label{cor:2.9}
    If $L$ is large enough, the number of $\phi_1$ which extend $\phi$ is equal to the number of distinct roots of $f(t)$ in $L$.
    This is equal to $|M_i:M| \iff \alpha$ is separable over $M$.
\end{ncor}

\begin{ncor}\label{cor:2.10}
    Let $K \leq M \leq N$ be finite field extensions, $K \leq L$. Let $\phi: M \to L$ be a $K$-homomorphism.
    Then the number of extensions of $\phi$ to maps $\theta:N \to L$ is $\leq \abs{N:M}$.
    Moreover, such a $\theta$ exists if $L$ is large enough.
\end{ncor}


\begin{nlemma}\label{lem:2.12}
    Let $K \leq N$ be a field extension with $\abs{N:K} = n$ and $N = K(\alpha_1, \dotsc, \alpha_r)$ say.
    Then the following are equivalent:
    \begin{enumerate}[label=(\roman*)]
        \item $N$ is separable over $K$.
        \item Each $\alpha_i$ is separable over $K(\alpha_1, \dotsc, \alpha_{i-1})$.
        \item If $K \leq L$ is large enough there are exactly $n$ distinct $K$-homomorphisms $N \to L$.
    \end{enumerate}
\end{nlemma}


\begin{ncor}\label{cor:2.14}
    A finite extension is separable $\iff$ it is separably generated.
\end{ncor}

\begin{nlemma}\label{lem:2.15}
    If $K \leq M \leq L$ finite field extensions, $M \leq L$, then
    \begin{equation*}
        K \leq M, \; \; M \leq L \text{ are both separable } \iff K \leq L \text{ is separable}
    \end{equation*}
\end{nlemma}


% new lec (9)

\begin{nthm}[Primitive Element Theorem]\index{Primitive element theorem}\label{thm:2.17}
    Any finite separable extension $K \leq M$ is a simple extension, that is, $M = K(\alpha)$ for some $\alpha$, called a primitive element.
\end{nthm}


\subsection{Trace and Norm}

\begin{nthm}\label{thm:2.19}
    With the above notation, suppose $f_\alpha(t) = t^s + a_{s-1} t^{s-1} + \dotsb + a_0$ is the minimal polynomial for $\alpha$ over $K$.
    Let $r = \abs{M:K(\alpha)}$, then the characteristic polynomial of $\theta_\alpha$ is $(f_\alpha(t))^r$.

    Note \begin{equation*}\abs{M:K} = \abs{M:K(\alpha)} |K(\alpha) : K| = rs.\end{equation*}
    Then $\Tr_{M/K} (\alpha) = -r a_{s-1}$ and $N_{M/K} = ((-1)^s a_0)^r$.
\end{nthm}

\begin{nthm}\label{thm:2.20}
    Let $K \leq M$ be a finite separable field extension and $|M:K| = n$, $\alpha \in M$.
    Let $K \leq L$ be large enough so that there are $n$ distinct $K$-homomorphisms
    \begin{equation*}
        \sigma_1, \sigma_2, \dotsc, \sigma_n: M \longrightarrow L.
    \end{equation*}
    Then the characteristic polynomial of $\theta_\alpha:M \to M$ (the multiplication map) is
    \begin{equation*}
        \prod_{i=1}^n (t-\sigma_i(\alpha))
    \end{equation*}
    hence
    \begin{equation*}
        \Tr_{M/K} (\alpha) = \sum_{i=1}^n \sigma_i(\alpha) \qquad \text{and} \qquad N_{M/K}(\alpha) = \prod_{i=1}^n \sigma_i(\alpha).
    \end{equation*}
\end{nthm}

% new lec (10)

\begin{nthm}\label{thm:2.21}
    Let $K \leq M$ be a finite separable extension.
    Then we define a $K$-bilinear form
    \begin{align*}
        T: M \times M &\rightarrow K \\
        (x, y) &\longmapsto \Tr_{M/K} (xy).
    \end{align*}
    Then this is non-degenerate and in particular the $K$-linear map $\Tr_{M/K}:M \to K$ is non-zero, and hence surjective.
\end{nthm}



\subsection{Normal extensions}









\begin{nlemma}\label{lem:2.23}
    \begin{equation*}
        \Aut_K(M) \leq \abs{M:K}.
    \end{equation*}
\end{nlemma}

\begin{nthm}\label{thm:2.24}
    Let $K \leq M$ be a finite field extension.
    Then $\abs{\Aut_K(M)} = \abs{M:K}$ iff the extension is both normal and separable.
\end{nthm}

% new lec (11)



% check this makes sense



\clearpage
\section{Fundamental Theorem of Galois Theory}
\subsection{Artin's Theorem}


\begin{nthm}[Fundamental Theorem of Galois Theory]\index{fundamental theorem}\label{thm:3.2}
    Let $K \leq L$ be a finite Galois extension.
    Then
    \begin{enumerate}[label=(\roman*)]
        \item there is a $1$ to $1$ correspondence
            \begin{align*}
                \{\text{intermediate subfields }K \leq M \leq L\} &\longleftrightarrow \{\text{subgroups $H$ of } \Gal(L/K)\} \\
                M &\longmapsto \Aut_M(L) \\
                L^H &\longmapsfrom H
            \end{align*}
            This is called the Galois correspondence.
        \item $H$ is a normal subgroup of $\Gal(L/K)$ iff $K \leq L^H$ is normal iff $K \leq L^H$ is Galois.
        \item If $H \lhd \Gal(L/K)$ then the map
            \begin{equation*}
                \theta: \Gal(L/K) \longrightarrow \Gal(L^H/K)
            \end{equation*}
            given by restriction to $L^H$ is a surjective group homomorphism with kernel $H$.
    \end{enumerate}
\end{nthm}


% new lec (12)

\begin{nthm}[Artin's Theorem]\index{Artin's theorem}\label{thm:3.3}
    Let $K \leq L$ be a field extension and $H$ a finite subgroup of $\Aut_K(L)$.
    Let $M = L^H$.
    Then $M \leq L$ is a finite Galois extension, and $H = \Gal(L/M)$.
\end{nthm}


\begin{nthm}\label{thm:3.4}
    Let $K \leq L$ be a finite field extension. Then the following are equivalent:
    \begin{enumerate}[label=(\roman*)]
        \item $K \leq L$ is Galois
        \item $L^H = K$ when $H = \Aut_K(L)$
    \end{enumerate}
\end{nthm}



\subsection{Galois groups of polynomials}

















\begin{nlemma}\label{lem:3.6}
    Suppose $f(t)$ is separable, $f(t) = g_1(t) \dotsm g_s(t)$ with $g_i(t)$ irreducible in $K[t]$ is a factorisation in $K[t]$.
    Then the orbits of $\Gal(f)$ on the roots of $f(t)$ correspond to the factors $g_j(t)$.
    \begin{equation*}
        \text{Two roots are in the same orbit} \iff \text{they are roots of the same } g_j(t).
    \end{equation*}
    In particular, if $f(t)$ is irreducible in $K[t]$ there is one orbit, i.e., $\Gal(f)$ acts transitively on the roots of $f(t)$.
\end{nlemma}



\begin{nlemma}\label{lem:3.7}
    The transitive subgroups of $S_n$ for $n \leq 5$ are
    \begin{center}
        \begin{tabular}{rl}
            $n=2$:  & $S_2 \; (\cong C_2)$ \\
            $n=3$:  & $A_3 \; (\cong C_3), \; S_3$ \\
            $n=4$:  & $C_4, \; V_4, \; D_8, \; A_4, \; S_4$ \\
            $n=5$:  & $C_5, \; D_{10}, \; H_{20}, \; A_5, \; S_5$\\
        \end{tabular}
    \end{center}
    where $H_{20}$ is generated by a 5-cycle and a 4-cycle.
\end{nlemma}

\begin{nthm}\label{thm:3.8}
    Let $p$ be a prime, and $f(t)$ irreducible $\in \Q[t]$ of degree $p$.
    Suppose $f(t)$ has exactly 2 non-real roots in $\C$.
    Then $\Gal(f)$ over $\Q \cong S_p$.
\end{nthm}





\begin{nlemma}\label{lem:3.11}
    Let $f(t)$ be separable $\in K[t]$ of degree $n$ with $\chara K \neq 2$.
    Then
    \begin{equation*}\Gal(f) \leq A_n \iff D(f)\text{ is a square in }K.\end{equation*}
\end{nlemma}









\begin{nthm}[Mod $p$ reduction]\label{thm:3.13}
    Let $f(t) \in \Z[t]$ be monic of degree $n$ with $n$ distinct roots in a splitting field.
    Let $p$ be a prime such that $\overline{f}(t)$, the reduction of $f(t)$ mod $p$ also has $n$ distinct roots in a splitting field.
    Let $\overline{f}(t) = \overline{g_1}(t) \dotsm \overline{g_s}(t)$ be the factorisation into irreducibles in $\F_p[t]$ with $n_j = \deg \overline{g_j}(t)$.
    Then $\Gal(\overline{f}) \hookrightarrow \Gal(f)$ and has an element of cycle type $(n_1, n_2, \dotsc, n_s)$.
\end{nthm}


\subsection{Galois Theory of Finite Fields}






















\begin{nthm}[Galois groups of finite fields]\label{thm:3.16}
    Let $\F$ be a finite field with $\abs{\F} = p^r$.
    Then $\F_p \leq \F$ is a Galois extension with $\Gal(\F/\F_p) = G$, a cyclic group with the Frobenius automorphism as generator.
\end{nthm}

\begin{ncor}\label{cor:3.17}
    Let $\F_p \leq M \leq \F$ be finite fields.
    Then $\Gal(\F/M)$ is cyclic, generated by $\phi^u$, where $\phi$ is the Frobenius automorphism and $\abs{M} = p^u$ and $M$ is the fixed field of $\langle \phi^u \rangle$.
\end{ncor}

\begin{nthm}[Existence of finite fields]\index{finite field!existence}\label{thm:3.18}
    Let $p$ be a prime and $u \geq 1$.
    Then there is a field of order $p^u$, unique up to isomorphism.
\end{nthm}


\clearpage
\section{Cyclotomic and Kummer extensions}
\subsection{Cyclotomic extensions}




























% can i change this to a lemma




\begin{nlemma}\label{lem:4.5}
    $\Phi_m(t) \in \Z[t]$ if $\chara K = 0$ (with $\Q \hookrightarrow K$, prime subfield).
    $\Phi_m(t) \in \F_p[t]$ if $\chara K = p$ (with $\F_p \hookrightarrow K$, prime subfield).
\end{nlemma}

\begin{nlemma}\label{lem:4.6}
    The homomorphism $\theta: G \to (\Z/m\Z)^\times$ defined in \cref{def:4.3} is an isomorphism iff $\Phi_m(t)$ is irreducible.
\end{nlemma}

\begin{nthm}\label{thm:4.7}
    Let $L$ be the $m$th cyclotomic extension of finite field $\F = \F_q$ where $q = p^n$.
    Then the Galois group $G = \Gal(L/\F)$ is isomorphic to the cyclic subgroup of $(\Z/m\Z)^\times$ generated by $q$.
\end{nthm}




\begin{nthm}\label{thm:4.8}
    For all $m > 0$, $\Phi_m(t)$ is irreducible in $\Z[t]$ and hence in $\Q[t]$.
    Thus $\theta$ in \cref{def:4.3} is an isomorphism and thus $\Gal(\Q(\xi)/\Q) \cong (\Z/m\Z)^\times$ where $\xi =$ primitive $m$th root of unity.
\end{nthm}




\subsection{Kummer Theory}






\begin{nthm}\label{thm:4.10}
    Let $f(t) = t^m - \lambda \in K[t]$ and $\chara K \nmid m$.
    Then the splitting field $L$ of $f(t)$ over $K$ contains a primitive $m$th root of unity $\xi$ and $\Gal(L/K(\xi))$ is cyclic of order dividing $m$.
    Moreover $f(t)$ is irreducible over $K(\xi)$ iff $\abs{L:K(\xi)} = m$.
\end{nthm}



% \begin{eg}
%     Take $f(t) = t^5 - 2$ over $\mathbb{Q}$, irreducible by \nameref{thm:1.16}, and $L$ the splitting field of $f(t)$ over $\mathbb{Q}$.
%     Let $\xi$ be a \hyperlink{def:primRoot}{primitive fifth root of unity}.
% \end{eg}


\begin{nthm}\label{thm:4.11}
    Suppose $K \leq M$ is a cyclic extension with $\abs{L:K} = m$, where $\chara K \nmid m$ and that $K$ contains a primitive $m$th root of unity.
    Then $\exists \lambda \in K$ such that $t^m - \lambda$ is irreducible over $K$ and $K$ is the splitting field of $t^m - \lambda$ over $K$.
    If $\beta$ is a root of $t^m - \lambda$ in $L$, then $L = K(\beta)$.
\end{nthm}


\begin{nlemma}\label{lem:4.13}
    Let $\phi_1, \dotsc, \phi_n$ be embeddings of a field $K$ into a field $L$.
    Then there do not exist $\lambda_1, \dotsc, \lambda_n$ not all zero such that $\lambda_1 \phi_1(x) + \dotsb + \lambda_n \phi_n(x) = 0 \; \forall x \in K$.
\end{nlemma}



% lecture 18
\subsection{Cubics}

































{
}





























\subsection{Quartics}







































% Lecture 19
























\subsection{Solubility by radicals}





















\begin{nlemma}\label{lem:4.16}
    A finite group $G$ is soluble if and only if we have
    \begin{equation*}
        \{e\} = G_m \lhd G_{m-1} \lhd \dotsb \lhd G_1 \lhd G_0 = G
    \end{equation*}
    with $G_i/G_{i+1}$ cyclic.
\end{nlemma}


\begin{nlemma}\label{lem:4.18}
    Let $K \lhd G$. Then $G/K$ abelian $\iff G' \leq K$.
\end{nlemma}



\begin{nlemma}\label{lem:4.20}
    For $G$ finite, $G$ is soluble $\iff G^{(m)} = \{e\}$ for some $m$.
\end{nlemma}

\begin{nlemma}\label{lem:4.21}\leavevmode
    \begin{enumerate}[label=(\roman*)]
        \item Let $H \leq G$, $G$ soluble. Then $H$ soluble.
        \item Let $H \lhd G$, then $G$ soluble $\iff H$ and $G/H$ both soluble.
    \end{enumerate}
\end{nlemma}


\begin{nthm}\label{thm:4.22}
    Let $K$ be a field and $f(t) \in K[t]$.
    Assume $\chara K= 0$. Then $f(t)$ is soluble by radicals over $K \iff \Gal f$ over $K$ is soluble.
\end{nthm}

\begin{ncor}\label{cor:4.23}
    If $f(t)$ is a monic irreducible polynomial $\in K[t]$ with $\Gal(f) \cong A_5$ or $S_5$ then $f(t)$ is not soluble by radicals (with $\chara K$ = 0).
\end{ncor}

\begin{nlemma}\label{lem:4.24}
    If $K \leq N$ is an extension by radicals then $\exists N'$ with $N \leq N'$ with $K \leq N'$ is an extension by radicals, with $K \leq N'$ a Galois extension.
\end{nlemma}




\clearpage
\section{Final Thoughts}
\subsection{Algebraic closure}



\begin{nlemma}\label{lem:5.3}
    If $K \leq L$ is algebraic and every polynomial in $K[t]$ splits completely over $L$, then $L$ is an algebraic closure of $K$.
\end{nlemma}



\begin{nlemma}[Zorn's Lemma]\index{Zorn's lemma}\label{lem:zorn}
    Let $(\mathcal{S},\leq)$ be a non-empty partially ordered set.
    Suppose that any chain has an upper bound in $\mathcal{S}$.
    Then $\mathcal{S}$ has a maximal element.
\end{nlemma}
\begin{nlemma}\label{lem:5.6}
    Let $R$ be a ring. Then $R$ has a maximal ideal.
\end{nlemma}

\begin{nthm}[Existence of algebraic closures]\index{algebraic closure!existence}\label{thm:5.7}
    For any field $K$ there is an algebraic closure.
\end{nthm}

\begin{nthm}\label{thm:5.8}
    Suppose $\theta: K \to L$ is a ring homomorphism and $L$ is algebraically closed.
    Suppose $K \leq M$ is an algebraic extension.
    Then $\theta$ can be extended to a homomorphism $\theta: M \to L$ (i.e.\ $\phi|_K = \theta$).
\end{nthm}

\begin{nthm}[Uniquness of algebraic closures]\index{algebraic closure!uniqueness}\label{thm:5.9}
    If $K \leq L_1$, $L \leq L_2$ are two algebraic closures of $K$ then there exists an isomorphism $\phi:L_1 \to L_2$.
\end{nthm}

\subsection{Symmetric polynomials and invariant theory}


















\begin{nthm}\label{thm:5.11}
    The fixed field $M = L^{s_n} = K(s_1, \dotsc, s_n)$ and the $s_1, \dotsc, s_n$ are algebraically independent over $K$ (in $L$).
\end{nthm}


\end{document}