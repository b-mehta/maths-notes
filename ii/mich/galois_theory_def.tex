\documentclass{article}

\def\npart {II}
\def\nyear {2017}
\def\nterm {Michaelmas}
\def\nlecturer{Dr C. Brookes}
\def\ncourse{Galois Theory}
\usepackage{imakeidx}
\ifx \nauthor\undefined
  \def\nauthor{Bhavik Mehta}
\else
\fi

\author{Based on lectures by \nlecturer \\\small Notes taken by \nauthor}
\date{\nterm\ \nyear}
\title{Part \npart\ -- \ncourse}

\usepackage[utf8]{inputenc}
\usepackage{amsmath}
\usepackage{amsthm}
\usepackage{amssymb}
\usepackage{enumerate}
\usepackage{mathtools}
\usepackage{graphicx}
\usepackage[dvipsnames]{xcolor}
\usepackage{tikz}
\usepackage{wrapfig}
\usepackage{centernot}
\usepackage{float}
\usepackage{braket}
\usepackage[hypcap=true]{caption}
\usepackage{enumitem}
\usepackage[colorlinks=true, linkcolor=mblue]{hyperref}
\usepackage[nameinlink,noabbrev]{cleveref}
\usepackage{nameref}
\usepackage[margin=1.5in]{geometry}

% Theorems
\theoremstyle{definition}
\newtheorem*{aim}{Aim}
\newtheorem*{axiom}{Axiom}
\newtheorem*{claim}{Claim}
\newtheorem*{cor}{Corollary}
\newtheorem*{conjecture}{Conjecture}
\newtheorem*{defi}{Definition}
\newtheorem*{eg}{Example}
\newtheorem*{ex}{Exercise}
\newtheorem*{fact}{Fact}
\newtheorem*{law}{Law}
\newtheorem*{lemma}{Lemma}
\newtheorem*{notation}{Notation}
\newtheorem*{prop}{Proposition}
\newtheorem*{question}{Question}
\newtheorem*{rrule}{Rule}
\newtheorem*{thm}{Theorem}
\newtheorem*{assumption}{Assumption}

\newtheorem*{remark}{Remark}
\newtheorem*{warning}{Warning}
\newtheorem*{exercise}{Exercise}

% \newcommand{\nthmautorefname}{Theorem}

\newtheorem{nthm}{Theorem}[section]
\newtheorem{nlemma}[nthm]{Lemma}
\newtheorem{nprop}[nthm]{Proposition}
\newtheorem{ncor}[nthm]{Corollary}
\newtheorem{ndef}[nthm]{Definition}

% Special sets
\newcommand{\C}{\mathbb{C}}
\newcommand{\N}{\mathbb{N}}
\newcommand{\Q}{\mathbb{Q}}
\newcommand{\R}{\mathbb{R}}
\newcommand{\Z}{\mathbb{Z}}

\newcommand{\abs}[1]{\left\lvert #1\right\rvert}
\newcommand{\norm}[1]{\left\lVert #1\right\rVert}
\renewcommand{\vec}[1]{\boldsymbol{\mathbf{#1}}}

\let\Im\relax
\let\Re\relax

\DeclareMathOperator{\Im}{Im}
\DeclareMathOperator{\Re}{Re}
\DeclareMathOperator{\id}{id}

\definecolor{mblue}{rgb}{0., 0.05, 0.6}

\makeindex[intoc]

% preamble
\setcounter{section}{-1}
\usepackage{tkz-euclide}
\usepackage{xfrac}
\usepackage{stmaryrd}
\SetSymbolFont{stmry}{bold}{U}{stmry}{m}{n}
\usetkzobj{all}
\usetikzlibrary{cd, backgrounds}

\DeclareMathOperator{\Aut}{Aut}
\DeclareMathOperator{\chara}{char}
\DeclareMathOperator{\Tr}{Tr}
\DeclareMathOperator{\Gal}{Gal}
\DeclareMathOperator{\Ker}{Ker}

\newtheorem{nexample}[nthm]{Example}
\newtheorem{nremark}[nthm]{Remark}
\newcommand{\F}{\mathbb{F}}

\newtheorem{manualinner}{}
\newenvironment{manual}[1]{%
    \renewcommand\themanualinner{#1}%
    \manualinner
}{\endmanualinner}
% and here we go!


\begin{document}
\maketitle

% lecture 1

\clearpage
\section{Introduction}

\subsection{Course overview}










% Galois' papers have been studied by Peter Neumann:
% The math writings of Evariste Galois, European Math Soc
% Different books: I. Steward Galois Theory, (something) and Hall
% contains a historcal introduction and covers almost all the syllabus.
% Artin Galois Theory
% Van der Waerden Modern Algebra (covers a lot more than Galois theory)
% Lang Algebra (late editions are preferred, covers a lot of algebra)
% Kaplansky Fields and Rings


\clearpage
\section{Field Extensions}\label{sec:1}
\begin{ndef}[Field extension]\index{field extension}\hypertarget{def:fieldExt}
    A \textbf{field extension} $K \leq L$ is the inclusion of a field $K$ into another field $L$ with the same $0$, $1$, and where the restriction of $+$ and $\cdot$ (in L) to $K$ gives the $+$ and $\cdot$ of $K$.
\end{ndef}






\begin{ndef}[Degree]\index{degree}\hypertarget{def:degreeOfFieldExt}
    The \textbf{degree} of $L$ over $K$ is $\dim_K L$, the $K$-vector space dimension of $L$. This may not be finite. We typically denote this by $\abs{L:K}$.
    If $\abs{L:K} < \infty$, then the extension is \textbf{finite}, otherwise the extension is \textbf{infinite}.
\end{ndef}












\subsection{Motivatory Example}\label{sec:motivEg}







% new lec (2)



























\subsection{Review of GRM}
\begin{ndef}[Algebraic]\index{algebraic}\hypertarget{def:algebraic}
    Suppose $K \leq L$ is a field extension. Take $\alpha \in L$ and define
    \begin{equation*}
        I_\alpha = \set{f \in K[t] | f(\alpha) = 0}
    \end{equation*}
    We say $\alpha$ is \textbf{algebraic} over $K$ if $I_\alpha \ne 0$.  Otherwise $\alpha$ is \textbf{transcendental}.
    We say $L$ is algebraic over $K$ if $\alpha$ is algebraic over $K$ for all $\alpha \in L$.
\end{ndef}




\begin{ndef}[Minimal polynomial]\index{minimal polynomial}\hypertarget{def:minimalPoly}
    The non-zero ideal $I_\alpha$ (where $\alpha$ is algebraic over $K$) is principal since $K[t]$ is a principal ideal domain.
    In particular, we can say $I_\alpha = (f_\alpha(t))$ where $f_\alpha(t)$ can be assumed to be monic.
    Such a monic $f_\alpha(t)$ is the \textbf{minimal polynomial} of $\alpha$ over $K$.
\end{ndef}




\begin{ndef}[Simple extension]\index{field extension!simple}\hypertarget{def:genField}
    Suppose $K\leq L$ is a field extension and $\alpha \in L$.  $K(\alpha)$ is defined to be the smallest subfield of $L$ containing $K$ and $\alpha$.
    It's called the field \textbf{generated} by $K$ and $\alpha$.  We say that $L$ is a \textbf{simple extension} if $L = K(\beta)$ for some $\beta \in L$.

    Given $\alpha_1, \dotsc, \alpha_n \in L$, $K \leq L$.  $K(\alpha_1, \dotsc \alpha_n)$ is the smallest field containing $\alpha_1, \dotsc, \alpha_n$.
    It is the field generated by $K$ and $\alpha_1, \dotsc, \alpha_n$.

    On the other hand $K[\alpha]$ is the ring generated by $K$ and $\alpha$, in particular the image of $K[t]$ under the map $f(t) \mapsto f(\alpha)$.
\end{ndef}
% new lec (3)







\subsection{Digression on (Non-)Constructibility}





\begin{ndef}[Constructible]\index{constructible}\hypertarget{def:constructible}
    The points of intersection of any two distinct lines or circles drawn using these operations are \textbf{constructible in one step} from $P_0$.
    More generally, a point $\vec{r} \in \R^2$ is \textbf{constructible} from $P_0$ if there is a finite sequence $\vec{r_1}, \vec{r_2}, \dotsc, \vec{r_n} = \vec{r}$ such that $\vec{r_i}$ is constructible in one step from $P_0 \cup \{\vec{r_1}, \dotsc, \vec{r_{i-1}}\}$.
\end{ndef}











% new lec (4)




















% back to non-constructibility




\subsection{Return to theory development}


% new lec (5)

\begin{ndef}[Homomorphism over a field]\index{K-homo@$K$-homomorphism}\hypertarget{def:homo}
    Suppose $K \leq L$, $K \leq L'$ are field extensions.
    A $K$-\textbf{homomorphism} $\phi: L \to L'$ is a ring homomorphism such that $\phi\vert_K = \id$.

    A $K$-homomorphism is a $K$-isomorphism if it is a ring isomorphism.
\end{ndef}


% \begin{notation}
%     If $K \leq L$ is a \hyperlink{def:fieldExt}{field extension} and $f(t) \in K[t]$, we denote the set of roots of $f$ in $L$ by $\Root_f(L)$.
% \end{notation}

\begin{ndef}[Splitting]\index{splitting field}\hypertarget{def:splitting}
    Let $K \leq L$ be a field extension and $f(t) \in K[t]$. We say \textbf{$f$ splits over $L$} if
    \begin{equation*}
        f(t) = a (t-\alpha_1) (t - \alpha_2) \dotsm (t - \alpha_n)
    \end{equation*}
    where $a \in K$ and $\alpha_1, \dotsc, \alpha_n \in L$.

    We say $L$ is a \textbf{splitting field for $f$ over $K$} if $L = K(\alpha_1, \dotsc, \alpha_n)$.
\end{ndef}





% new lec (6)

% need to make sure this makes sense

\begin{ndef}[Normal extension]\label{def:1.25}\index{field extension!normal}\hypertarget{def:normal}
    A field extension $K \leq L$ is \textbf{normal} if for every $\alpha \in L$ the minimal polynomial $f_\alpha(t)$ of $\alpha$ over $K$ splits over $L$.
\end{ndef}





\clearpage
\section{Separable, normal and Galois extensions}
\begin{ndef}[Separable polynomial]\index{separable polynomial}\hypertarget{def:separablePoly}
    Let $K$ be a field and $f(t) \in K[t]$. Suppose $f(t)$ is irreducible in $K[t]$ and $L$ is a splitting field for $f(t)$ over $K$.
    Then $f(t)$ is \textbf{separable} over $K$ if $f(t)$ has no repeated roots in $L$.

    For general $f(t)$ we say $f(t)$ is separable over $K$ if every irreducible factor in $K[t]$ is separable over $K$.

    All constant polynomials are separable.
\end{ndef}
% new lec (7)

\begin{ndef}[Formal differentiation]\index{formal derivative}\hypertarget{def:diff}
    If $K$ is a field then \textbf{formal differentiation}
    \begin{align*}
        D: K[t] &\to K[t] \\
        t^n &\mapsto n t^{n-1}
    \end{align*}
    is a $K$-linear map.
    We denote this by $D(f(t)) = f'(t)$.
\end{ndef}




\begin{ndef}[Separable extension]\index{field extension!separable}\hypertarget{def:separableExt}
    We say \textbf{$\alpha \in L$ is separable over $K$} if its minimal polynomial is separable over $K$.

    \textbf{$L$ is separable over $K$} if all $\alpha \in L$ are separable over $K$.

    If $f_\alpha(t) = (t-\alpha)^n = t^n - \alpha^n$ where $n$ is a power of $p (=\chara K)$, we say that $\alpha$ is \textbf{purely inseparable over $K$}.
\end{ndef}




% new lec (8)














\begin{ndef}[Separably generated]\index{field extension!separably generated}\hypertarget{def:separableGen}
    We say $M = K(\alpha_1, \dotsc, \alpha_r)$ is \textbf{separably generated} by $\alpha_1, \dotsc, \alpha_r$ over $K$ if each $\alpha_i$ is separable over $K$.
\end{ndef}





% new lec (9)




\subsection{Trace and Norm}
\begin{ndef}[Trace and norm]\index{trace}\index{norm}\hypertarget{def:trNorm}
    Let $K \leq M$ be a finite field extension, and $\alpha \in M$.
    Multiplication by $\alpha$ gives a $K$-linear map $\theta_\alpha: M \to M$.

    Then we define
    \begin{itemize}[label={}]
        \item \textbf{Trace of $\alpha$ over $K$} is given by $\Tr_{M/K}(\alpha) =$ trace of $\theta_\alpha$  $\in K$.
        \item \textbf{Norm of $\alpha$ over $K$} is given by $N_{M/K}(\alpha) =$ determinant of $\theta_\alpha$  $\in K$.
    \end{itemize}

    Note these are dependent on the field extension.
\end{ndef}




% new lec (10)





\subsection{Normal extensions}






\begin{ndef}[Automorphism group]\index{automorphism group}\hypertarget{def:autGroup}
    Let $K \leq M$ be a finite field extension.
    Its \textbf{$K$-automorphism} group is $\Aut_K(M) = \set{\phi | \phi \text{ a $K$-homomorphism } M \to M}$.
\end{ndef}





\begin{ndef}[Galois extension]\index{Galois extension}\hypertarget{def:galoisExt}
    A finite field extension that is normal and separable is a \textbf{Galois extension}.
\end{ndef}
% new lec (11)

\begin{ndef}[Galois group]\index{Galois group}\hypertarget{def:galoisGroup}
    Let $K \leq M$ be a Galois extension.
    Then, the $K$-automorphism group of $M$ is the \textbf{Galois group} of $M$ over $K$.
    Write this as $\Gal(M / K)$.
\end{ndef}

% check this makes sense



\clearpage
\section{Fundamental Theorem of Galois Theory}
\subsection{Artin's Theorem}
\begin{ndef}[Fixed field]\index{fixed field}\hypertarget{def:fixedField}
    Let $K \leq L$ be a field extension and $H \leq \Aut_K(L)$.
    The \textbf{fixed field} of $H$ is,
    \begin{equation*}
        L^H \coloneqq \set{\alpha \in L | \sigma(\alpha) = \alpha \text{ for all } \sigma \in H}
    \end{equation*}
\end{ndef}




% new lec (12)








\subsection{Galois groups of polynomials}
\begin{ndef}[Galois group of polynomial]\index{Galois group!of polynomial}\hypertarget{def:ggPoly}
    Let $f(t)$ be a separable polynomial $\in K[t]$ and let $K \leq L$ with $L$ a splitting field for $f(t)$.
    Then the \textbf{Galois group of $f(t)$} over $K$ is
    \begin{equation*}
        \Gal(f) \coloneqq \Gal(L/K).
    \end{equation*}
\end{ndef}


























\begin{ndef}[Discriminant]\index{discriminant}\hypertarget{def:disc}
    Let $f(t) \in K[t]$ with distinct roots $\alpha_1, \dotsc, \alpha_n$ in a splitting field (with $f(t)$ not necessarily irreducible).
    Let
    \begin{equation*}
        \Delta = \prod_{i < j} (\alpha_i- \alpha_j).
    \end{equation*}
    Then the \textbf{discriminant} $D = D(f)$ of $f$ is
    \begin{align*}
        D = \Delta^2 &= \prod_{i < j} (\alpha_i - \alpha_j)^2 \\
                          &= (-1)^\frac{n(n-1)}{2} \prod_{i \neq j} (\alpha_i - \alpha_j).
    \end{align*}
\end{ndef}














\subsection{Galois Theory of Finite Fields}




















\begin{ndef}[Frobenius automorphism]\index{frobenius automorphism}\hypertarget{def:frob}
    Let $\F$ be a finite field of characteristic $p$.
    Then the \textbf{Frobenius automorphism} of $\F$ is
    \begin{align*}
        \phi: \F &\longrightarrow \F \\
        \alpha &\longmapsto \alpha^p.
    \end{align*}
\end{ndef}








\clearpage
\section{Cyclotomic and Kummer extensions}
\subsection{Cyclotomic extensions}
\begin{ndef}[Cyclotomic extension]\index{field extension!cyclotomic}\hypertarget{def:cycloExt}
    Suppose $\chara K = 0$ or $p$ prime where $p \nmid m$.
    The $m$th \textbf{cyclotomic extension} of $K$ is the splitting field $L$ of $t^m - 1$.
\end{ndef}

\begin{ndef}[Primitive $m$th root of unity]\index{primitive root}\hypertarget{def:primRoot}
    An element $\xi \in \mu_m$ is a \textbf{primitive $m$th root of unity} if $\mu_m = \langle \xi \rangle$.
\end{ndef}

























% can i change this to a lemma
\begin{ndef}[Group of roots of unity]\label{def:4.3}
    \begin{equation*}
        \theta: G \longrightarrow (\Z/m\Z)^\times
    \end{equation*}
    This is a group homomorphism:
    If $\sigma(\xi) = \xi^c, \phi(\xi) = \xi^j$ then $(\sigma\phi)(\xi) = \sigma(\xi^j) = \xi^{ij}$.
    Hence $G$ is abelian.

    Thus we regard $G$ as a subgroup of $(\Z/m\Z)^\times$.
\end{ndef}
\begin{ndef}[Cyclotomic polynomial]\index{cyclotomic polynomial}\hypertarget{def:cycloPoly}
    The $m$th \textbf{cyclotomic polynomial} is
    \begin{equation*}
        \Phi_m(t) = \prod _{i \in (\Z/m\Z)^\times} (t - \xi^i),
    \end{equation*}
    the product of the linear factors of $t^m-1$ corresponding to the primitive $m$th roots of unity.
\end{ndef}














\begin{ndef}[Cyclic, abelian extension]\index{field extension!cyclic}\hypertarget{def:cyclic}
    An extension $K \leq L$ is \textbf{cyclic} if the extension is Galois and $\Gal(L/K)$ is cyclic.
    Similarly, it is called \textbf{abelian} if $\Gal(L/K)$ is abelian.
\end{ndef}

\subsection{Kummer Theory}










% \begin{eg}
%     Take $f(t) = t^5 - 2$ over $\mathbb{Q}$, irreducible by \nameref{thm:1.16}, and $L$ the splitting field of $f(t)$ over $\mathbb{Q}$.
%     Let $\xi$ be a \hyperlink{def:primRoot}{primitive fifth root of unity}.
% \end{eg}



\begin{ndef}[Kummer extension]\index{field extension!Kummer}\hypertarget{def:kummerExt}
    A cyclic extension $K \leq L$ with $\abs{L:K} = m$, where $\chara K \nmid m$ and $K$ contains a primitive $m$th root of unity is a \textbf{Kummer extension}.
\end{ndef}




\begin{ndef}[Extension by radicals]\index{field extension!by radicals}\hypertarget{def:radicals}
    A field extension $K \leq L$ is an \textbf{extension by radicals} if $\exists K = L_0 \leq L_1 \leq \dotsb \leq L_n = L$ such that each $L_i \leq L_{i+1}$ is either cyclotomic or Kummer extension.
    A polynomial $f(t) \in K[t]$ is \textbf{soluble by radicals} if its splitting field lies in an extension by radicals.
\end{ndef}
% lecture 18
\subsection{Cubics}

































{
}





























\subsection{Quartics}







































% Lecture 19
























\subsection{Solubility by radicals}



















\begin{ndef}[Soluble group]\index{soluble}\hypertarget{def:soluble}
    A group is \textbf{soluble} if there is a chain of subgroups
    \begin{equation*}
        \{e\} = G_m \lhd G_{m-1} \lhd \dotsb \lhd G_1 \lhd G_0 = G
    \end{equation*}
    with $G_i / G_{i+1}$ abelian.
\end{ndef}



\begin{ndef}[Derived subgroup]\index{derived subgroup}\hypertarget{def:derivedSub}
    The \textbf{derived subgroup} $G'$ of a group $G$ is the subgroup generated by all the commutators $g_1 g_2 g_1^{-1} g_2^{-1}$ for $g_1, g_2 \in G$.
\end{ndef}



\begin{ndef}[Derived series]\index{derived series}\hypertarget{def:derivedSer}
    The \textbf{derived series} $\{G^{(m)}\}$ of $G$ is defined inductively:
    \begin{align*}
        G^{(0)} &= G \\
        G^{(1)} &= G' \\
        G^{(2)} &= (G')' \\
        G^{(j+1)} &= (G^{(j)})' \\
    \end{align*}
    Thus $G = G^{(0)} \rhd G^{(1)} \rhd G^{(2)} \rhd \dotsb$ with $G^{(j)}/G^{(j+1)}$ abelian.
\end{ndef}














\clearpage
\section{Final Thoughts}
\subsection{Algebraic closure}
\begin{ndef}[Algebraically closed]\index{algebraically closed}\hypertarget{def:closed}
    A field $L$ is \textbf{algebraically closed} if any $f(t) \in L[t]$ splits into a product of linear factors in $L[t]$.
\end{ndef}

\begin{ndef}[Algebraic closure]\index{algebraic closure}\hypertarget{def:closure}
    An extension $K \leq L$ is an algebraic closure of $K$ if $K \leq L$ is algebraic and $L$ is algebraically closed.
\end{ndef}



\begin{ndef}[Partial order]\index{poset}\hypertarget{def:poset}
    $(\mathcal{S},\leq)$ is a \textbf{partial order} on $\mathcal{S}$ if
    \begin{enumerate}[label=(\roman*)]
        \item $\forall x \in \mathcal{S}$ $x \leq x$
        \item $x \leq y$ and $y \leq z \implies x \leq z$.
        \item if $x\leq y$ and $y \leq x$ then $x=y$.
    \end{enumerate}
    $\mathcal{S}$ is \textbf{totally ordered} if for any $x,y \in S$ either $x \leq y$ or $y \leq x$.
    A \textbf{chain} is a partially ordered set $(\mathcal{S},\leq)$ that is a totally ordered subset.
\end{ndef}









\subsection{Symmetric polynomials and invariant theory}

















\begin{ndef}[Elementary symmetric polynomials]\index{elementary symmetric polynomials}\hypertarget{def:esp}
    These $s_i$ are the elementary symmetric polynomials.
\end{ndef}

\begin{ndef}\hypertarget{def:indep}
    $\alpha_1, \dotsc, \alpha_n$ are \textbf{algebraically independent} over $K$ if the ring homomorphism $K[Y_1, \dotsc, Y_n] \to K[\alpha_1, \dotsc, \alpha_n] \leq L$ is an isomorphism where $K[Y_1, \dotsc, Y_n]$ is the polynomial ring in $Y_1, \dotsc, Y_n$.
\end{ndef}

\end{document}