\documentclass{letter}
\usepackage[a4paper,left=2.5cm, right=2.5cm, top=1.5cm, bottom=2.5cm]{geometry}
\begin{document}

\begin{letter}{}
\opening{To the Teaching Committee and Examiners,}

As Part II mathematicians who took either entirely or predominantly pure courses, we are writing to express our serious concerns regarding what our year group has seen as a major disparity in difficulty between the pure courses we studied and the applied/applicable courses taken by many of our peers.

Most of us feared the worst within minutes of opening Paper 1.
After finishing the first examination we found that virtually all pure mathematicians had massively struggled, whilst we learned that others' courses had questions set to a level of difficulty in line with what they were expecting.
The experience was much the same for the remaining examinations.

The Teaching Committee's examination reports from the previous two years expressed concern that there had been a discrepancy in the other direction, with applied students being at a disadvantage in those years.
If the intention was to level the playing field in 2018 then unfortunately these efforts have been seriously overdone - even the best prepared amongst us, i.e.\ those who revised a decade of the most recent Tripos papers, found that nothing could have prepared us for the overall nature of questions this year.

As third year mathematicians we of course expect to be thoroughly challenged in Tripos and to take on unseen problems, but we were dismayed by the dearth of the usual testing of bookwork or ``typical computations'' in problems, which are always key parts of personal revision and revision supervisions.
This abnormality in comparison to the previous decade of Part II papers held for the majority of courses ranging all the way from analysis to abstract algebra, and so pure mathematicians found that a level of preparation that would have set them up very well in any other year was suddenly inadequate for the 2018 papers.
Another common theme was that the minority of questions that felt in some sense familiar were generally far longer than comparable past questions, so that working through them was a huge opportunity cost in a timed examination.
It is not an exaggeration to say that a number of questions appeared to be almost twice as long as typical past questions on comparable topics.

Ultimately the extent of this skew against the pure courses will be apparent when the scripts are marked, because individually the pure mathematicians will rank significantly lower than we did in parts IA and IB, and for many of us this will change our grades and impact our futures.
Many of our exam strategies were derailed: to illustrate, a candidate aiming for a strong performance may sensibly choose to focus largely on (pure) D courses, but then unfortunately find in 2018 that their time in exams would have been better spent scrapping for fewer marks from C courses.
This is compounded by the fact that falling on the wrong side of the M1/M2 classification criteria threshold can cause a focus on Section II questions (which ought to be a good strategy for many students) to severely backfire.
We therefore believe it is critical that the examiners take serious meaningful action to somehow ensure this years' candidates are not disadvantaged by the sort of mathematics they study.
After three years of study we hope everyone's final grades will represent our capability as undergraduate mathematicians, rather than the fluctuations in difficulty between courses.

If the examiners feel an early discussion with students would be helpful then we would appreciate the opportunity to discuss these concerns in more depth.
Thank you for taking the time to read this letter.

\closing{Signed,}

\end{letter}
\end{document}
