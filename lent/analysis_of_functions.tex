\documentclass{article}

\def\npart {II}
\def\nyear {2017}
\def\nterm {Lent}
\def\nlecturer{Prof C.\ Mouhot}
\def\ncourse{Analysis of Functions}
\ifx \nauthor\undefined
  \def\nauthor{Bhavik Mehta}
\else
\fi

\author{Based on lectures by \nlecturer \\\small Notes taken by \nauthor}
\date{\nterm\ \nyear}
\title{Part \npart\ -- \ncourse}

\usepackage[utf8]{inputenc}
\usepackage{amsmath}
\usepackage{amsthm}
\usepackage{amssymb}
\usepackage{enumerate}
\usepackage{mathtools}
\usepackage{graphicx}
\usepackage[dvipsnames]{xcolor}
\usepackage{tikz}
\usepackage{wrapfig}
\usepackage{centernot}
\usepackage{float}
\usepackage{braket}
\usepackage[hypcap=true]{caption}
\usepackage{enumitem}
\usepackage[colorlinks=true, linkcolor=mblue]{hyperref}
\usepackage[nameinlink,noabbrev]{cleveref}
\usepackage{nameref}
\usepackage[margin=1.5in]{geometry}

% Theorems
\theoremstyle{definition}
\newtheorem*{aim}{Aim}
\newtheorem*{axiom}{Axiom}
\newtheorem*{claim}{Claim}
\newtheorem*{cor}{Corollary}
\newtheorem*{conjecture}{Conjecture}
\newtheorem*{defi}{Definition}
\newtheorem*{eg}{Example}
\newtheorem*{ex}{Exercise}
\newtheorem*{fact}{Fact}
\newtheorem*{law}{Law}
\newtheorem*{lemma}{Lemma}
\newtheorem*{notation}{Notation}
\newtheorem*{prop}{Proposition}
\newtheorem*{question}{Question}
\newtheorem*{rrule}{Rule}
\newtheorem*{thm}{Theorem}
\newtheorem*{assumption}{Assumption}

\newtheorem*{remark}{Remark}
\newtheorem*{warning}{Warning}
\newtheorem*{exercise}{Exercise}

% \newcommand{\nthmautorefname}{Theorem}

\newtheorem{nthm}{Theorem}[section]
\newtheorem{nlemma}[nthm]{Lemma}
\newtheorem{nprop}[nthm]{Proposition}
\newtheorem{ncor}[nthm]{Corollary}
\newtheorem{ndef}[nthm]{Definition}

% Special sets
\newcommand{\C}{\mathbb{C}}
\newcommand{\N}{\mathbb{N}}
\newcommand{\Q}{\mathbb{Q}}
\newcommand{\R}{\mathbb{R}}
\newcommand{\Z}{\mathbb{Z}}

\newcommand{\abs}[1]{\left\lvert #1\right\rvert}
\newcommand{\norm}[1]{\left\lVert #1\right\rVert}
\renewcommand{\vec}[1]{\boldsymbol{\mathbf{#1}}}

\let\Im\relax
\let\Re\relax

\DeclareMathOperator{\Im}{Im}
\DeclareMathOperator{\Re}{Re}
\DeclareMathOperator{\id}{id}

\definecolor{mblue}{rgb}{0., 0.05, 0.6}


% preamble
\usepackage{tikz}
\usepackage{mathrsfs}
% \newcommand{\powerset}{\raisebox{.15\baselineskip}{\Large\ensuremath{\wp}}}
\newcommand{\powerset}{\mathscr{P}}
%\setcounter{section}{-1}
% and here we go!

\begin{document}
\maketitle
\tableofcontents

% Chapter 1: Integration of functions
% Chapter 2: Vector spaces of functions
% Chapter 3: Fourier decomposition of functions
% Chapter 4: Generalised derivatives of functions (and spaces using them)

% Chapter 1:
    % I. Lebesgue measure/integration
    % II. Integrability and convergence

% I)
\section{Lebesgue theory}
\begin{ex}
    Show pointwise limit of Riemann-integrable functions is not necessarily Riemann-integrable.
\end{ex}
(Hint: Dirichlet function).

% 1)
\subsection{Recap of measure theory}
Consider a set $X$ and $\powerset(X)$ subsets of $X$.
\begin{defi}[Algebra]\hypertarget{def:algebra}
    $\mathscr{A} \subset \powerset(X)$ is an \textbf{algebra} if it is
    \begin{enumerate}[label=(\roman*)]
        \item stable under finite union
        \item stable under absolute difference
        \item $X \in \mathscr{A}$.
    \end{enumerate}
\end{defi}

\begin{defi}[$\sigma$-algebra]\hypertarget{def:sigAlg}
    $\mathscr{A} \subset \powerset(X)$ is a $\sigma$-\textbf{algebra} if it is
    \begin{enumerate}[label=(\roman*)]
        \item stable under countable union
        \item stable under absolute difference
        \item $X \in \mathscr{A}$.
    \end{enumerate}
\end{defi}

\begin{remark}
    Topologies $\mathscr{T} \subset \powerset(X)$ are (i) stable under \textit{any} union, (ii) finite intersection, (ii) include $X$ and $\emptyset$.
\end{remark}

\begin{remark}
    The property of being a \hyperlink{def:sigAlg}{$\sigma$-algebra} is stable under intersection.\hypertarget{def:borelSet}
    The notion of smaller $\sigma$-algebra containing some topology $\mathscr{T}$ are called \textbf{Borel} sets, written $\mathcal{B}(X)$.
\end{remark}

\begin{defi}\hypertarget{def:measure}
    Consider $(X, \mathscr{A})$, where $\mathscr{A}$ is a \hyperlink{def:sigAlg}{$\sigma$-algebra}.
    A \textbf{measure} $\mu$ is a function $\mathscr{A} \to [0, +\infty]$ such that $\mu(\emptyset) = 0$.
    It is \textbf{$\sigma$-additive} \hypertarget{def:sigAdd} if
    \begin{equation*}
        \mu\left( \bigcup_{n=1}^\infty A_n\right) = \mu\left( \bigsqcup_{n=1}^\infty A_n\right) = \sum_{n=1}^\infty \mu(A_n).
    \end{equation*}

    Then $X, \mathscr{A}, \mu)$ is a \textbf{measure space}.
    It is called \textbf{complete} \hypertarget{def:completeMeasure} if $A \in \mathscr{A}$ with $B \subset A$ and $\mu(A) = 0$, then $B \in \mathscr{A}$ and $\mu(B)=0$.
\end{defi}

\begin{ex}
    Show \hyperlink{def:sigAdd}{$\sigma$-additivity} is implied by either of the following properties:
    \begin{itemize}
        \item finite additivity and continuity from below
        \item finite $\mu(X) < +\infty$ and finite additivity and continuity from above at $\emptyset$
    \end{itemize}
    where
    \begin{itemize}
        \item continuity from below:
            \begin{equation*}
                A_n \in \mathscr{A}, \mu\left(\bigcup_{k=1}^\infty A_k\right)  = \lim_{n \to +\infty} \mu\left(\bigcup_{k=1}^n A_k\right)
            \end{equation*}
        \item continuity from above
            \begin{equation*}
                A_n \in \mathscr{A}, \mu(A_1) < +\infty, \mu\left(\bigcap_{k=1}^\infty A_k\right)  = \lim_{\mathclap{n \to +\infty}} \mu\left(\bigcap_{k=1}^n A_k\right)
            \end{equation*}
    \end{itemize}
\end{ex}

\begin{ex}
    Find the cardinality of $\mathscr{T}(\R)$, $\mathcal{B}(\R)$, $\mathscr{L}(\R)$ where $\mathscr{L}(\R)$ are the Lebesgue sets, defined by adding all subsets of null sets to $\mathcal{B}(\R)$.
\end{ex}

\begin{thm}
    There is a unique \hyperlink{def:measure}{measure} on $(\R^n, \mathcal{B}(\R^n))$ such that
    \begin{equation*}
        \mu\left(\prod_{i=1}^n [a_i, b_i]\right) = \prod_{i=1}^n (b_i - a_i) \quad a_i \leq b_i \in \R
    \end{equation*}
    called the Lebesgue measure\hypertarget{def:lebMeas}.
\end{thm}
\begin{proof}
    See Probability and Measure.
\end{proof}

\begin{remark}
    \hyperlink{def:lebMeas}{Lebesgue measure} is \hypertarget{def:sigFinite}{$\sigma$-finite}: $\exists$ a countable increasing sequence of sets with finite measure covering $\R^n$.
\end{remark}

\begin{defi}[Measurable function]\hypertarget{def:measFunc}
    Take $(X, \mathscr{A})$, $(Y, \mathscr{B})$ two spaces with \hyperlink{def:sigAlg}{$\sigma$-algebras}.
    A function $f: X \to Y$ is said to be \textbf{measurable} if $\forall B \in \mathscr{B}$, $f^{-1}(B) \in \mathscr{A}$.
\end{defi}

\begin{prop}
    Take $(X, \mathscr{A}), (Y, \mathscr{B})$ two spaces with \hyperlink{def:sigAlg}{$\sigma$-algebras} where $Y$ is a metric space and $\mathscr{B}$ is the collection of \hyperlink{def:borelSet}{Borel sets}.
    Let $f_k: X \to Y$ be a sequence of \hyperlink{def:measFunc}{measurable functions} which converge pointwise to $f:X \to Y$.
    Then $f$ is measurable.
\end{prop}

\begin{proof}
    Since $B$ is formed from open sets through countable union/intersection and difference, it is enough to prove $\forall U \in \mathscr{T}(Y)$, $f^{-1}(U) \in \mathscr{A}$. (Exercise: Check.)

    Let
    \begin{align*}
        U_n &= \Set{y \in Y | d(y, Y \setminus U) > \frac{1}{n}} \\
        F_n &= \Set{y \in Y | d(y, Y \setminus U) \geq \frac{1}{n}}
    \end{align*}
    so that
    \begin{equation*}
        U_n \subset F_n \subset U_{n+1} \subset \dotsb \subset U
    \end{equation*}
    and $F_n$ are closed.

    We can see $U = \bigcup_{n \geq 1} U_n = \bigcup_{n \geq 1} F_n$ ($U$ open). % why U open matters?
    Hence,
    \begin{equation*}
        f^{-1}(U) = f^{-1} \left(\bigcup_{n \geq 1} U_n\right) = \bigcup_{n \geq 1} f^{-1}(U_n) \subset \bigcup_{n \geq 1} \bigcup_{l \geq 1} \bigcap_{k \geq l} f^{-1}_k(U_n).
    \end{equation*}
    We used the fact that
    \begin{equation*}f^{-1}(U_n) \subset \bigcup_{l \geq 1} \bigcap_{k \geq l} f_k^{-1}(U_n)\end{equation*}
    To show this, take $x \in f^{-1}(U_n)$, so $f(x) = y \in U_n$.
    We know
    \begin{equation*}
        f_k(x) \xrightarrow{k \to \infty} f(x).
    \end{equation*}
    Since $U_n$ open, $\exists l_x \geq 1$ such that $\forall k \geq l_x, f_k(x) \in U_n$ giving $x \in \bigcap_{k \geq l} f_k^{-1}(U_n)$.

    Continuing,
    \begin{align*}
        f^{-1}(U) &\subset \bigcup_{n \geq 1} \bigcup_{l \geq 1} \bigcap_{k \geq l} f_k^{-1}(U_n). \\
        &\subset \bigcup_{n \geq 1} \bigcup_{l \geq 1} \bigcap_{k \geq l} f_k^{-1}(F_n).
    \end{align*}
    $F_n$ closed, so
    \begin{equation*}
        \bigcup_{l \geq 1} \bigcap_{k \geq l} f_k^{-1} (F_n) \subset f^{-1}(F_n).
    \end{equation*}
    In particular, if $x \in$ LHS, $\exists l \geq 1$ such that $\forall k \geq l$, $f_k(x) \in F_n$.
    Pass to the limit, and $f_n$ closed gives $f(x) \in F_n$, $x \in f^{-1}(F_n)$.

    In conclusion,
    \begin{align*}
        f^{-1}(U) &\subset \bigcup_{n \geq 1} \bigcup_{l \geq 1} \bigcap_{k \geq l} f_k^{-1}(U_n) \subset \bigcup_{n \geq 1} \bigcup_{l \geq 1} \bigcap_{k \geq l} f_k^{-1}(F_n) \\
                  & \subset \bigcup_{n \geq 1} f^{-1}(F_n) = f^{-1}\left(\bigcup_{n \geq 1} F_n\right) = f^{-1}(U).
    \end{align*}
    So, all inclusions are equality: $f^{-1}(U)$ is formed of countable intersections and unions of preimages of sets in $\mathscr{B}$, hence $f^{-1}(U) \in \mathscr{A}$.
\end{proof}
\end{document}
