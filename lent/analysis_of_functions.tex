\documentclass{article}

\def\npart {II}
\def\nyear {2017}
\def\nterm {Lent}
\def\nlecturer{Prof C.\ Mouhot}
\def\ncourse{Analysis of Functions}
\ifx \nauthor\undefined
  \def\nauthor{Bhavik Mehta}
\else
\fi

\author{Based on lectures by \nlecturer \\\small Notes taken by \nauthor}
\date{\nterm\ \nyear}
\title{Part \npart\ -- \ncourse}

\usepackage[utf8]{inputenc}
\usepackage{amsmath}
\usepackage{amsthm}
\usepackage{amssymb}
\usepackage{enumerate}
\usepackage{mathtools}
\usepackage{graphicx}
\usepackage[dvipsnames]{xcolor}
\usepackage{tikz}
\usepackage{wrapfig}
\usepackage{centernot}
\usepackage{float}
\usepackage{braket}
\usepackage[hypcap=true]{caption}
\usepackage{enumitem}
\usepackage[colorlinks=true, linkcolor=mblue]{hyperref}
\usepackage[nameinlink,noabbrev]{cleveref}
\usepackage{nameref}
\usepackage[margin=1.5in]{geometry}

% Theorems
\theoremstyle{definition}
\newtheorem*{aim}{Aim}
\newtheorem*{axiom}{Axiom}
\newtheorem*{claim}{Claim}
\newtheorem*{cor}{Corollary}
\newtheorem*{conjecture}{Conjecture}
\newtheorem*{defi}{Definition}
\newtheorem*{eg}{Example}
\newtheorem*{ex}{Exercise}
\newtheorem*{fact}{Fact}
\newtheorem*{law}{Law}
\newtheorem*{lemma}{Lemma}
\newtheorem*{notation}{Notation}
\newtheorem*{prop}{Proposition}
\newtheorem*{question}{Question}
\newtheorem*{rrule}{Rule}
\newtheorem*{thm}{Theorem}
\newtheorem*{assumption}{Assumption}

\newtheorem*{remark}{Remark}
\newtheorem*{warning}{Warning}
\newtheorem*{exercise}{Exercise}

% \newcommand{\nthmautorefname}{Theorem}

\newtheorem{nthm}{Theorem}[section]
\newtheorem{nlemma}[nthm]{Lemma}
\newtheorem{nprop}[nthm]{Proposition}
\newtheorem{ncor}[nthm]{Corollary}
\newtheorem{ndef}[nthm]{Definition}

% Special sets
\newcommand{\C}{\mathbb{C}}
\newcommand{\N}{\mathbb{N}}
\newcommand{\Q}{\mathbb{Q}}
\newcommand{\R}{\mathbb{R}}
\newcommand{\Z}{\mathbb{Z}}

\newcommand{\abs}[1]{\left\lvert #1\right\rvert}
\newcommand{\norm}[1]{\left\lVert #1\right\rVert}
\renewcommand{\vec}[1]{\boldsymbol{\mathbf{#1}}}

\let\Im\relax
\let\Re\relax

\DeclareMathOperator{\Im}{Im}
\DeclareMathOperator{\Re}{Re}
\DeclareMathOperator{\id}{id}

\definecolor{mblue}{rgb}{0., 0.05, 0.6}


% preamble
\usepackage{tikz}
\usepackage{mathrsfs}
% \newcommand{\powerset}{\raisebox{.15\baselineskip}{\Large\ensuremath{\wp}}}
\newcommand{\powerset}{\mathscr{P}}
%\setcounter{section}{-1}
% and here we go!

\begin{document}
\maketitle
\tableofcontents

% Chapter 1: Integration of functions
% Chapter 2: Vector spaces of functions
% Chapter 3: Fourier decomposition of functions
% Chapter 4: Generalised derivatives of functions (and spaces using them)

% Chapter 1:
    % I. Lebesgue measure/integration
    % II. Integrability and convergence

% I)
\section{Lebesgue theory}
\begin{ex}
    Show pointwise limit of Riemann-integrable functions is not necessarily Riemann-integrable.
\end{ex}
(Hint: Dirichlet function).

% 1)
\subsection{Recap of measure theory}
Consider a set $X$ and $\powerset(X)$ subsets of $X$.
\begin{defi}[Algebra]\hypertarget{def:algebra}
    $\mathscr{A} \subset \powerset(X)$ is an \textbf{algebra} if it is
    \begin{enumerate}[label=(\roman*)]
        \item stable under finite union
        \item stable under absolute difference
        \item $X \in \mathscr{A}$.
    \end{enumerate}
\end{defi}

\begin{defi}[$\sigma$-algebra]\hypertarget{def:sigAlg}
    $\mathscr{A} \subset \powerset(X)$ is a $\sigma$-\textbf{algebra} if it is
    \begin{enumerate}[label=(\roman*)]
        \item stable under countable union
        \item stable under absolute difference
        \item $X \in \mathscr{A}$.
    \end{enumerate}
\end{defi}

\begin{remark}
    Topologies $\mathscr{T} \subset \powerset(X)$ are (i) stable under \textit{any} union, (ii) finite intersection, (ii) include $X$ and $\emptyset$.
\end{remark}

\begin{remark}
    The property of being a \hyperlink{def:sigAlg}{$\sigma$-algebra} is stable under intersection.\hypertarget{def:borelSet}
    The notion of smaller $\sigma$-algebra containing some topology $\mathscr{T}$ are called \textbf{Borel} sets, written $\mathcal{B}(X)$.
\end{remark}

\begin{defi}\hypertarget{def:measure}
    Consider $(X, \mathscr{A})$, where $\mathscr{A}$ is a \hyperlink{def:sigAlg}{$\sigma$-algebra}.
    A \textbf{measure} $\mu$ is a function $\mathscr{A} \to [0, +\infty]$ such that $\mu(\emptyset) = 0$.
    It is \textbf{$\sigma$-additive} \hypertarget{def:sigAdd} if
    \begin{equation*}
        \mu\left( \bigcup_{n=1}^\infty A_n\right) = \mu\left( \bigsqcup_{n=1}^\infty A_n\right) = \sum_{n=1}^\infty \mu(A_n).
    \end{equation*}

    Then $X, \mathscr{A}, \mu)$ is a \textbf{measure space}.
    It is called \textbf{complete} \hypertarget{def:completeMeasure} if $A \in \mathscr{A}$ with $B \subset A$ and $\mu(A) = 0$, then $B \in \mathscr{A}$ and $\mu(B)=0$.
\end{defi}

\begin{ex}
    Show \hyperlink{def:sigAdd}{$\sigma$-additivity} is implied by either of the following properties:
    \begin{itemize}
        \item finite additivity and continuity from below
        \item finite $\mu(X) < +\infty$ and finite additivity and continuity from above at $\emptyset$
    \end{itemize}
    where
    \begin{itemize}
        \item continuity from below:
            \begin{equation*}
                A_n \in \mathscr{A}, \mu\left(\bigcup_{k=1}^\infty A_k\right)  = \lim_{n \to +\infty} \mu\left(\bigcup_{k=1}^n A_k\right)
            \end{equation*}
        \item continuity from above
            \begin{equation*}
                A_n \in \mathscr{A}, \mu(A_1) < +\infty, \mu\left(\bigcap_{k=1}^\infty A_k\right)  = \lim_{\mathclap{n \to +\infty}} \mu\left(\bigcap_{k=1}^n A_k\right)
            \end{equation*}
    \end{itemize}
\end{ex}

\begin{ex}
    Find the cardinality of $\mathscr{T}(\R)$, $\mathcal{B}(\R)$, $\mathscr{L}(\R)$ where $\mathscr{L}(\R)$ are the Lebesgue sets, defined by adding all subsets of null sets to $\mathcal{B}(\R)$.
\end{ex}

\begin{thm}
    There is a unique \hyperlink{def:measure}{measure} on $(\R^n, \mathcal{B}(\R^n))$ such that
    \begin{equation*}
        \mu\left(\prod_{i=1}^n [a_i, b_i]\right) = \prod_{i=1}^n (b_i - a_i) \quad a_i \leq b_i \in \R
    \end{equation*}
    called the Lebesgue measure\hypertarget{def:lebMeas}.
\end{thm}
\begin{proof}
    See Probability and Measure.
\end{proof}

\begin{remark}
    \hyperlink{def:lebMeas}{Lebesgue measure} is \hypertarget{def:sigFinite}{$\sigma$-finite}: $\exists$ a countable increasing sequence of sets with finite measure covering $\R^n$.
\end{remark}

\begin{defi}[Measurable function]\hypertarget{def:measFunc}
    Take $(X, \mathscr{A})$, $(Y, \mathscr{B})$ two spaces with \hyperlink{def:sigAlg}{$\sigma$-algebras}.
    A function $f: X \to Y$ is said to be \textbf{measurable} if $\forall B \in \mathscr{B}$, $f^{-1}(B) \in \mathscr{A}$.
\end{defi}

\begin{prop}
    Take $(X, \mathscr{A}), (Y, \mathscr{B})$ two spaces with \hyperlink{def:sigAlg}{$\sigma$-algebras} where $Y$ is a metric space and $\mathscr{B}$ is the collection of \hyperlink{def:borelSet}{Borel sets}.
    Let $f_k: X \to Y$ be a sequence of \hyperlink{def:measFunc}{measurable functions} which converge pointwise to $f:X \to Y$.
    Then $f$ is measurable.
\end{prop}

\begin{proof}
    Since $B$ is formed from open sets through countable union/intersection and difference, it is enough to prove $\forall U \in \mathscr{T}(Y)$, $f^{-1}(U) \in \mathscr{A}$. (Exercise: Check.)

    Let
    \begin{align*}
        U_n &= \Set{y \in Y | d(y, Y \setminus U) > \frac{1}{n}} \\
        F_n &= \Set{y \in Y | d(y, Y \setminus U) \geq \frac{1}{n}}
    \end{align*}
    so that
    \begin{equation*}
        U_n \subset F_n \subset U_{n+1} \subset \dotsb \subset U
    \end{equation*}
    and $F_n$ are closed.

    We can see $U = \bigcup_{n \geq 1} U_n = \bigcup_{n \geq 1} F_n$ ($U$ open). % why U open matters?
    Hence,
    \begin{equation*}
        f^{-1}(U) = f^{-1} \left(\bigcup_{n \geq 1} U_n\right) = \bigcup_{n \geq 1} f^{-1}(U_n) \subset \bigcup_{n \geq 1} \bigcup_{l \geq 1} \bigcap_{k \geq l} f^{-1}_k(U_n).
    \end{equation*}
    We used the fact that
    \begin{equation*}f^{-1}(U_n) \subset \bigcup_{l \geq 1} \bigcap_{k \geq l} f_k^{-1}(U_n)\end{equation*}
    To show this, take $x \in f^{-1}(U_n)$, so $f(x) = y \in U_n$.
    We know
    \begin{equation*}
        f_k(x) \xrightarrow{k \to \infty} f(x).
    \end{equation*}
    Since $U_n$ open, $\exists l_x \geq 1$ such that $\forall k \geq l_x, f_k(x) \in U_n$ giving $x \in \bigcap_{k \geq l} f_k^{-1}(U_n)$.

    Continuing,
    \begin{align*}
        f^{-1}(U) &\subset \bigcup_{n \geq 1} \bigcup_{l \geq 1} \bigcap_{k \geq l} f_k^{-1}(U_n). \\
        &\subset \bigcup_{n \geq 1} \bigcup_{l \geq 1} \bigcap_{k \geq l} f_k^{-1}(F_n).
    \end{align*}
    $F_n$ closed, so
    \begin{equation*}
        \bigcup_{l \geq 1} \bigcap_{k \geq l} f_k^{-1} (F_n) \subset f^{-1}(F_n).
    \end{equation*}
    In particular, if $x \in$ LHS, $\exists l \geq 1$ such that $\forall k \geq l$, $f_k(x) \in F_n$.
    Pass to the limit, and $f_n$ closed gives $f(x) \in F_n$, $x \in f^{-1}(F_n)$.

    In conclusion,
    \begin{align*}
        f^{-1}(U) &\subset \bigcup_{n \geq 1} \bigcup_{l \geq 1} \bigcap_{k \geq l} f_k^{-1}(U_n) \subset \bigcup_{n \geq 1} \bigcup_{l \geq 1} \bigcap_{k \geq l} f_k^{-1}(F_n) \\
                  & \subset \bigcup_{n \geq 1} f^{-1}(F_n) = f^{-1}\left(\bigcup_{n \geq 1} F_n\right) = f^{-1}(U).
    \end{align*}
    So, all inclusions are equality: $f^{-1}(U)$ is formed of countable intersections and unions of preimages of sets in $\mathscr{B}$, hence $f^{-1}(U) \in \mathscr{A}$.
\end{proof}

\subsection{Lebesgue integration}
The important result from measures is the existence of Lebesgue measure, and that the `theory' is closed for pointwise convergence.

We move now from Riemann integration to Lebesgue integration.
In Riemann's theory of integration, we approximate the integral with Darboux sums, by dividing the domain. We require the domain to have a total order, while the codomain must be a Banach space.
Conversely, in Lebesgue integration we divide the codomain (again, needing a total order) while the domain must simply be a measure space.

\begin{defi}[Simple]\hypertarget{def:simple}
    $f: (X, \mathscr{A}) \to (\R, \mathcal{B}(R))$ is \textbf{simple} if it is measurable and takes a finite number of values in $[0, +\infty)$.
\end{defi}
From here on in, when working on the real line, subsets thereof, or the extended real line the \hyperlink{def:sigAlg}{$\sigma$-algebra} will be the \hyperlink{def:borelSet}{Borel sets}.
\begin{remark}
    $A \subset X$, $\chi_A$ is \hyperlink{def:measFunc}{measurable} iff $A \in \mathscr{A}$ is simple.
    The general form of a simple function is $s = \sum \alpha_i \chi_{A_i}$.
\end{remark}
\begin{prop}
    Let $f: (X, \mathscr{A}) \to [0, +\infty]$ measurable.
    Here $[0, +\infty] = [0, +\infty) \cup \{+\infty\}$, so the neighbourhoods of $\infty$ are $(a, +\infty])$, and we can have a metric $d(x,y) = \abs{\arctan x - \arctan y}$
    There is $(s_k)$, a sequence of simple functions $s_{k+1} \geq s_k$ converging pointwise to $f$.
\end{prop}
\begin{proof}
    For $n \geq 1$, define
    \begin{align*}
        B_n &= \set{x | f(x) \geq n}  \\
        A_n^i &= \Set{x | f(x) \in \left[\frac{i-1}{2^n}, \frac{i}{2^n}\right]} \quad i=0, \dotsc, n 2^n
    \end{align*}
    Also, set

    \begin{equation*}
    s_n = \begin{cases*}\frac{i-1}{2^n} & on $A_n^i$ \\ n & on $B_n$\end{cases*}
    \end{equation*}
    Check
    \begin{itemize}
        \item $s_{n+1} \geq s_n$ $[A_n^i = A_{n+1}^{2i} \cup A_{n+1}^{2i+1}]$
        \item on $B_n$, $\abs{s_n - f} \leq \frac{1}{2^n}$
        \item on $x \in \bigcap_{k \geq 1} B_n$, $s_n(x) \to +\infty$.
    \end{itemize}
\end{proof}

\begin{defi}
    Take  $(X, \mathscr{A}, \mu)$ and $s$ a simple function on it given by $s=\sum_{i=1}^n \alpha_i \chi_{A_i}$, $\alpha_i \in [0, \infty)$.
    For $E \in \mathscr{A}$, define
    \begin{equation*}
        \int_E s \, d\mu = \sum_{i=1}^n \alpha_i \mu(A_i \cap E).
    \end{equation*}
\end{defi}
\begin{remark}
    This induces a new measure on $\mathscr{A}$, with $E \mapsto \int_E s \, d\mu$.
\end{remark}

\begin{defi}
    Take $f:(X, \mathscr{A}, \mu) \to [0, \infty]$ measurable and $E \in \mathscr{A}$.
    Then define
    \begin{equation*}
        \int_E f \, d \mu \coloneqq \sup\Set{\int_E s \, d\mu | s \leq f} \in [0, +\infty]
    \end{equation*}
\end{defi}
\begin{remark}
    This integral always makes sense. Also, $\int_E f \, d\mu = 0$ if $\mu(E) = 0$
\end{remark}
\begin{ex}
    Check linearity of the integral. Show Chebyshev's inequality:
    \begin{equation*}
        \mu(\set{x | f(x) \geq \alpha}) \leq \alpha^{-1} \int_X f \, d\mu
    \end{equation*}
    for $\alpha>0$.
    If $f$ is measurable, $X \to [0, +\infty]$ satisfies $\int_X f \, d\mu < +\infty$, then $\mu(\set{x | f(x) = +\infty})$
\end{ex}

\begin{thm}[Beppo-Levi monotone convergence]
    Take $f_k: (X, \mathscr{A}, \mu) \to [0, +\infty]$ measurable, converging pointwise to $f$ with $f_k \leq f_{k+1}$.
    Then $\forall E \in \mathscr{A}$, $\int_E f_k \, d\mu \xrightarrow{k \to +\infty} \int_E f \, d\mu$.
\end{thm}
\begin{proof}
    Reduce to $E = X$ by consdering $f_k \chi_E, f \chi_E, \dotsc$.
    Then $\left(\int_X f_k \, d\mu\right)_{k \geq 1}$ is a sequence in $[0, +\infty]$, non-decreasing.

    By monotonicity, $f_k \nearrow f$, so $\int_X f_k \, d\mu \leq \int_X f \, d\mu$.
    Let
    \begin{equation*}
        \alpha \coloneqq \lim_{k \to +\infty} \int_X f_k \, d\mu \leq \int_X f \, d\mu.
    \end{equation*}
    Consider a simple function $s \leq f$ and $c \in (0, 1)$.
    \begin{gather*}
        E_k = \set{x\in X | f_k(x) \geq c s(x)} \in \mathscr{A} \text{ (using that $f_k$, $s$ are measurable)}\\
        E_k \subset E_{k+1}, \bigcup_{k \geq 1} E_k = X \text{ (by pointwise convergence)}
    \end{gather*}
    Thus $\int_X s \, d\mu = \lim_{k \to +\infty} \int_{E_k} s \, d\mu$ (by continuity from below of $\mu$).
    \begin{equation*}
        \int_X f_k \, d\mu \geq \int_{E_k} f_k \, d \mu \geq c \int_{E_k} s \, d \mu
    \end{equation*}
    Take $k \to +\infty$. $\alpha \geq c \int_X s \, \mu$, and let $c \nearrow 1$, giving $\int_X s \, d\mu$.
    Taking the supremum over $s \leq f$ for $s$ simple,
    \begin{equation*}
        \alpha \geq \int_X f `, d\mu.\qedhere
    \end{equation*}
\end{proof}

\begin{ex}
    Taking $f_k$ as above, show
    \begin{equation*}
        \int_X \left(\sum_{k \geq 1} f_k\right) \, d\mu = \sum_{k \geq 1} \int_X f_k \, d\mu
    \end{equation*}
    Let $\nu: A \in \mathscr{A} \mapsto \int_A f \, d\mu$ (for $f$ a measure $X \to [0, +\infty]$).
    Show that for any measure $g: (X, \mathscr{A}, \mu) \to [0, +\infty]$, $\int_X g \, d\mu = \int_X fg \, d\mu$.
\end{ex}
\begin{thm}[Fatou's lemma]
    Take $f_k$ as above, then
    \begin{equation*}
        \int_X (\liminf f_k) \, d\mu \leq \liminf \left(\int_X f_k \, d\mu\right)
    \end{equation*}
\end{thm}

\begin{proof}
    Let $F_k = \inf\set{f_l | l \geq k}$, non-decreasing, valued in $[0, +\infty]$.
    These are measurable: $\{F_k \geq a\} = \bigcap_{l \geq k} \{f_l \geq a\}$.
    Observe that $\int \min(f, g) \, d\mu \leq \min(\int f \, d\mu, \int g \, d\mu)$.
    Now, by \hyperlink{thm:Beppo-Levi},
    \begin{align*}
        \int_X (\liminf f_k) \, d\mu &= \int_X (\lim F_k) \, d\mu = \lim_{k \to +\infty} \left(\int_X F_k \, d\mu\right) \\
                                     &= \lim_{k \to \infty} \left(\int_X \left( \inf_{l \geq k} f_l\right) \, d\mu\right) \\
                                     &\leq \lim_{k \to \infty} \inf_{l \geq k} \left(\int_X f_l \, d\mu\right)\\
                                     &\leq \liminf_{k \to \infty} \int f_k \, d\mu.\qedhere
    \end{align*}
\end{proof}
\begin{defi}[Integrable]
    $f: (X, \mathscr{A}, \mu) \to \C$ measurable is integrable if $\abs{f}: X \to [0, +\infty)$ satisfies $\int_X \abs{f} \, d\mu < +\infty$.
\end{defi}
Compute by splitting $f$ into real and imaginary parts, and each into nonnegative and nonpositive parts.

\begin{thm}[Lebesgue's Dominated Convergence]
    Take $f_k: (X, \mathscr{A}, \mu) \to \C$ where
    \begin{itemize}
        \item convergence: $f_k$ converges pointwise to $f$
        \item domination: $\exists g$ integrable
    \end{itemize}
\end{thm}

\begin{proof}
    Let $h_k = 2g - \abs{f - f_k}$, taking values in $[0, +\infty]$.
    Then $h_k \to 2g$ pointwise.
    \begin{gather*}
        \int_X 2g \, \mu = \int_X (\lim h_k) \, d\mu \leq \liminf_{k \to \infty} \left(\int_X h_k \, d\mu\right).
        \text{Recall }\int_X h_k \, d\mu = \int_X 2g \, d\mu - \int_X \abs{f_k-f} \, d\mu
        \int_X 2g \, d\mu \leq \int_X 2g \, d\mu - \limsup_{k \to \infty} \int \abs{f_k - f} \, d\mu.
    \end{gather*}
\end{proof}
\end{document}
