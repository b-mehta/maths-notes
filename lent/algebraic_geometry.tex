\documentclass{article}

\def\npart {II}
\def\nyear {2017}
\def\nterm {Lent}
\def\nlecturer{Prof. I. Grojnowski}
\def\ncourse{Algebraic Geometry}
\ifx \nauthor\undefined
  \def\nauthor{Bhavik Mehta}
\else
\fi

\author{Based on lectures by \nlecturer \\\small Notes taken by \nauthor}
\date{\nterm\ \nyear}
\title{Part \npart\ -- \ncourse}

\usepackage[utf8]{inputenc}
\usepackage{amsmath}
\usepackage{amsthm}
\usepackage{amssymb}
\usepackage{enumerate}
\usepackage{mathtools}
\usepackage{graphicx}
\usepackage[dvipsnames]{xcolor}
\usepackage{tikz}
\usepackage{wrapfig}
\usepackage{centernot}
\usepackage{float}
\usepackage{braket}
\usepackage[hypcap=true]{caption}
\usepackage{enumitem}
\usepackage[colorlinks=true, linkcolor=mblue]{hyperref}
\usepackage[nameinlink,noabbrev]{cleveref}
\usepackage{nameref}
\usepackage[margin=1.5in]{geometry}

% Theorems
\theoremstyle{definition}
\newtheorem*{aim}{Aim}
\newtheorem*{axiom}{Axiom}
\newtheorem*{claim}{Claim}
\newtheorem*{cor}{Corollary}
\newtheorem*{conjecture}{Conjecture}
\newtheorem*{defi}{Definition}
\newtheorem*{eg}{Example}
\newtheorem*{ex}{Exercise}
\newtheorem*{fact}{Fact}
\newtheorem*{law}{Law}
\newtheorem*{lemma}{Lemma}
\newtheorem*{notation}{Notation}
\newtheorem*{prop}{Proposition}
\newtheorem*{question}{Question}
\newtheorem*{rrule}{Rule}
\newtheorem*{thm}{Theorem}
\newtheorem*{assumption}{Assumption}

\newtheorem*{remark}{Remark}
\newtheorem*{warning}{Warning}
\newtheorem*{exercise}{Exercise}

% \newcommand{\nthmautorefname}{Theorem}

\newtheorem{nthm}{Theorem}[section]
\newtheorem{nlemma}[nthm]{Lemma}
\newtheorem{nprop}[nthm]{Proposition}
\newtheorem{ncor}[nthm]{Corollary}
\newtheorem{ndef}[nthm]{Definition}

% Special sets
\newcommand{\C}{\mathbb{C}}
\newcommand{\N}{\mathbb{N}}
\newcommand{\Q}{\mathbb{Q}}
\newcommand{\R}{\mathbb{R}}
\newcommand{\Z}{\mathbb{Z}}

\newcommand{\abs}[1]{\left\lvert #1\right\rvert}
\newcommand{\norm}[1]{\left\lVert #1\right\rVert}
\renewcommand{\vec}[1]{\boldsymbol{\mathbf{#1}}}

\let\Im\relax
\let\Re\relax

\DeclareMathOperator{\Im}{Im}
\DeclareMathOperator{\Re}{Re}
\DeclareMathOperator{\id}{id}

\definecolor{mblue}{rgb}{0., 0.05, 0.6}


% preamble
\usepackage{tikz}
\usepackage{tikz-cd}
\newcommand{\A}{\mathbb{A}}
\DeclareMathOperator{\Mat}{Mat}
\DeclareMathOperator{\chara}{char}
%\setcounter{section}{-1}
% and here we go!

\begin{document}
\maketitle

Consider $E = \set{(x, y) \in \C^2 | y^2 = x^3 - x}$. Let's first draw this when $(x, y) \in \R^2$.
If $y \in \R$, $y^2 \geq 0$, so if $x \in \R$, $x^3 - x = x(x^2 - 1) \geq 0$ so $x \geq 1$ or $-1 \leq x \leq 0$.

% draw picture!

Now consider $(x, y) \in \C$. In general, this is tricky.
Here, define $p: E \to \C$ given by $(x, y) \mapsto x$ most of the time ($x \notin \{0, 1, -1\}$), $p^{-1}(x)$ is two points.
This doesn't help us visualise.

\begin{equation*}
    \Gamma = \set{(x, y) \in \C^2 | y \in \R, x \in [-1, 0] \cup [1, \infty)}
\end{equation*}
Claim: $E \setminus \Gamma$ is disconnected and has two pieces.
Proof: Exercise.

So, $E \setminus \Gamma$ is two copies of
% pic
glued together. To glue, turn one of the pieces over (this ruins the representation as a double cover, but is the right gluing).
Think of (pic) by adding a point at $\infty$, so it lives on the Riemann surface.

Take another copy, flip it over and glue back.

% I
\section{Dictionary between algebra and geometry}
% 1.
\subsection{Basic notions}
\begin{defi}[Affine space]
    \textbf{Affine $n$-space} is $\A^n = \A^n(k) \coloneqq k^n$ for $k$ a field.
\end{defi}
Write $k[\A^n] = k[x_1, \dotsc, x_n]$ polynomials in $n$ variables.
Any $f \in k[\A^n]$ defines a function $f: \A^n =k^n \to k$ by $(\lambda_1, \dotsc, \lambda_n) \mapsto f(\lambda_1, \dotsc, \lambda_n)$ by evaluation.

Let $S \subseteq k[x_1, \dotsc, x_n]$ be any subset of polynomials.
\begin{defi}[Affine variety]
    \begin{equation*}
        Z(S) = \set{\lambda = (\lambda_1, \dotsc, \lambda_n) \in k^n | f(\lambda_1, \dotsc, \lambda_n) = 0 \text{ for all } f \in S}
    \end{equation*}
    is called the \textbf{affine variety defined by} $S$, the simultaneous zeros of all functions in $S$.
    $Z(S)$ is called an affine subvariety of $\A^n$.
\end{defi}

\begin{eg}
    \begin{enumerate}[label=(\Roman*)]
        \item $\A^n = Z(0)$.
        \item On $\A^1$, $Z(x) = \{0\}$, $Z(x-7) = \{7\}$.  If $f(x) = (x - \lambda_1)\dotsc (x-\lambda_n)$, $Z(f(x)) = \{\lambda_1, \dotsc, \lambda_n\}$.
            Affine subvarieties of $\A^1$ are $\A^1$ and finite subsets of $\A^1$.
        \item in $\A^2$, $E=Z(y^2-x^3+x)$ we have sketched when $k=C$ and $k=R$.
            % diagrams of Z(x,y), Z(xy), Z(y) and Z(y(y-1),x(y-1))

    \end{enumerate}
\end{eg}

\begin{remark}
    If $f \in k[A^n]$ then $Z(f)$ is called a \textbf{hypersurface}.
\end{remark}

Observe that if $J$ is the ideal generated by $S$
\begin{equation*}
    J = \Set{\sum a_l f_l | a_l \in k[x_1, \dotsc, x_l], f_l \in S}
\end{equation*}
then $Z(J) = Z(S)$.  Hence,
\begin{thm}
    If $Z(S)$ is an affine subvariety of $\A^n$, there is a finite set $f_1, \dotsc, f_r$ of polynomials with $Z(S) = Z(f_1, \dotsc, f_r)$.
\end{thm}

\begin{proof}
    $J = \langle f_1, \dotsc, f_r \rangle$ for some $f_1, \dotsc, f_r$ by Hilbert basis theorem.
\end{proof}

\begin{lemma}
    \leavevmode
    \begin{enumerate}[label=(\roman*)]
        \item if $I \subseteq J$, $Z(J) \subseteq Z(I)$
        \item $Z(0) = \A^n$, $Z(k[x_1, \dotsc, x_n]) = \emptyset$.
        \item $Z\left(\bigcup J_L\right) = Z(\sum J_L) = \bigcap Z(J_L)$ for any possibly infinite family of ideals
        \item $Z(I \cap J) = Z(I) \cap Z(J)$ if $I, J$ ideals
    \end{enumerate}
\end{lemma}
\begin{proof}
    (i), (ii), (iii) are clear.
    (iv): $\supseteq$ holds by (i). Conversely, if $x \notin Z(I)$ then $\exists f_1 \in I$ such that $f_1(X) \neq 0$.
    So if $x \notin Z(J)$ also, $\exists f_2 \in J$ with $f_2(x) \neq 0$ also.
    Hence $f_1 f_2(x) = f_1(x) f_2(x) \neq 0$, so $x \notin Z(f_1 f_2)$. But $f_1 f_2 \in I \cap J$, as $I, J$ ideals so $x \notin Z(I \cap J)$.
\end{proof}

Looking at these results, $Z(I)$ form closed subsets of a topology on $\A^n$, called the \hypertarget{def:zariski}{`Zariski topology'}.

If $Z \subset \A^n$ is any subset, let $I(Z) = \set{f \in k[\A^n] | f(p) = 0, \forall p \in \Z}$.
Observe that $I(Z)$ is an ideal: if $g \in k[\A^n]$, $f(p) = 0$ then $(gf)(p) = 0$.
\begin{lemma}
    \leavevmode
    \begin{enumerate}[label=(\roman*)]
        \item $Z \subseteq Z' \implies I(Z') \subseteq I(Z)$
        \item for any $Y \subseteq \A^n$, $Y \subseteq Z(I(Y))$,
        \item if $V = Z(J)$ is a subvariety of $\A^n$, then $V = Z(I(V))$.
        \item if $J \lhd k[\A^n] = k[x_1, \dotsc, x_n]$ an ideal, then $J \subseteq I(Z(J))$.
    \end{enumerate}
\end{lemma}
\begin{proof}
    (i), (ii), (iv) are clear.
    For (iii), first show $\supseteq$. $I(V) = I(Z(J))\supseteq J$ by (iv) so $Z(I(V)) \subseteq Z(J)=V$ by (i). $\subseteq$ follows by (iv).
\end{proof}
Hence (ii) and (iii) show that $Z(I(Y))$ is the smallest affine subvariety of $\A^n$ containing $Y$, i.e. it is the closure of $Y$ in the Zariski topology.

Take $\Z \subseteq \C = \A^1$, $k = \C$ the closure of $\Z$ in Zariski topology is $\C$. $I(\Z) = \{0\}$ as if a poly vanishhes at every integer it is 0.
Note if $k = \C$, $f \in \C[x_1, \dotsc, x_n]$, then $f$ is continuous in the usual topology, so
\begin{equation*}
    Z(J) = \bigcap_{f \in J} Z(f) = \bigcap_{f \in J} f^{-1}(0)
\end{equation*}
is a closed set in the usual topology, i.e. Zariski closed $Z \implies$ clsoed in the usual topology.
So,
\begin{equation*}
\begin{tikzcd}
    \{\text{Zariski closed subvarieties of} \A^n\} & \{\text{ideals in } k[x_1, \dotsc, x_n]\}
\end{tikzcd}
\end{equation*}
But this is not a bijection
$Z(X) = Z(X^2) = Z(X^3) = \dotsc = \{0\} \subseteq \A^1$. $Z(\langle f_1^{a_1}, \dotsc, f_r^{a_r}\rangle) = Z(f_1, f_2, \dotsc, f_r)$.
but it turns out this kind of thing is the only problem % Nullstellensatz
\begin{defi}
    An affine variety $Y$ is \textbf{reducible} if $\exists$ affine varieties $Y_1, Y_2$, $Y_i \neq Y$ with $Y = Y_1 \cup Y_2$, and irreducible otherwise, and disconnected if $Y_1 \cap Y_2 = \emptyset$.
\end{defi}
% picsss
So $Z(xy) = Z(x) \cup Z(y)$, reducible.
$Z(y(y-1), x(y-1)) = Z(xy) \cup Z(y-1)$ reducible and disconnected.
\begin{prop}
    Any affine variety is a finite union of irreducible affine varieties.
\end{prop}
\begin{remark}
    This is very different from usual manifolds.
\end{remark}
\begin{proof}
    If not, $Y$ is not irreducible, so $Y = Y_1 \cup Y_1'$ and one of $Y_1, Y_1'$, (say $Y_1$) is not the finite union of irreducible affine varieties, so
    \begin{equation*}
        Y_1 = Y_2 \cup Y_2', \dotsc
    \end{equation*}
    and so we get an infinite chain of affine varities $Y \supsetneq Y_1 \supsetneq Y_2 \supsetneq \dotsb$.
    But each $Y_i = Z(I_i)$ for some ideal $I_l$. Let $W = \bigcap Y_l = Z(\sum I_i) = Z(I)$.
    $I = \sum I_i$ is an ideal. As the ideal $I$ is finitely generated $I = \langle f_1, \dotsc, f_r \rangle$ for some $f_i$.
    $f_i \in I_{a_i}$ for some $a_1, \dotsc, a_r$ so $I = I_{a_1} + \dotsb + I_{a_r}$, $W = Y_{i_1} \cap \dotsb \cap Y_{i_r}$ contradicting $Y_N \subsetneq Y_{a_1} \cap \dotsb \cap Y_{a_r}$ if $N > r$.
\end{proof}
\begin{ex}
    If $Y$ is a subvariety of $\A^\N$, $Y = Y_1 \cup \dotsb \cup Y_r$ with $Y_i$ irreducible, and $r$ minimal is unique up to reordering. Call the $Y_l$ the irreducible components of $Y$.
\end{ex}
\begin{prop}
    $Y$ is irreducible $\iff I(Y)$ is a prime ideal in $k[\A^n] = k[x_1, \dotsc, x_n]$.
\end{prop}
\begin{eg}
    \begin{enumerate}[label=(\roman*)]
        \item $(x y)$ is not a prime ideal.
        \item Exercise: Let $R$ be a UFD, $f \in R$, $f \neq 0$, $f$ irreducible $\iff (f)$ a prime ideal.
        \item Exercise: $k[x_1, \dotsc, x_n]$ is a UFD.
            hence $Z(y^2 - x^3 + x)$ is irreducible, $Z(y-x^2)$ is irreducible.
    \end{enumerate}
\end{eg}
\begin{proof}
    \begin{center}
        \begin{tikzpicture}
            \draw plot [smooth] coordinates {(-1, 1) (1, -1) (1.4, -2)};
            \draw plot [smooth] coordinates {(-1, -1) (1, 1) (1.5, 1.2)};
            % \draw plot [smooth] coordinates {<coordinate1> <coordinate2> <coordinate3> ...};
        \end{tikzpicture}
    \end{center}
    If $Y = Y_1 \cup Y_2$ is reducible, $\exists p \in Y_1 \setminus Y_2$ so $\exists f \in I(Y_2)$ such that $f(p) \neq 0$ and similarly, $\exists q \in Y_2 \setminus Y_1$ so $\exists g \in I(Y_1)$ such that $g(q) \neq 0$.
    Then $fg \in I(Y_1) \cap I(Y_2) = I(Y)$. But $f \notin I(Y)$, $g \notin I(Y)$ so not prime.

    Conversely, if $I(Y)$ is not prime $\exists f_1 f_2 \in k[\A^n]$ such that $f_1, f_2 \notin I(Y)$ but $f_1 f_2 \in I(Y)$.
    Let $Y_i = Y_n \cap Z(f_i) = \set{p \in Y | f_i(p) = 0}$. $Y_1 \cup Y_2 = Y$, as $p \in Y \implies f_1 f_2 (p) = 0 \implies f_1(p) = 0$ or $f_2(p) = 0$.
    $Y_i \neq Y$ as $f_i \notin I(Y)$ (i.e. $\exists p_l \in Y$ such that $f_i(p_i) \neq 0$ so $p_i \notin Y_i$).
\end{proof}
\begin{lemma}
    $X$ irreducible affine subvariety of $\A^n$, $\mathcal{U} \subseteq X$ open and non-empty $\implies \overline{\mathcal{U}} = X$.
\end{lemma}
\begin{proof}
    Let $Y = X - \mathcal{U}$, closed. Then $\overline{\mathcal{U}} \cup Y = X$, and $U \neq \emptyset \implies Y \neq X$. But $X$ is irreducible, so $\overline{\mathcal{U}} = X$.
\end{proof}

Application: Cayley-Hamilton Theorem
$A \in \Mat_n(k)$, an $n \times n$ matrix, with
\begin{equation*}
    \chara_A(x) = \det(x I - A) \in k[x]
\end{equation*}
the characteristic polynomial.
This gives a function $\chara_A: \Mat_n(k) \to \Mat_n(k)$ $B \mapsto \chara_A(B)$.
Cayley-Hamilton theorem says that $\forall A \in \Mat_n(k)$, $\chara_A(A) = 0$. Notice this is an equality of matrices, so it is $n^2$ equations.
\begin{proof}
    Let $X = \A^{n^2} = \Mat_n(k)$, affine space, hence irreducible algebraic variety.
    Consider $CH = \set{A \in \Mat_n(k) | \chara_A(A) = 0}$.
    Claim: this is a Zariski closed subvariety of $\A^{n^2}$, cut out by $n^2$ equations, $\chara_A(A)_y = 0$.
    We must check that these equations are polynomials in the matrix coefficients of $A$.
\end{proof}
\end{document}
