\documentclass{article}

% preamble
\def\npart{III}
\def\nyear{2019}
\def\nterm{Lent}
\def\draft{Unfinished course}
\def\nlecturer{Dr T.\ Bloom}
\def\ncourse{Analytic Number Theory}

\usepackage{imakeidx}
\usepackage{marginnote}
\usepackage{chngcntr}
\usepackage[intoc, refpage]{nomencl}

\ifx \nauthor\undefined
  \def\nauthor{Bhavik Mehta}
\else
\fi

\author{Based on lectures by \nlecturer \\\small Notes taken by \nauthor}
\date{\nterm\ \nyear}
\title{Part \npart\ -- \ncourse}

\usepackage[utf8]{inputenc}
\usepackage{amsmath}
\usepackage{amsthm}
\usepackage{amssymb}
\usepackage{enumerate}
\usepackage{mathtools}
\usepackage{graphicx}
\usepackage[dvipsnames]{xcolor}
\usepackage{tikz}
\usepackage{wrapfig}
\usepackage{centernot}
\usepackage{float}
\usepackage{braket}
\usepackage[hypcap=true]{caption}
\usepackage{enumitem}
\usepackage[colorlinks=true, linkcolor=mblue]{hyperref}
\usepackage[nameinlink,noabbrev]{cleveref}
\usepackage{nameref}
\usepackage[margin=1.5in]{geometry}

% Theorems
\theoremstyle{definition}
\newtheorem*{aim}{Aim}
\newtheorem*{axiom}{Axiom}
\newtheorem*{claim}{Claim}
\newtheorem*{cor}{Corollary}
\newtheorem*{conjecture}{Conjecture}
\newtheorem*{defi}{Definition}
\newtheorem*{eg}{Example}
\newtheorem*{ex}{Exercise}
\newtheorem*{fact}{Fact}
\newtheorem*{law}{Law}
\newtheorem*{lemma}{Lemma}
\newtheorem*{notation}{Notation}
\newtheorem*{prop}{Proposition}
\newtheorem*{question}{Question}
\newtheorem*{rrule}{Rule}
\newtheorem*{thm}{Theorem}
\newtheorem*{assumption}{Assumption}

\newtheorem*{remark}{Remark}
\newtheorem*{warning}{Warning}
\newtheorem*{exercise}{Exercise}

% \newcommand{\nthmautorefname}{Theorem}

\newtheorem{nthm}{Theorem}[section]
\newtheorem{nlemma}[nthm]{Lemma}
\newtheorem{nprop}[nthm]{Proposition}
\newtheorem{ncor}[nthm]{Corollary}
\newtheorem{ndef}[nthm]{Definition}

% Special sets
\newcommand{\C}{\mathbb{C}}
\newcommand{\N}{\mathbb{N}}
\newcommand{\Q}{\mathbb{Q}}
\newcommand{\R}{\mathbb{R}}
\newcommand{\Z}{\mathbb{Z}}

\newcommand{\abs}[1]{\left\lvert #1\right\rvert}
\newcommand{\norm}[1]{\left\lVert #1\right\rVert}
\renewcommand{\vec}[1]{\boldsymbol{\mathbf{#1}}}

\let\Im\relax
\let\Re\relax

\DeclareMathOperator{\Im}{Im}
\DeclareMathOperator{\Re}{Re}
\DeclareMathOperator{\id}{id}

\definecolor{mblue}{rgb}{0., 0.05, 0.6}


\reversemarginpar

\makeindex[intoc]

\makenomenclature
\renewcommand{\pagedeclaration}[1]{, \hyperlink{page.#1}{#1}}
\renewcommand{\nomname}{Index of Notation}

\renewcommand{\nompreamble}{\begin{multicols}{2}}
\renewcommand{\nompostamble}{\end{multicols}}

\newcommand{\named}[1]{\textbf{#1}\index{#1}}
\newcommand{\bonusnamed}[1]{\textbf{#1}\index{#1@*#1}}
\newcommand{\bigO}{\mathcal{O}}

\DeclarePairedDelimiter\floor{\lfloor}{\rfloor}

\setcounter{section}{-1}
\counterwithout{nthm}{section}

% and here we go!
\begin{document}
\maketitle

\tableofcontents

\clearpage
\section{Introduction}
Analytic \marginnote{\emph{Lecture 1}}[0cm]Number Theory is the study of numbers using analysis. It is a fascinating field because because a number - in particular in this course an integer - is discrete, whilst analysis involves the real/complex numbers which are continuous.

In this course, we will ask quantitative questions.

\begin{eg}\leavevmode
  \begin{enumerate}
    \item How many primes? We can define the function $\pi(x) = |\{n \mid n \leq x \text{ and } n \text{ is prime} \}|$. Then the prime number theorem, which we will prove in this course states $$\pi(x) \sim \frac{x}{\log x}. $$
      (We will always take `numbers' to mean natural numbers, not including zero).
    \item  How many twin primes are there? That is, where $p,\,p+2$ are both prime.
      It is not known whether there are infinitely many but since 2014, there has been immense progress by Zhang, Maynard and a Polymath project which has determined there are infinitely many primes at most 246 apart.
      Guess: there are $\approx \frac{x}{(\log x)^2}$ many $\leq x$.
    \item How many primes are there $\equiv a \bmod q $ where $(a,q) = 1$. We know, by Dirichlet's theorem proven in the 20th century, that there are infinitely many such.
      The guess for how many is
      \begin{equation*}
        \frac{1}{\varphi(q)} \frac{x}{\log x}.
      \end{equation*}
      This is known for small $q$. Recall $\varphi(n) = |\{1\leq m \leq n \ | \ (m,n) =1 \}|$
  \end{enumerate}
\end{eg}
The course will be split up into 4 (roughly equal) parts
\begin{enumerate}
  \item Elementary techniques (real analysis)
  \item Sieve methods
  \item Riemann Zeta function, Prime Number Theorem (complex analysis)
  \item Primes in arithmetic progressions
\end{enumerate}
\clearpage

\section{Elementary Techniques}
We begin with a review of asymptotic notations:
\begin{itemize}\hypertarget{def:asymp}
  \item \nomenclature{$\bigO$}{Big $\bigO$ notation; Landau notation}$f(x) = \bigO(g(x))$ if there is $C>0$ such that $|f(x)| \leq C|g(x)|$ for all large enough $x$. (Landau notation)
  \item \nomenclature{$\ll$}{Vinogradov notation}$f \ll g$ is the same as $f = \bigO(g)$ (Vinogradov notation)
  \item \nomenclature{$\sim$}{asymptotic equality}$f\sim g$ if $\lim_{x \to \infty}\frac{f(x)}{g(x)} = 1 $ (i.e.\ $f = (1+o(1))g$).
  \item \nomenclature{$o$}{Little $o$ notation}$f= o(g)$ if $\lim_{x \to \infty}\frac{f(x)}{g(x)} = 0$
\end{itemize}

\subsection{Arithmetic Functions}
\begin{defi}[Arithmetic function]\hypertarget{def:arith}
  An \named{arithmetic function} is just a function $f : \mathbb{N} \to \mathbb{C}$.
\end{defi}
\begin{defi}[Convolution]\hypertarget{def:conv}
  An important operation for multiplicative number theory is the \textbf{multiplicative convolution\index{convolution}}
  \nomenclature{$*$}{convolution}
  \begin{equation*}f*g(n) \coloneqq \sum_{ab = n} f(a)g(b).\end{equation*}
\end{defi}
\begin{eg}\leavevmode
  \begin{itemize}
    \item \hypertarget{def:1}{$1(n) \coloneqq 1 \; \; \forall n$}. Caution: $1*f \neq f$.
    \item \hypertarget{def:mu}M\"obius function:\index{M\"obius function}\nomenclature{$\mu$}{M\"obius function}
      \begin{equation*}
        \mu(n) =
        \begin{cases*}
          (-1)^k & if $n = p_1\dotsm p_k$ \\
          0 & if $n$ not squarefree
        \end{cases*}
      \end{equation*}
    \item \hypertarget{def:lamb}Liouville function:\index{Liouville function}\nomenclature{$\lambda$}{Liouville function}
      \begin{equation*}
        \lambda(n) = (-1)^k \text{ if } n = p_1\dotsm p_k, \text{ not necessarily distinct}
      \end{equation*}
    \item \hypertarget{def:tau}Divisor function:\index{divisor function}\nomenclature{$\tau$}{divisor function}
      \begin{align*}
        \tau(n) &= | \{ d \mid d\text{ a factor of }n \} | \\
        \tau &= 1 \hyperlink{def:conv}{*}1
      \end{align*}
  \end{itemize}
\end{eg}
\begin{defi}[Multiplicative function]\hypertarget{def:multi}
  An \hyperlink{def:arith}{arithmetic function} is a \named{multiplicative function} if $f(nm) = f(n)f(m)$ for $(n,m) =1$.
  In particular, a multiplicative function is determined by its values on prime powers $f(p^k)$.
\end{defi}
\begin{fact}
  If $f,g$ are \hyperlink{def:multi}{multiplicative}, then so is $\hyperlink{def:conv}{f*g}$.
  $\log n $ is not \hyperlink{def:multi}{multiplicative}.
  Note, almost all \hyperlink{def:arith}{arithmetic functions} are not multiplicative.
\end{fact}
\begin{fact}[\hypertarget{def:mobinv}M\"obius inversion]\index{M\"obius function}
  \begin{equation*}
    \hyperlink{def:1}{1} \hyperlink{def:conv}{*} f = g \iff \hyperlink{def:mu}{\mu} * g = f.
  \end{equation*}
\end{fact}

\begin{proof}
  \begin{equation*}
    \sum_{d \mid n} \hyperlink{def:mu}{\mu(d)} =
    \begin{cases*}
      1 & if $n=1$ \\
      0 & otherwise
    \end{cases*}
  \end{equation*}
  Note the left hand side is $1\hyperlink{def:conv}{*}\mu$.
  Since $1,\mu$ are \hyperlink{def:multi}{multiplicative}, $1*\mu$ is multiplicative.
  Hence it is enough to check the identity for prime powers:
  If $n= p^k$, then $\{d \mid d\text{ divides }n\} = \{1,p,\ldots,p^k\}$ so the left hand side is $1-1 + 0 + \ldots + 0 = 0$, unless $k=0$ when the left hand side is $\mu(1) =1$.

  The right hand side is the identity of \hyperlink{def:conv}{convolution}, and convolution is associative, giving the required result.
\end{proof}

Our ultimate goal is to study the primes. This would suggest that we should work with
\begin{equation*}
  1_p(n) = \begin{cases*}
    1 &if $n$ prime \\
    0 & otherwise
  \end{cases*}
\end{equation*}
For example $\pi(x) = \displaystyle \sum_{1\leq n\leq x} 1_p(n)$.
This is an awkward function to work with.
Instead, we work with the \nomenclature{$\Lambda$}{von Mangoldt function}\named{von Mangoldt function}
\begin{equation*}
  \hypertarget{def:vonMang}\Lambda(n) =
  \begin{cases*}
    \log p & if $n$ is a prime power\\
    0 & otherwise.
  \end{cases*}
\end{equation*}
This function is easier to understand. Why?
\begin{lemma}
  \begin{equation*}
    \hyperlink{def:1}{1} \hyperlink{def:conv}{*} \hyperlink{def:vonMang}{\Lambda} = \log \quad \text{and} \quad \hyperlink{def:mu}{\mu} * \log = \Lambda
  \end{equation*}
\end{lemma}
\begin{proof}
  The second part follows immediately by \hyperlink{def:mobinv}{M\"obius inversion}.
  \begin{align*}
    1*\hyperlink{def:vonMang}{\Lambda}(n) & = \sum_{d \mid n}\Lambda(d) \quad \text{so if } n = p_1^{k_1}\ldots p_k^{n_k} \\
                 &= \sum_{i=1}^r \sum_{j=1}^{k_i} \Lambda(p_i^j) \\
                 &= \sum_{i=1}^r \sum_{j=1}^{k_i} \log p_i \\
                 &= \sum_{i=1}^{r}k_i\log p_i
                 = \sum_{i=1}^{r}\log p_i^{k_i}
                 = \log n. \qedhere
  \end{align*}
\end{proof}
We can write
\begin{align*}
  \Lambda(n) &= \sum_{d|n}\mu(d)\log\left(\frac{n}{d}\right) \\
             &= \log n \sum_{d|n}\mu(d) - \sum_{d|n}\mu(d)\log d \\
             &= - \sum_{d|n}\mu(d)\log d.
\end{align*}

\begin{eg}
  \begin{align*}
    \sum_{\mathclap{1\leq n \leq x}}\Lambda(n) &= -\sum_{1\leq n\leq x}\sum_{d|n}\mu(d)\log d \\
                                    &= - \sum_{d\leq x}\mu(d)\log(d)\left(\sum_{\substack{1\leq n\leq x \\ d|n}}1\right) \\
                                    &= - x \sum_{d\leq x}\mu(d) \frac{\log d}{d} + \hyperlink{def:asymp}{O}\left(\sum_{d\leq x}\mu(d)\log d \right)
  \end{align*}
  since
  \begin{equation*}
    \sum_{\mathclap{\substack{1\leq n\leq x \\ d|n}}}1 = \floor*{\frac{x}{d}}= \frac{x}{d} + O(1).
  \end{equation*}
\end{eg}
\clearpage

% lecture 2
\subsection{Summation}
Given an arithmetic function, we can ask for estimates of $\sum_{n \leq x} f(n)$.
We say that $f$ has average order $g$ if $\sum_{1 \leq n \leq x} f(n) \sim x g(x)$ (average size of $f$ is $g$).

For example, if $f \equiv 1$,
\begin{equation*}
  \sum_{1 \leq n \leq x} f(n) = \floor{x} = x + O(1) \sim x
\end{equation*}
So average order of $f$ is $1$.

Now take $f(n) = n$,
\begin{equation*}
  \sum_{1 \leq n \leq x} n \sim \frac{x^2}{2}
\end{equation*}
so order of $n$ is $\frac{n}{2}$.

\begin{nlemma}[Partial summation]\label{lem:1}
  If $(a_n)$ is a sequence of complex numbers and $f$ is such that $f'$ is continuous, then
  \begin{equation*}
    \sum_{1 \leq n \leq x} a_n f(n) = A(x) f(x) - \int_1^x A(t) f'(t) \, dt
  \end{equation*}
  where $A(x) = \sum_{1 \leq n \leq x} a_n$.
\end{nlemma}

\begin{proof}
  Suppose $x = N$ is an integer. Note that $a_n = A(n) - A(n-1)$.
  So
  \begin{align*}
    \sum_{1 \leq n \leq N} a_n f(n) &= \sum_{1 \leq n \leq N} f(n) (A(n) - A(n-1)) \\
    \shortintertext{(note $A(0) = 0$)}
    &= A(N) f(N) + \sum_{n=1}^{N-1} A(n) (f(n+1)-f(n)).
  \end{align*}
  Now
  \begin{align*}f(n+1)-f(n) = \int_{n}^{n+1} f'(t) \, dt.\end{align*}
  So
  \begin{align*}
    \sum_{1 \leq n \leq N} a_n f(n) &= A(N) f(N) - \sum_{n=1}^{N-1} f'(t) \, dt \\
                                    &= A(N) f(N) - \int_1^N A(t) f'(t) \, dt
  \end{align*}
  where we set $A(n) = A(t) \; \forall t \in [n, n+1)$ %]
  If $N > \floor{x}$, i.e.\ $x$ not an integer,
  \begin{align*}
    A(x) f(x) &= A(N) f(x) \\
              &= A(N) \left(f(N) + \int_N^x f'(t) \, dt\right). \qedhere
  \end{align*}
\end{proof}
\begin{nlemma}\label{lem:2}
  \begin{align*}
    \sum_{1 \leq n \leq x} \frac{1}{n} = \log x + \gamma + \bigO\left(\tfrac{1}{x}\right)
  \end{align*}
\end{nlemma}
\begin{proof}
  Partial summation with $f(x) = \frac{1}{x}$ and $a_n = 1$ so $A(x) = \floor{x}$. So
  \begin{align*}
    \sum_{1 \leq n \leq x} \frac{1}{n} &= \frac{\floor{x}}{x} + \int_1^x \frac{\floor{t}}{t^2}\,dt \\
                                       &= 1 + O\left(\tfrac{1}{x}\right) + \int_1^x \frac{1}{t}\,dt - \int_1^x \frac{\{t\}}{t^2}\,dt \\
                                       &= 1 + O\left(\tfrac{1}{x}\right) + \int_1^\infty \frac{\{t\}}{t^2} \, dt + \underbrace{\int_x^\infty \frac{\{t\}}{t^2}\,dt}_{\leq \int_x^\infty \frac{1}{t^2}\,dt \leq \frac{1}{x}} \\
                                       &=\gamma + O\left(\tfrac{1}{x}\right) + \log x + O\left(\tfrac{1}{x}\right) \\
                                       &= \log x + \gamma + O\left(\tfrac{1}{x}\right)
  \end{align*}
  where $\gamma = 1 - \int_1^\infty \frac{\{t\}}{t^2} \, dt$.
\end{proof}
This $\gamma$ is called Euler's constant (Euler-Mascheroni). $\gamma \approx 0.577\dots$ but we don't know if $\gamma$ is irrational or not.

\begin{nlemma}
  \begin{equation*}
    \sum_{1 \leq n \leq x} \log n = x \log x - x + O(\log x).
  \end{equation*}
\end{nlemma}
\begin{proof}
  Partial summation with $f(x) = \log x$, $a_n = 1$, $A(x) = \floor{x}$.
  \begin{align*}
    \sum_{1 \leq n \leq x} \log n &= \floor{x} \log x - \int_1^x \frac{\floor{t}}{t} \, dt \\
                                  &= x \log x + O(\log x) - \int_1^x 1 \, dt + \bigO\left(\int_1^x \tfrac{1}{t} \, dt\right) \\
                                  &= x \log x + O(\log x) - x + O(\log x) \\
                                  &= x \log x - x + O(\log x). \qedhere
  \end{align*}
\end{proof}
This is not really Number Theory - we haven't really used multiplication yet.
\subsection{Divisor function}
Recall that
\begin{equation*}\tau(n) = 1 \star 1 (n) = \sum_{ab \mid n} 1 = \sum_{d \mid n} 1\end{equation*}
We will analyse how many divisors an integer has.
\begin{nthm}\label{thm:4}
  \begin{align*}
    \sum_{1 \leq n \leq x} \tau(n) = x \log x + (2 \gamma - 1) x + O(x^{\frac{1}{2}})
  \end{align*}
  So average order of $\tau$ is $\log x$.
\end{nthm}
\begin{proof}
  Partial summation involves turning a sum $\sum a_n \rightsquigarrow \sum a_n f(n)$, but what does $\tau(\frac{1}{2})$ even mean?
  So there is no continuous function to use.

  Instead, play around with the definition
  \begin{align*}
    \sum_{1 \leq n \leq x} \tau(n) &= \sum_{1 \leq n \leq x} \sum_{d \mid x} 1 \\
                                   &= \sum_{1 \leq d \leq x} \sum_{\substack{1 \leq n \leq x \\ d \mid n}} 1
                                   \shortintertext{note that $\sum_{\substack{1 \leq n \leq x \\ d \mid n}} 1 = \floor{\frac{x}{d}}$}
                                   &= \sum_{1 \leq d \leq x} \floor*{\frac{x}{d}} = \sum_{1 \leq d \leq x} \frac{x}{d} + O(x) \\
                                   &= x \sum_{1 \leq d \leq x} \frac{1}{d} + O(x) \\
                                   &= x \log x + \gamma x + O(x)
  \end{align*}
  using \cref{lem:2}.
  To reduce the error term, we use (Dirichlet's) hyperbola trick.
  \begin{equation*}
    \sum \tau(n) = \sum_{1 \leq n \leq x} \sum_{a b = n} 1 = \sum_{ab \leq x} 1 = \sum_{a \leq x} \sum_{b \leq \frac xa} 1
  \end{equation*}
  % hyperbola picture
  When summing over $ab \leq x$, we can sum over $a \leq x^{\frac{1}{2}}$, $b \leq x^{\frac{1}{2}}$
  \begin{align*}
    \sum_{1 \leq n \leq x} \tau(n) &= \sum_{a \leq x^{\frac{1}{2}}} \sum_{b \leq \frac{x}{a}} 1 + \sum_{b \leq x^{\frac{1}{2}}} \sum_{a \leq \frac{x}{b}} 1 - \sum_{a,b \leq x^{\frac{1}{2}}} 1 \\
                                   &= 2 \sum_{a \leq x^{\frac{1}{2}}} \floor*{\frac{x}{a}} - \underbrace{\floor*{x^{\frac{1}{2}}}}_{\mathclap{=\left(x^{\frac{1}{2}} + o(1)\right)^2}} = 2 \sum_{a \leq x^{\frac{1}{2}}} \frac{x}{a} + O(x^{\frac{1}{2}}) - x + O(x^{\frac{1}{2}}) \\
                                   &= 2 x \log x^{\frac{1}{2}} + 2 \gamma x - x + O(x^{\frac{1}{2}}) \\
                                   &= x \log x + (2 \gamma - 1) x + O(x^{\frac{1}{2}}). \qedhere
  \end{align*}
\end{proof}
Analytic Number Theory is mostly just controlling the error term.
\begin{remark}
  Improving this $O(x^{\frac{1}{2}})$ error term is a famous and hard problem! Probably, $O(x^{\frac{1}{4} + \epsilon})$. Best know is $O(x^{0.3148})$.

  This does not mean that $\tau(n) = \log n$: the average order does not give any information about specific values.
\end{remark}
% lecture 3
\begin{nthm}\label{thm:5}
  For any $n \geq 1$,
  \begin{equation*}
    \tau(n) \leq n^{O\left(\frac{1}{\log \log n}\right)}
  \end{equation*}
  In particular,
  \begin{equation*}
    \tau(n) \ll_\epsilon n^\epsilon \; \forall \epsilon > 0
  \end{equation*}
  i.e.\ ($\forall e > 0$, $\exists C(\epsilon) > 0$ such that $\tau(n) \leq C n^\epsilon$).
\end{nthm}
\begin{proof}
  $\tau$ is multiplicative, so enough to calculate at prime powers. $\tau(p^k) = k+1$, so if $n = p_1^{k_1} \dotsm p_r^{k_r}$ then
  \begin{equation*}
    \tau(n) = \prod_{i=1}^r (k_i + 1).
  \end{equation*}
  Let $\epsilon > 0$ be chosen later and consider $\frac{\tau(n)}{n^\epsilon}$.
  \begin{equation*}
    \frac{\tau(n)}{n^\epsilon} = \prod_{i=1}^r \frac{k_i+1}{p^{k_i \epsilon}}.
  \end{equation*}
  Note as $p$ is large, $\frac{k+1}{p^{k\epsilon}} \to 0$. In particular, if $p \geq 2^{\frac{1}{\epsilon}}$, then $\frac{k+1}{p^{k\epsilon}} \leq \frac{k+1}{2^k} \leq 1$.

  What about small $p$? Can't do better than $p \geq 2$.
  In this case, $\frac{k+1}{p^{k\epsilon}} \leq \frac{k+1}{2^{k\epsilon}} \leq \frac{1}{\epsilon}$.
  Why? Rearrange to say $\epsilon k + \epsilon \leq 2^{k \epsilon}$ (if $\epsilon \leq \frac{1}{2}$), which follows from $x + \frac{1}{2} \leq 2^x \; \forall x \geq 0$

  So
  \begin{equation*}
    \frac{\tau(n)}{n^\epsilon} \leq \prod_{\substack{i=1 \\ p_i < 2^{\frac{1}{\epsilon}}}} \frac{k_i + 1}{p^{k_i \epsilon}} \leq \left(\frac{1}{\epsilon}\right)^{\pi(2^{\frac{1}{\epsilon}})} \leq \left(\frac{1}{\epsilon}\right)^{2^{\frac{1}{\epsilon}}}.
  \end{equation*}

  Now choose optimal $\epsilon$.
  (Trick: if you want to choose $x$ to minimise $f(x) + g(x)$, choose $x$ such that $f(x) = g(x)$).

  So have,
  \begin{equation*}
    \tau(n) \leq n^\epsilon \epsilon^{-2^{\frac{1}{\epsilon}}} = \exp\left(\epsilon \log n + 2^{\frac{1}{\epsilon}} \log \frac{1}{\epsilon}\right).
  \end{equation*}
  Choose $\epsilon$ such that $\log n \approx 2^{\frac{1}{\epsilon}}$, i.e.\ $\epsilon \approx \frac{1}{\log \log n}$.
  \begin{equation*}
    \tau(n) \leq n^{\frac{1}{\log \log n}} (\log \log n)^{2^{\log \log n}} = n^{\frac{1}{\log \log n}} e^{(\log n)^{\log 2} \log \log \log n} \leq n^{O(\frac{1}{\log \log n})}. \qedhere
  \end{equation*}
\end{proof}
\subsection{Estimates for the primes}
Recall
\begin{align*}
  \pi(x) &= \abs{\set{p \leq x}} = \sum_{1\leq n \leq x} 1_p(n)
  \shortintertext{and}
  \psi(x) = \sum_{1 \leq n \leq x} \Lambda(n).
\end{align*}
The Prime Number Theorem is $\pi(x) \sim \frac{x}{\log x}$ or equivalently $\psi(x) \sim x$.
It was 1850 before the correct magnitude of $\pi(x)$ was proved.
Chebyshev showed $p(x) \asymp \frac{x}{\log x}$, where ($f \asymp g$ means $g \ll f \ll g$).

\begin{nthm}[Chebyshev]\label{thm:6}
  \begin{equation*}
    \psi(x) \asymp x
  \end{equation*}
\end{nthm}
\begin{proof}
  First we'll prove the lower bound, i.e.\ that $\psi(x) \gg x$.
  \begin{equation*}
    \psi(x) = \sum_{n \leq x} \Lambda(n).
  \end{equation*}
  $x \log x$ is a trivial upper bound for this, we'd like to remove the factor of $\log x$.
  Recall $1 \star \Lambda = \log$, i.e.\ $\sum_{ab=n} \Lambda(a) = \log n$.
  The trick is to find a sum $\Sigma$ such that $\Sigma \leq 1$. We'll use the identity $\floor{x} \leq 2 \floor{\frac{x}{2}} + 1$ ($\forall x \geq 0$).
  (Say $\frac{x}{2} = n + \theta$, with $\theta \in [0,1)$, so $\floor{\frac{x}{2}} = n$ then $x = 2n + 2\theta$ so $\floor{x} = 2n$ or $2n+1$.)

  So
  \begin{align*}
    \psi(x) &\geq \sum_{n \leq x} \Lambda(n) \left(\floor{\tfrac{x}{n}} - 2 \floor{\tfrac{x}{2n}}\right) \\
    \shortintertext{Note $\floor{\frac{x}{n}} = \sum_{m \leq \frac{x}{n}} 1$}.
            &= \sum_{n \leq x} \Lambda(n) \sum_{m \leq \frac xn} 1 - 2 \sum_{n \leq x} \Lambda(n) \sum_{m \leq \frac x {2n}} 1 \\
            &= \sum_{n \leq x} \Lambda(n) - 2 \sum_{nm \leq \frac x2} \Lambda(n) \\
            &= \sum_{d \leq x} 1 \star \Lambda(d) - 2 \sum_{d \leq \frac x2} 1 \star \Lambda(d) \\
            &= \sum_{d \leq x} \log d - 2 \sum_{d \leq \frac{x}{2}} \log d \\
            &= x \log x - x + O(\log x) - 2 \left(\frac{x}{2} \log \frac{x}{2} - \frac{x}{2} + O(\log x)\right) \\
            &= (\log 2) x + O(\log x) \gg x.
  \end{align*}

  For the upper bound, note $\floor{x} = 2 \floor{\frac{x}{2}} + 1$ for $x \in (1,2)$ so
  \begin{equation*}
    \sum_{\frac{x}{2} < n < x} \Lambda(n) = \sum_{\frac{x}{2} < n < x} \Lambda(n) \left(\floor{\tfrac{x}{n}} - 2 \floor{\tfrac{x}{2n}}\right) \leq \sum_{1 \leq n \leq x} \Lambda(n) \left(\floor{\tfrac{x}{n}} - 2 \floor{\tfrac{x}{2n}}\right)
  \end{equation*}
  Thus
  \begin{align*}
    \psi(x) - \psi\left(\frac{x}{2}\right) &\leq (\log 2) x + \bigO(\log x). \\
    \psi(x) &= \left(\psi(x) - \psi\left(\frac{x}{2}\right)\right) + \left(\psi(\frac{x}{2}) - \psi(\frac{x}{4})\right) + \dotsb  \\
    &\leq \log 2\left(x + \frac{x}{2} + \frac{x}{4} + \dotsb\right) = 2 \log 2 x
  \end{align*}
  We ignored a $\bigO((\log x)^2)$ error term along the way.
\end{proof}
\begin{nlemma}\label{lem:7}
  \begin{equation*}
    \sum_{p \leq x} \frac{\log p}{p} = \log x + O(1).
  \end{equation*}
\end{nlemma}
\begin{proof}
  Recall $\log = 1 \star \Lambda$. So
  \begin{align*}
    \sum_{n \leq x} \log n &= \sum_{ab \leq x} \Lambda(a) = \sum_{a \leq x} \Lambda(a) \sum_{b \leq \frac{x}{a}} 1 = \sum_{a \leq x} \Lambda(a) \floor{\tfrac{x}{a}} = x \sum_{a \leq x} \frac{\Lambda(a)}{a} + \bigO(\psi(x)) \\
&= x \sum_{a \leq x} \frac{\Lambda(a)}{a} + \bigO(x) \\
&= x \log x - x + O(\log x) \\
    \text{So }\sum_{n \leq x} \frac{\Lambda(n)}{n} &= \log x - 1 + \mathcal{O}(\tfrac{\log x}{x}) + \bigO(1) = \log x + \bigO(1).
  \end{align*}
  Remains to note
  \begin{align*}
    \sum_{p \leq x} \sum_{n=2}^\infty \frac{\log p}{p^k} = \sum_{p \leq x} \log p \sum_{k=2}^\infty \frac{1}{p^k} = \sum_{p \leq x} \frac{\log p}{p^2 - p} \leq \sum_{p=2}^\infty \frac{1}{p^{\frac{3}{2}}} = \bigO(1).
  \end{align*}
  So  $\sum_{n \leq x} \frac{\Lambda(n)}n = \sum_{p \leq x} \frac{\log p}{p} + \bigO(1)$.
\end{proof}
\begin{nlemma}\label{lem:8}
  \begin{equation*}
    \pi(x) = \frac{\psi(x)}{\log x} + \bigO\left(\frac{x}{(\log x)^2}\right).
  \end{equation*}
\end{nlemma}
In particular, $\pi(x) \asymp \frac{x}{\log x}$ and PNT: $\pi(x) \sim \frac{x}{\log x}$ is equivaelnt to $\psi(x) \sim x$.
\begin{proof}
  Idea is to use partial summation:
  \begin{equation*}
    \theta(x) = \sum_{p \leq x} \log p = \pi(x) \log x - \int_1^x \frac{\pi(t)}{t} \, dt
  \end{equation*}
  whereas $\psi(x) = \sum_{n \leq x} \Lambda(n) = \sum_{p^k \leq x} \log p$.
  \begin{align*}
    \psi(x) - \theta(x) = \sum_{k=2}^\infty \sum_{p^k \leq x} \log p = \sum_{k=2}^\infty \theta(x^{\frac{1}{k}}) \leq \sum_{k=2}^{\log x} \psi(x^{\frac{1}{k}}) \ll \sum_{k=2}^{\log x} x^{\frac{1}{k}} \ll x^{\frac{1}{2}} \log x
  \end{align*}
  Thus,
  \begin{align*}
    \psi(x) &= \pi(x) \log x + O(x^{\frac{1}{2}} \log x) - \int_1^x \frac{\pi(t)}{t} \, dt \\
            &= \pi(x) \log x + O(x^{\frac{1}{2}}) + O(\int_1^x \frac{1}{\log t} \, dt) = \pi(x) \log x + O(\frac{x}{\log x})
  \end{align*}
  where we used the fact that $\pi(x) \ll \frac{t}{\log t}$:
  Trivially, $\pi(x) \leq t$, so
  \begin{equation*}
    \psi(x) = \pi(x) \log x + O(x^{\frac{1}{2}} \log x) + O(x)
  \end{equation*}
  so $\pi(x) \log x = O(x)$).
\end{proof}
\begin{nlemma}\label{lem:9}
  \begin{equation*}
    \sum_{p \leq x} \frac{1}{p} = \log \log x + b + O(\frac{1}{\log x})
  \end{equation*}
  where $b$ is some constant.
\end{nlemma}
\begin{proof}
  We use partial summation. Let $A(x) = \sum_{p \leq x} \frac{\log p}{p} = \log x + R(x)$ so $R(x) \ll 1$).
  \begin{equation*}.
    \sum_{2 \leq p\leq x} \frac{1}{p} = \frac{A(x)}{\log x} + \int_2^x \frac{A(t)}{t (\log t)^2} \, dt = 1 + O(\frac{1}{\log x}) + \int_2^x \frac{1}{t \log t} \, dt + \int_2^x \frac{R(t)}{t (\log t)^2} \, dt
  \end{equation*}
  Note $\int_2^\infty \frac{R(t)}{t (\log t)^2} \, dt$ exists, say it is $c$.
  \begin{align*}
    \sum_{2 \leq p \leq x} \frac{1}{p} &= 1 + c + O(\frac{1}{\log x}) + \log \log x - \log \log 2 + O\left(\int_x^\infty \frac{1}{t (\log t)^2}\right) \\
                                       &= \log \log x + b + O\left(\frac{1}{\log x}\right). \qedhere
  \end{align*}
\end{proof}
\begin{nthm}[Chebyshev]\label{thm:10}
  If
  \begin{equation*}
    \pi(x) \sim c \frac{x}{\log x}
  \end{equation*}
  then $c = 1$.
\end{nthm}
Chebyshev also showed if $\pi(x) \sim \frac{x}{\log x - A(x)}$ then $A \sim 1$, which was a surprise since it was believed $A \sim 1.08\dots$
\begin{proof}
  Partial summation on $\sum_{p \leq x} \frac{1}{p}$.
  \begin{align*}
    \sum_{p \leq x} &= \frac{1}{p} \sum_{\pi(x)}{x} + \int_1^x \frac{\pi(t)}{t^2} \, dt.
    \shortintertext{If $\pi(x) = (c + o(1)) \frac{x}{\log x}$ then }
                    &= \frac{c}{\log x} + o\left(\frac{1}{\log x}\right) + (c + o(1)) \int_1^x \frac{1}{t \log t} \, dt \\
                    &= \bigO\left(\frac{1}{\log x}\right) + (c+o(1)) \log \log x.
  \end{align*}
  But $\sum_{p \leq x} \frac{1}{p} = (1 + o(1)) \log \log x$. Hence $c = 1$.
\end{proof}
\begin{nlemma}\label{lem:11}
  \begin{equation*}
    \prod_{p \leq x} \left(1 - \frac{1}{p}\right)^{-1} = c \log x + \bigO(1)
  \end{equation*}
  where $c$ is some constant.
\end{nlemma}
\begin{proof}
  \begin{align*}
    \log \left(\prod_{p \leq x} \left(1 - \frac{1}{p}\right)^{-1}\right) &= - \sum_{p \leq x} \log \left(1 - \frac{1}{p}\right) \\
                                                                         &= \sum_{p \leq x} \sum_{k} \frac{1}{k p^k} \\
                                                                         &= \sum_{p \leq x} \frac 1p + \sum_{k \geq 2} \sum_{p \leq x} \frac{1}{k p^k} \\
                                                                         &= \log \log x + c' + O\left(\frac{1}{\log x}\right).
  \end{align*}
  Now note that $e^x = 1 + O(x)$ for $|x| \leq 1$.
  So
  \begin{align*}
    \prod_{p \leq x} \left(1 - \frac{1}{p}\right)^{-1} = c \log x\ e^{O(\frac{1}{\log x})} &= c \log x\ (1 + O(\frac{1}{\log x})) \\
                                                                                            &= c \log x + O(1). \qedhere
  \end{align*}
\end{proof}
It turns out that $c = e^\gamma = 1.78\dots$

Probabilistic heuristic: `probability' that $p \mid n$ is $\frac{1}{p}$.
What is the `probability' that $n$ is prime?
\begin{equation*}
  n\text{ is prime} \iff n\text{ has no prime divisors } p \leq n^{\frac{1}{2}}.
\end{equation*}
Guess is that the events `divisible by $p$' are independent so $\mathbb{P}(p \nmid n) = 1 - \frac{1}{p}$.
\begin{equation*}
  \mathbb{P}(n \text{ is prime}) \approx \prod_{p \leq n^{\frac{1}{2}}} \left(1 - \frac{1}{p}\right) \approx \frac{1}{c \log n^{\frac{1}{2}}} = \frac{2}{c} \frac{1}{\log n}.
\end{equation*}
So $\pi(x) = \sum_{n \leq x} 1_{n \text{ prime}} \approx \frac{2}{c} \sum_{n \leq x} \frac{1}{\log n} \approx \frac{2}{c} \frac{x}{\log x} \approx 2 e^{-\gamma} \frac{x}{\log x}$.

$2 e^{-\gamma} \approx 1.122\dots$, so this heuristic says there are around 12\% more primes than there are.

The PNT is hard.
Recall that $1 \star \Lambda = \log$ so $\mu \star \log = \Lambda$.
So
\begin{equation*}
  \psi(x) = \sum_{n \leq x} \Lambda(n) = \sum_{ab \leq x} \mu(a) \log b = \sum_{a \leq x} \mu(a) \left(\sum_{b \leq \frac{x}{a}} \log b\right).
\end{equation*}
Recall that
\begin{align*}
  \sum_{m \leq x} \log m &= x \log x - x + O(\log x) \\
  \sum_{m \leq x} \tau(m) &= x \log x + (2 \gamma - 1) x + O(x^{\frac{1}{2}})
\end{align*}
Thus
\begin{align*}
  \psi(x) &= \sum_{a \leq x} \mu(a) \left(\sum_{b \leq \frac{x}{a}} \tau(b) - 2 \gamma \frac{x}{a} + O\left(\frac{x^{\frac{1}{2}}}{a^{\frac{1}{2}}}\right)\right) \\
          &= \sum_{ab \leq x} \mu(a) \tau(b) = \sum_{abc \leq x} \mu(a) = \sum_{b \leq x} \sum_{ac \leq \frac{x}{b}} \mu(a) = \sum_{b \leq x} \sum_{d \leq \frac{x}{b}} \mu \star 1(d) \\
          &= \floor{x} = x + O(1).
\end{align*}
This gives an error term of
\begin{equation*}
  -2 \gamma \sum_{a \leq x} \mu(a) \frac{x}{a} = \bigO(x \sum_{a \leq x} \frac{\mu(a)}{a})
\end{equation*}
so still need to show that $\sum_{a \leq x} \frac{\mu(a)}{a} = o(1)$.
\clearpage
\printnomenclature
\printindex
\end{document}
