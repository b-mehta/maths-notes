\documentclass{article}

% preamble
\def\npart{III}
\def\nyear{2019}
\def\nterm{Lent}
\def\draft{Unfinished course}
\def\nlecturer{Dr T.\ Bloom}
\def\ncourse{Analytic Number Theory}

\usepackage{imakeidx}
\usepackage{marginnote}
\usepackage[intoc, refpage]{nomencl}

\ifx \nauthor\undefined
  \def\nauthor{Bhavik Mehta}
\else
\fi

\author{Based on lectures by \nlecturer \\\small Notes taken by \nauthor}
\date{\nterm\ \nyear}
\title{Part \npart\ -- \ncourse}

\usepackage[utf8]{inputenc}
\usepackage{amsmath}
\usepackage{amsthm}
\usepackage{amssymb}
\usepackage{enumerate}
\usepackage{mathtools}
\usepackage{graphicx}
\usepackage[dvipsnames]{xcolor}
\usepackage{tikz}
\usepackage{wrapfig}
\usepackage{centernot}
\usepackage{float}
\usepackage{braket}
\usepackage[hypcap=true]{caption}
\usepackage{enumitem}
\usepackage[colorlinks=true, linkcolor=mblue]{hyperref}
\usepackage[nameinlink,noabbrev]{cleveref}
\usepackage{nameref}
\usepackage[margin=1.5in]{geometry}

% Theorems
\theoremstyle{definition}
\newtheorem*{aim}{Aim}
\newtheorem*{axiom}{Axiom}
\newtheorem*{claim}{Claim}
\newtheorem*{cor}{Corollary}
\newtheorem*{conjecture}{Conjecture}
\newtheorem*{defi}{Definition}
\newtheorem*{eg}{Example}
\newtheorem*{ex}{Exercise}
\newtheorem*{fact}{Fact}
\newtheorem*{law}{Law}
\newtheorem*{lemma}{Lemma}
\newtheorem*{notation}{Notation}
\newtheorem*{prop}{Proposition}
\newtheorem*{question}{Question}
\newtheorem*{rrule}{Rule}
\newtheorem*{thm}{Theorem}
\newtheorem*{assumption}{Assumption}

\newtheorem*{remark}{Remark}
\newtheorem*{warning}{Warning}
\newtheorem*{exercise}{Exercise}

% \newcommand{\nthmautorefname}{Theorem}

\newtheorem{nthm}{Theorem}[section]
\newtheorem{nlemma}[nthm]{Lemma}
\newtheorem{nprop}[nthm]{Proposition}
\newtheorem{ncor}[nthm]{Corollary}
\newtheorem{ndef}[nthm]{Definition}

% Special sets
\newcommand{\C}{\mathbb{C}}
\newcommand{\N}{\mathbb{N}}
\newcommand{\Q}{\mathbb{Q}}
\newcommand{\R}{\mathbb{R}}
\newcommand{\Z}{\mathbb{Z}}

\newcommand{\abs}[1]{\left\lvert #1\right\rvert}
\newcommand{\norm}[1]{\left\lVert #1\right\rVert}
\renewcommand{\vec}[1]{\boldsymbol{\mathbf{#1}}}

\let\Im\relax
\let\Re\relax

\DeclareMathOperator{\Im}{Im}
\DeclareMathOperator{\Re}{Re}
\DeclareMathOperator{\id}{id}

\definecolor{mblue}{rgb}{0., 0.05, 0.6}


\reversemarginpar

\makeindex[intoc]

\makenomenclature
\renewcommand{\pagedeclaration}[1]{, \hyperlink{page.#1}{#1}}
\renewcommand{\nomname}{Index of Notation}

\renewcommand{\nompreamble}{\begin{multicols}{2}}
\renewcommand{\nompostamble}{\end{multicols}}

\newcommand{\named}[1]{\textbf{#1}\index{#1}}
\newcommand{\bonusnamed}[1]{\textbf{#1}\index{#1@*#1}}
\newcommand{\bigO}{\mathcal{O}}

\DeclarePairedDelimiter\floor{\lfloor}{\rfloor}

\setcounter{section}{-1}

% and here we go!
\begin{document}
\maketitle

\tableofcontents

\clearpage
\section{Introduction}
Analytic \marginnote{\emph{Lecture 1}}[0cm]Number Theory is the study of numbers using analysis. It is a fascinating field because because a number - in particular in this course an integer - is discrete, whilst analysis involves the real/complex numbers which are continuous.

In this course, we will ask quantitative questions.

\begin{eg}\leavevmode
  \begin{enumerate}
    \item How many primes? We can define the function $\pi(x) = |\{n \mid n \leq x \text{ and } n \text{ is prime} \}|$. Then the prime number theorem, which we will prove in this course states $$\pi(x) \sim \frac{x}{\log x}. $$
      (We will always take `numbers' to mean natural numbers, not including zero).
    \item  How many twin primes are there? That is, where $p,\,p+2$ are both prime.
      It is not known whether there are infinitely many but since 2014, there has been immense progress by Zhang, Maynard and a Polymath project which has determined there are infinitely many primes at most 246 apart.
      Guess: there are $\approx \frac{x}{(\log x)^2}$ many $\leq x$.
    \item How many primes are there $\equiv a \bmod q $ where $(a,q) = 1$. We know, by Dirichlet's theorem proven in the 20th century, that there are infinitely many such.
      The guess for how many is
      \begin{equation*}
        \frac{1}{\varphi(q)} \frac{x}{\log x}.
      \end{equation*}
      This is known for small $q$. Recall $\varphi(n) = |\{1\leq m \leq n \ | \ (m,n) =1 \}|$
  \end{enumerate}
\end{eg}
The course will be split up into 4 (roughly equal) parts
\begin{enumerate}
  \item Elementary techniques (real analysis)
  \item Sieve methods
  \item Riemann Zeta function, Prime Number Theorem (complex analysis)
  \item Primes in arithmetic progressions
\end{enumerate}
\clearpage

\section{Elementary Techniques}
We begin with a review of asymptotic notations:
\begin{itemize}\hypertarget{def:asymp}
  \item \nomenclature{$\bigO$}{Big $\bigO$ notation; Landau notation}$f(x) = \bigO(g(x))$ if there is $C>0$ such that $|f(x)| \leq C|g(x)|$ for all large enough $x$. (Landau notation)
  \item \nomenclature{$\ll$}{Vinogradov notation}$f \ll g$ is the same as $f = \bigO(g)$ (Vinogradov notation)
  \item \nomenclature{$\sim$}{asymptotic equality}$f\sim g$ if $\lim_{x \to \infty}\frac{f(x)}{g(x)} = 1 $ (i.e.\ $f = (1+o(1))g$).
  \item \nomenclature{$o$}{Little $o$ notation}$f= o(g)$ if $\lim_{x \to \infty}\frac{f(x)}{g(x)} = 0$
\end{itemize}

\subsection{Arithmetic Functions}
\begin{defi}[Arithmetic function]\hypertarget{def:arith}
  An \named{arithmetic function} is just a function $f : \mathbb{N} \to \mathbb{C}$.
\end{defi}
\begin{defi}[Convolution]\hypertarget{def:conv}
  An important operation for multiplicative number theory is the \textbf{multiplicative convolution\index{convolution}}
  \nomenclature{$*$}{convolution}
  \begin{equation*}f*g(n) \coloneqq \sum_{ab = n} f(a)g(b).\end{equation*}
\end{defi}
\begin{eg}\leavevmode
  \begin{itemize}
    \item \hypertarget{def:1}{$1(n) \coloneqq 1 \; \; \forall n$}. Caution: $1*f \neq f$.
    \item \hypertarget{def:mu}M\"obius function:\index{M\"obius function}\nomenclature{$\mu$}{M\"obius function}
      \begin{equation*}
        \mu(n) =
        \begin{cases*}
          (-1)^k & if $n = p_1\dotsm p_k$ \\
          0 & if $n$ not squarefree
        \end{cases*}
      \end{equation*}
    \item \hypertarget{def:lamb}Liouville function:\index{Liouville function}\nomenclature{$\lambda$}{Liouville function}
      \begin{equation*}
        \lambda(n) = (-1)^k \text{ if } n = p_1\dotsm p_k, \text{ not necessarily distinct}
      \end{equation*}
    \item \hypertarget{def:tau}Divisor function:\index{divisor function}\nomenclature{$\tau$}{divisor function}
      \begin{align*}
        \tau(n) &= | \{ d \mid d\text{ a factor of }n \} | \\
        \tau &= 1 \hyperlink{def:conv}{*}1
      \end{align*}
  \end{itemize}
\end{eg}
\begin{defi}[Multiplicative function]\hypertarget{def:multi}
  An \hyperlink{def:arith}{arithmetic function} is a \named{multiplicative function} if $f(nm) = f(n)f(m)$ for $(n,m) =1$.
  In particular, a multiplicative function is determined by its values on prime powers $f(p^k)$.
\end{defi}
\begin{fact}
  If $f,g$ are \hyperlink{def:multi}{multiplicative}, then so is $\hyperlink{def:conv}{f*g}$.
  $\log n $ is not \hyperlink{def:multi}{multiplicative}.
  Note, almost all \hyperlink{def:arith}{arithmetic functions} are not multiplicative.
\end{fact}
\begin{lemma}[\hypertarget{def:mobinv}M\"obius inversion]\index{M\"obius function}
  \begin{equation*}
    \hyperlink{def:1}{1} \hyperlink{def:conv}{*} f = g \iff \hyperlink{def:mu}{\mu} * g = f.
  \end{equation*}
\end{lemma}

\begin{proof}
  \begin{equation*}
    \sum_{d \mid n} \hyperlink{def:mu}{\mu(d)} =
    \begin{cases*}
      1 & if $n=1$ \\
      0 & otherwise
    \end{cases*}
  \end{equation*}
  Note the left hand side is $1\hyperlink{def:conv}{*}\mu$.
  Since $1,\mu$ are \hyperlink{def:multi}{multiplicative}, $1*\mu$ is multiplicative.
  Hence it is enough to check the identity for prime powers:
  If $n= p^k$, then $\{d \mid d\text{ divides }n\} = \{1,p,\ldots,p^k\}$ so the left hand side is $1-1 + 0 + \ldots + 0 = 0$, unless $k=0$ when the left hand side is $\mu(1) =1$.

  The right hand side is the identity of \hyperlink{def:conv}{convolution}, and convolution is associative, giving the required result.
\end{proof}

Our ultimate goal is to study the primes. This would suggest that we should work with
\begin{equation*}
  1_p(n) = \begin{cases*}
    1 &if $n$ prime \\
    0 & otherwise
  \end{cases*}
\end{equation*}
For example $\pi(x) = \displaystyle \sum_{1\leq n\leq x} 1_p(n)$.
This is an awkward function to work with.
Instead, we work with the \nomenclature{$\Lambda$}{von Mangoldt function}\named{von Mangoldt function}
\begin{equation*}
  \hypertarget{def:vonMang}\Lambda(n) =
  \begin{cases*}
    \log p & if $n$ is a prime power\\
    0 & otherwise.
  \end{cases*}
\end{equation*}
This function is easier to understand. Why?
\begin{nlemma}
  \begin{equation*}
    \hyperlink{def:1}{1} \hyperlink{def:conv}{*} \hyperlink{def:vonMang}{\Lambda} = \log \quad \text{and} \quad \hyperlink{def:mu}{\mu} * \log = \Lambda
  \end{equation*}
\end{nlemma}
\begin{proof}
  The second part follows immediately by \hyperlink{def:mobinv}{M\"obius inversion}.
  \begin{align*}
    1*\hyperlink{def:vonMang}{\Lambda}(n) & = \sum_{d \mid n}\Lambda(d) \quad \text{so if } n = p_1^{k_1}\ldots p_k^{n_k} \\
                 &= \sum_{i=1}^r \sum_{j=1}^{k_i} \Lambda(p_i^j) \\
                 &= \sum_{i=1}^r \sum_{j=1}^{k_i} \log p_i \\
                 &= \sum_{i=1}^{r}k_i\log p_i
                 = \sum_{i=1}^{r}\log p_i^{k_i}
                 = \log n. \qedhere
  \end{align*}
\end{proof}
We can write
\begin{align*}
  \Lambda(n) &= \sum_{d|n}\mu(d)\log\left(\frac{n}{d}\right) \\
             &= \log n \sum_{d|n}\mu(d) - \sum_{d|n}\mu(d)\log d \\
             &= - \sum_{d|n}\mu(d)\log d.
\end{align*}

\begin{eg}
  \begin{align*}
    \sum_{\mathclap{1\leq n \leq x}}\Lambda(n) &= -\sum_{1\leq n\leq x}\sum_{d|n}\mu(d)\log d \\
                                    &= - \sum_{d\leq x}\mu(d)\log(d)\left(\sum_{\substack{1\leq n\leq x \\ d|n}}1\right) \\
                                    &= - x \sum_{d\leq x}\mu(d) \frac{\log d}{d} + \hyperlink{def:asymp}{O}\left(\sum_{d\leq x}\mu(d)\log d \right)
  \end{align*}
  since
  \begin{equation*}
    \sum_{\mathclap{\substack{1\leq n\leq x \\ d|n}}}1 = \floor*{\frac{x}{d}}= \frac{x}{d} + O(1).
  \end{equation*}
\end{eg}
\clearpage
\printnomenclature
\printindex
\end{document}
