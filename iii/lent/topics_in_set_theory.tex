\documentclass{article}

% preamble
\def\npart{III}
\def\nyear{2019}
\def\nterm{Lent}
\def\draft{Unfinished course}
\def\nlecturer{Professor B. L\"owe}
\def\ncourse{Topics in Set Theory}

\usepackage{mathrsfs}
\usepackage{imakeidx}
\usepackage{marginnote}
\usepackage{mathdots}
\usepackage{tabularx}

\ifx \nauthor\undefined
  \def\nauthor{Bhavik Mehta}
\else
\fi

\author{Based on lectures by \nlecturer \\\small Notes taken by \nauthor}
\date{\nterm\ \nyear}
\title{Part \npart\ -- \ncourse}

\usepackage[utf8]{inputenc}
\usepackage{amsmath}
\usepackage{amsthm}
\usepackage{amssymb}
\usepackage{enumerate}
\usepackage{mathtools}
\usepackage{graphicx}
\usepackage[dvipsnames]{xcolor}
\usepackage{tikz}
\usepackage{wrapfig}
\usepackage{centernot}
\usepackage{float}
\usepackage{braket}
\usepackage[hypcap=true]{caption}
\usepackage{enumitem}
\usepackage[colorlinks=true, linkcolor=mblue]{hyperref}
\usepackage[nameinlink,noabbrev]{cleveref}
\usepackage{nameref}
\usepackage[margin=1.5in]{geometry}

% Theorems
\theoremstyle{definition}
\newtheorem*{aim}{Aim}
\newtheorem*{axiom}{Axiom}
\newtheorem*{claim}{Claim}
\newtheorem*{cor}{Corollary}
\newtheorem*{conjecture}{Conjecture}
\newtheorem*{defi}{Definition}
\newtheorem*{eg}{Example}
\newtheorem*{ex}{Exercise}
\newtheorem*{fact}{Fact}
\newtheorem*{law}{Law}
\newtheorem*{lemma}{Lemma}
\newtheorem*{notation}{Notation}
\newtheorem*{prop}{Proposition}
\newtheorem*{question}{Question}
\newtheorem*{rrule}{Rule}
\newtheorem*{thm}{Theorem}
\newtheorem*{assumption}{Assumption}

\newtheorem*{remark}{Remark}
\newtheorem*{warning}{Warning}
\newtheorem*{exercise}{Exercise}

% \newcommand{\nthmautorefname}{Theorem}

\newtheorem{nthm}{Theorem}[section]
\newtheorem{nlemma}[nthm]{Lemma}
\newtheorem{nprop}[nthm]{Proposition}
\newtheorem{ncor}[nthm]{Corollary}
\newtheorem{ndef}[nthm]{Definition}

% Special sets
\newcommand{\C}{\mathbb{C}}
\newcommand{\N}{\mathbb{N}}
\newcommand{\Q}{\mathbb{Q}}
\newcommand{\R}{\mathbb{R}}
\newcommand{\Z}{\mathbb{Z}}

\newcommand{\abs}[1]{\left\lvert #1\right\rvert}
\newcommand{\norm}[1]{\left\lVert #1\right\rVert}
\renewcommand{\vec}[1]{\boldsymbol{\mathbf{#1}}}

\let\Im\relax
\let\Re\relax

\DeclareMathOperator{\Im}{Im}
\DeclareMathOperator{\Re}{Re}
\DeclareMathOperator{\id}{id}

\definecolor{mblue}{rgb}{0., 0.05, 0.6}

\swapnumbers
\reversemarginpar

\makeindex[intoc]

\newcommand{\named}[1]{\textbf{#1}\index{#1}}
\newcommand{\bonusnamed}[1]{\textbf{#1}\index{#1@*#1}}

\setcounter{section}{-1}

% and here we go!
\begin{document}
\maketitle

\tableofcontents

\clearpage
\section{Introduction}
The \marginnote{\emph{Lecture 1}}[0cm]main `topic in set theory' covered in this course will be one of the most important: solving the Continuum Problem.
A priori, set theory does not seem intrinsically related to logic, but the continuum hypothesis showed that logic was a very important tool in set theory.
In contrast to many other disciplines of mathematics, in set theory we typically try to prove things are \emph{impossible}, rather than showing what is possible.

The second international congress of mathematicians in 1900 was in Paris, where Hilbert spoke. At that time, Hilbert was a `universal' mathematician, and had worked in every major field of mathematics. He gave a list of problems for the century, the 23 Hilbert Problems. The first on this list was the Continuum Problem.

\subsection{Continuum Hypothesis}\index{Continuum hypothesis}
Here is Hilbert's formulation of the Continuum hypothesis (CH):
Every set of infinitely many real numbers is either equinumerous with the set of natural numbers or the set of real numbers.
More formally, we might write
\begin{equation*}
  \forall X \subseteq \mathbb{R}, (X \text{ is infinite} \Rightarrow X \sim \mathbb{N} \text{ or } X \sim \mathbb{R})
\end{equation*}

In more modern terms, we write this as the claim $2^{\aleph_0} = \aleph_1$.
These two statements are equivalent (in ZFC).

Assume that $2^{\aleph_0} > \aleph_1$, in particular $2^{\aleph_0} \geq \aleph_2$. Since $2^{\aleph_0} \sim \mathbb{R}$, we get an injection $i: \aleph_2 \to \mathbb{R}$.
Consider $X \coloneqq i[\aleph_1] \subseteq \mathbb{R}$. Clearly, $i|_{\aleph_1}$ is a bijection between $\aleph_1$ and $X$, so $X \sim \aleph_1$.
But $\aleph_1 \nsim \mathbb{N}$ and $\aleph_1 \nsim \mathbb{R}$.
Thus $X$ refutes CH (in its earlier formulation).
So: $2^{\aleph_0} \neq \aleph_1 \implies \neg \text{CH}$.

If $2^{\aleph_0} = \aleph_1$.
Let $X \subseteq \mathbb{R}$. Consider $b: 2^{\aleph_0} \to \mathbb{R}$ a bijection. If $X$ is infinite, then $b^{-1}[X] \subseteq 2^{\aleph_0}$.
Thus the cardinality of $X$ is either $\aleph_0$, i.e.\ $X \sim \mathbb{N}$ or $\aleph_1$, i.e.\ $X \sim \mathbb{R}$.
So, $2^{\aleph_0} = \aleph_1 \implies \text{CH}$.

\subsection{History of CH}
\begin{itemize}[label=--]
  \item 1938, G\"odel: ZFC does not prove $\neg$CH.
  \item 1963, Cohen: ZFC does not prove CH.
\end{itemize}
G\"odel's proof used the technique of inner models; Cohen's proof used forcing, sometimes referred to as outer models.

G\"odel's Completeness Theorem:
\begin{equation*}
  \text{Cons}(T) \iff \exists(M, E) (M, E) \models T
\end{equation*}
From this, we might guess that G\"odel's and Cohen's proof will show there is a model of ZFC + CH, and a model of ZFC + $\neg$CH, but by the incompleteness phenomenon, we cannot prove there is a model of ZFC!
So, we are not going to be able to prove Cons(ZFC+CH), but instead
\begin{equation*}
  \text{Cons(ZFC)} \rightarrow \text{Cons(ZFC+CH)}
\end{equation*}
Or, equivalently,
\begin{equation*}
  \text{if } M \models \text{ZFC, then there is } N \models \text{ZFC + CH}.
\end{equation*}

\clearpage
\section{Model theory of set theory}
Let's assume for a moment that
\begin{equation*}
  (M,\epsilon) \models ZFC.
\end{equation*}
We refer to the canonical objects in $M$ by the usual symbols, e.g., $0,1,2,3,4,\dotsc, \omega, \omega+1, \dotsc$

What would an ``inner model'' be?
Take $A \subseteq M$, and consider $(A, \epsilon)$. This is a substructure of $(M, \epsilon)$.

Note: the language of set theory has no function or constant symbols. But we write down
\begin{equation*}
  X = \emptyset,\ X = \{Y\},\ X = \{Y,Z\},\ X = \bigcup Z,\ X = \mathcal{P}(Z)
\end{equation*}
which appear to use function or constant symbols. These are technically not part of the language of set theory; they are abbreviations:
\begin{align*}
  X &= \emptyset & &\text{ abbreviates } & &\forall w (\neg w \in X) \\
  X &= \{Y\} & & \text{ abbreviates } & &\forall w (w \in X \leftrightarrow w = Y) \\
  X &\subseteq Y & & \text{ abbreviates } & &\forall w (w \in X \rightarrow w \in Y)
\end{align*}
and so on.

\begin{defi}\hypertarget{def:abso}
  If $\varphi$ is a formula in $n$ free variables. We say
  \begin{enumerate}[label=(\arabic*)]
    \item $\varphi$ is \named{upwards absolute} between $A$ and $M$ if
      \begin{equation*}
        \text{for all } a_1, \dotsc, a_n \in A,\quad (A,\epsilon) \models \varphi(a_1, \dotsc, a_n) \implies (M, \epsilon) \models \varphi(a_1, \dotsc, a_n)
      \end{equation*}
    \item $\varphi$ is \named{downwards absolute} between $A$ and $M$ if
      \begin{equation*}
        \text{for all } a_1, \dotsc, a_n \in A,\quad (M,\epsilon) \models \varphi(a_1, \dotsc, a_n) \implies (A, \epsilon) \models \varphi(a_1, \dotsc, a_n)
      \end{equation*}
    \item $\varphi$ is \named{absolute} between $A$ and $M$ if it is upwards absolute and downwards absolute.
  \end{enumerate}
\end{defi}

\begin{defi}\hypertarget{def:sigmaPi}
      We say that a formula is $\Sigma_1$ if it is of the form
      \begin{equation*}
        \exists x_1 \dotsc \exists x_n\ \varphi(x_1, \dotsc, x_n) \text{ where } \varphi \text{ is quantifier-free}
      \end{equation*}
      or $\Pi_1$ if it is of the form
      \begin{equation*}
        \forall x_1 \dotsc \forall x_n\ \varphi(x_1, \dotsc, x_n) \text{ where } \varphi \text{ is quantifier-free}
      \end{equation*}
\end{defi}

\begin{remark}
  \leavevmode
  \begin{enumerate}[label=(\alph*)]
    \item If $\varphi$ is quantifier-free, then $\varphi$ is \hyperlink{def:abso}{absolute} between $A$ and $M$.
    \item If $\varphi$ is \hyperlink{def:sigmaPi}{$\Pi_1$}, then it's downward absolute
    \item If $\varphi$ is \hyperlink{def:sigmaPi}{$\Sigma_1$}, then it's upward absolute
  \end{enumerate}
\end{remark}
\printindex
\end{document}
