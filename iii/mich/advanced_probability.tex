\documentclass{article}

\def\npart{III}
\def\nyear{2019}
\def\nterm{Michaelmas}
\def\nlecturer{Dr. S. Andres}
\def\ncourse{Advanced Probability}
\def\draft{Incomplete}

% preamble
\usepackage{imakeidx}
\usepackage{marginnote}
\usepackage{chngcntr}
\usepackage{bbm}
\usepackage{xfrac}

\ifx \nauthor\undefined
  \def\nauthor{Bhavik Mehta}
\else
\fi

\author{Based on lectures by \nlecturer \\\small Notes taken by \nauthor}
\date{\nterm\ \nyear}
\title{Part \npart\ -- \ncourse}

\usepackage[utf8]{inputenc}
\usepackage{amsmath}
\usepackage{amsthm}
\usepackage{amssymb}
\usepackage{enumerate}
\usepackage{mathtools}
\usepackage{graphicx}
\usepackage[dvipsnames]{xcolor}
\usepackage{tikz}
\usepackage{wrapfig}
\usepackage{centernot}
\usepackage{float}
\usepackage{braket}
\usepackage[hypcap=true]{caption}
\usepackage{enumitem}
\usepackage[colorlinks=true, linkcolor=mblue]{hyperref}
\usepackage[nameinlink,noabbrev]{cleveref}
\usepackage{nameref}
\usepackage[margin=1.5in]{geometry}

% Theorems
\theoremstyle{definition}
\newtheorem*{aim}{Aim}
\newtheorem*{axiom}{Axiom}
\newtheorem*{claim}{Claim}
\newtheorem*{cor}{Corollary}
\newtheorem*{conjecture}{Conjecture}
\newtheorem*{defi}{Definition}
\newtheorem*{eg}{Example}
\newtheorem*{ex}{Exercise}
\newtheorem*{fact}{Fact}
\newtheorem*{law}{Law}
\newtheorem*{lemma}{Lemma}
\newtheorem*{notation}{Notation}
\newtheorem*{prop}{Proposition}
\newtheorem*{question}{Question}
\newtheorem*{rrule}{Rule}
\newtheorem*{thm}{Theorem}
\newtheorem*{assumption}{Assumption}

\newtheorem*{remark}{Remark}
\newtheorem*{warning}{Warning}
\newtheorem*{exercise}{Exercise}

% \newcommand{\nthmautorefname}{Theorem}

\newtheorem{nthm}{Theorem}[section]
\newtheorem{nlemma}[nthm]{Lemma}
\newtheorem{nprop}[nthm]{Proposition}
\newtheorem{ncor}[nthm]{Corollary}
\newtheorem{ndef}[nthm]{Definition}

% Special sets
\newcommand{\C}{\mathbb{C}}
\newcommand{\N}{\mathbb{N}}
\newcommand{\Q}{\mathbb{Q}}
\newcommand{\R}{\mathbb{R}}
\newcommand{\Z}{\mathbb{Z}}

\newcommand{\abs}[1]{\left\lvert #1\right\rvert}
\newcommand{\norm}[1]{\left\lVert #1\right\rVert}
\renewcommand{\vec}[1]{\boldsymbol{\mathbf{#1}}}

\let\Im\relax
\let\Re\relax

\DeclareMathOperator{\Im}{Im}
\DeclareMathOperator{\Re}{Re}
\DeclareMathOperator{\id}{id}

\definecolor{mblue}{rgb}{0., 0.05, 0.6}

\makeindex[intoc]
\reversemarginpar

\newtheorem{nremark}[nthm]{Remark}

\newcommand{\named}[1]{\textbf{#1}\index{#1}}

\newcommand{\F}{\mathcal{F}}
\newcommand{\G}{\mathcal{G}}
\newcommand{\1}[1]{\mathbbm{1}_{#1}}
\newcommand{\Prob}{\mathbb{P}}
\newcommand{\E}{\mathbb{E}}

% and here we go!

\begin{document}
\maketitle

\tableofcontents

\clearpage
\newlec
\section{Conditional Expectations}
\newlec
Take a \named{probability space} $(\Omega, \mathcal{F}, \mathbb{P})$, meaning $\mathcal{F}$ is a $\sigma$-algebra and $\mathbb{P}$ is a probability measure, with $\mathbb{P}(\Omega) = 1$.
\hypertarget{def:as}We use the term `\named{almost surely}' (or a.s.) to mean almost everywhere.

\hypertarget{def:e}Take $X$ to be a random variable, i.e.\ $X: \Omega \to \mathbb{R}$ which is $\mathcal{F}$-measurable and write
\begin{equation*}
  \mathbb{E}[X] = \int X\, d\mathbb{P}
\end{equation*}
for the \named{expectation} of $X$.
We write also
\begin{equation*}
  \mathbb{E}[X \1A] = \int_A X \, d\mathbb{P}
\end{equation*}
for $A \in \mathcal{F}$.
\begin{ndef}
  Let $B \in \mathcal{F}$ with $\mathbb{P}[B] > 0$. We know
  \begin{equation*}
    \mathbb{P}[A | B] = \frac{\mathbb{P}[A \cap B]}{\mathbb{P}[B]},
  \end{equation*}
  the \named{conditional probability} of $A$ given $B$.
  Similarly,
  \begin{equation*}
    \mathbb{E}[X | B] = \frac{\mathbb{E}[X \1B]}{\mathbb{P}[B]}
  \end{equation*}
  the \named{conditional expectation} of $X$ given $B$.
\end{ndef}
There is a significant restriction to this definition: that $\mathbb{P}[B] > 0$. By the end of this lecture, we will generalise this definition to any $\sigma$-algebra of events, rather than just one.

\begin{aim}
  Improve the prediction of $X$ if additional information (given as a sub-$\sigma$-algebra $\mathcal{G} \subseteq \mathcal{F}$) is available.
\end{aim}

\subsection{Discrete case}
Take $B_1, B_2, \dotsc \in \mathcal{F}$ a disjoint decomposition of $\Omega$.
We take
\begin{equation*}
  \mathcal{G} = \sigma(B_1, B_2, \dotsc) = \left\{\bigcup_{i \in J} B_i : J \subseteq \mathbb{N}\right\} \subseteq \F.
\end{equation*}
That is, the `extra information' of $\G$ is that we know which of the disjoint events $B_i$ we fall into.

Then,
\begin{equation*}
  \E[X | \G] (\omega) \coloneqq \sum_{i : \Prob[B_i] > 0} \E[X | B_i] \1{B_i}(\omega)
\end{equation*}
is the conditional expectation of $X$ given $\G$.

It is easy to see that $\E[X | \G]$ is a $\G$-measurable random variable, and
\begin{equation*}
  \E[\1{A} X] = \E[\1{A} \E[X|\G]] \quad \forall A \in \G
.\end{equation*}

\begin{eg}\leavevmode
  \begin{enumerate}[(i)]
    \item
      Take now $\Omega = (0, 1]$, and $\F = \mathcal{B} (\Omega)$, and $\Prob$ to be Lebesgue measure.
      Use $X$ as shown below, and use
      \begin{equation*}
        \G = \sigma\left(\left(\tfrac{k}{m}, \tfrac{k+1}m\right] : k = 0, \dotsc, m-1\right)
      .\end{equation*}
      In the picture, we take $m=8$, and the conditional expectation $\E(X | \G)$ is shown.
      \begin{center}
        \begin{tikzpicture}[scale=5]
          \draw [->] (0,0) -- (0,0.7);
          \draw [->] (0,0) -- (1.1,0);
          \draw plot [smooth, tension=1] coordinates {(0,0.4) (0.3,0.6) (0.6,0.2) (1,0.3)};
          \foreach \x in {0.125,0.25,...,1} {
            \draw (\x,-0.02) -- (\x,0.02);
          }
          \node at (-0.07, 0.7) {$X$};
          \node at (1, -0.07) {$1$};
          \node at (1.14, -0.07) {$\Omega$};
          \draw (0,0.49) -- (0.125,0.49);
          \draw (0.125,0.59) -- (0.25,0.59);
          \draw (0.25,0.585) -- (0.375,0.585);
          \draw (0.375,0.42) -- (0.5,0.42);
          \draw (0.5,0.225) -- (0.625,0.225);
          \draw (0.625,0.182) -- (0.75,0.182);
          \draw (0.75,0.195) -- (0.875,0.195);
          \draw (0.875,0.255) -- (1,0.255);
        \end{tikzpicture}
      \end{center}
    \item Take a random variable $Z: \Omega \to \{z_1, z_2, \dotsc\} \subseteq \mathbb{R}$, and use $\G = \sigma(Z) = \sigma(\{Z = z_i\} : i = 1,2, \dotsc)$.
      Then,
      \begin{align*}
        \E[X | Z]  &\coloneqq \E[X | \sigma(Z)]  \\
      &= \sum_{{i : \Prob[Z = z_i] > 0}} \E[X | Z=z_i] \1{\{Z=z_i\}}
      .\end{align*}
  \end{enumerate}
\end{eg}
This is not satisfactory quite yet: if $Z$ has an absolutely continuous distribution (eg $\mathcal{N}(0,1)$), i.e.\ $\mathcal{P}[Z =z] = 0$ for every $z$, then $\E[X | Z]$ is not defined yet!

\subsection{General case}
\begin{ndef}\label{def:1.2}
  \hypertarget{def:eg}Let $X \in L^1 (\Omega, \F, \Prob)$, $\mathcal{G} \subseteq \mathcal{F}$ a $\sigma$-algebra. A random variable $Y$ is called (a version of) the \named{conditional expectation} of $X$ given $\G$ if
  \begin{enumerate}[(i)]
    \item $Y$ is $\G$-measurable
    \item $\E[X \1A] = \E[Y \1A]$ for all $A \in \G$.
  \end{enumerate}
  We notate $Y = \E[X | \G]$.
\end{ndef}
\begin{nremark}\leavevmode
  \begin{enumerate}[(a)]
    \item We took $X \in L^1$, but this can be changed to $X \geq 0$ throughout.
    \item If $\G = \sigma(\mathcal{C})$ for some $\mathcal{C} \subseteq \F$, it suffices to check (ii) for all $A \in \mathcal{C}$.
    \item If $\G = \sigma(Z)$ where $Z$ is a random variable, we write $\E[X | Z] \coloneqq \hyperlink{def:eg}{\E[X | \sigma(Z)]}$. This is $\sigma(Z)$ measurable by (i), so it's of the form $f(Z)$ for some function $f$.
      It's then common to define $\E[X | Z=z] = f(z)$.
  \end{enumerate}
\end{nremark}
\begin{nthm}[Existence and uniqueness]
  Let $X \in L^1(\Omega, \F, \Prob)$, and $\G \subseteq \F$ a $\sigma$-algebra.
  \begin{enumerate}[(i)]
    \item $\hyperlink{def:eg}{E[X | \G]}$ exists
    \item Any two versions of $\E[X | \G]$ coincide $\Prob$-almost surely.
  \end{enumerate}
\end{nthm}
\begin{proof}\leavevmode
  \begin{enumerate}[(i)]
    \item[(ii)] Uniqueness. Let $Y$ be as in \cref{def:1.2}, and let $Y'$ satisfy \cref*{def:1.2}(i) and (ii) for some $X' \in L^1$ with $X \leq X'$ almost surely.
      Let $Z = (Y-Y') \1A$ with $A \coloneqq \{Y \geq Y'\} \in \G$.
      \begin{equation*}
        \E[Y \1A] = % def 1.2 ii)
        E[X \1A] \leq % X \leq X' a.s.
        \E[X' \1A] =
        \E[Y' \1A] < \infty
      \end{equation*}
      and note that $\E[X' \1A] < \infty$, so $\E[Y' \1A] < \infty$.

      By definition of $Z$, this means $\E[Z] \leq 0$. But $Z \geq 0$ almost surely, so $Z=0$ a.s.\ therefore $Y \leq Y'$ a.s.
      (This shows monotonicity of conditional expectation.)
      If $X = X'$, we can run the same argument to show that $Y = Y'$ almost surely (using $A = \{Y > Y'\}$ and $A = \{Y < Y'\}$, we see both sets are measure zero).
    \item Existence. Step 1: Assume first $X \in L^2(\F)$. Since $L^2(\G)$ is a complete subspace of $L^2(\F)$, $X$ has an orthogonal projection $Y$ on $L^2(\G)$, i.e.\ there is $Y \in L^2(\G)$ such that $\E[(X-Y)Z] = 0$ for every $Z \in L^2(\G)$.
      Choosing $Z = \1A$ for $A \in \G$ we get $\E[X \1A] = \E[Y \1A]$ so $Y$ satisfies the conditions of \cref{def:1.2}.

      Step 2: Assume $X \geq 0$. Then $X_n = X \wedge n \in L^2(\F)$ and $0 \leq X_n \nearrow X$ as $n \to \infty$.
      By Step 1, we can find $Y_n \in L^2(\G)$ such that $\E[X_n \1A] = \E[Y_n \1A]$ for all $A \in \G$ and $0 \leq Y_n \leq Y_{n+1}$ almost surely (from the proof of (ii)).
      Let $Y_\infty = \lim_n Y_n \1{\Omega_0}$ with
      \begin{equation*}
        \Omega_0 = \{\omega \in \Omega : 0 \leq Y_n(\omega) \leq Y_{n+1}(\omega) \ \forall n\}.
      \end{equation*}
      Then $Y_\infty$ is a non-negative random variable, is $\G$-measurable as a limit of $\G$-measurable r.v.s and by monotone convergence $\E[X \1A] = \E[Y_\infty \1A]$ for every $A \in \G$.
      Taking $A = \Omega$, $\E[Y_\infty] = \E[X] < \infty$, since $X \in L_1$.
      So $Y_\infty < \infty$ almost surely and $Y \coloneqq Y_\infty \1{\{Y_\infty < \infty\}}$ satisfies \cref{def:1.2}(i) and (ii).

      Step 3: For general $X \in L^1$, apply Step 2 on $X^+$ and $X^-$ to obtain $Y^+$ and $Y^-$. Then $Y = Y^+ - Y^-$ satisfies the conditions of \cref{def:1.2}. \qedhere
  \end{enumerate}
\end{proof}
\printindex
\end{document}
