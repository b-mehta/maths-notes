\documentclass{article}

% preamble
\def\npart{III}
\def\nyear{2019}
\def\nterm{Michaelmas}
\def\nlecturer{Professor I.\ Leader}
\def\ncourse{Ramsey Theory}

\def\draft{Incomplete}

\usepackage{mathrsfs}
\usepackage{imakeidx}
\usepackage{marginnote}
\usepackage{mathdots}
\usepackage{tabularx}
\usepackage{ifthen}

\ifx \nauthor\undefined
  \def\nauthor{Bhavik Mehta}
\else
\fi

\author{Based on lectures by \nlecturer \\\small Notes taken by \nauthor}
\date{\nterm\ \nyear}
\title{Part \npart\ -- \ncourse}

\usepackage[utf8]{inputenc}
\usepackage{amsmath}
\usepackage{amsthm}
\usepackage{amssymb}
\usepackage{enumerate}
\usepackage{mathtools}
\usepackage{graphicx}
\usepackage[dvipsnames]{xcolor}
\usepackage{tikz}
\usepackage{wrapfig}
\usepackage{centernot}
\usepackage{float}
\usepackage{braket}
\usepackage[hypcap=true]{caption}
\usepackage{enumitem}
\usepackage[colorlinks=true, linkcolor=mblue]{hyperref}
\usepackage[nameinlink,noabbrev]{cleveref}
\usepackage{nameref}
\usepackage[margin=1.5in]{geometry}

% Theorems
\theoremstyle{definition}
\newtheorem*{aim}{Aim}
\newtheorem*{axiom}{Axiom}
\newtheorem*{claim}{Claim}
\newtheorem*{cor}{Corollary}
\newtheorem*{conjecture}{Conjecture}
\newtheorem*{defi}{Definition}
\newtheorem*{eg}{Example}
\newtheorem*{ex}{Exercise}
\newtheorem*{fact}{Fact}
\newtheorem*{law}{Law}
\newtheorem*{lemma}{Lemma}
\newtheorem*{notation}{Notation}
\newtheorem*{prop}{Proposition}
\newtheorem*{question}{Question}
\newtheorem*{rrule}{Rule}
\newtheorem*{thm}{Theorem}
\newtheorem*{assumption}{Assumption}

\newtheorem*{remark}{Remark}
\newtheorem*{warning}{Warning}
\newtheorem*{exercise}{Exercise}

% \newcommand{\nthmautorefname}{Theorem}

\newtheorem{nthm}{Theorem}[section]
\newtheorem{nlemma}[nthm]{Lemma}
\newtheorem{nprop}[nthm]{Proposition}
\newtheorem{ncor}[nthm]{Corollary}
\newtheorem{ndef}[nthm]{Definition}

% Special sets
\newcommand{\C}{\mathbb{C}}
\newcommand{\N}{\mathbb{N}}
\newcommand{\Q}{\mathbb{Q}}
\newcommand{\R}{\mathbb{R}}
\newcommand{\Z}{\mathbb{Z}}

\newcommand{\abs}[1]{\left\lvert #1\right\rvert}
\newcommand{\norm}[1]{\left\lVert #1\right\rVert}
\renewcommand{\vec}[1]{\boldsymbol{\mathbf{#1}}}

\let\Im\relax
\let\Re\relax

\DeclareMathOperator{\Im}{Im}
\DeclareMathOperator{\Re}{Re}
\DeclareMathOperator{\id}{id}

\definecolor{mblue}{rgb}{0., 0.05, 0.6}

\swapnumbers
\reversemarginpar

\usetikzlibrary{positioning, decorations.pathmorphing, decorations.text, calc, backgrounds, fadings}
\tikzset{node/.style = {circle,draw,inner sep=0.8mm}}

\makeindex[intoc]

\newcommand{\named}[1]{\textbf{#1}\index{#1}}
\newcommand{\bonusnamed}[1]{\textbf{#1}\index{#1@*#1}}

\setcounter{section}{-1}

\newcommand{\red}[1]{\textcolor{bred}{#1}}
\newcommand{\green}[1]{\textcolor{bgreen}{#1}}
\newcommand{\blue}[1]{\textcolor{bblue}{#1}}
\newcommand{\yellow}[1]{\textcolor{byellow}{#1}}
\newcommand{\orange}[1]{\textcolor{borange}{#1}}
\newcommand{\purple}[1]{\textcolor{bpurple}{#1}}

% and here we go!
\begin{document}
\maketitle

\tableofcontents

\clearpage
\section{Introduction}

\newlec
If you liked Graph Theory, you'll almost certainly like Ramsey Theory. If you didn't like Graph Theory, you probably won't like Ramsey Theory.
Ramsey theory is an unusual part of maths, in that it's all about answering one question. The basic question is:
\begin{quotation}
    \textit{Can we find some order in enough disorder?}
\end{quotation}

As usual in discrete mathematics, the key ideas of the course are in the proofs rather than in the definitions.

The course is structured into three sections.
\begin{enumerate}[label=\bfseries Chapter \arabic*:, leftmargin=*]
  \item Monochromatic systems (abstract and concrete)
  \item Partition regular equations (concrete)
  \item Infinite Ramsey Theory (abstract)
\end{enumerate}
There are not many prerequisites to this course, only basic concepts of topology (compact spaces).

No single book covers all of the course, but there are two books which cover the relevant content:
\begin{itemize}
  \item Bollob\'as, \textit{Combinatorics}, C.U.P., 1986 (for chapter 3). An excellent survey of the material.
  \item Graham, Rothschild, Spencer, \textit{Ramsey Theory}, Wiley, 1990 (for chapters 1,2).
\end{itemize}

As well as lots of nice proofs in the area, there are many open problems we will come across.

\clearpage
\section{Monochromatic systems}
\subsection{Ramsey's Theorem}
\hypertarget{not:ns}Write $\mathbb{N} = \{1, 2, 3, \dotsc\}$, and write $[n]$ for $\{1, \dotsc, n\}$.\index{natural numbers}
\hypertarget{not:xr}For any set $X$, write \begin{equation*}X^{(r)} = \{A \subseteq X : \abs{A} = r\}.\end{equation*}

Suppose we have the \hyperlink{not:ns}{natural numbers} listed, and each pair of naturals is connected by an edge coloured either \red{red} or \blue{blue}.
\begin{center}
  \begin{tikzpicture}
    \begin{scope}[every node/.style={node, inner sep=0.4mm}]
    \foreach \x in {1,...,5} {
      \node [label=below:$\x$] (\x) at (\x,0) {};
    }
    \end{scope}
    \draw [bred]  (1) to[bend left=40] (2);
    \draw [bred]  (1) to[bend left=40] (3);
    \draw [bred]  (3) to[bend left=40] (5);
    \draw [bblue] (2) to[bend left=40] (3);
    \draw [bblue] (2) to[bend left=40] (4);
    \node at (6,0) {\ldots};
  \end{tikzpicture}
\end{center}
\hypertarget{def:2colouring}Formally, we have a 2-colouring $c$ of $\hyperlink{not:ns}{\mathbb{N}}^{\hyperlink{not:xr}{(2)}}$, i.e.\ $c: \mathbb{N}^{(2)} \to \{\red{1},\blue{2}\}$.\index{colouring}
Can we always find an infinite set $M$ that is \named{monochromatic}, i.e.\ $c$ is constant on $M^{(2)}$?

\begin{eg}
  \leavevmode
  \begin{enumerate}[(i)]
    \item Colour $ij$ \red{red} if $i+j$ even and \blue{blue} if it is odd.
      Then we can find an $M$ that works, by using the evens.
    \item Colour $ij$ \red{red} if $\max\{n : 2^n \mid i+j\}$ even, and \blue{blue} otherwise.

      Again yes, we can use $M = \{4^0, 4^1, 4^2, \dotsc\}$ or $M = \{x : x \equiv 1 \pmod{4}\}$.
    \item Colour $ij$ \red{red} if $i+j$ has an even number of (distinct) prime factors, and \blue{blue} if odd.
      Now the answer is less clear...
  \end{enumerate}
\end{eg}
It turns out that the answer is always yes.
\begin{nthm}[Ramsey's Theorem]\index{Ramsey's theorem}\label{thm:ramsey}
  Let $c$ be a 2-colouring of $\mathbb{N}^{(2)}$. Then $c$ has an infinite monochromatic set.
\end{nthm}
\begin{proof}\leavevmode

\begin{center}
  \begin{tikzpicture}[scale=0.7, every node/.style={node, fill=white, inner sep=0.4mm}]
    \coordinate  (a) at ( -7,  0) {};
    \coordinate  (b) at ( -4,  1) {};
    \coordinate  (c) at ( -1, -1) {};
    \coordinate  (d) at (1.7,  0) {};

    \foreach \theta in {38, 36, ..., -36}{
      \pgfmathsetmacro\rand{rand}
      \draw [bred]    (a) -- ++(\theta    +\rand*1.4   :  3+\rand*0.5);
    }
    \foreach \theta in {30, 28, ..., -28}{
      \pgfmathsetmacro\rand{rand}
      \draw [bblue]  (b) -- ++(\theta*1.1+\rand*1.3-15:  3+\rand*0.5);
    }
    \foreach \theta in {24, 22, ..., -22}{
      \pgfmathsetmacro\rand{rand}
      \draw [bred] (c) -- ++(\theta*1.4+\rand*1.2+20:2.5+\rand*0.5);
    }
    \foreach \theta in {14, 12, ..., -12}{
      \pgfmathsetmacro\rand{rand}
      \draw [bblue]   (d) -- ++(\theta*1.6+\rand*1.2+10:2.3+\rand*0.5);
    }
    \draw [bred]    (a) -- (b);
    \draw [bblue]  (b) -- (c);
    \draw [bred] (c) -- (d);

    \draw plot [smooth cycle, tension=0.8, xshift=2.0cm, xscale=2.5, scale=3.0] coordinates {(-180:1.0) (-120:1.2) (-60:0.8) (0:0.9) (60:1.1) (120:1.3)};
    \draw plot [smooth cycle, tension=0.8, xshift=2.8cm, xscale=2.0, scale=2.6] coordinates {(-180:1.0) (-120:1.2) (-60:0.7) (0:0.9) (60:0.9) (120:1.3)};
    \draw plot [smooth cycle, tension=0.8, xshift=2.9cm, xscale=1.7, scale=1.8] coordinates {(-180:1.0) (-120:1.3) (-60:0.8) (0:0.9) (60:1.1) (120:1.2)};
    \draw plot [smooth cycle, tension=0.8, xshift=3.9cm, xscale=1.2, scale=1.0] coordinates {(-180:0.9) (-120:0.9) (-60:0.8) (0:1.1) (60:1.4) (120:1.4)};
    \node [label=left:$a_1$] at (a) {};
    \node [label=below:$a_2$] at (b) {};
    \node [label=below:$a_3$] at (c) {};
    \node [label=below:$a_4$] at (d) {};
  \end{tikzpicture}
\end{center}
% call the first blob $B_1$, the second $B_2$ and so on
  Pick $a_1 \in \mathbb{N}$. There are infinitely many edges from $a$, so there is an infinite set $B_1 \subseteq \mathbb{N}-\{a_1\}$ such that all edges from $a_1$ to $B_1$ have the same colour, say $\red{C_1}$.

  Pick $a_2 \in B_1$. There are infinitely maby edges from $a_2$ inside $B_1$, so there is an infinite set $B_2 \subset B_1 - \{a_2\}$ such that all edges from $a_2$ to $B_2$ have same colour, say $\blue{C_2}$.

  Continue inductively. We obtain distinct points $a_1, a_2, \dotsc$ and colours $C_1, C_2, \dotsc$ such that $a_i a_j$ (for $i<j$) has colour $C_i$.
  \begin{center}
    \begin{tikzpicture}
      \begin{scope}[every node/.style={node, inner sep=0.4mm}]
        \foreach \x in {1,...,5} {
          \node [label=below:$a_\x$] (\x) at (\x,0) {};
        }
        \draw [bred] (1) to[bend left] (2);
        \draw [bred] (1) to[bend left] (3);
        \draw [bred] (1) to[bend left] (4);
        \draw [bred] (1) to[bend left] (5);
        \draw [bred] (3) to[bend left] (4);
        \draw [bred] (3) to[bend left] (5);
        \draw [bblue] (2) to[bend right] (3);
        \draw [bblue] (2) to[bend right] (4);
        \draw [bblue] (2) to[bend right] (5);
        \draw [bblue] (4) to[bend right] (5);
        % TODO: do this properly
      \end{scope}
    \end{tikzpicture}
  \end{center}

  We must have $C_{i_1} = C_{i_2} = C_{i_3} = \dotsb$ for some $i_1 < i_2 < \dotsb$ (as there are only two colours), so $\{a_{i_1}, a_{i_2}, \dotsc\}$ is monochromatic.
\end{proof}
\begin{remark}\leavevmode
  \begin{enumerate}[(i)]
    \item This is called a two-pass proof.
    \item In example 3, no explicit example is known.
    \item What about a $k$-colouring? (i.e.\ $c: \mathbb{N}^{(2)} \to [k]$). The same proof would show there is an infinite monochromatic set.
      Alternatively, we can deduce this from \nameref{thm:ramsey}, by `turquoise spectacles': view our colouring as a $2$-colouring by colours `1' and `2 or 3 or ... or $k$' and apply \nameref*{thm:ramsey} and induction.
    \item Asking for an infinite monochromatic set is much more than asking for arbitrarily large finite monochromatic sets, e.g. in
      \begin{center}
        \begin{tikzpicture}[every node/.style={node, fill=white, inner sep=0.5mm}, scale=0.6]
          \path[clip] (5,-7) rectangle (27,7);
          \fill [white, scope fading=east] (10,-7) rectangle (27,7);
          \begin{scope}[transparency group=knockout]
            \foreach \n in {2,...,13} {
              \foreach \m in {1,...,\n}{
                \coordinate[xshift=1.77*\n cm] (\n\m) at (360/\n*\m+90: 1) {};
              }
            }
            \foreach \a in {3,...,13} {
              \pgfmathsetmacro\biggest{\a-1}
              \foreach \b in {2,...,\biggest} {
                \foreach \c in {1,...,\a} {
                  \foreach \d in {1,...,\b} {
                    \pgfmathsetmacro\adist{cos(360/\a*\c)}
                    \pgfmathsetmacro\bdist{cos(360/\b*\d)}
                    \pgfmathsetmacro\atop{\adist >  0.2?1:0}
                    \pgfmathsetmacro\abot{\adist < -0.2?1:0}
                    \pgfmathsetmacro\btop{\bdist >  0.2?1:0}
                    \pgfmathsetmacro\bbot{\bdist < -0.2?1:0}
                    \pgfmathsetmacro\top{\atop+\btop==2?1:0}
                    \pgfmathsetmacro\bot{\abot+\bbot==2?1:0}
                    \pgfmathsetmacro\importance{2/(\a*\b)}
                    \begin{scope}[opacity=\importance]
                    \ifthenelse{\top=1}
                      {\draw [bblue] (\a\c) to [bend right] (\b\d)} {};
                        {\ifthenelse{\bot=1}
                          {\draw [bblue] (\a\c) to [bend left] (\b\d)}
                            {\draw [bblue] (\a\c) to (\b\d)};
                          };
                      \end{scope}
                    }
                  }
                }
              }
              \foreach \n in {2,...,13} {
                \foreach \x in {1,...,\n}{
                  \foreach \y in {1,...,\x} {
                    \draw[bred, thick] (\n\x) -- (\n\y);
                    \foreach \m in {1,...,\n}{
                      \node at (\n\m) {};
                    }
                  }
                }
              }
            \end{scope}
          \end{tikzpicture}
        \end{center}
  we have no infinite red set, but arbitrarily large finite red sets.
  \end{enumerate}
\end{remark}
\begin{eg}
  Any sequence $x_1, x_2, \dotsc$ in $\mathbb{R}$ (or in any totally ordered set) has a monotone subsequence.
  Indeed, 2-colour $\mathbb{N}^{(2)}$ by giving $ij$ (for $i<j$) colour \red{up} if $x_i < x_j$ and colour \blue{down} if $x_i \geq x_j$ and apply \nameref{thm:ramsey}.
\end{eg}
\printindex
\end{document}
