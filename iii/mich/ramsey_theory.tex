\documentclass{article}

% preamble
\def\npart{III}
\def\nyear{2019}
\def\nterm{Michaelmas}
\def\nlecturer{Professor I.\ Leader}
\def\ncourse{Ramsey Theory}

\def\draft{Incomplete}

\usepackage{mathrsfs}
\usepackage{imakeidx}
\usepackage{marginnote}
\usepackage{mathdots}
\usepackage{tabularx}
\usepackage{ifthen}

\ifx \nauthor\undefined
  \def\nauthor{Bhavik Mehta}
\else
\fi

\author{Based on lectures by \nlecturer \\\small Notes taken by \nauthor}
\date{\nterm\ \nyear}
\title{Part \npart\ -- \ncourse}

\usepackage[utf8]{inputenc}
\usepackage{amsmath}
\usepackage{amsthm}
\usepackage{amssymb}
\usepackage{enumerate}
\usepackage{mathtools}
\usepackage{graphicx}
\usepackage[dvipsnames]{xcolor}
\usepackage{tikz}
\usepackage{wrapfig}
\usepackage{centernot}
\usepackage{float}
\usepackage{braket}
\usepackage[hypcap=true]{caption}
\usepackage{enumitem}
\usepackage[colorlinks=true, linkcolor=mblue]{hyperref}
\usepackage[nameinlink,noabbrev]{cleveref}
\usepackage{nameref}
\usepackage[margin=1.5in]{geometry}

% Theorems
\theoremstyle{definition}
\newtheorem*{aim}{Aim}
\newtheorem*{axiom}{Axiom}
\newtheorem*{claim}{Claim}
\newtheorem*{cor}{Corollary}
\newtheorem*{conjecture}{Conjecture}
\newtheorem*{defi}{Definition}
\newtheorem*{eg}{Example}
\newtheorem*{ex}{Exercise}
\newtheorem*{fact}{Fact}
\newtheorem*{law}{Law}
\newtheorem*{lemma}{Lemma}
\newtheorem*{notation}{Notation}
\newtheorem*{prop}{Proposition}
\newtheorem*{question}{Question}
\newtheorem*{rrule}{Rule}
\newtheorem*{thm}{Theorem}
\newtheorem*{assumption}{Assumption}

\newtheorem*{remark}{Remark}
\newtheorem*{warning}{Warning}
\newtheorem*{exercise}{Exercise}

% \newcommand{\nthmautorefname}{Theorem}

\newtheorem{nthm}{Theorem}[section]
\newtheorem{nlemma}[nthm]{Lemma}
\newtheorem{nprop}[nthm]{Proposition}
\newtheorem{ncor}[nthm]{Corollary}
\newtheorem{ndef}[nthm]{Definition}

% Special sets
\newcommand{\C}{\mathbb{C}}
\newcommand{\N}{\mathbb{N}}
\newcommand{\Q}{\mathbb{Q}}
\newcommand{\R}{\mathbb{R}}
\newcommand{\Z}{\mathbb{Z}}

\newcommand{\abs}[1]{\left\lvert #1\right\rvert}
\newcommand{\norm}[1]{\left\lVert #1\right\rVert}
\renewcommand{\vec}[1]{\boldsymbol{\mathbf{#1}}}

\let\Im\relax
\let\Re\relax

\DeclareMathOperator{\Im}{Im}
\DeclareMathOperator{\Re}{Re}
\DeclareMathOperator{\id}{id}

\definecolor{mblue}{rgb}{0., 0.05, 0.6}

\swapnumbers
\reversemarginpar

\usetikzlibrary{positioning, decorations.pathmorphing, decorations.text, calc, backgrounds, fadings}
\tikzset{node/.style = {circle,draw,inner sep=0.8mm}}

\makeindex[intoc]

\newcommand{\named}[1]{\textbf{#1}\index{#1}}
\newcommand{\bonusnamed}[1]{\textbf{#1}\index{#1@*#1}}

\setcounter{section}{-1}

\newcommand{\red}[1]{\textcolor{bred}{#1}}
\newcommand{\green}[1]{\textcolor{bgreen}{#1}}
\newcommand{\blue}[1]{\textcolor{bblue}{#1}}
\newcommand{\yellow}[1]{\textcolor{byellow}{#1}}
\newcommand{\orange}[1]{\textcolor{borange}{#1}}
\newcommand{\purple}[1]{\textcolor{bpurple}{#1}}

% and here we go!
\begin{document}
\maketitle

\tableofcontents

\clearpage
\section{Introduction}

\newlec
If you liked Graph Theory, you'll almost certainly like Ramsey Theory. If you didn't like Graph Theory, you probably won't like Ramsey Theory.
Ramsey theory is an unusual part of maths, in that it's all about answering one question. The basic question is:
\begin{quotation}
    \textit{Can we find some order in enough disorder?}
\end{quotation}

As usual in discrete mathematics, the key ideas of the course are in the proofs rather than in the definitions.

The course is structured into three sections.
\begin{enumerate}[label=\bfseries Chapter \arabic*:, leftmargin=*]
  \item Monochromatic systems (abstract and concrete)
  \item Partition regular equations (concrete)
  \item Infinite Ramsey Theory (abstract)
\end{enumerate}
There are not many prerequisites to this course, only basic concepts of topology (compact spaces).

No single book covers all of the course, but there are two books which cover the relevant content:
\begin{itemize}
  \item Bollob\'as, \textit{Combinatorics}, C.U.P., 1986 (for chapter 3). An excellent survey of the material.
  \item Graham, Rothschild, Spencer, \textit{Ramsey Theory}, Wiley, 1990 (for chapters 1,2).
\end{itemize}

As well as lots of nice proofs in the area, there are many open problems we will come across.

\clearpage
\section{Monochromatic systems}
\subsection{Ramsey's Theorem}
\hypertarget{not:ns}Write $\mathbb{N} = \{1, 2, 3, \dotsc\}$, and write $[n]$ for $\{1, \dotsc, n\}$.\index{natural numbers}
\hypertarget{not:xr}For any set $X$, write \begin{equation*}X^{(r)} = \{A \subseteq X : \abs{A} = r\}.\end{equation*}

Suppose we have the \hyperlink{not:ns}{natural numbers} listed, and each pair of naturals is connected by an edge coloured either \red{red} or \blue{blue}.
\begin{center}
  \begin{tikzpicture}
    \begin{scope}[every node/.style={node, inner sep=0.4mm}]
    \foreach \x in {1,...,5} {
      \node [label=below:$\x$] (\x) at (\x,0) {};
    }
    \end{scope}
    \draw [bred]  (1) to[bend left=40] (2);
    \draw [bred]  (1) to[bend left=40] (3);
    \draw [bred]  (3) to[bend left=40] (5);
    \draw [bblue] (2) to[bend left=40] (3);
    \draw [bblue] (2) to[bend left=40] (4);
    \node at (6,0) {\ldots};
  \end{tikzpicture}
\end{center}
\hypertarget{def:2colouring}Formally, we have a 2-colouring $c$ of $\hyperlink{not:ns}{\mathbb{N}}^{\hyperlink{not:xr}{(2)}}$, i.e.\ $c: \mathbb{N}^{(2)} \to \{\red{1},\blue{2}\}$.\index{colouring}
Can we always find an infinite set $M$ that is \named{monochromatic}, i.e.\ $c$ is constant on $M^{(2)}$?

\begin{eg}
  \leavevmode
  \begin{enumerate}[(i)]
    \item Colour $ij$ \red{red} if $i+j$ even and \blue{blue} if it is odd.
      Then we can find an $M$ that works, by using the evens.
    \item Colour $ij$ \red{red} if $\max\{n : 2^n \mid i+j\}$ even, and \blue{blue} otherwise.

      Again yes, we can use $M = \{4^0, 4^1, 4^2, \dotsc\}$ or $M = \{x : x \equiv 1 \pmod{4}\}$.
    \item Colour $ij$ \red{red} if $i+j$ has an even number of (distinct) prime factors, and \blue{blue} if odd.
      Now the answer is less clear...
  \end{enumerate}
\end{eg}
It turns out that the answer is always yes.
\begin{nthm}[Ramsey's Theorem]\index{Ramsey's theorem}\label{thm:ramsey}
  Let $c$ be a 2-colouring of $\mathbb{N}^{(2)}$. Then $c$ has an infinite monochromatic set.
\end{nthm}
\begin{proof}\leavevmode

\begin{center}
  \begin{tikzpicture}[scale=0.7, every node/.style={node, fill=white, inner sep=0.4mm}]
    \coordinate  (a) at ( -7,  0) {};
    \coordinate  (b) at ( -4,  1) {};
    \coordinate  (c) at ( -1, -1) {};
    \coordinate  (d) at (1.7,  0) {};

    \foreach \theta in {38, 36, ..., -36}{
      \pgfmathsetmacro\rand{rand}
      \draw [bred]    (a) -- ++(\theta    +\rand*1.4   :  3+\rand*0.5);
    }
    \foreach \theta in {30, 28, ..., -28}{
      \pgfmathsetmacro\rand{rand}
      \draw [bblue]  (b) -- ++(\theta*1.1+\rand*1.3-15:  3+\rand*0.5);
    }
    \foreach \theta in {24, 22, ..., -22}{
      \pgfmathsetmacro\rand{rand}
      \draw [bred] (c) -- ++(\theta*1.4+\rand*1.2+20:2.5+\rand*0.5);
    }
    \foreach \theta in {14, 12, ..., -12}{
      \pgfmathsetmacro\rand{rand}
      \draw [bblue]   (d) -- ++(\theta*1.6+\rand*1.2+10:2.3+\rand*0.5);
    }
    \draw [bred]    (a) -- (b);
    \draw [bblue]  (b) -- (c);
    \draw [bred] (c) -- (d);

    \draw plot [smooth cycle, tension=0.8, xshift=2.0cm, xscale=2.5, scale=3.0] coordinates {(-180:1.0) (-120:1.2) (-60:0.8) (0:0.9) (60:1.1) (120:1.3)};
    \draw plot [smooth cycle, tension=0.8, xshift=2.8cm, xscale=2.0, scale=2.6] coordinates {(-180:1.0) (-120:1.2) (-60:0.7) (0:0.9) (60:0.9) (120:1.3)};
    \draw plot [smooth cycle, tension=0.8, xshift=2.9cm, xscale=1.7, scale=1.8] coordinates {(-180:1.0) (-120:1.3) (-60:0.8) (0:0.9) (60:1.1) (120:1.2)};
    \draw plot [smooth cycle, tension=0.8, xshift=3.9cm, xscale=1.2, scale=1.0] coordinates {(-180:0.9) (-120:0.9) (-60:0.8) (0:1.1) (60:1.4) (120:1.4)};
    \node [label=left:$a_1$] at (a) {};
    \node [label=below:$a_2$] at (b) {};
    \node [label=below:$a_3$] at (c) {};
    \node [label=below:$a_4$] at (d) {};
  \end{tikzpicture}
\end{center}
% call the first blob $B_1$, the second $B_2$ and so on
  Pick $a_1 \in \mathbb{N}$. There are infinitely many edges from $a$, so there is an infinite set $B_1 \subseteq \mathbb{N}-\{a_1\}$ such that all edges from $a_1$ to $B_1$ have the same colour, say $\red{C_1}$.

  Pick $a_2 \in B_1$. There are infinitely maby edges from $a_2$ inside $B_1$, so there is an infinite set $B_2 \subset B_1 - \{a_2\}$ such that all edges from $a_2$ to $B_2$ have same colour, say $\blue{C_2}$.

  Continue inductively. We obtain distinct points $a_1, a_2, \dotsc$ and colours $C_1, C_2, \dotsc$ such that $a_i a_j$ (for $i<j$) has colour $C_i$.
  \begin{center}
    \begin{tikzpicture}
      \begin{scope}[every node/.style={node, inner sep=0.4mm}]
        \foreach \x in {1,...,5} {
          \node [label=below:$a_\x$] (\x) at (\x,0) {};
        }
        \draw [bred] (1) to[bend left] (2);
        \draw [bred] (1) to[bend left] (3);
        \draw [bred] (1) to[bend left] (4);
        \draw [bred] (1) to[bend left] (5);
        \draw [bred] (3) to[bend left] (4);
        \draw [bred] (3) to[bend left] (5);
        \draw [bblue] (2) to[bend right] (3);
        \draw [bblue] (2) to[bend right] (4);
        \draw [bblue] (2) to[bend right] (5);
        \draw [bblue] (4) to[bend right] (5);
        % TODO: do this properly
      \end{scope}
    \end{tikzpicture}
  \end{center}

  We must have $C_{i_1} = C_{i_2} = C_{i_3} = \dotsb$ for some $i_1 < i_2 < \dotsb$ (as there are only two colours), so $\{a_{i_1}, a_{i_2}, \dotsc\}$ is monochromatic.
\end{proof}
\begin{remark}\leavevmode
  \begin{enumerate}[(i)]
    \item This is called a two-pass proof.
    \item In example 3, no explicit example is known.
    \item What about a $k$-colouring? (i.e.\ $c: \mathbb{N}^{(2)} \to [k]$). The same proof would show there is an infinite monochromatic set.
      Alternatively, we can deduce this from \nameref{thm:ramsey}, by `turquoise spectacles': view our colouring as a $2$-colouring by colours `1' and `2 or 3 or ... or $k$' and apply \nameref*{thm:ramsey} and induction.
    \item Asking for an infinite monochromatic set is much more than asking for arbitrarily large finite monochromatic sets, e.g. in
      \begin{center}
        \begin{tikzpicture}[every node/.style={node, fill=white, inner sep=0.5mm}, scale=0.6]
          \path[clip] (5,-7) rectangle (27,7);
          \fill [white, scope fading=east] (10,-7) rectangle (27,7);
          \begin{scope}[transparency group=knockout]
            \foreach \n in {2,...,13} {
              \foreach \m in {1,...,\n}{
                \coordinate[xshift=1.77*\n cm] (\n-\m) at (360/\n*\m+90: 1) {};
              }
            }
            \foreach \a in {3,...,13} {
              \pgfmathsetmacro\biggest{\a-1}
              \foreach \b in {2,...,\biggest} {
                \foreach \c in {1,...,\a} {
                  \foreach \d in {1,...,\b} {
                    \pgfmathsetmacro\adist{cos(360/\a*\c)}
                    \pgfmathsetmacro\bdist{cos(360/\b*\d)}
                    \pgfmathsetmacro\atop{\adist >  0.2?1:0}
                    \pgfmathsetmacro\abot{\adist < -0.2?1:0}
                    \pgfmathsetmacro\btop{\bdist >  0.2?1:0}
                    \pgfmathsetmacro\bbot{\bdist < -0.2?1:0}
                    \pgfmathsetmacro\top{\atop+\btop==2?1:0}
                    \pgfmathsetmacro\bot{\abot+\bbot==2?1:0}
                    \pgfmathsetmacro\importance{2/(\a*\b)}
                    \begin{scope}[opacity=\importance, every path/.style=bblue]
                      \ifthenelse{\top=1}
                      {\draw (\a-\c) to [bend right] (\b-\d)}
                      {\ifthenelse{\bot=1}
                        {\draw (\a-\c) to [bend left] (\b-\d)}
                          {\draw (\a-\c) to (\b-\d)};
                        };
                      \end{scope}
                    }
                  }
                }
              }
              \foreach \n in {2,...,10} {
                \foreach \x in {1,...,\n}{
                  \foreach \y in {1,...,\x} {
                    \draw [bred] (\n-\x) -- (\n-\y);
                  }
                }
                \foreach \m in {1,...,\n}{
                  \node at (\n-\m) {};
                }
              }
            \end{scope}
          \end{tikzpicture}
        \end{center}
  we have no infinite red set, but arbitrarily large finite red sets.
  \end{enumerate}
\end{remark}
\begin{eg}
  Any sequence $x_1, x_2, \dotsc$ in $\mathbb{R}$ (or in any totally ordered set) has a monotone subsequence.
  Indeed, 2-colour $\mathbb{N}^{(2)}$ by giving $ij$ (for $i<j$) colour \red{up} if $x_i < x_j$ and colour \blue{down} if $x_i \geq x_j$ and apply \nameref{thm:ramsey}.
\end{eg}
\newlec
What if we 2-colour $\hyperlink{not:ns}{\mathbb{N}}^{\hyperlink{not:xr}{(r)}}$, for some $r=3, 4, \dotsc$? Do we always get an infinite monochromatic set?
\begin{center}
  \begin{tikzpicture}
    \begin{scope}[every node/.style={node, inner sep=0.4mm}]
    \foreach \x in {1,...,5} {
      \node [label=below:$\x$] (\x) at (\x,0) {};
    }
    \end{scope}
    \draw [bred]  (1) to[bend left=40] (2);
    \draw [bred]  (1) to[bend left=40] (3);
    \draw [bred]  (2) to[bend left=40] (3);
    \draw [bred]  (1) to[bend right=40] (2);
    \draw [bblue] (1) to[bend right=40] (4);
    \draw [bblue] (2) to[bend right=40] (4);
    \node at (6,0) {\ldots};
  \end{tikzpicture}
\end{center}
For example, colour $ijk$ (for $i < j < k$) \red{red} if $i | j +k$, and \blue{blue} if not.
In this case we can find a such a set, e.g. take $M = \{2^0, 2^1, 2^2, \dotsc\}$.

Let's try to prove this in general.
\begin{nthm}[Ramsey for $r$-sets]\label{thm:ramseyr}
  Whenever $\hyperlink{not:ns}{\mathbb{N}}^{\hyperlink{not:xr}{(r)}}$ is 2-coloured, there is an infinite monochromatic set.
\end{nthm}
\begin{proof}
  Induction on $r$. $r=1$ is easy, by pigeonhole. Alternatively, we could say $r=2$ is true by \cref{thm:ramsey}.

  Given $r$ and a \hyperlink{def:2colouring}{2-colouring} $c: \mathbb{N}^{(r)} \to \{1,2\}$, pick $a_1 \in \mathbb{N}$. Induce a 2-colouring $c'$ of $(\mathbb{N}-\{a_1\})^{(r-1)}$ by
  \begin{equation*}
    c'(F) = c(F \cup \{a_1\})
  \end{equation*}
  for each $F \in (\mathbb{N} - \{a_1\})^{(r-1)}$.

  By induction, there is an infinite monochromatic set $B_1$, for $c'$, i.e.\ $c(F \cup \{a_1\}) = c_1$, for each $F \in B_1^{(r-1)}$.
  Repeat: choose $a_2 \in B_1$ and obtain an infinite set $B_2 \subset B_1 - \{a_2\}$ such that $c(F \cup \{a_2\}) = c_2$, for each $F \in B_2^{(r-1)}$.
  Continue inductively, we obtain distinct $a_1, a_2, \dotsc \in \mathbb{N}$ and colours $c_1, c_2, \dotsc$ such that $c(a_{i_1}, \dotsc, a_{i_r}) = c_{i_1}$, for $i_1 < \dotsb < i_r$.

  We must have $c_{i_1} = c_{i_2} = \dotsb$ for some sequence $i_1, i_2, \dotsc$ and so $\{a_{i_1}, a_{i_2}, \dotsc\}$ is monochromatic, as required.
\end{proof}

We saw that, given $(1, x_1), (2, x_2), (3, x_3) \in \mathbb{R}^2$ there is a subsequence for which the induced (piecewise-linear) function is monotone.
We could actually insist that the induced function is convex ($\smile$) or concave ($\frown$).
Indeed, 2-colour $\mathbb{N}^{(3)}$ by colouring $ijk$ according to whether $(i, x_i), (j, x_j), (k, x_k)$ are % draw these pics
\begin{tikzpicture}[baseline=2pt, scale=0.2]
  \node (A) at (0,0) {$\times$};
  \node (B) at (2,1) {$\times$};
  \node (C) at (4,0) {$\times$};
\end{tikzpicture}
or
\begin{tikzpicture}[baseline=2pt, scale=0.2]
  \node (A) at (0,1) {$\times$};
  \node (B) at (2,0) {$\times$};
  \node (C) at (4,1) {$\times$};
\end{tikzpicture}
and apply \cref{thm:ramseyr} for $r=3$.


Rather unexpectedly, the infinite Ramsey implies the finite Ramsey.
\begin{nthm}[Finite Ramsey]\label{thm:finiteramsey}
  For any $m,r$ there is an $n$ such that whenever $\hyperlink{not:ns}{[n]}^{\hyperlink{not:xr}{(r)}}$ is \hyperlink{def:2colouring}{2-coloured}, there is a monochromatic $m$-set.
\end{nthm}
\begin{proof}
  Suppose not: so for some fixed $m,r$ we have for \emph{each} $n$, a 2-colouring $c_n: [n]^{(r)} \to \{1,2\}$ with no monochromatic $m$-set.

  We'll obtain a 2-colouring of $\mathbb{N}^{(r)}$ with no monochromatic $m$-set, contradicting \cref{thm:ramseyr}.
  [We want to `put the $c_n$ together' - but can only do this if the $c_n$ are \emph{nested}, meaning $c_{n+1} | [n]^{(r)} = c_n$ for every $n$.]

  There are only finitely many ways (2, in fact) to colour $[r]^{(r)}$, so infinitely many of the $c_n$ agree on $[r]^{(r)}$: say $c_n | [r]^{(r)} = d_r$ for every $n \in A_1$ ($A_1$ infinite, $d_r$ is a 2-colouring of $[r]^{(r)}$).

  There are only finitely many ways to $2$-colour $[r+1]^{(r)}$, so infinitely many of the $c_n, n \in A_1$ agree on $[r+1]^{(r)}$, say $c_n | [r+1]^{(r)} = d_{r+1}$ for every $n \in A_2$.
  (There is an infinite $A_2 \subseteq A_1$, $d_{r+1}$ is a 2-colouring of $[r+1]^{(r)}$).

  Continue. We obtain $d_{r}, d_{r+1}, \dotsc$ (where $d_n: [n]^{(r)} \to \{1,2\}$) such that
  \begin{itemize}
    \item no $d_n$ has a mono $m$-set ($d_n = c_{n'} | [n]^{(r)}$ for some $n'$).
    \item The $d_n$ are nested (because $A_2 \subset A_1$ etc.)
  \end{itemize}
  so can find $c: \mathbb{N}^{(r)}\to \{1,2\}$ by setting $c(F) = d_n(F)$ for any $n \geq \max F$.
  Then $c$ has no monochromatic $m$-set, a contradiction.
\end{proof}

\begin{remark}\leavevmode
  \begin{enumerate}[1.]
    \item The proof gives absolutely no information about the least possible value of $n = n(m,r)$, but there are other proofs that do give bounds.
    \item This proof is often called a `compactness' proof. We actually showed that the space of all infinite sequences from $\{1,2\}$ with the topology given by the metric
      \begin{equation*}
        d(f,g) = \frac{1}{\min\{n : f_n \neq g_n\}}
      \end{equation*}
      is compact. (This topology is also called the product topology.)
  \end{enumerate}
\end{remark}

What about infinitely many colours? What if we have $c: \mathbb{N}^{(2)} \to X$ for some arbitrary set $X$.
Do we get an infinite monochromatic set? Definitely not, we could just give every edge a different colour.
Could it be that we can always find an infinite set on which $c$ is either constant or injective?

\begin{center}
  \begin{tikzpicture}
    \begin{scope}[every node/.style={node, inner sep=0.4mm}]
    \foreach \x in {1,...,8} {
      \node (\x) at (\x,0) {};
    }
    \end{scope}
    \foreach \x in {2,...,8} {
      \draw [bred] (1) to[bend left=40] (\x);
    }
    \foreach \x in {3,...,8} {
      \draw [bgreen] (2) to[bend left=40] (\x);
    }
    \foreach \x in {4,...,8} {
      \draw [bblue] (3) to[bend right=40] (\x);
    }
    \foreach \x in {5,...,8} {
      \draw [byellow] (4) to[bend right=40] (\x);
    }
    \node at (9,0) {\ldots};
  \end{tikzpicture}
\end{center}
No.
\printindex
\end{document}
