\documentclass{article}

\def\npart{III}
\def\nyear{2018}
\def\nterm{Michaelmas}
\def\nlecturer{Professor W.\ T.\ Gowers}
\def\ncourse{Introduction to Discrete Analysis}
\def\draft{Incomplete}
\usepackage{scalerel}
\usepackage{bbm}
\usepackage{chngcntr}
\ifx \nauthor\undefined
  \def\nauthor{Bhavik Mehta}
\else
\fi

\author{Based on lectures by \nlecturer \\\small Notes taken by \nauthor}
\date{\nterm\ \nyear}
\title{Part \npart\ -- \ncourse}

\usepackage[utf8]{inputenc}
\usepackage{amsmath}
\usepackage{amsthm}
\usepackage{amssymb}
\usepackage{enumerate}
\usepackage{mathtools}
\usepackage{graphicx}
\usepackage[dvipsnames]{xcolor}
\usepackage{tikz}
\usepackage{wrapfig}
\usepackage{centernot}
\usepackage{float}
\usepackage{braket}
\usepackage[hypcap=true]{caption}
\usepackage{enumitem}
\usepackage[colorlinks=true, linkcolor=mblue]{hyperref}
\usepackage[nameinlink,noabbrev]{cleveref}
\usepackage{nameref}
\usepackage[margin=1.5in]{geometry}

% Theorems
\theoremstyle{definition}
\newtheorem*{aim}{Aim}
\newtheorem*{axiom}{Axiom}
\newtheorem*{claim}{Claim}
\newtheorem*{cor}{Corollary}
\newtheorem*{conjecture}{Conjecture}
\newtheorem*{defi}{Definition}
\newtheorem*{eg}{Example}
\newtheorem*{ex}{Exercise}
\newtheorem*{fact}{Fact}
\newtheorem*{law}{Law}
\newtheorem*{lemma}{Lemma}
\newtheorem*{notation}{Notation}
\newtheorem*{prop}{Proposition}
\newtheorem*{question}{Question}
\newtheorem*{rrule}{Rule}
\newtheorem*{thm}{Theorem}
\newtheorem*{assumption}{Assumption}

\newtheorem*{remark}{Remark}
\newtheorem*{warning}{Warning}
\newtheorem*{exercise}{Exercise}

% \newcommand{\nthmautorefname}{Theorem}

\newtheorem{nthm}{Theorem}[section]
\newtheorem{nlemma}[nthm]{Lemma}
\newtheorem{nprop}[nthm]{Proposition}
\newtheorem{ncor}[nthm]{Corollary}
\newtheorem{ndef}[nthm]{Definition}

% Special sets
\newcommand{\C}{\mathbb{C}}
\newcommand{\N}{\mathbb{N}}
\newcommand{\Q}{\mathbb{Q}}
\newcommand{\R}{\mathbb{R}}
\newcommand{\Z}{\mathbb{Z}}

\newcommand{\abs}[1]{\left\lvert #1\right\rvert}
\newcommand{\norm}[1]{\left\lVert #1\right\rVert}
\renewcommand{\vec}[1]{\boldsymbol{\mathbf{#1}}}

\let\Im\relax
\let\Re\relax

\DeclareMathOperator{\Im}{Im}
\DeclareMathOperator{\Re}{Re}
\DeclareMathOperator{\id}{id}

\definecolor{mblue}{rgb}{0., 0.05, 0.6}


\DeclareMathOperator*{\E}{\scalerel*{\mathbb{E}}{\textstyle\sum}}
\newcommand{\1}[1]{\mathbbm{1}_{#1}}
\counterwithout{nthm}{section}

% \DeclareMathOperator*{\E}{\text{\Large $\mathbb{E}$}}

% preamble

%\setcounter{section}{-1}

%\DeclarePairedDelimiter\ceil{\lceil}{\rceil}
%\DeclarePairedDelimiter\floor{\lfloor}{\rfloor}

%\newtheorem{manualtheoreminner}{Theorem}
%\newenvironment{manualtheorem}[1]{%
%    \renewcommand\themanualtheoreminner{#1}%
%    \manualtheoreminner
%}{\endmanualtheoreminner}

%\newcommand{\red}[1]{\textcolor{bred}{#1}}
%\newcommand{\green}[1]{\textcolor{bgreen}{#1}}
%\newcommand{\blue}[1]{\textcolor{bblue}{#1}}
%\newcommand{\yellow}[1]{\textcolor{byellow}{#1}}
%\newcommand{\orange}[1]{\textcolor{borange}{#1}}
%\newcommand{\purple}[1]{\textcolor{bpurple}{#1}}

% and here we go!

\begin{document}
\maketitle

\tableofcontents

\clearpage
\section{The discrete Fourier transform}
Let $N$ be some fixed positive integer. Write $\omega$ for $e^{\frac{2\pi i}{N}}$, and $\mathbb{Z}_N$ for $\mathbb{Z}/N\mathbb{Z}$.

\begin{defi}[Discrete Fourier transform]\hypertarget{def:dft}
  Let $f: \mathbb{Z}_N \to \mathbb{C}$.
  Given $r \in \mathbb{Z}_N$, define $\hat{f}(r)$ to be
  \begin{equation*}
    \frac{1}{N} \sum_{x \in \mathbb{Z}_N} f(x) \omega^{-r x}.
  \end{equation*}
\end{defi}

\begin{notation}
  From now on, we shall use notation $\E_{x \in \mathbb{Z}_N}$ for $\frac{1}{N} \sum_{x \in \mathbb{Z}_N}$, where the subscript is omitted when it is clear from context.
\end{notation}

Notice we can write
\begin{equation*}
  \hat{f}(r) = \E_x f(x) e^{-\frac{2 \pi i r x}{N}},
\end{equation*}
highlighting the similarity with the usual Fourier transform.

% say something about cyclic groups and well-definedness?
If we write $\omega_r$ for the function $x \mapsto \omega^{r x}$, and $\langle f, g \rangle$ for $\E_x f(x) \overline{g(x)}$, then $\hat{f}(r) = \langle f, \omega_r \rangle$.
Let us write $\| f \|_p$ for $\left(\E_x |f(x)|^p\right)^{\frac{1}{p}}$ and call the resulting space $L_p(\mathbb{Z}_N)$.

\paragraph{Important convention.} We use \emph{averages} for the `original functions' in `physical space' and \emph{sums} for their Fourier transforms in `frequency space'

% Lemma 1.
\begin{nlemma}[Parseval's identity]
  If $f,g: \mathbb{Z}_N \to \mathbb{C}$, then $\langle \hat{f}, \hat{g} \rangle = \langle f, g \rangle$.
\end{nlemma}
\begin{proof}
  \begin{align*}
    \langle \hat{f}, \hat{g} \rangle &= \sum_r \hat{f}(r) \overline{\hat{g}(r)} \\
                                     &= \sum_r (\E_x f(x) \omega^{-r x}) \overline{(\E_y g(y) \omega^{-r y})} \\
                                     &= \E_x \E_y f(x) \overline{g(y)} \sum_r \omega^{-r(x-y)} \\
                                     &= \E_x \E_y f(x) \overline{g(y)} \Delta_{xy} \\
                                     &= \E_x f(x) \E_y \overline{g(y)} \Delta_{xy} \\
                                     &= \E_x f(x) \overline{g(x)} = \langle f, g \rangle
  \end{align*}
  where
  \begin{equation*}
    \Delta_{xy} =
    \begin{cases}
      N & x = y \\
      0 & x \neq y.
    \end{cases} \qedhere
  \end{equation*}
   % reformat this...
\end{proof}

\begin{defi}[Convolution]\hypertarget{def:conv}
  The convolution $\widehat{f * g}(x)$ is defined to be \begin{equation*}\E_{y + z = x} f(y) g(z) = \E_y f(y) g(x-y).\end{equation*}
\end{defi}

\begin{nlemma}[Convolution identity]
  \begin{equation*}
    \widehat{\hyperlink{def:conv}{f * g}}(r) = \hat{f}(r) \hat{g}(r).
  \end{equation*}
\end{nlemma}
\begin{proof}
  \begin{align*}
    \widehat{f * g}(r) &= \E_x f * g(x) \omega^{-r x} \\
                       &= \E_x \E_{y + z = x} f(y) g(z) \omega^{- r x} \\
                       &= \E_x \E_{y + z = x} f(y) g(z) \omega^{- r y} \omega^{- r z} \\
                       &= \E_y f(y) \omega^{-r y} \E_z g(z) \omega^{-r z} = \hat{f}(r) \hat{g}(r). \qedhere
  \end{align*}
\end{proof}

\begin{nlemma}[Inversion formula]
  \begin{equation*}
    f(x) = \sum_r \hat{f}(r) \omega^{r x}
  \end{equation*}
\end{nlemma}
\begin{proof}
  \begin{align*}
    \sum_r \hat{f}(r) \omega^{r x} &= \sum_r \E_y f(y) \omega^{r (x-y)} \\
                                   &= \E_y f(y) \sum_r \omega^{r (x-y)} \\
                                   &= \E_y f(y) \Delta_{x y} = f(x). \qedhere
  \end{align*}
\end{proof}

Further observations:
\begin{enumerate}
  \item If $f$ is real-valued, then $\hat{f}(-r) = \E_x f(x) \omega^{r x} = \overline{\E_x f(x) \omega^{- r x}} = \overline{\hat{f}(r)}$.
  \item If $A \subset \mathbb{Z}_n$, write $A$ (instead of $\1{A}$ or $\chi_A$) for the characteristic function of $A$.
    Then $\hat{A}(0) = \E_x A(x) = \frac{|A|}{N}$, the density of $A$.
  \item Also, $\|\hat{A}\|^2_2 = \langle \hat{A}, \hat{A} \rangle = \langle A, A \rangle = \E_x A(x)^2 = \E_x A(x) = \frac{A}{N}$.
\end{enumerate}

Let $f : \mathbb{Z}_N \to \mathbb{C}$. Given $\mu \in \mathbb{Z}_N$ with $(\mu,N) = 1$, define $f_\mu(x)$ to be $f(\mu^{-1}x)$.
Then
\begin{align*}
  \hat{f_\mu}(r) &= \E_x f_\mu(x) \omega^{-r x} \\
                 &= \E_x f(x / \mu) \omega^{-r x}  \\
                 &= \E_x f(x) \omega^{- r \mu x}  \\
                 &= \hat{f}(\mu r).
\end{align*}

\subsection{Roth's Theorem}
\begin{nthm}
  For every $\delta > 0$, there exists $N$ such that if $A \subseteq \{1, \dotsc, N\}$ is a set of size at least $\delta N$ then $A$ must contain an arithmetic progression of length 3.
\end{nthm}
This is the $k=3$ case of Szemer\'{e}di's theorem.

Basic strategy: show that if $A$ has density $\geq \delta$ and no arithmetic progression of length 3, then there is a long arithmetic progression $P \subseteq \{1, \dotsc, N\}$ such that
\begin{equation*}
  |A \cap P| \geq (\delta + c(\delta)) |P|.
\end{equation*}
In particular, we have that $|P| \to \infty$ as $N \to \infty$.

The proof we give will produce a bound $\frac{C}{\log \log N}$, but this is not the best known.
If the bound was reduced to $\frac{1}{\log N}$, this produces a combinatorial proof of the fact that there are arbitrarily long arithmetic progressions in the primes.
The best known bound is $\frac{(\log \log N)^4}{\log N}$ by Thomas Bloom.
In the other direction, \textbf{add the bound}.
\end{document}
