\documentclass{article}

\def\npart{III}
\def\nyear{2018}
\def\nterm{Michaelmas}
\def\nlecturer{Professor W.\ T.\ Gowers}
\def\ncourse{Introduction to Discrete Analysis}
\def\draft{Ongoing course, rough}
\usepackage{scalerel}
\usepackage{bbm}
\usepackage{chngcntr}
\ifx \nauthor\undefined
  \def\nauthor{Bhavik Mehta}
\else
\fi

\author{Based on lectures by \nlecturer \\\small Notes taken by \nauthor}
\date{\nterm\ \nyear}
\title{Part \npart\ -- \ncourse}

\usepackage[utf8]{inputenc}
\usepackage{amsmath}
\usepackage{amsthm}
\usepackage{amssymb}
\usepackage{enumerate}
\usepackage{mathtools}
\usepackage{graphicx}
\usepackage[dvipsnames]{xcolor}
\usepackage{tikz}
\usepackage{wrapfig}
\usepackage{centernot}
\usepackage{float}
\usepackage{braket}
\usepackage[hypcap=true]{caption}
\usepackage{enumitem}
\usepackage[colorlinks=true, linkcolor=mblue]{hyperref}
\usepackage[nameinlink,noabbrev]{cleveref}
\usepackage{nameref}
\usepackage[margin=1.5in]{geometry}

% Theorems
\theoremstyle{definition}
\newtheorem*{aim}{Aim}
\newtheorem*{axiom}{Axiom}
\newtheorem*{claim}{Claim}
\newtheorem*{cor}{Corollary}
\newtheorem*{conjecture}{Conjecture}
\newtheorem*{defi}{Definition}
\newtheorem*{eg}{Example}
\newtheorem*{ex}{Exercise}
\newtheorem*{fact}{Fact}
\newtheorem*{law}{Law}
\newtheorem*{lemma}{Lemma}
\newtheorem*{notation}{Notation}
\newtheorem*{prop}{Proposition}
\newtheorem*{question}{Question}
\newtheorem*{rrule}{Rule}
\newtheorem*{thm}{Theorem}
\newtheorem*{assumption}{Assumption}

\newtheorem*{remark}{Remark}
\newtheorem*{warning}{Warning}
\newtheorem*{exercise}{Exercise}

% \newcommand{\nthmautorefname}{Theorem}

\newtheorem{nthm}{Theorem}[section]
\newtheorem{nlemma}[nthm]{Lemma}
\newtheorem{nprop}[nthm]{Proposition}
\newtheorem{ncor}[nthm]{Corollary}
\newtheorem{ndef}[nthm]{Definition}

% Special sets
\newcommand{\C}{\mathbb{C}}
\newcommand{\N}{\mathbb{N}}
\newcommand{\Q}{\mathbb{Q}}
\newcommand{\R}{\mathbb{R}}
\newcommand{\Z}{\mathbb{Z}}

\newcommand{\abs}[1]{\left\lvert #1\right\rvert}
\newcommand{\norm}[1]{\left\lVert #1\right\rVert}
\renewcommand{\vec}[1]{\boldsymbol{\mathbf{#1}}}

\let\Im\relax
\let\Re\relax

\DeclareMathOperator{\Im}{Im}
\DeclareMathOperator{\Re}{Re}
\DeclareMathOperator{\id}{id}

\definecolor{mblue}{rgb}{0., 0.05, 0.6}


% https://tex.stackexchange.com/questions/100574/really-wide-hat-symbol
\usepackage{scalerel,stackengine}
\stackMath
\newcommand\reallywidehat[1]{%
\savestack{\tmpbox}{\stretchto{%
  \scaleto{%
    \scalerel*[\widthof{\ensuremath{#1}}]{\kern.1pt\mathchar"0362\kern.1pt}%
    {\rule{0ex}{\textheight}}%WIDTH-LIMITED CIRCUMFLEX
  }{\textheight}%
}{2.4ex}}%
\stackon[-6.9pt]{#1}{\tmpbox}%
}
\parskip 1ex

\usetikzlibrary{cd}

\DeclareMathOperator*{\E}{\scalerel*{\mathbb{E}}{\textstyle\sum}}
\newcommand{\1}[1]{\mathbbm{1}_{#1}}
%\counterwithout{nthm}{section}
\DeclarePairedDelimiter\ceil{\lceil}{\rceil}
\DeclarePairedDelimiter\floor{\lfloor}{\rfloor}

\DeclareMathOperator{\osc}{osc}
\DeclareMathOperator{\diam}{diam}

% preamble

%\setcounter{section}{-1}

%\DeclarePairedDelimiter\ceil{\lceil}{\rceil}
%\DeclarePairedDelimiter\floor{\lfloor}{\rfloor}

%\newtheorem{manualtheoreminner}{Theorem}
%\newenvironment{manualtheorem}[1]{%
%    \renewcommand\themanualtheoreminner{#1}%
%    \manualtheoreminner
%}{\endmanualtheoreminner}

%\newcommand{\red}[1]{\textcolor{bred}{#1}}
%\newcommand{\green}[1]{\textcolor{bgreen}{#1}}
%\newcommand{\blue}[1]{\textcolor{bblue}{#1}}
%\newcommand{\yellow}[1]{\textcolor{byellow}{#1}}
%\newcommand{\orange}[1]{\textcolor{borange}{#1}}
%\newcommand{\purple}[1]{\textcolor{bpurple}{#1}}

% and here we go!

\begin{document}
\maketitle

\tableofcontents

\clearpage
\section{The discrete Fourier transform}
Let $N$ be some fixed positive integer. Write $\omega$ for $e^{\frac{2\pi i}{N}}$, and $\mathbb{Z}_N$ for $\mathbb{Z}/N\mathbb{Z}$.

\begin{defi}[Discrete Fourier transform]\hypertarget{def:dft}
  Let $f: \mathbb{Z}_N \to \mathbb{C}$.
  Given $r \in \mathbb{Z}_N$, define $\hat{f}(r)$ to be
  \begin{equation*}
    \frac{1}{N} \sum_{x \in \mathbb{Z}_N} f(x) \omega^{-r x}.
  \end{equation*}
\end{defi}

\begin{notation}
  From now on, we shall use notation $\E_{x \in \mathbb{Z}_N}$ for $\frac{1}{N} \sum_{x \in \mathbb{Z}_N}$, where the subscript is omitted when it is clear from context.
\end{notation}

Notice we can write
\begin{equation*}
  \hat{f}(r) = \E_x f(x) e^{-\frac{2 \pi i r x}{N}},
\end{equation*}
highlighting the similarity with the usual Fourier transform.

% say something about cyclic groups and well-definedness?
If we write $\omega_r$ for the function $x \mapsto \omega^{r x}$, and $\langle f, g \rangle$ for $\E_x f(x) \overline{g(x)}$, then $\hat{f}(r) = \langle f, \omega_r \rangle$.
Let us write $\| f \|_p$ for $\left(\E_x |f(x)|^p\right)^{\frac{1}{p}}$ and call the resulting space $L_p(\mathbb{Z}_N)$.

\paragraph{Important convention.} We use \emph{averages} for the `original functions' in `physical space' and \emph{sums} for their Fourier transforms in `frequency space'.

% Lemma 1.
\begin{nlemma}[Parseval's identity]
  If $f,g: \mathbb{Z}_N \to \mathbb{C}$, then $\langle \hat{f}, \hat{g} \rangle = \langle f, g \rangle$.
\end{nlemma}
\begin{proof}
  \begin{align*}
    \langle \hat{f}, \hat{g} \rangle &= \sum_r \hat{f}(r) \overline{\hat{g}(r)} \\
                                     &= \sum_r (\E_x f(x) \omega^{-r x}) \overline{(\E_y g(y) \omega^{-r y})} \\
                                     &= \E_x \E_y f(x) \overline{g(y)} \sum_r \omega^{-r(x-y)} \\
                                     &= \E_x \E_y f(x) \overline{g(y)} \Delta_{xy} \\
                                     &= \E_x f(x) \E_y \overline{g(y)} \Delta_{xy} \\
                                     &= \E_x f(x) \overline{g(x)} = \langle f, g \rangle
  \end{align*}
  where
  \begin{equation*}
    \Delta_{xy} =
    \begin{cases}
      N & x = y \\
      0 & x \neq y.
    \end{cases} \qedhere
  \end{equation*}
   % reformat this...
\end{proof}

\begin{defi}[Convolution]\hypertarget{def:conv}
  The convolution $\widehat{f * g}(x)$ is defined to be \begin{equation*}\E_{y + z = x} f(y) g(z) = \E_y f(y) g(x-y).\end{equation*}
\end{defi}

\begin{nlemma}[Convolution identity]
  \begin{equation*}
    \widehat{\hyperlink{def:conv}{f * g}}(r) = \hat{f}(r) \hat{g}(r).
  \end{equation*}
\end{nlemma}
\begin{proof}
  \begin{align*}
    \widehat{f * g}(r) &= \E_x f * g(x) \omega^{-r x} \\
                       &= \E_x \E_{y + z = x} f(y) g(z) \omega^{- r x} \\
                       &= \E_x \E_{y + z = x} f(y) g(z) \omega^{- r y} \omega^{- r z} \\
                       &= \E_y f(y) \omega^{-r y} \E_z g(z) \omega^{-r z} = \hat{f}(r) \hat{g}(r). \qedhere
  \end{align*}
\end{proof}

\begin{nlemma}[Inversion formula]\label{lem:3}
  \begin{equation*}
    f(x) = \sum_r \hat{f}(r) \omega^{r x}
  \end{equation*}
\end{nlemma}
\begin{proof}
  \begin{align*}
    \sum_r \hat{f}(r) \omega^{r x} &= \sum_r \E_y f(y) \omega^{r (x-y)} \\
                                   &= \E_y f(y) \sum_r \omega^{r (x-y)} \\
                                   &= \E_y f(y) \Delta_{x y} = f(x). \qedhere
  \end{align*}
\end{proof}

Further observations:
\begin{itemize}
  \item If $f$ is real-valued, then $\hat{f}(-r) = \E_x f(x) \omega^{r x} = \overline{\E_x f(x) \omega^{- r x}} = \overline{\hat{f}(r)}$.
  \item If $A \subset \mathbb{Z}_n$, write $A$ (instead of $\1{A}$ or $\chi_A$) for the characteristic function of $A$.
    Then $\hat{A}(0) = \E_x A(x) = \frac{|A|}{N}$, the density of $A$.
  \item Also, $\|\hat{A}\|^2_2 = \langle \hat{A}, \hat{A} \rangle = \langle A, A \rangle = \E_x A(x)^2 = \E_x A(x) = \frac{A}{N}$.
\end{itemize}

Let $f : \mathbb{Z}_N \to \mathbb{C}$. Given $\mu \in \mathbb{Z}_N$ with $(\mu,N) = 1$, define $f_\mu(x)$ to be $f(\mu^{-1}x)$.
Then
\begin{align*}
  \hat{f_\mu}(r) &= \E_x f_\mu(x) \omega^{-r x} \\
                 &= \E_x f(x / \mu) \omega^{-r x}  \\
                 &= \E_x f(x) \omega^{- r \mu x}  \\
                 &= \hat{f}(\mu r).
\end{align*}

\subsection{Roth's Theorem}
\begin{nthm}\label{thm:1.4}
  For every $\delta > 0$, there exists $N$ such that if $A \subseteq \{1, \dotsc, N\}$ is a set of size at least $\delta N$ then $A$ must contain an arithmetic progression of length 3.
\end{nthm}
This is the $k=3$ case of Szemer\'{e}di's theorem.

Basic strategy: show that if $A$ has density $\geq \delta$ and no arithmetic progression of length 3, then there is a long arithmetic progression $P \subseteq \{1, \dotsc, N\}$ such that
\begin{equation*}
  |A \cap P| \geq (\delta + c(\delta)) |P|.
\end{equation*}
In particular, we have that $|P| \to \infty$ as $N \to \infty$.

The proof we give will produce a bound $\delta \geq \frac{C}{\log \log N}$, but this is not the best known.
If the bound was reduced to $\frac{1}{\log N}$, this produces a combinatorial proof of the fact that there are arbitrarily long arithmetic progressions in the primes.
The best known bound is $\frac{(\log \log N)^4}{\log N}$ by Thomas Bloom.
In the other direction, we know $e^{- \sqrt{\log N}}$ does not work.

% lecture 2
\begin{nlemma}\label{lem:1.5}
  Let $A, B, C \subset \mathbb{Z}_N$ have densities $\alpha, \beta, \gamma$, for $N$ odd.
  If $\max_{r \neq 0} |\hat{A}(r)| \leq \frac{\alpha(\beta \gamma)^\frac{1}{2}}{2}$ and $\frac{\alpha\beta\gamma}{2} > \frac{1}{N}$ then there exists $x,d \in \mathbb{Z}_N$ with $d \neq 0$ such that $(x, x+d, x+2d) \in A \times B \times C$.
\end{nlemma}
\begin{proof}
  \begin{align*}
    \E_{x,d} A(x) B(x+d) C(x+2d) &= \E_{\mathclap{x + z = 2y}} A(x) B(y) C(z) \\
                                 &= \E_u (\E_{x+z=u} A(x) C(z)) \E_{2y=u} B(y) \\
                                 &= \E_u (A * C)(u) B_2(u) = \langle A * C, B_2 \rangle \\
                                 &= \langle \widehat{A * C}, \hat{B} _2 \rangle \\
                                 &= \langle \hat{A} \hat{C}, \hat{B}_2 \rangle \\
                                 &= \sum_r \hat{A}(r) \hat{C}(r) \hat{B}(-2r) \\
                                 &= \alpha \beta \gamma + \sum_{r\neq 0} \hat{A}(r) \hat{C}(r) \hat{B}(-2r).
  \end{align*}
  We have a lower bound on the left term, so focus on the right.
  \begin{align*}
    \left| \sum_{r\neq0} \hat{A}(r) \hat{B}(-2r) \hat{C}(r) \right| &\leq \frac{\alpha(\beta\gamma)^\frac{1}{2}}{2} \sum_{r\neq0} |\hat{B}(-2r) \hat{C}(r)| \\
                                                      &\leq \frac{\alpha(\beta\gamma)^\frac{1}{2}}{2} \left(\sum_r |\hat{B}(-2r)|^2\right)^\frac{1}{2} \left(\sum_r |\hat{C}(r)|^2\right)^\frac{1}{2} \\
                                                      &= \frac{\alpha(\beta\gamma)^\frac{1}{2}}{2} \|\hat{B}\|_2 \|\hat{C}\|_2 = \frac{\alpha(\beta\gamma)^\frac{1}{2}}{2} \|B\|_2 \|C\|_2 \\
                                                      &= \frac{\alpha\beta\gamma}{2}.
  \end{align*}
  The contribution to $\E_{x,d} A(x) B(x+d) C(x+2d)$ from $d=0$ is at most $\frac{1}{N}$, so if $\frac{\alpha\beta\gamma}{2} > \frac{1}{N}$, we are done.
\end{proof}

Now let $A$ be a subset of $\{1, \dotsc, N\}$ of density $\geq \delta$ and let $B = C = A \cap (\frac{N}{3}, \frac{2N}{3}]$.
If $B$ has density $< \frac{\delta}{5}$, then either $A \cap [1, \frac{N}{3}]$ or $A \cap [\frac{2N}{3}, N]$ has density at least $\frac{2 \delta}{5}$.
So in that case we find an AP $P$ of length about $\frac{N}{3}$ such that $\frac{|A \cap P|}{|P|} \geq \frac{6 \delta}{5}$.

Otherwise, we find that if $\max_{r \neq 0} |\hat{A}(r)| \leq \frac{\delta^2}{10}$ and $\frac{\delta^3}{50} > \frac{1}{N}$ then $A \times B \times C$ contains a 3AP $\implies A$ contains a 3AP.
So if $A$ does not contain a 3AP, then either we find $P$ of length about $\frac{N}{3}$ with $\frac{|A \cap P|}{|P|} \geq \frac{6 \delta}{5}$ or $\exists r \neq 0$ such that $|\hat{A}(r)| \geq \frac{\delta^2}{10}$.

\begin{defi}
  If $X$ is a finite set and $f: X \to \mathbb{C}$, $Y \subseteq X$, write $\osc(f|_Y)$ to mean $\max_{y_1, y_2 \in Y} |f(y_1) - f(y_2)|$.
\end{defi}
\begin{nlemma}\label{lem:1.6}
  Let $r \in \hat{\mathbb{Z}}_n$ and let $\epsilon > 0$. Then there is a partition of $\{1,2,\dotsc,N\}$ into arithmetic progressions $P_i$ of length at least $c(\epsilon) \sqrt{N}$ such that $\osc(\omega_r|_{P_i}) \leq \epsilon$ for each $i$.
\end{nlemma}
\begin{proof}
  Let $t = \floor{\sqrt{N}}$. Of the numbers $1, \omega^r, \omega^{2r}, \dotsc, \omega^{tr}$ there must be two that differ by at most $\frac{2\pi}{t}$.
  If $|\omega^{ar} - \omega^{br}| \leq \frac{2\pi}{t}$ with $a < b$, then $|1 - \omega^{dr}| \leq \frac{2\pi}{t}$ where $d = b-a$.
  Now, by the triangle inequality, if $u < v$, then
  \begin{equation*}
    |\omega^{u rd} - \omega^{v r d}| \leq |\omega^{urd} - \omega^{(u+1) r d}| + |\omega^{urd} - \omega^{(u+1) r d}| + \dotsb + |\omega^{urd} - \omega^{(u+1) r d}| \leq \frac{2\pi}{t} (v-u).
  \end{equation*}
  So if $P$ is a progression with common difference $d$ and length $l$, then $\osc(\omega_r|_P) \leq \frac{2\pi l}{t}$.
  So divide up $\{1,\dotsc,N\}$ into residue classes mod $d$ and partition each residue class into parts of length between $\frac{\epsilon t}{4 \pi}$ and $\frac{\epsilon t}{2 \pi}$ (possible, since $d \leq t \leq \sqrt{N}$).
  We are done, with $c(\epsilon) = \frac{\epsilon}{16}$.
\end{proof}
Now let us use the information that $r \neq 0$ and $|\hat{A}(r)| \geq \frac{\delta^2}{10}$.
Define the balanced function $f$ of $A$ by $f(x) = A(x) - \frac{|A|}{N}$ for each $x$.

Note that $\hat{f}(0) = 0$ and $\hat{f}(r) = \hat{A}(r)$ for all $r \neq 0$.
Now let $P_1, \dotsc, P_m$ be given by \cref{lem:1.6} with $\epsilon = \frac{\delta^2}{20}$.
Then
\begin{align*}
  \frac{\delta^2}{10} \leq \frac{1}{N} \left| \sum_x f(x) \omega^{-rx} \right| &\leq \frac{1}{N} \sum_{i=1}^m \left|\sum_{x\in P_i} f(x) \omega^{-r x}\right| \\
                                                                      &\leq \frac{1}{N} \sum_{i=1}^m \left[\left|\sum_{x \in P_i} f(x) \omega^{-r x_i}\right| + \left|\sum_{x \in P_i} f(x) (\omega^{-r x} - \omega^{-r x_i})\right|\right] \\
                                                                      \shortintertext{where $x_i \in P_i$ is arbitrary }
                                                                      &\leq \frac{1}{N} \sum_{i=1}^m \left|\sum_{x \in P_i} f(x)\right| + \frac{\delta^2}{20}
                                                                      \shortintertext{So}
  \sum_{i=1}^N \left|\sum_{x \in P_i} f(x) \right| &\geq \frac{\delta^2 N}{20}.
\end{align*}
Also,
\begin{equation*}
  \sum_{i=1}^m \sum_{x \in P_i} f(x) = 0.
\end{equation*}
So
\begin{equation*}
  \sum_{i=1} \left( \left|\sum_{x \in P_i} f(x) \right| + \sum_{x \in P_i} f(x) \right)  \geq \frac{\delta^2}{20}  \sum_{i =1}^m |P_i|
\end{equation*}
Therefore, $\exists i$ such that
\begin{align*}
  \left| \sum_{x \in P_i} f(x) \right| + \sum_{x \in P_i} f(x) &\geq \frac{\delta^2}{20} |P_i| \\
  \implies \sum_{x \in P_i} f(x) \geq \frac{\delta}{40} |P_i| \\
  \implies |A \cap P_i| \geq \left(\delta + \frac{\delta^2}{40}\right) |P_i|
\end{align*}

% lecture 3
So now, either
\begin{enumerate}
  \item $A$ contains a $3AP$
  \item $N$ is even
  \item $\exists P \subset \{1,\dotsc,N\}$, $|P| \geq \frac{N}{3}$ such that $|A \cap P| \geq \frac{6 \delta}{5}|P|$
  \item $\exists P \subset \{1,\dotsc,N\}$, $|P| \geq \frac{\delta^2}{320} \sqrt{N}$ such that $|A \cap P| \geq \left(\delta + \frac{\delta^2}{40}\right) |P|$
\end{enumerate}

If 2 holds, write $N = N_1 + N_2$ with $N_1, N_2$ odd, $N_1 + N_2 \approx \frac{N}{2}$.
Then $A$ has density at least $\delta$ in one of $\{1,\dotsc, N_1\}$ or $\{N_1 + 1, \dotsc, N_1 + N_2\}$.

If 4 holds (note $3 \Rightarrow 4$) then we pass to $P$ and start again.
After $\frac{40}{\delta}$ iterations, the density at least doubles.
So the total number of iterations we can have is $\leq \frac{40}{\delta} + \frac{40}{2 \delta} + \frac{40}{4 \delta} + \dotsc \leq \frac{80}{\delta}$.

If $\frac{\delta^2}{320} \sqrt{N} \geq N^\frac{1}{3}$ at each iteration, and $\frac{\delta^3}{25} > N^{-1}$ (which follows from the first condition) then after $\frac{80}{\delta}$ iterations we have $N \geq N^{\left(\frac{1}{3}\right)^\frac{80}{\delta}}$.
So the argument works provided
\begin{align*}
  N^{\left(\frac{1}{3}\right)^\frac{80}{\delta}} \geq \left(\frac{320}{\delta^2}\right)^6 &\impliedby \left(\frac{1}{3}\right)^{\frac{80}{\delta}} \log N \geq 6\left(\log 320 + 2 \log \frac{1}{\delta}\right) \\
                                                                             &\impliedby - \frac{80}{\delta} \log 3 + \log \log N \geq \log 6 + \log \left(\log 320 + 2 \log \frac{1}{\delta}\right) \\
                                                                             &\impliedby \log \log N \geq \frac{160}{\delta} \impliedby \delta \geq \frac{160}{\log \log N}.
\end{align*}
% change 320 to 640

\subsection{Bogolyubov's method}
Let $K \subset \hat{\mathbb{Z}}_N$ and let $\delta > 0$.
\begin{defi}[Bohr set]\hypertarget{def:bohr}
  The \textbf{Bohr set} $B(K, \delta)$ has two definitions.
  \begin{enumerate}
    \item $B(K, \delta) = \set{x \in \mathbb{Z}_N | rx \in [-\delta N, \delta N] \ \forall r \in K}$ (arc-length definition)
    \item $B(K, \delta) = \set{x \in \mathbb{Z}_N | |1 - \omega^{r x}| < \delta \ \forall r \in K}$ (chord-length definition)
  \end{enumerate}
\end{defi}
\begin{defi}
  Let $G$ be an abelian group and let $A,B$ be subsets of $G$. Then
  \begin{align*}
    A + B &= \set{a + b | a \in A, b \in B} \\
    A - B &= \set{a - b | a \in A, b \in B} \\
    rA &= \set{a_1 + \dotsb + a_r | a_1, \dotsc, a_r \in A}
  \end{align*}
\end{defi}

\begin{nlemma}\label{lem:1.7}
  Let $A \subset \mathbb{Z}_N$ be a set of density $\alpha$. Then $2A - 2A$ contains a \hyperlink{def:bohr}{Bohr set} (arc) with $|K| \geq \alpha^{-2}$.
\end{nlemma}
\begin{proof}
  Observe that $x \in 2A - 2A$ iff $A * A * (-A) * (-A)(x) \neq 0$.
  But
  \begin{align*}
    A * A * (-A) * (-A) (x) &= \sum_r \reallywidehat{A * A * (-A) * (-A)} (r) \omega^{r x} \\
                            &= \sum_r |\hat{A}(r)|^4 \omega^{r x}.
  \end{align*}
  Let $K = \set{r | |\hat{A}(r)| \geq \alpha^\frac{3}{2}}$.
  Then $\alpha = \|\hat{A}\|^2_2 = \sum_r |\hat{A}(r)|^2 \geq \alpha^3 |K|$
  So $|K| \leq \alpha^{-2}$.

  Now suppose that $x \in B(K, \frac{1}{4})$.
  Then
  \begin{align*}
    \sum_r |\hat{A}(r)|^4 \omega^{r x} = \alpha^4 + \sum_{\mathclap{r \in K \setminus \{0\}}} |\hat{A}(r)|^4 \omega^{rx} + \sum_{r \notin K} |\hat{A}(r)|^4 \omega^{r x}.
  \end{align*}
  The real part of the second term is non-negative, since $r x \in \left[-\frac{N}{4}, \frac{N}{4}\right]$ when $r \in K$.
  Also
  \begin{equation*}
    \left|\sum_{r \notin K} |\hat{A}(r)|^4 \omega^{r x}\right| \leq \sum_{r \notin K} |\hat{A}(r)|^4 < \alpha^3 \sum_{r \notin K} |\hat{A}(r)|^2 \leq \alpha^4.
  \end{equation*}

  It follows that $\Re \left(\sum_r |\hat{A}(r)|^4 \omega^{r x}\right) > 0$, so $x \in 2A - 2A$.
\end{proof}
\begin{nlemma}\label{lem:1.8}
  Let $K \subset \mathbb{Z}_N$ and let $\delta > 0$. Then
  \begin{enumerate}[label=(\roman*)]
    \item $B(K, \delta)$ (arc) has density at least $\delta^{|K|}$
    \item $B(K, \delta)$ contains a mod-$N$ arithmetic progression of length $\geq \delta N^\frac{1}{|K|}$
  \end{enumerate}
\end{nlemma}
\begin{proof}\leavevmode
  \begin{enumerate}[label=(\roman*)]
    \item Let $K = \{r_1, \dotsc, r_k\}$. Consider the $N$ $k$-tuples $(r_1 x, r_2 x, \dotsc, r_k x) \in \mathbb{Z}_N^k$.
      If we intersect this set of $k$-tuples with a random `box' $[t_1, t_1 + \delta N] \times \dotsm \times [t_k, t_k + \delta N]$
      then the expected number of the $k$-tuples in the box is $\delta^k N$ (since each one has a probability $\delta^k$).
      But if $(r_1 x, \dotsc, r_k x)$ and $(r_1 y, \dotsc, r_k y)$ belong to this box, then $x - y \in B(K, \delta)$.
    \item If we take $\eta > N^{\frac{1}{2}}$, then by (i) we get that $|B(K, \eta) > 1$, so $\exists x \in B(K, \eta)$ such that $x \neq 0$.
      But then $d x \in B(K, d \eta)$ for every $d$.
      So if $d \eta \leq \delta$ then $d x \in B(K, \delta)$. That gives us an AP of length at least $\frac{\delta}{\eta}$.
      So we get one of length at least $\delta N^\frac{1}{k}$. \qedhere
  \end{enumerate}
\end{proof}
\begin{defi}[Freiman homomorphism]
  Let $A,B$ be subsets of Abelian groups and let $\phi : A \to B$.
  Then $\phi$ is a \textbf{Freiman homomorphism of order $k$} if
  \begin{equation*}
    a_1 + \dotsb + a_k = a_{k+1} + \dotsc + a_{2k} \implies \phi(a_1) + \dotsb + \phi(a_k) = \phi(a_{k+1}) + \dotsb + \phi(a_{2k}).
  \end{equation*}
  If $k=2$, we call this a \textbf{Freiman homomorphism}.
  In that case, the condition is equivalent to $a - b = c - d \implies \phi(a) - \phi(b) = \phi(c) - \phi(d)$.

  If $\phi$ has an inverse which is also a Freiman homomorphism of order $k$, then $\phi$ is a Freiman isomorphism of order $k$.
\end{defi}
\begin{nlemma}\label{lem:1.9}
  Assume $0 \notin K$ and $N$ prime.
  If $\delta < \frac{1}{4}$, then $B(K, \delta)$ (arc) is Freiman isomorphic to the intersection in $\mathbb{R}^{K}$ of $[-\delta N, \delta N]^{|K|}$ with some lattice $\Delta$.
\end{nlemma}
\begin{proof}
  Let $K = \{r_1, \dotsc, r_k\}$ and let $\Lambda = N \mathbb{Z}^k + \set{(r_1 x, \dotsc, r_k x) | x \in \mathbb{Z}}$.
  Write $\mathbf{r}$ for $(r_1, \dotsc, r_k)$.
  Claim that $B(K, \delta) \cong \Lambda \cap [-\delta N, \delta N]^k$.
  Define a map $\phi: B(K, \delta) \to \Lambda \cap [-\delta N, \delta N]^k$ by sending $x$ to $(\langle r_1 x \rangle, \dotsc, \langle r_k x \rangle)$ where $\langle u \rangle$ means the least-modulus residue of $u \bmod N$.

  If $x + y = z + w$, then $\mathbf{r} x + \mathbf{r} y = \mathbf{r} z + \mathbf{r} w$ in $\mathbb{Z}_N^k$.
  But for each $i$, $\langle r_i x \rangle + \langle r_i y \rangle - \langle r_i z \rangle - \langle r_i w \rangle \in [-4 \delta N, 4 \delta N]$.
  Since $\delta < \frac{1}{4}$, that implies that $\langle r_i x \rangle + \langle r_i y\rangle - \langle r_i z \rangle - \langle r_i w \rangle = 0$.
  So $\langle \mathbf{r} x \rangle + \langle \mathbf{r} y \rangle = \langle \mathbf{r} z \rangle + \langle \mathbf{r} w \rangle$.

  That already implies that $\phi$ is an injection.
  If $\mathbf{r} x + \mathbf{a} N \in [-\delta N, \delta N]^k$ then $r_i x \in [-\delta N, \delta N] \bmod{N}$ for each $i$, so $x \in B(K, \delta)$ and $\phi(x) = \mathbf{r} x + \mathbf{a} N$.
  So $\phi$ is a surjection.

  If $\mathbf{r} x + \mathbf{a} N + \mathbf{r} y + \mathbf{b} N = \mathbf{r} z + \mathbf{c} N + \mathbf{r} w + \mathbf{d} N$, then $r_1 (x+y) = r_1 (z + w) \bmod{N}$, so $x + y = z + w \bmod{N}$.
  So the inverse of $\phi$ is also a Freiman homomorphism.
\end{proof}
\begin{nlemma}\label{lem:1.10}
  Let $\Lambda$ be a lattice and let $C$ be a symmetric convex body, both in $\mathbb{R}^k$.
  Then $\Lambda \cap C \leq 5^k |\Lambda \cap \frac{C}{2}|$.
\end{nlemma}
\begin{proof}
  Let $x_1, \dotsc, x_n$ be a maximal subset of $\Lambda \cap C$ such that for all $i \neq j$, $x_j \notin x_i + \frac{C}{2}$.
  Then by maximality, the sets $x_i + \frac{C}{2}$ cover all of $\Lambda \cap C$.
  Also, the sets $x_i + \frac{C}{4}$ are disjoint subsets of $\mathbb{R}^k$, and they are all contained in $C + \frac{C}{4} = \frac{5}{4} C$.
  So
  \begin{equation*}
    m \leq \frac{\operatorname{vol}(\frac{5}{4} C)}{\operatorname{vol}(\frac{1}{4} C)} = 5^k.
  \end{equation*}
\end{proof}
\begin{ncor}
  If $N$ is prime, $0 \notin K$, $|K| = k$, $\delta < \frac{1}{4}$, then $|B(K,\delta)| \leq 5^k |B(K, \frac{\delta}{2})$.
\end{ncor}

\clearpage
\section{Sumsets and their structure}
The idea is to show that for $A \subset \mathbb{Z}$, if $|A + A| \leq K|A|$ then $|rA - sA| \leq K^{r+s} |A|$.
\begin{nlemma}[Petridis]\label{lem:2.1}
  Let $A_0, B$ be finite subsets of an abelian group such that $|A_0 + B| \leq K_0 |A_0|$.
  Then there exists a non-empty subset $A \subset A_0$ and $K \leq K_0$ such that $|A + B + C| \leq K|A + C|$ for every finite subset $C$ of the group.
\end{nlemma}
\begin{proof}
  Let $A$ minimise the ratio $\frac{|A+B|}{|A|}$ and let the minimal ratio be $K$.
  Claim: this works. We prove this by induction on $C$.

  If $C = \emptyset$, then the result holds.
  Now assume it for $C$ and let $x \notin C$.
  Then \begin{equation*}A + (C \cup \{x\}) = (A + C) \cup (A + \{x\}) = (A + C) \cup \left[(A + x) \setminus (A' + x)\right]\end{equation*} where $A' = \set{a \in A | a + x \in A + C}$.
  This is a disjoint union, so
  \begin{equation*}
    |A + (C \cup \{x\})| = |A + C| + |A| - |A'|.
  \end{equation*}
  Similarly,
  \begin{align*}
    A + B + (C \cup \{x\}) &= (A + B + C) \cup ((A + B + x) \setminus (A' + B + x)) \\
    \shortintertext{(since if $a+x \in A + C$ then $a + B + x \subset A + B + C$)}
    \implies |A + B + (C \cup \{x\})| &\leq |A + B + C| + |A + B| - |A' + B| \\
                                      &\leq K |A + C| + K|A| - K|A'|
  \end{align*}
  by induction and minimality property of $A$.
\end{proof}
% new lecture
\begin{ncor}\label{cor:2.2}
  If $A,B$ are finite subsets of an Abelian group and $|A + B| \leq K|A|$, then there exists $A' \subseteq A$, $A' \neq \emptyset$ such that $|A' + rB| \leq K^r |A'|$ for every positive integer $r$.
\end{ncor}
\begin{proof}
  Choose $A'$ as we chose $A$ in the proof of \cref{lem:2.1}.
  Then
  \begin{equation*}
    |A' + rB| = |A'+B+(r-1)B| \leq K |A' + (r-1) B|
  \end{equation*}
  and $|A'+B| \leq K|A'|$
  so we are done by induction.
\end{proof}
\begin{ncor}\label{cor:2.3}
  If $|A+A| \leq K|A|$ or $|A - A| \leq K|A|$, then $|rA| \leq K^r |A|$.
\end{ncor}
\begin{proof}
  Set $B = A$ or $-A$ in \cref{cor:2.2}
\end{proof}

\begin{nlemma}[Rusza inequality]\label{lem:2.4}
  Let $A,B,C$ be finite subsets of an abelian group.
  Then $|A||B-C| \leq |A-B||A-C|$.
\end{nlemma}
\begin{proof}
  Define a map $\phi: A \times (B-C) \to (A-B) \times (A-C)$ as follows.
  Given $(a,x)$ with $a\in A, x \in B-C$, choose, somehow, $b(x) \in B$ and $c(x) \in C$ such that $b(x) - c(x) = x$ and set $\phi(a,x) = (a - b(x), a-c(x))$.

  Note that $(a - c(x)) - (a - b(x)) = b(x) - c(x) = x$.
  Having worked out $x$, we know $b(x)$ and $a = a - b(x) + b(x)$, so $a$ is determined too.
  So $\phi$ is an injection.
\end{proof}

Why is it called a triangle inequality?
We can write it as
\begin{equation*}
  \frac{|B-C|}{|B|^{\frac{1}{2}} |C|^{\frac{1}{2}}} \leq \frac{|A-B|}{|A|^{\frac{1}{2}} |B|^{\frac{1}{2}}} + \frac{|A-C|}{|A|^{\frac{1}{2}} |C|^{\frac{1}{2}}}
\end{equation*}
so if we define the Rusza distance $d(A,B)$ to be
\begin{equation*}
  \frac{|A-B|}{|A|^{\frac{1}{2}} |B|^{\frac{1}{2}}},
\end{equation*}
then the inequality says $d(B,C) \leq d(A,B) d(A,C)$.

\begin{ncor}\label{cor:2.5}
  If $|A - B| \leq K|A|$, then $|rB - sB| \leq K^{r+s}|A|$ for all $r,s$.
\end{ncor}
\begin{proof}
  Pick $A'$ as before. Then by \cref{cor:2.2} with $B$ replaced by $-B$, $|A'-rB| \leq K'|A'|$ and $|A' - sB| \leq K^s |A'|$.
  Therefore, by \nameref{lem:2.4},
  \begin{equation*}
    |A'| |rB - sB| \leq K^{r+s} |A'|^2 \implies |rB - sB| \leq K^{r+s} |A|. \qedhere
  \end{equation*}
\end{proof}

\begin{ncor}[Plünnecke's theorem]\label{cor:2.6}
  If $|A+A| \leq K|A|$ or $|A-A| \leq K|A|$, then $|rA-sA| \leq K^{r+s}|A|$.
\end{ncor}
\begin{proof}
  Apply \cref{cor:2.5} with $B=-A$ or $B=A$.
\end{proof}

\begin{nlemma}[Ruzsa's embedding theorem]\label{lem:2.7}
  Let $A \subseteq \mathbb{Z}$ be finite and suppose that $|kA - kA| \leq C|A|$.
  Then there exists a prime $p \leq 4 C |A|$ and a subset $A' \subseteq A$ of size at least $|A|/k$ such that $A'$ is Freiman isomorphic of order $k$ to a subset of $\mathbb{Z}_p$.
\end{nlemma}
\begin{proof}
  Consider the following composition of maps
  \begin{equation*}
    \begin{tikzcd}[column sep=6em]
      \mathbb{Z} \rar{\text{reduce mod }q}
    & \mathbb{Z}_q \arrow[r, "\times\text{ by some}", "\text{non-zero }r"']
    & \mathbb{Z}_q \arrow[r, "\text{least non-negative}", "\text{residue}"'] & \mathbb{Z} \rar{\text{reduce mod }p} & \mathbb{Z}_p
    \end{tikzcd}
  \end{equation*}
  where $q$ is a prime bigger than $\diam A$ and $p$ is a prime $\in (2 C |A|, 4 C |A|]$.

  Let $\phi$ be the composition. The first, second and fourth parts are group homomorphisms, and thus Freiman homomorphisms of all orders.
  Also, the third map is a Freiman homomorphism of order $k$ if you restrict to a subinterval of $[0, q-1]$ of length $\leq \frac{q}{k}$.
  To see this, write $\langle u \rangle$ for the least non-negative residue.
  Then if $I$ has length $\leq \frac{q}{k}$ (and therefore $< \frac{q}{k}$) and $u_1, \dotsc, u_{2k} \in I$, then if $u_1 + \dotsb + u_k - u_{k+1} - \dotsb - u_{2k} = 0$, then
  \begin{equation*}
    \langle u_1 \rangle + \dotsb + \langle u_k \rangle - \langle u_{k+1} \rangle - \dotsb - \langle u_{2k} \rangle \equiv 0 \pmod{q}
  \end{equation*}
  and also has modulus less than $q$. So it is zero.

  By the pigeonhole principe, for any $r$ we can find $I$ of length $\leq \frac{q}{k}$ such that
  \begin{equation*}
    A' = \set{a \in A | ra \in I}
  \end{equation*}
  has size at least $|A|/k$.

  Remains to prove that $\phi$ is an isomorphism to its image. That is, we must show that if
  \begin{equation*}
    a_1 + \dotsb + a_k - a_{k+1} - \dotsb a_{2k} \equiv 0 \quad (a_i \in A)
  \end{equation*}
  then
  \begin{equation*}
    \langle r a_1 \rangle + \dotsb + \langle r a_k \rangle - \langle r a_{k+1} \rangle - \langle r a_{2k} \rangle \neq 0 \pmod{p}
  \end{equation*}
  But if the $a_i$ are chosen such that the $r a_i$ all belong to the same interval of length $\frac{q}{k}$,
  \begin{equation*}
    |\langle r a_1 \rangle + \dotsb \langle r a_k \rangle - \langle r a_{k+1} \rangle - \dotsb - \langle r a_{2k} \rangle| < q \\
  \end{equation*}
  and
  \begin{equation*}
    \langle r a_1 \rangle + \dotsb + \langle r a_k \rangle - \langle r a_{k+1} \rangle - \dotsb - \langle r a_{2k} \rangle \equiv r(a_1 + \dotsb + a_k - a_{k+1} - \dotsb - a_{2k}) \bmod{q}
  \end{equation*}
  So all that can go wrong is if $r(a_1 + \dotsb + a_k - a_{k+1} - \dotsb - a_{2k})$ is $xp$ for some $x \neq 0$ with $|x| < \frac{q}{p}$.
  The number of values to avoid is at most $\frac{2q}{p}$, so for each $a_1 + \dotsb + a_k - a_{k+1} - \dotsc - a_{2k}$ the probability of going wrong if $r$ is chosen randomly is at most $\frac{2}{p}$.
  So since $|kA - kA| \leq C|A|$, the probability of going wrong is at most $\frac{2}{p} C |A|$.
  Since $p > 2 C |A|$, there exists $r$ such that we get a Freiman isomorphism of order $k$.
\end{proof}

% new lecture
\subsection{Freiman's theorem (a version of)}
We shall now start with a set $A \subseteq \mathbb{Z}$ with $|A+A| \leq C|A|$ and put together several of the previous results to say a lot about the structure of $A$.

By Pl\"unnecke's theorem, $|8A - 8A| \leq C^{16}|A|$. By Ruzsa's embedding lemma, $A$ has a subset $A'$ of size at least $\frac{A}{8}$ that is $8$-isomorphic to a subset $A'' \subset \mathbb{Z}_p$ with $p \leq 4 C^{16} |A|$.
The density of $A''$ in $\mathbb{Z}_p$ is $\alpha \geq \frac{1}{32 C^{16}}$.
By Bogolubov's lemma, $2A'' - 2A''$ contains a Bohr set $B(K, \frac{1}{4})$ with $|K| \leq \alpha^{-2}$, which is $2$-isomorphic to a set $B'$ that is the intersection of a symmetric convex body with a lattice of dimension at most $\alpha^{-2}$.

On the example sheet, prove that if $A$ is $8$-isomorphic to $B$, then $2A-2A$ is $2$-isomorphic to $2B - 2B$.
Thus $2A'' - 2A'' \overset{2}{\cong} 2A' - 2A'$, and therefore $2A' - 2A'$ has a subset $B$ that is isomorphic to $B'$.

Now let $X \subset A$ be maximal such that the sets $x + B$ with $x \in X$ are disjoint.
Then $A \subset X + B - B$, by maximality.
Also, $|X| |B| = |X + B| \leq |3A - 2A| \leq C^5 |A| \implies |X| \leq C^5 \frac{|A|}{|B|}$.
But by basic facts about Bohr sets, $|B| \geq 4^{-\alpha^{-2}} |A|$, so $|X| \leq 4^{\alpha^{-2}} C^5$.
So $A$ is the union of at most $4^{1024 C^{32}} C^5$ translates of $B-B$.
If $B = \Lambda \cap K_0$ then $B - B \subset \Lambda \cap 2K_0$ and also $|B-B| \leq 5^{\alpha^{-2}} |B|$.

\subsection{The Balog-Szemer\'edi-Gowers theorem}
\begin{defi}[Additive quadruple]
  Let $A$ be a subset of an Abelian group. An additive quadruple in $A$ is a quadruple $(a,b,c,d) \in A^4$ such that $a+b = c+d$.
\end{defi}
(Equivalently, it's a quadruple such that $a - b = c-d$.)

If $|A|=n$, then the number of additve quadruples in $A$ is at most $n^3$.
We shall show that if $A^4$ contains at least $cn^3$ additive quadruples, then $A$ has a subset $A'$ of size at least $c' n$ with $|A' - A'| \leq C|A|$ where $c'$ and $C$ depend (nicely) on $c$ only.
\end{document}
