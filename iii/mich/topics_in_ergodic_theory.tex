\documentclass{article}

\def\npart{III}
\def\nyear{2018}
\def\nterm{Michaelmas}
\def\nlecturer{Dr. P. Varj\'{u}}
\def\ncourse{Topics in Ergodic Theory}
\def\draft{Ongoing course, rough}
\usepackage{imakeidx}
\ifx \nauthor\undefined
  \def\nauthor{Bhavik Mehta}
\else
\fi

\author{Based on lectures by \nlecturer \\\small Notes taken by \nauthor}
\date{\nterm\ \nyear}
\title{Part \npart\ -- \ncourse}

\usepackage[utf8]{inputenc}
\usepackage{amsmath}
\usepackage{amsthm}
\usepackage{amssymb}
\usepackage{enumerate}
\usepackage{mathtools}
\usepackage{graphicx}
\usepackage[dvipsnames]{xcolor}
\usepackage{tikz}
\usepackage{wrapfig}
\usepackage{centernot}
\usepackage{float}
\usepackage{braket}
\usepackage[hypcap=true]{caption}
\usepackage{enumitem}
\usepackage[colorlinks=true, linkcolor=mblue]{hyperref}
\usepackage[nameinlink,noabbrev]{cleveref}
\usepackage{nameref}
\usepackage[margin=1.5in]{geometry}

% Theorems
\theoremstyle{definition}
\newtheorem*{aim}{Aim}
\newtheorem*{axiom}{Axiom}
\newtheorem*{claim}{Claim}
\newtheorem*{cor}{Corollary}
\newtheorem*{conjecture}{Conjecture}
\newtheorem*{defi}{Definition}
\newtheorem*{eg}{Example}
\newtheorem*{ex}{Exercise}
\newtheorem*{fact}{Fact}
\newtheorem*{law}{Law}
\newtheorem*{lemma}{Lemma}
\newtheorem*{notation}{Notation}
\newtheorem*{prop}{Proposition}
\newtheorem*{question}{Question}
\newtheorem*{rrule}{Rule}
\newtheorem*{thm}{Theorem}
\newtheorem*{assumption}{Assumption}

\newtheorem*{remark}{Remark}
\newtheorem*{warning}{Warning}
\newtheorem*{exercise}{Exercise}

% \newcommand{\nthmautorefname}{Theorem}

\newtheorem{nthm}{Theorem}[section]
\newtheorem{nlemma}[nthm]{Lemma}
\newtheorem{nprop}[nthm]{Proposition}
\newtheorem{ncor}[nthm]{Corollary}
\newtheorem{ndef}[nthm]{Definition}

% Special sets
\newcommand{\C}{\mathbb{C}}
\newcommand{\N}{\mathbb{N}}
\newcommand{\Q}{\mathbb{Q}}
\newcommand{\R}{\mathbb{R}}
\newcommand{\Z}{\mathbb{Z}}

\newcommand{\abs}[1]{\left\lvert #1\right\rvert}
\newcommand{\norm}[1]{\left\lVert #1\right\rVert}
\renewcommand{\vec}[1]{\boldsymbol{\mathbf{#1}}}

\let\Im\relax
\let\Re\relax

\DeclareMathOperator{\Im}{Im}
\DeclareMathOperator{\Re}{Re}
\DeclareMathOperator{\id}{id}

\definecolor{mblue}{rgb}{0., 0.05, 0.6}

\makeindex[intoc]

% preamble

% and here we go!

\begin{document}
\maketitle

\tableofcontents

\clearpage
Ergodic theory is all about measure preserving systems.
\begin{defi}[Measure preserving system]\index{measure preserving!system}\hypertarget{def:mps}
  A \textbf{measure preserving system} $(X, \mathcal{B}, \mu, T)$ with $X$ a set, $\mathcal{B}$ a $\sigma$-algebra, $\mu$ a probability measure ($\mu(A) \geq 0$ $\forall A \in \mathcal{B}$ and $\mu(X) = 1$) and $T$ is a measure preserving transformation.
  \index{measure preserving!transformation}Recall a measure preserving transformation $T : X \to X$ is a measurable function such that $\mu(T^{-1}(A)) = \mu(A)$ $\forall A \in \mathcal{B}$.
\end{defi}
If $Y$ is a random element of $X$ with distribution $\mu$, then $T(Y)$ also has distribution $\mu$.

\begin{eg}
  \index{rotation map}For example, consider a circle rotation. We have $X = \mathbb{R}/\mathbb{Z}$, $\mathcal{B}$ is the Borel sets, $\mu$ the Lebesgue measure, and $T = R_\alpha$, with $x \mapsto x + \alpha$ and $\alpha \in \mathbb{R}/\mathbb{Z}$ is a parameter.

  \index{doubling map}\hypertarget{def:doubling}We also have the `times 2 map', with the same $X, \mathcal{B}, \mu$ and $T = T_2$, $x \mapsto 2 \cdot x$.
\end{eg}
\begin{proof}[Proof that \hyperlink{def:doubling}{$T_2$} is \hyperlink{def:mps}{measure preserving}]
  First check for intervals: Let $I = (a,b)$, then $\mu(I) = b-a$.
  Also, $\mu(T_2^{-1}I) = \mu\left((\frac{a}{2},\frac{b}{2}) \cup (\frac{a}{2} + \frac{1}{2}, \frac{b}{2} + \frac{1}{2})\right) = \frac{b}{2} - \frac{a}{2} + \frac{b}{2} - \frac{a}{2} = b - a$, as required.

  Now, let $U \subset \mathbb{R}/\mathbb{Z}$ be open. Then $U = I_1 \sqcup I_2 \sqcup \dotsb$ is a disjoint union of intervals:
  \begin{align*}
    \mu(T^{-1} U) &= \mu\left(\bigcup T^{-1} I_j\right) \\
                  &= \sum \mu(T^{-1} I_j) \\
                  &= \sum \mu(I_j) \\
                  &= \mu(U).
  \end{align*}

  Let $K \subset \mathbb{R}/\mathbb{Z}$ be a compact set.
  \begin{align*}
    \mu(T^{-1} K) = 1 - \mu((T^{-1} K)^c) = 1 - \mu(T^{-1} K^c) = 1 - \mu(K^c) = \mu(K).
  \end{align*}
  Now let $A \in \mathcal{B}$ be arbitrary. Let $\epsilon > 0$. $\exists U$ open and $\exists K$ compact such that $K \subset A \subset U$ and $\mu(U \setminus K) < \epsilon$.
  \begin{align*}
    \mu(K) = \mu(T^{-1} K) \leq \mu(T^{-1} A) \leq \mu(T^{-1} U) = \mu(U).
  \end{align*}
  We also have $\mu(K) \leq \mu(A) \leq \mu(U)$.
  Since $\mu(U) - \mu(K) < \epsilon$, $|\mu(A) - \mu(T^{-1}A)| < \epsilon$. $\epsilon$ was arbitrary, so $\mu(A) = \mu(T^{-1} A)$.
\end{proof}

The two examples generalise to the Haar measure on a topological group and to endomorphisms respectively.

In ergodic theory, we study the long term behaviour of orbits.
\begin{defi}[Orbit]\index{orbit}\hypertarget{def:orbit}
  The orbit of $x \in X$ is the sequence
  \begin{equation*}
    x, Tx, T^2 x, \dotsc
  \end{equation*}
\end{defi}
Some questions we might ask are:
\begin{itemize}
  \item Let $A \in \mathcal{B}$ and $x \in A$. Does the orbit of $x$ visit $A$ infinitely often? (Recurrence)
  \item What is the proportion of times $n$ such that $T^n x \in A$?
  \item What is $\mu(\set{x \in A | T^n x \in A})$ if $n$ is large? (Mixing property)
\end{itemize}

\begin{eg}
  Let $A = [0, \frac{1}{4}) \subset \mathbb{R}/\mathbb{Z}$. %]
  Then $\hyperlink{def:doubling}{T_2}^n x \in A \iff $ the $n+1$th and $n+2$th `binary digits' of $x$ are $0$.

  For some $x = 0.x_1 x_2 x_3 \dots_2$, $x \in A$ corresponds to $x_1, x_2$ both being 0 and the doubling map sends $x$ to $T_2x = x_2 x_3 \dots_2$, giving the required property above.

  For example, $x = \frac{1}{6} = 0.00101010\dots_2$ starts in $A$ but never comes back to $A$.
  Also, we have $\mu(\set{x \in A | T_2^n x}) = \frac{1}{16}$ if $n \geq 2$.
\end{eg}

\begin{eg}[Markov shift]\index{markov shift}\hypertarget{def:markovshift}
  Let $P_1, P_2, \dotsc, P_n$ be a probability vector. Let $A \in \mathbb{R}^{n \times n}_{\geq 0}$ be the `matrix of transition probabilities'.
  Assume
  \begin{equation*}
    A
    \begin{pmatrix}
      1 \\ 1 \\ \vdots \\ 1
    \end{pmatrix}
    =
    \begin{pmatrix}
      1 \\ 1 \\ \vdots \\ 1
    \end{pmatrix},
    \begin{pmatrix}
      P_1 & P_2 & \dots & P_n
    \end{pmatrix}
    A =
    \begin{pmatrix}
      P_1 & P_2 & \dots & P_n
    \end{pmatrix}
  \end{equation*}
  Take $X = \{1, \dotsc, n\}^\mathbb{Z}$, $\mathcal{B}$ the Borel $\sigma$-algebra generated by the product topology of the discrete topology on $\{1, \dotsc, n\}$, $T = \sigma$ the shift map: $(\sigma x)_m = x_{m+1}$.
  Finally, set the measure
  \begin{equation*}
    \mu(\set{x \in X | x_m = i_0, x_{m+1} = i_1, \dotsc, x_{m+n} = i_n}) = P_{i_0} a_{i_0 i_1} \dotsm a_{i_{n-1} i_n}.
  \end{equation*}
\end{eg}

\begin{thm}[Szemer\'{e}di]\index{Szemer\'{e}rdi's theorem}\hypertarget{thm:sze}
  Let $S \subset \mathbb{Z}$ of positive upper Banach density. That is,
  \begin{equation*}
    \bar{d}(S) \coloneqq \limsup_{N,M: M - N \to \infty} \frac{1}{M-N} \big| S \cap [N,M-1] \big|
  \end{equation*}
  and $\bar{d}(S) > 0$.
  Then $S$ contains arbitrarily long arithmetic progressions. That is, $\forall l, \exists a \in \mathbb{Z}, d \in \mathbb{Z}_{> 0}$,
  \begin{equation*}a, a+d, \dotsc, a+(l-1)d \in S.\end{equation*}
\end{thm}

\begin{thm}[Furstenberg, multiple recurrence]\index{Furstenberg's theorem}\hypertarget{thm:furs}
  Let $(X, \mathcal{B}, \mu, T)$ be a \hyperlink{def:mps}{measure preserving system}. Let $A \in \mathcal{B}$ such that $\mu(A) > 0$. Let $l \in \mathbb{Z}_{>0}$.
  Then
  \begin{equation*}
    \liminf_{N \to \infty} \frac{1}{N} \sum_{n=1}^N \mu(A \cap T^{-n} A \cap \dotsb \cap T^{-(l-1) n} A) > 0.
  \end{equation*}
\end{thm}

Let
\begin{itemize}
  \item $X = \{0, 1\}^\Z$
  \item $\mathcal{B}=$ Borel $\sigma$-algebra
  \item $\sigma=$ the \hyperlink{def:markovshift}{shift} map $\vec{x} \mapsto (x_{n+1})_n$
\end{itemize}
Let $ \vec{x}^S \in X$ be defined by
\begin{equation*}
  \vec{x}_n^S =
  \begin{cases}
    1 & n\in S\\
    0 & n \notin S.
  \end{cases}
\end{equation*}
Also let $A\in\beta$ be given by $A = \set{x\in X | x_0 = 1}$.
Observe then that
\begin{equation*}
  \vec{x}^S_n=1\iff n\in S\iff \sigma^n\vec{x}^S\in A\iff (\sigma^n\vec{x}^S)_0=1.
\end{equation*}

Let $\{M_m\}$ and $\{N_m\}$ be sequences s.t. $ M_n-N_m\to\infty $ and
\begin{equation*} \bar{d}(S) = \lim_{m\to\infty}\frac{1}{M_m-N_m}\left|S\cap[N_m,M_m-1]\right| \end{equation*}
Let
\begin{equation*}
  \mu_m = \frac{1}{M_m-N_m}\sum_{n=N_m}^{M_m-1}\delta_{\sigma^n\vec{x}^S}
\end{equation*}
where $\delta_x$ is a measure on $X$ defined as
\begin{equation*}
  \delta_x(B) =
  \begin{cases}
    1 & x\in B\\
    0 & x\notin B
  \end{cases}
\end{equation*}

Let $\mu$ be the weak limit of \emph{a} subsequence of $\mu_m$.
Note how the $\mu$ could be different dependent on subsequence choice.

\begin{defi}[Weak limit]
  Let $X$ be a compact metric space.
  Let $\mu_m$ be a sequence of Borel measures on $X$, and let $\mu$ be another Borel measure.
  Then $\mu_m$ converges weakly to $\mu$ if for any $f\in C(X)$, we have
  \begin{equation*}
    \lim _{n \to \infty} \int_X f \,d \mu_{n} = \int_X f \,d \mu.
  \end{equation*}
\end{defi}

\begin{thm}(Banach-Alaoglu, or Helly)
  Let $X$ be a compact metric space.
  Then $\mathcal{M}(X)$, the set of Borel probability measures on $X$, endowed with the topology of weak convergence, is compact and metrizable.
  That is, there is a weakly convergent subsequence in any sequence of Borel probability measures.
\end{thm}

\begin{lemma}
  $(X, \mathcal{B}, \mu, \sigma)$ as defined above is a \hyperlink{def:mps}{measure preserving system}.
\end{lemma}
\begin{proof}[Proof sketch]
  Let $B \in \mathcal{B}$. Then
  \begin{align*}
    \mu_m(B) &= \frac{1}{M_m - N_m} \left|\set{n \in [N_m, M_m -1] | \sigma^n \vec{x}^S \in \mathcal{B}} \right| \\
    \mu_m(\sigma^{-1} B) &= \frac{1}{M_m - N_m} \left|\set{n \in [N_m, M_m -1] | \sigma^n \vec{x}^S \in \sigma^{-1}\mathcal{B}} \right| \\
                         &= \frac{1}{M_m - N_m} \left|\set{n \in [N_m+1, M_m] | \sigma^n \vec{x}^S \in \mathcal{B}} \right|
  \end{align*}

  So the difference is such that
  \[ \left|\mu_m(B) - \mu_m(\sigma^{-1}B)\right| \leq \frac{1}{M_m-N_m}\to 0 \]
  It can be shown that we can pass to the limit on $ m $ and conclude that $ \mu(B) = \mu(\theta^{-1}B)$.
\end{proof}

\begin{remark}
  If $B$ is a cylinder set, i.e.\ $\exists L \in \mathbb{Z}_{>0}$ and $\tilde{B} \subseteq \{0,1\}^{2L+1}$ such that
  \begin{equation*}
    B = \set{x \in X | (x_{-L}, \dotsc, x_L) \in \tilde{B}},
  \end{equation*}
  then $B$ is both closed and open.
  Therefore $\chi_B$, the characteristic function of $B$ is continuous.
  Hence $\lim_{n \to \infty} \mu_m(B) = \mu(B)$, since $\mu_m(B) = \int \chi_B \, d \mu_m$ and $\mu(B) = \int \chi_B \, d \mu$.

  Approximating any Borel set by such cylinder sets would help complete the proof, but we in fact can get this result on spaces where $ \chi $ is not continuous on nice set of sets. So we leave full proof till a more general theorem.
\end{remark}

\begin{prop}
  Let $S \subseteq \mathbb{Z}$, let $\vec{x}^S, A, (X, \mathcal{B}, \mu, \sigma)$ as defined above.
  Let $l \in \mathbb{Z}_{> 0}$.
  Suppose that $\exists n \in \mathbb{Z}_{>0}$ such that
  \begin{equation*}
    \mu\left(A \cap \sigma^{-n}(A) \cap \dotsb \cap \sigma^{-n(l-1)}(A)\right) > 0.
  \end{equation*}
  Then $S$ contains an arithmetic progression of length $l$.
\end{prop}
\begin{proof}
  Without loss of generality, we can assume $\mu = \lim \mu_m$ - if not, pass to a subsequence.
  Let $B = A \cap \sigma^{-n} A \cap \dotsb \cap \sigma^{-n (l-1)} (A).$
  Observe that $B$ is a cylinder set.
  Then by the earlier remark, $\mu(B) = \lim \mu_m(B)$, hence $\exists m$ such that $\mu_m(B) > 0$.

  By definition of $\mu_m$, $\exists k \in [N_m, M_m - 1]$ such that $\sigma^k \vec{x}^S \in B$.
  Hence
  \begin{equation*}\sigma^k \vec{x}^S \in A, \sigma^k \vec{x}^S \in \sigma^{-n}(A), \dotsc, \sigma^k \vec{x}^S \in \sigma^{-n(l-1)}(A).\end{equation*}
  Thus, $k, k+n, \dotsc, k + n(l-1) \in S$.
\end{proof}
Returning to the overall proof, we note $ A $ is a cylinder set.
Then $ \mu_m(A)\to\mu(A) $, i.e.\
\begin{equation*} \mu(A) = \underbrace{\lim_{m\to\infty}\frac{1}{M_m-N_m}\left|\set{n\in[N_m,M_m-1]: n\in S}\right|}_{\bar{d}(S)} > 0 \end{equation*}
where the inequality comes from satisfying the conditions of Furstenberg.

\printindex
\end{document}
