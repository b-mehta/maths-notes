\documentclass{article}

\def\npart{III}
\def\nyear{2018}
\def\nterm{Michaelmas}
\def\nlecturer{Professor P.\ T.\ Johnstone}
\def\ncourse{Category Theory}
\def\draft{Ongoing course, rough}

\usepackage{mathrsfs}
\usepackage{imakeidx}

\ifx \nauthor\undefined
  \def\nauthor{Bhavik Mehta}
\else
\fi

\author{Based on lectures by \nlecturer \\\small Notes taken by \nauthor}
\date{\nterm\ \nyear}
\title{Part \npart\ -- \ncourse}

\usepackage[utf8]{inputenc}
\usepackage{amsmath}
\usepackage{amsthm}
\usepackage{amssymb}
\usepackage{enumerate}
\usepackage{mathtools}
\usepackage{graphicx}
\usepackage[dvipsnames]{xcolor}
\usepackage{tikz}
\usepackage{wrapfig}
\usepackage{centernot}
\usepackage{float}
\usepackage{braket}
\usepackage[hypcap=true]{caption}
\usepackage{enumitem}
\usepackage[colorlinks=true, linkcolor=mblue]{hyperref}
\usepackage[nameinlink,noabbrev]{cleveref}
\usepackage{nameref}
\usepackage[margin=1.5in]{geometry}

% Theorems
\theoremstyle{definition}
\newtheorem*{aim}{Aim}
\newtheorem*{axiom}{Axiom}
\newtheorem*{claim}{Claim}
\newtheorem*{cor}{Corollary}
\newtheorem*{conjecture}{Conjecture}
\newtheorem*{defi}{Definition}
\newtheorem*{eg}{Example}
\newtheorem*{ex}{Exercise}
\newtheorem*{fact}{Fact}
\newtheorem*{law}{Law}
\newtheorem*{lemma}{Lemma}
\newtheorem*{notation}{Notation}
\newtheorem*{prop}{Proposition}
\newtheorem*{question}{Question}
\newtheorem*{rrule}{Rule}
\newtheorem*{thm}{Theorem}
\newtheorem*{assumption}{Assumption}

\newtheorem*{remark}{Remark}
\newtheorem*{warning}{Warning}
\newtheorem*{exercise}{Exercise}

% \newcommand{\nthmautorefname}{Theorem}

\newtheorem{nthm}{Theorem}[section]
\newtheorem{nlemma}[nthm]{Lemma}
\newtheorem{nprop}[nthm]{Proposition}
\newtheorem{ncor}[nthm]{Corollary}
\newtheorem{ndef}[nthm]{Definition}

% Special sets
\newcommand{\C}{\mathbb{C}}
\newcommand{\N}{\mathbb{N}}
\newcommand{\Q}{\mathbb{Q}}
\newcommand{\R}{\mathbb{R}}
\newcommand{\Z}{\mathbb{Z}}

\newcommand{\abs}[1]{\left\lvert #1\right\rvert}
\newcommand{\norm}[1]{\left\lVert #1\right\rVert}
\renewcommand{\vec}[1]{\boldsymbol{\mathbf{#1}}}

\let\Im\relax
\let\Re\relax

\DeclareMathOperator{\Im}{Im}
\DeclareMathOperator{\Re}{Re}
\DeclareMathOperator{\id}{id}

\definecolor{mblue}{rgb}{0., 0.05, 0.6}

\usetikzlibrary{cd}
\swapnumbers

\makeindex[intoc]

\DeclareMathOperator{\ob}{ob}
\DeclareMathOperator{\mor}{mor}
\DeclareMathOperator{\dom}{dom}
\DeclareMathOperator{\cod}{cod}

% preamble

\setcounter{section}{-1}

\newtheorem{manualeginner}{Examples}
\newenvironment{manualeg}[1]{%
    \renewcommand\themanualeginner{#1}%
    \manualeginner
}{\endmanualeginner}

%\let\nthm\undefined
%\let\nlemma\undefined
%\let\nprop\undefined
%\let\ncor\undefined
%\let\ndef\undefined
%\newtheorem{nthm}{Theorem}[section]
%\newtheorem{nlemma}[nthm]{Lemma}
%\newtheorem{nprop}[nthm]{Proposition}
%\newtheorem{ncor}[nthm]{Corollary}
%\newtheorem{ndef}[nthm]{Definition}
\newtheorem{nremark}[nthm]{Remark}
\newtheorem{nexample}[nthm]{Examples}

% and here we go!

\begin{document}
\maketitle

\tableofcontents

\clearpage
\section{Introduction}
Category theory is like a language spoken by many different people, with many different dialects.
Specifically, different parts of category theory are used in different branches of mathematics.
In this course, we aim to speak the language of category theory, without an accent - a broad overview of all aspects of category theory.
There will be many examples, some of which may not be understandable.
As long as some examples make sense, it is not a point of concern that some examples seem unfamiliar.

\section{Definitions and Examples}
\begin{ndef}[Category]\index{category}\hypertarget{def:cat}
  A \textbf{category} $\mathscr{C}$ consists of
  \begin{enumerate}[label=(\alph*)]
    \item \index{object}a collection $\mathscr{C}$ of \textbf{objects} $A,B,C,\dotsc$
    \item \index{morphism}a collection $\mor \mathscr{C}$ of \textbf{morphism} $f,g,h,\dotsc$
    \item \index{domain}\index{codomain}two operations $\dom, \cod$ assigning to each $f \in \mor \mathscr{C}$ a pair of objects, its \textbf{domain} and \textbf{codomain}. We write $A \xrightarrow{f} B$ to mean `$f$ is a morphism and $\dom f = A$ and $\cod f = B$'.
    \item \index{identity}an operation assigning to each $A \in \ob \mathscr{C}$ a morphism $A \xrightarrow{1_A} A$, called its \textbf{identity}.
    \item \index{composition}a partial binary operation \textbf{composition} $(f,g) \mapsto f g$ on morphisms, such that $fg$ is defined iff $\dom f = \cod g$ and $\dom(f g) = \dom g$, $\cod (f g) = \cod f$ if $fg$ is defined.
  \end{enumerate}
  satisfying
  \begin{enumerate}[label=(\alph*)] \setcounter{enumi}{5}
    \item \index{category!axioms}$f 1_A = f = 1_B f$ for any $A \xrightarrow{f} B$
    \item $(fg)h = f (gh)$ whenever $fg$ and $gh$ are defined
  \end{enumerate}
\end{ndef}

\begin{nremark}\leavevmode
  \begin{enumerate}[label=(\alph*)]
    \item \index{category!small}\hypertarget{def:small}This definition is independent of a model of set theory. If we're given a particular model of set theory, we call the \hyperlink{def:cat}{category $\mathscr{C}$} \textbf{small} if $\ob \mathscr{C}$ and $\mor \mathscr{C}$ are sets.
    \item Some texts say $fg$ means `$f$ followed by $g$', i.e.\ $fg$ defined $\iff \hyperlink{def:cat}{\cod} f = \dom g$.
      \item Note that a morphism $f$ is an \hyperlink{def:cat}{identity} iff $fg = g$ and $hf = h$ whenever the compositions are defined.
        So we could formulate the definition entirely in terms of morphisms.
  \end{enumerate}
\end{nremark}

\begin{nexample}\leavevmode
  \begin{enumerate}[label=(\alph*)]\hypertarget{def:categ}
    \item The \hyperlink{def:cat}{category} \textbf{Set} has all sets as objects, and all functions between sets as morphisms.
      (Strictly, morphisms $A \to B$ are pairs $(f,B)$ where $f$ is a set-theoretic function.)
    \item The category \textbf{Gp} has all groups as objects, and group homomorphisms as morphisms. Similarly, \textbf{Rng} is the category of rings, \textbf{Mod}$_R$ the category of $R$-modules.
    \item The category \textbf{Top} has all topological spaces as objects and continuous functions as morphisms. Similarly \textbf{Unif} has uniform spaces and uniformly continuous functions, and \textbf{Mf} has manifolds and smooth maps.
    \item The category \textbf{Htpy} has the same objects as \textbf{Top}, but morphisms are homotopy classes of continuous functions.
      More generally, given $\mathscr{C}$, we call an equivalence relation $\simeq$ on $\mor \mathscr{C}$ a \textbf{congruence} if $f \simeq g \implies \dom f = \dom g$ and $\cod f = \cod g$, and $f \simeq g \implies f h \simeq g h$ and $k f \simeq k g$ whenever the composites are defined. Then we have a category $\mathscr{C}/\simeq$ with the same objects as $\mathscr{C}$, but congruence classes as morphisms.
    \item \index{duality}\hypertarget{def:duality}Given $\mathscr{C}$, the \textbf{opposite category} $\mathscr{C}^{op}$ has the same objects and morphisms as $\mathscr{C}$, but $\dom$ and $\cod$ are interchanged, and $f g $ in $\mathscr{C}^{op}$ is $gf$ in $\mathscr{C}$.
      This leads to the \textbf{Duality principle} if $P$ is a true statement about categories, so is the statement $P^*$ obtained from $P$ by reversing all arrows.
    \item A \hyperlink{def:small}{small} category with one object is a \textbf{monoid}, i.e.\ a semigroup with $1$. In particular, a group is a small category with one object, in which every morphism is an isomorphism (i.e.\ for all $f$, $\exists g$ such that $fg$ and $gf$ are identities).
    \item A \textbf{groupoid} is a category in which every morphism is an isomorphism. For a topological space $X$, the fundamental groupoid $\pi(X)$ has all points of $X$ as objects and morphisms $x \to y$ are homotopy classes rel $\{0,1\}$ of paths $u: [0,1] \to X$ with $u(0) = x$, $u(1) = y$.
      (If you know how to prove that the fundamental group is a group, you can prove that $\pi(X)$ is a groupoid.)
    \item A \textbf{discrete} category is one whose only morphisms are identities. A \textbf{preorder} is a category $\mathscr{C}$ in which, for any pair $(A,B)$ there is at most $1$ morphism $A \to B$.
      A small preorder is a set equipped with a binary relation which is reflexive and transitive.
      In particular, a partially ordered set is a small preorder in which the only isomorphisms are identities.
    \item The category \textbf{Rel} has the same objects as \textbf{Set}, but morphisms $A \to B$ are arbitrary relations $R \subseteq A \times B$.
      Given $R$ and $S \subseteq B \times C$, we define
      \begin{equation*}
        S \circ R = \set{(a,c) \in A \times C | (\exists b \in B) ((a,b) \in R \wedge (b,c) \in S)}.
      \end{equation*}
      The identity $1_A: A \to A$ is $\set{(a,a) | a \in A}$.

      Similarly, the category \textbf{Part} of sets and partial functions (i.e.\ relations such that $(a,b) \in R, (a,b') \in R \implies b = b'$).
    \item Let $K$ be a field. The category $\mathbf{Mat}_K$ has natural numbers as objects, and morphisms $n \to p$ are $(p \times n)$ matrices with entries from $K$.
      Composition is matrix multiplication.
  \end{enumerate}
\end{nexample}
\begin{ndef}[Functor]\index{functor}\hypertarget{def:funct}
  Let $\mathscr{C}, \mathscr{D}$ be \hyperlink{def:cat}{categories}. A \textbf{functor} $F: \mathscr{C} \to \mathscr{D}$ consists of
  \begin{enumerate}[label=(\alph*)]
    \item a mapping $A \mapsto FA$ from $\ob \mathscr{C}$ to $\ob \mathscr{D}$
    \item a mapping $f \mapsto Ff$ from $\mor \mathscr{C}$ to $\mor \mathscr{D}$
  \end{enumerate}
  such that $\dom(Ff) = F (\dom f)$, $\cod(F f)=F(\cod f)$, $1_{FA} = F(1_A)$ and $(Ff)(Fg) = F(fg)$ whenever $fg$ is defined.
\end{ndef}
\begin{manualeg}{1.3}[\emph{Continued}]\leavevmode
  \begin{enumerate}[label=(\alph*)]\setcounter{enumi}{10}
    \item We write \textbf{Cat} for the category whose objects are all \hyperlink{def:small}{small} \hyperlink{def:cat}{categories}, and whose morphisms are \hyperlink{def:funct}{functors} between them.
  \end{enumerate}
\end{manualeg}
\begin{nexample}\leavevmode
  \begin{enumerate}[label=(\alph*)]
    \item \index{functor!forgetful}\hypertarget{def:forgFunc}We have \textbf{forgetful \hyperlink{def:funct}{functors}} $\mathbf{\hyperlink{def:categ}{Gp}} \xrightarrow{U} \mathbf{Set}$, $\mathbf{Rng} \to \mathbf{Set}$, $\mathbf{Top} \to \mathbf{Set}$, $\mathbf{Rng} \to \mathbf{Ab Gp}$ (forgetting $\times$), $\mathbf{Rng} \to \mathbf{Mon}$ (forgetting $+$).
    \item Given a set $A$, the free group $FA$ has the property: given any group $G$ and any function $A \xrightarrow{f} UG$, there's a unique homomorphism $FA \xrightarrow{f} G$ extending $f$.
      $F$ is a functor $\mathbf{Set} \to \mathbf{Gp}$: given $A \xrightarrow{f} B$, we define $Ff$ to be the unique homomorphism extending $A \xrightarrow{f} B \hookrightarrow UFB$.

      Functoriality follows from uniqueness: given $B \xrightarrow{g} C$, $F(gf)$ and $(Fg)(Ff)$ are both homoms extending $A \xrightarrow{f} B \xrightarrow{g} C\hookrightarrow UFC$.
    \item Given a set $A$, we write $PA$ for the set of all subsets of $A$. We can make $P$ into a functor $\mathbf{Set} \to \mathbf{Set}$: given $A \xrightarrow{f} B$, we define $Pf(A') = \set{f(a) | a \in A'}$ for $A' \subseteq A$.
      But we also have a functor $P^*: \mathbf{Set} \to \mathbf{Set}^{op}$ defined on objects by $P$, but $P^*f(B') = \set{a \in A | f(a) \in B'}$ for $B' \subseteq B$.

      \index{functor!contravariant}\hypertarget{def:contrFunct}By a \textbf{contravariant} functor $\mathscr{C} \to \mathscr{D}$. we mean a \hyperlink{def:funct}{functor} $\mathscr{C} \to \mathscr{D}^{\hyperlink{def:duality}{op}}$ (or $\mathscr{C}^{op} \to \mathscr{D}$). (A \textbf{covariant} functor is one that doesn't reverse arrows).
    \item Let $K$ be a field. We have a functor $*: \mathbf{Mod}_K \to \mathbf{Mod}_K^{op}$ defined by $V^* = \{\text{linear maps }V \to K\}$ and if $V \xrightarrow{f} W$, $f^*(\theta:W \to K) = \theta f$.
    \item We have a functor $op: \mathbf{Cat} \to \mathbf{Cat}$ which is the `identity' on morphisms. (Note that this is \hyperlink{def:contrFunct}{covariant}).
    \item A functor between monoids is a monoid homomorphism.
    \item A functor between posets is an order-preseving map.
    \item Let $G$ be a group. A functor $F: G \to \mathbf{Set}$ consists of a set $A = F*$ together with an action of $G$ on $A$, i.e.\ a permutation representation of $G$. (Use $*$ to refer to the unique object of the group).
      Similarly a functor $G \to \mathbf{Mod}_K$ is a $K$-linear representation of $G$.
    \item The construction of a the fundamental group $\pi_1(X,x)$ of a space $X$ with basepoint $x$ is a functor $\mathbf{Top}_* \to \mathbf{Gp}$ where $\mathbf{Top}_*$ is the set of spaces with a chosen basepoint.
      Similarly, the fundamental groupoid is a functor $\mathbf{Top} \to \mathbf{Gpd}$ where $\mathbf{Gpd}$ is the category of groupoids and functors between them.
  \end{enumerate}
\end{nexample}

\begin{ndef}[Natural transformation]\index{natural transformations}\hypertarget{def:nattrans}
  Let $\mathscr{C}, \mathscr{D}$ be \hyperlink{def:cat}{categories} and $F,G: \mathscr{C} \rightrightarrows \mathscr{D}$ two \hyperlink{def:funct}{functors}.
  A \textbf{natural transformation} $\alpha: F \to G$ consists of an assignment $A \mapsto \alpha_A$ from $\ob \mathscr{C}$ to $\mor \mathscr{D}$, such that $\dom \alpha_A \ FA$ and $\cod \alpha_A = GA$ for all $A$, and for all $A \xrightarrow{f} B$ in $\mathscr{C}$ the square
  \begin{equation*}
    \begin{tikzcd}
      FA \rar{Ff} \dar{\alpha_a} & FB \dar{\alpha_B} \\
      GA \rar{Gf} & GB
    \end{tikzcd}
  \end{equation*}
  commutes (i.e.\ $\alpha_B(Ff) = (Gf) \alpha_A$).
\end{ndef}
\begin{manualeg}{1.3}[\emph{Continued}]\leavevmode
  \begin{enumerate}[label=(\alph*)]\setcounter{enumi}{11}
    \item Given categories $\mathscr{C}, \mathscr{D}$, we write $[\mathscr{C}, \mathscr{D}]$ for the category whose objects are functors$\mathscr{C}\to \mathscr{D}$, and whose morphism are \hyperlink{def:nattrans}{natural transformations}.
  \end{enumerate}
\end{manualeg}
\begin{nexample}
  \begin{enumerate}[label=(\alph*)]

    \item Let $K$ be a field, $V$ a vector space over $K$. There is al linear map $\alpha_V: V \to V^{**}$ given by
      \begin{equation*}
        \alpha_V(v)(\theta) = \theta(v)
      \end{equation*}
      for $\theta \in V^*$.
      This is the $V$-component of a natural transformation
      \begin{equation*}
        1_{\mathbf{Mod}_K} \to ** : \mathbf{Mod}_K \to \mathbf{Mod}_K
      \end{equation*}
  \end{enumerate}
\end{nexample}
\printindex
\end{document}
